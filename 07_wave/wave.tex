\documentclass[class=article, crop=false, 12pt]{standalone}
\usepackage[subpreambles=true]{standalone}
\usepackage{../.common/common}


\author{Tony Shing}
%\pretitle{Supplementary}

\topic{Note 07 (Mechanics)}
\title{Wave Equation}

\version{2025} % leave blank for omitting

\begin{document}

\maketitle


\begin{overview}

    \begin{center}
        \red{\bf{\ul{The wave equation is a partial differential equation}}}
        \aleq{
            \pdvv[2]{x}y(x,t) = \inv{v^2}\pdvv[2]{t}y(x,t)
        }
        is an equation of $y(x,t)$ -  
        the wave's magnitude as a function of position $x$ and time $t$.\\
        We are going to derive and study its solutions.
    \end{center}

    \begin{itemize}
        \item Derive wave equation - transverse and longitudinal wave
        \item Initial value problem \it{\scriptsize (Not so important)}
        \item Boundary value problem - \cul[red]{What are "Modes" of standing wave}
        \it{\red{\scriptsize (Main focus)}}
    \end{itemize}
\end{overview}



% content begins here
% Section %%%%%%%%%%%%%%%%%%%%%%%%%%%%%%%%%%%%%%%%%%%%%%%%%%%%
\section{Model of Transverse Wave}

We usually use an elastic string to visualize transverse wave travel.

\begin{itemize}
    \item When the string lies flat - Each string segment has a width $\Delta x$.
    
    \insertFig{flat string}
    \item When the string shakes - The segment jumps up and down,
    horizontal length remains the same, but gain a vertial length.
    
    \insertFig{curve string}
\end{itemize}

Transverse wave is \red{height of string segments} at different position / time,
which is described by the function $y(x,t)$.\\

\bf{\ul{Deriving wave eqauation}}

\begin{enumerate}
    \item Equations of forces by Newton's \nth{2} Law
    
    \insertFig{tension}

    Tension \blue{$\vvec{F}$} must be a function of $x$ because 
    it must be different everywhere along the string.\\

    Separate the tensions' horizontal and vertical components:
    \aleq{
        \bcase{
            \rightarrow : &\quad F_\rightarrow(x+\Delta x,t) - F_\rightarrow(x,t) = \tkn{horzT0}{\cul[red]{0}} \\
            \uparrow : &\quad F_\uparrow(x+\Delta x, t) - F_\uparrow(x,t) = (\tkn{mu}{\cul[blue]{\mu}}\Delta x) a_\uparrow
        }
    }
    \addArrow[red]{horzT0}{(10ex,0)}{\scriptsize Horizontal acceleration = 0\\[-1ex]\scriptsize because the string segment\\[-1ex]\scriptsize only jump up \& down}
    {(1.5ex,0.5ex)}{(8ex,0)}
    \addArrow[blue]{mu}{(0,-3ex)}{\scriptsize $\mu = $ Density per unit length\\[-1ex]\scriptsize $\Rightarrow \mu\Delta x = $ Mass of the string segment}
    {(0,-1.5ex)}{(0,-1ex)}


    \hfill\\[1em]
    \bf{\ul{Note}}: There must be no gravity, or the \nth{2} law becomes
    \aleq{
        F_\uparrow(x+\Delta x, t) - F_\uparrow(x,t) - \cus[blue]{(\mu \Delta x)g}{\text{Extra term}}= (\mu \Delta x) a_\uparrow
    }
 
    \item Analysis by the string's geometry
    
    \insertFig{string geometry}

    Tension \blue{$\vvec{F}$} must be parallel to the slope 
    at the 2 end points of the segment.\\

    Beause the graph of string's height variation = the graph of $y(x,t)$ 
    at some fix time $t$, 
    \aleq{
        \text{Slope of the graph}=\pdvv{x}y(x,t) 
    }

    And the inclination of tension $\vvec{F}$ can be calculated as $\dfrac{F_\uparrow}{F_\rightarrow}$.
    \aleq{
        \Rightarrow \text{ Relation at end points : } \bcase{
            \frac{F_\uparrow(x+\Delta x, t)}{F_\rightarrow(x+\Delta x, t)} &= \pdvv{x}y(x,t)\eval_{\text{at }x+\Delta x}\\
            \frac{F_\uparrow(x, t)}{F_\rightarrow(x, t)} &= \pdvv{x}y(x,t)\eval_{\text{at }x}\\
        }
    }

    \item Substitute the above two results:
    \addArrow[green]{Fleftright}{(4ex,9.5ex)}
    {\scriptsize Can be grouped together\\[-1ex]\scriptsize because they are equal.\\[-1ex]\scriptsize This is from the Newton's \nth{2} Law\\[-1ex]\scriptsize for horizontal direction}
    {(-0.5ex,3ex)}{(-18ex,-6ex)}
    \addArrow[green]{Fleftright}{(50ex,9.5ex)}{}
    {(0.5ex,3ex)}
    \addArrow[red]{dt2y}{(5ex,10ex)}
    {\scriptsize Vertical acceleration\\[-1ex]\scriptsize = \nth{2} derivative of \\[-1ex]\scriptsize segment's height over $t$}
    {(0,3ex)}{(-14ex,-5ex)}
    \addBentArrow*[blue]{dx2y}{(12.5ex,6ex)}
    {\scriptsize This is just the derivative $\frac{f(x+\Delta x)-f(x)}{\Delta x}$\\[-1ex]\scriptsize $\Rightarrow$ become \nth{2} derivative over $x$}
    {(6ex,0.5ex)}{(12ex,-5ex)}
    \aleq{
        (\mu\Delta x)a_\uparrow &= F_\uparrow(x+\Delta x, t) - F_\uparrow(x,t)\\
        %
        &= \cul[green]{F_\rightarrow(x+\Delta x, t)}\qty[\pdvv{x}y(x,t)\eval_{\text{at }x+\Delta x}] 
            - \cul[green]{F_\rightarrow(x,t)}\qty[\pdvv{x}y(x,t)\eval_{\text{at }x}]\\[2em]
        %
        &= \tkn{Fleftright}{\cul[green]{F_\leftrightarrow}}\qty[\pdvv{x}y(x,t)\eval_{\text{at }x+\Delta x} 
            - \pdvv{x}y(x,t)\eval_{\text{at }x}]\\
        %
        \mu \cul[red]{a_\uparrow} &= {F_\leftrightarrow}\cub[blue]{\qty[\frac{\pdvv{x}y(x,t)\eval_{\cbox[blue]{\scriptstyle \text{at }x+\Delta x}} 
            - \pdvv{x}y(x,t)\eval_{\cbox[blue]{\scriptstyle \text{at }x}}}{\cbox[blue]{\Delta x}}]}{}\\
        %
        \mu\tkn{dt2y}{\cul[red]{\pdvv[2]{t}y(x,t)}} &= F_\leftrightarrow \tkn{dx2y}{\cul[blue]{\pdvv[2]{x}y(x,t)}}\\
        %
        \Aboxed{
            \frac{\mu}{F_\leftrightarrow} \pdvv[2]{t}y(x,t) &= \pdvv[2]{x}y(x,t)
        }
    }
    

\end{enumerate}

Compare with the general form of wave equation $\dinv{v^2}\pdvv[2]{t}y(x,t) = \pdvv[2]{x}y(x,t)$,
we can identify the wave speed of transverse wave as 
\aleq{
    v=\sqrt{\frac{F_\leftrightarrow}{\mu}} = \sqrt{\frac{\text{Horizontal Tension}}{\text{Mass per Length}}}
}


\linesep
% Section %%%%%%%%%%%%%%%%%%%%%%%%%%%%%%%%%%%%%%%%%%%%%%%%%%%%
\section{Model of Longitudinal Wave}

We usually use a slinky to visualize longitudinal wave travel.

\begin{itemize}
    \item When the slinky is static - Each peak are of equal spacing.
    
    \insertFig{static slinky}
    \item When the slinky shakes - the peaks become unevenly distributed.
    
    \insertFig{shake slinky}
\end{itemize}

Longitudinal wave is \red{displacement of slinky segments} at different position / time,
which is described by the function $\Psi(x,t)$.\\

\bf{\ul{Deriving wave eqauation}}

\begin{enumerate}

    \item Although the slinky is continuous, 
    we can divide it into many very small segments and only look at the motions of the center of these segments - 
    using their centers as nodes to represent the motion of the segment.
    Under this approximation, we assume

    \begin{itemize}
        \item The segment is "rigid". 
        The whole segment moves at the same velocity / acceleration as its center node.

        \insertFig{rigid segment}

        \item Elastic forces are only between the center nodes. 
        Think of it as a spring-mass system composed of many nodes.

        \insertFig{force is only between node}

    \end{itemize}


    \item Consider 3 neighbouring nodes.
    When a longitudinal wave is travelling through, 
    their displacements are described by the function $\Psi(x,t)$. 
    We can compute the change of separation between nodes.

    \hfill\\[-3em]
    \begin{center}
        \begin{tikzpicture}
            \matrix(m)[matrix of nodes,
                nodes={align=center},
                %row 1/.style={anchor=south},
                %column 1/.style={nodes={text width=2cm,align=right}}
            ]{
                Node's position & Left node & Center node & Right node \\
                $\mstack{\text{When static}}$ & $x-\Delta x$ & $x$ & $x+\Delta x$ \\
                $\mstack{\text{When a wave is}\\\text{travelling through}}$ & $x-\Delta x +\yellow{\Psi(x-\Delta x,t)}$ 
                    & $x+\yellow{\Psi(x,t)}$ & $x+\Delta x + \yellow{\Psi(x+\Delta x,t)}$\\
            };
            \draw (m-1-1.south west) to (m-1-4.south east);
            \draw (m-1-1.north east -| m-3-1.south east) to (m-3-1.south east);
            \draw (m-1-2.north east -| m-3-2.south east) to (m-3-2.south east);
            \draw (m-1-3.north east -| m-3-4.south west) to (m-3-4.south west);
            %
            \draw[draw=blue] ($(m-3-2.south) + (-8ex,0)$) to ($(m-3-2.south) + (-6ex,-2ex)$) 
                to node[pos=0.5,blue,below]{\begin{tabular}{c} Their distance in-between\\[-1.5ex] increases by\\[-1.5ex] $\Psi(x,t)-\Psi(x-\Delta x, t)$ \end{tabular}} 
                ($(m-3-3.south) + (-4ex,-2ex)$) to ($(m-3-3.south) + (-2ex,0)$);
            \draw[draw=blue] ($(m-3-4.south) + (8ex,0)$) to ($(m-3-4.south) + (6ex,-2ex)$) 
                to node[pos=0.5,blue,below]{\begin{tabular}{c} Their distance in-between\\[-1.5ex] increases by\\[-1.5ex] $\Psi(x+\Delta x,t)-\Psi(x, t)$ \end{tabular}} 
                ($(m-3-3.south) + (4ex,-2ex)$) to ($(m-3-3.south) + (2ex,0)$);
        \end{tikzpicture}
    \end{center}

    \item The elastic force on the center node is proportional to the separation change with its neighbouring nodes,
    just like the elastic force in spring, $F=-k(\Delta L)$.  
    But here we express the elastic force using \bf{Young's modulus}:
    \aleq{
        (\text{Elastic force}) = F = -Y\cdot \qty(\frac{\Delta L}{L}) = -Y\cdot \qty(\frac{\text{Change in length}}{\text{Original length}})
    }

    So the elastic forces on the two sides of the center node are:
    \aleq{
        \bcase{
            F_L &= -Y\cdot \frac{\Psi(x,t)-\Psi(x-\Delta x, t)}{\Delta x}\\[1ex]
            %
            F_R &= -Y\cdot \frac{\Psi(x+\Delta x,t)-\Psi(x, t)}{\Delta x}
        }
    }
    


    \item The Newton's \nth{2} Law on the center segment's center is therefore
    
    \insertFig{free body center segment}

    \addArrow[red]{dt2psi}{(2ex,4ex)}
    {\scriptsize Horizontal acceleration\\[-1ex]\scriptsize = \nth{2} derivative of \\[-1ex]\scriptsize segment's displacement over $t$}
    {(2ex,3ex)}{(-15ex,0)}
    \addArrow[yellow]{mu2}{(-2ex,-4ex)}
    {\scriptsize $\mu = $ Density per unit length\\[-1ex]\scriptsize $\Rightarrow \mu\Delta x = $ Mass of the segment}
    {(0,-1ex)}{(-3ex,-1ex)}
    \addArrow[green]{dxpsi1}{(3ex,5.5ex)}{\scriptsize \nth{1} derivative of $x$}{(0,3ex)}{(8ex,-3ex)}
    \addArrow[green]{dxpsi2}{(12ex,5.5ex)}{}{(0,3ex)}
    \addBentArrow*[blue]{dx2psi}{(15ex,6ex)}
    {\scriptsize This is just the derivative $\frac{f(x+\Delta x)-f(x)}{\Delta x}$\\[-1ex]\scriptsize $\Rightarrow$ become \nth{2} derivative over $x$}
    {(4ex,0.5ex)}{(12ex,-5ex)}
    \aleq{
        m\cul[red]{a_\rightarrow} &= F_L - F_R \\[1em]
        %
        (\tkn{mu2}{\cul[yellow]{\mu}}\Delta x)\tkn{dt2psi}{\cul[red]{\pdvv[2]{t}\Psi(x,t)}} 
            &= Y \qty[\cul[green]{\frac{\Psi(x+\Delta x,t)-\Psi(x, t)}{\Delta x}} - \cul[green]{\frac{\Psi(x,t)-\Psi(x-\Delta x, t)}{\Delta x}}]\\[1em]
        %
        &= Y \qty[\tkn{dxpsi1}{\cul[green]{\pdv{x}\Psi(x,t)\eval_{\text{at }x+\Delta x}}} - \tkn{dxpsi2}{\cul[green]{\pdv{x}\Psi(x,t)\eval_{\text{at }x}}}] \\[1em]
        %
        \mu \pdvv[2]{t}\Psi(x,t) 
        &= Y \cub[blue]{\qty[\frac{\pdv{x}\Psi(x,t)\eval_{\cbox[blue]{\text{at }x+\Delta x}} 
            - \pdv{x}\Psi(x,t)\eval_{\cbox[blue]{\text{at }x}}}{\cbox[blue]{\Delta x}}]}{} \\
        %
        &= Y \pdvv[2]{x}\tkn{dx2psi}{\cul[blue]{\Psi(x,t)}}\\[1ex]
        %
        \Aboxed{
            \frac{\mu}{Y}\pdvv[2]{t}\Psi(x,t) &= \pdvv[2]{x}\Psi(x,t)
        }
    }
    
    
\end{enumerate}

Compare with the general form of wave equation $\dinv{v^2}\pdvv[2]{t}y(x,t) = \pdvv[2]{x}y(x,t)$,
we can identify the wave speed of longitudinal wave as 
\aleq{
    v=\sqrt{\frac{\mu}{Y}} = \sqrt{\frac{\text{Mass per Length}}{\text{Young's modulus}}}
}


\begin{notation}[Side note:]

    Spring constant $k$ depends on the length of the material. 
    For example, we can compute the equivalent spring constants for two springs in series to be half of the original.

    \insertFig{spring in series}

    If we stack more springs in series, we can see that $k\propto \qty(\inv{\text{Length of material}})$. 
    To remove this dependency on length, we define the Young's modulus, 
    which is a property that only depends on the material's type. 
    \aleq{
        \Aboxed{
            F = -k(\Delta L) = -\frac{Y}{L}(\Delta L) 
        }
    }


\end{notation}

\linesep
% Section %%%%%%%%%%%%%%%%%%%%%%%%%%%%%%%%%%%%%%%%%%%%%%%%%%%%
\section{Wave Equation \& Initial Value Problem}

The initial value problem is asking the follow: 
If we are told the state of a system at the start, 
how will system evolve at later time?\\

For example in wave propagation, 
given that at $t=0$, 
a string is hold to a shape described by the function $\Psi(x,0)$. 
After releasing, how will the waveform evolves? 

\insertFig{release string}

i.e. We would like to solve $\cub[blue]{\Psi(x,t)}{\substack{\text{the waveform}\\\text{in the future}}}$ 
by the given $\cub[blue]{\Psi(x,0)}{\substack{\text{the waveform}\\\text{at the start}}}$ 
and $\cub[blue]{\pdvv{t}\Psi(x,t)\eval_{t=0}}{\substack{\text{velocity at each point}\\\text{at the start}}}$.


%%%%%%%%%%%%%%
\subsection{General Solution to Wave Equation}

The wave equation is a \bf{partial differential equation} (PDE):
\aleq{
    \pdvv[2]{x}\Psi(x,t) = \dinv{v^2}\pdvv[2]{t}\Psi(x,t)
}

We can show that the general solution is
\aleq{
    \Aboxed{
        \Psi(x,t) = f(x+vt) + g(x-vt)
    }
}

where $f(\cdots)$ and $g(\cdots)$ are any single variable function,
and then we substitute $x+vt$ or $x-vt$ as the inputs.
For example,
\aleq{
    f(u) = \sin u + u^2 \quad\Rightarrow\quad f(x+vt) = \sin(x+vt) + (x+vt)^2
}

\begin{proof}
    By differentiation with chain rule. 
    \begin{center}
        \begin{minipage}{0.45\textwidth}
            \centering
            \ul{L.H.S.}
            \aleq{
                \pdvv{x} f(x+vt) 
                &=  \pdvv{f(u)}{u}\eval_{u=x+vt}\pdvv{(x+vt)}{x} \\[1ex]
                &= \pdvv{f(u)}{u}\eval_{u=x+vt} \cdot 1 \\[1em]
                %
                \pdvv[2]{x} f(x+vt) 
                &=  \pdvv[2]{f(u)}{u}\eval_{u=x+vt}\pdvv{(x+vt)}{x} \\[1ex]
                &= \pdvv[2]{f(u)}{u}\eval_{u=x+vt} \cdot 1
            }
        \end{minipage}
        \hfill
        \vline
        \hfill
        \begin{minipage}{0.45\textwidth}
            \centering
            \ul{R.H.S.}
            \aleq{
                \dinv{v^2}\pdvv{t} f(x+vt) 
                &= \dinv{v^2}\pdvv{f(u)}{u}\eval_{u=x+vt}\pdvv{(x+vt)}{t} \\[1ex]
                &= \dinv{v^2}\pdvv{f(u)}{u}\eval_{u=x+vt} \cdot v \\[1em]
                %
                \dinv{v^2}\pdvv[2]{t} f(x+vt) 
                &=  \dinv{v}\pdvv[2]{f(u)}{u}\eval_{u=x+vt}\pdvv{(x+vt)}{t} \\[1ex] 
                &= \dinv{v}\pdvv[2]{f(u)}{u}\eval_{u=x+vt} \cdot v
            }
        \end{minipage}
    \end{center}
    \hfill\\
    Obviously L.H.S = R.H.S.. You can also prove the same for $g(x-vt)$.
\end{proof}


\bf{\ul{Physical Interpretation}}\\

\begin{itemize}
    \item $f(x+vt) =$ A waveform travelling to the left ($-x$ direction).
    \item $g(x-vt) =$ A waveform travelling to the right ($+x$ direction).
\end{itemize}

When they are added together, they form superposition.

\insertFig{wave superposition}





%%%%%%%%%%%%%%
\subsection{Solution to Initial Value Problem}

\it{(The derivation is long but not important. Welcome to skip to the result.)}

\begin{enumerate}
    \item From the initial conditions, break them down by $\Psi=f+g$.
    \aleqr{
        \Psi(x,0) &= f(x+0) + g(x-0) \label{eq:fg0}\\[1ex]
        \pdvv{t}\Psi(x,t)\eval_{t=0} &= v\dvv{f(u)}{u}\eval_{u=x+0} - v\dvv{g(u)}{u}\eval_{u=x-0} \label{eq:dfg0}
    }

    \item Differentiate Eq.\eqref{eq:fg0} with respect to $x$:
    \aleqr{
        \dvv{x}\Psi(x,0) = \dvv{f(u)}{u}\eval_{u=x+0} + \dvv{g(u)}{u}\eval_{u=x-0} \label{eq:fg1}
    }

    \item Isolate $\dvv{f(u)}{u}\eval_{u=x+0}$ and $\dvv{g(u)}{u}\eval_{u=x-0}$ 
    from Eq.\eqref{eq:fg1} and Eq.\eqref{eq:dfg0}.
    \aleqr{
        \text{Eq.\eqref{eq:fg1}} +\dinv{v} (\text{Eq.\eqref{eq:dfg0}})
        &\qquad\Rightarrow\qquad
        \dvv{f(u)}{u}\eval_{u=x+0} = \half\qty[\dvv{x}\Psi(x,0) + \dinv{v}\pdvv{t}\Psi(x,t)\eval_{t=0}]
        \label{eq:df0}
        \\[1em]
        %
        \text{Eq.\eqref{eq:fg1}} -\dinv{v} (\text{Eq.\eqref{eq:dfg0}})
        &\qquad\Rightarrow\qquad
        \dvv{g(u)}{u}\eval_{u=x-0} = \half\qty[\dvv{x}\Psi(x,0) - \dinv{v}\pdvv{t}\Psi(x,t)\eval_{t=0}]
        \label{eq:dg0}
    }

    \item Integrate both Eq.\eqref{eq:df0} and Eq.\eqref{eq:dg0}.
    \aleq{
        \text{Eq.\eqref{eq:df0}} 
        \qquad\Rightarrow\qquad 
        \int \dvv{f(u)}{u}\eval_{u=x+0} \dd{x} 
        &= \half\int\qty[\cul[red]{\dvv{x}\Psi(x,0)} + \dinv{v}\pdvv{t}\Psi(x,t)\eval_{t=0}]\dd{x} \\[1.5em]
        %
        \int \dvv{f(x)}{x} \dd{x} 
        &= \half\qty[\tkn{C1}{\cul[red]{\Psi(x,0) + C_1}} + \dinv{v}\int\qty[\pdvv{t}\Psi(x,t)\eval_{t=0}]\dd{x}] \\[1em]
        %
        f(x) &= \half \biggl[\Psi(x,0) + C_1 
            + \dinv{v}\cub[green]{\int_{\green{s=0}}^{\green{s=x}}\qty[\pdvv{t}\Psi(\green{s},t)\eval_{t=0}]\dd{\green{s}}}
            {\substack{s \text{ = A dummy variable to replace }x\\\text{For convenience in later steps}}}\biggr]\\[2em]
        %
        \text{Eq.\eqref{eq:dg0}} 
        \qquad\Rightarrow\qquad 
        \int \dvv{g(u)}{u}\eval_{u=x-0} \dd{x} 
        &= \half\int\qty[\cul[red]{\dvv{x}\Psi(x,0)} - \dinv{v}\pdvv{t}\Psi(x,t)\eval_{t=0}]\dd{x} \\[1.5em]
        %
        \int \dvv{g(x)}{x} \dd{x} 
        &= \half\qty[\tkn{C2}{\cul[red]{\Psi(x,0) + C_2}} - \dinv{v}\int\qty[\pdvv{t}\Psi(x,t)\eval_{t=0}]\dd{x}] \\[1em]
        %
        g(x) &= \half \biggl[\Psi(x,0) + C_2 
            - \dinv{v}\cub[green]{\int_{\green{s=0}}^{\green{s=x}}\qty[\pdvv{t}\Psi(\green{s},t)\eval_{t=0}]\dd{\green{s}}}
            {\substack{s \text{ = A dummy variable to replace }x\\\text{For convenience in later steps}}}\biggr]
    }
    \addArrow[red]{C1}{(0,6ex)}{\scriptsize $C_1 = $ some integration constant}
    {(2ex,3ex)}{(12ex,-3.5ex)}
    \addArrow[red]{C2}{(0,6ex)}{\scriptsize $C_2 = $ some integration constant}
    {(2ex,3ex)}{(12ex,-3.5ex)}

    \item Replace the "$x$" in $f(x)$ by "$x+vt$", and the "$x$" in $g(x)$ by "$x-vt$".
    \aleq{
        f(\blue{x+vt}) &= \half \qty[\Psi(\blue{x+vt},0) + C_1 
            + \dinv{v}\int_{s=0}^{s=\blue{x+vt}}\qty[\pdvv{t}\Psi(s,t)\eval_{t=0}]\dd{s}]\\[1em]
        %
        g(\blue{x-vt}) &= \half \qty[\Psi(\blue{x-vt},0) + C_1 
            - \dinv{v}\int_{s=0}^{s=\blue{x-vt}}\qty[\pdvv{t}\Psi(s,t)\eval_{t=0}]\dd{s}]
    }

    \item Add these two expression together to yield $\Psi(x,t)$
    \aleq{
        \Psi(x,t) &= f(x+vt) + g(x-vt) \\
        %
        &= \half\qty[\Psi(x+vt,0) + \Psi(x-vt,0)] + \half (C_1+C_2) \\
        &\qquad + \dinv{2v} \int_{s=0}^{s=x+vt} \qty[\pdvv{t}\Psi(s,t)\eval_{t=0}]\dd{s} 
            \cub[yellow]{- \dinv{2v} \int_{s=0}^{s=x-vt}}{\substack{\text{Can switch}\\\text{upper/lower bound}\\\text{and change sign}}} 
                \qty[\pdvv{t}\Psi(s,t)\eval_{t=0}]\dd{s}\\
        %
        &= \half\qty[\Psi(x+vt,0) + \Psi(x-vt,0)] + \half (C_1+C_2) \\
        &\qquad + \dinv{2v} \int_{s=0}^{s=x+vt} \qty[\pdvv{t}\Psi(s,t)\eval_{t=0}]\dd{s} 
            \yellow{+} \dinv{2v} \int_{\yellow{s=x-vt}}^{\yellow{s=0}} \qty[\pdvv{t}\Psi(s,t)\eval_{t=0}]\dd{s}\\[1.5em]
        %
        &= \half\qty[\Psi(x+vt,0) + \Psi(x-vt,0)] + \half (C_1+C_2) + \dinv{2v}\tkn{combine_bound}{\cul[yellow]{\int_{s=x-vt}^{s=x+vt}}} \qty[\pdvv{t}\Psi(s,t)\eval_{t=0}]\dd{s} 
    }
    \addArrow[yellow]{combine_bound}{(-8ex,8ex)}{\scriptsize Combine integral by their bounds}
    {(-3ex,2ex)}{(18ex,-4.5ex)}
    \addArrow[yellow]{combine_bound}{(-56ex,9ex)}{}{(-3.5ex,1ex)}{(19ex,-2.5ex)}

    \item Finally, substitute $t=0$ to find out what $C_1+C_2$ is.
    \aleq{
        \Psi(x,\red{0}) &= \half\qty[\Psi(x+\red{0},\red{0}) + \Psi(x-\red{0},\red{0})] 
            + \half (C_1+C_2) + \dinv{2v}\int_{s=x-\red{0}}^{s=x+\red{0}} \qty[\pdvv{t}\Psi(s,t)\eval_{t=0}]\dd{s} \\[1em]
        %
        &= \half\qty[\Psi(x,\red{0}) + \Psi(x,\red{0})] + \half (C_1+C_2) + 0 \\[1em]
        %
        C_1+C_2 &= 0
    }

\end{enumerate}

Finally, we reach the solution to the initial value problem of wave equation.
\aleq{
    \Aboxed{
        \cub[blue]{\Psi(x,t)}{\substack{\text{the waveform}\\\text{in the future}}} 
        = \half\cub[red]{\qty[\Psi(x+vt,0) + \Psi(x-vt,0)]}
            {\substack{\text{Derived from the}\\\text{initial waveform }\Psi(x,0)}} 
            + \dinv{2v}\cub[green]{\int_{s=x-vt}^{s=x+vt} \qty[\pdvv{t}\Psi(s,t)\eval_{t=0}]\dd{s}}
            {\substack{\text{Derived from the}\\\text{initial velocity }\pdv{t}\Psi(s,t)\eval_{t=0}}}
    }
}


\begin{example}
    Given the initial waveform of the string as
    \aleq{
        \Psi(x,0) = \begin{cases}
            b-\dfrac{b\abs{x}}{a} & \text{for }\abs{x}\leq a \\[1ex]
            0 & \text{for }\abs{x}>a
        \end{cases}
    }
    and knowing that the string is static at the beginning (i.e. velocity $= 0$ everywhere).\\

    \insertFig{ivp eg}

    We can find how the wave will evolve by direct substituting these info into the general solution.
    \aleq{
        \Psi(x,t) &= \half\qty[\Psi(x+vt) + \Psi(x-vt)] 
            + \dinv{2v}\int_{x-vt}^{x+vt} \cut[yellow]{\ccancelto[yellow]{0}{\qty[\pdvv{t}\Psi(s,t)\eval_{t=0}]}}{\text{Because no initial velocity}}\dd{s}\\[1ex]
        %
        &= \cub[blue]{\half\qty(b-\frac{b\abs{x+vt}}{a})}{\substack{\text{This is a function of }x+vt\\\text{i.e. the waveform travelling}\\\text{in }-\text{ve direction}}} 
        + \cub[green]{\half\qty(b-\frac{b\abs{x-vt}}{a})}{\substack{\text{This is a function of }x-vt\\\text{i.e. the waveform travelling}\\\text{in }+\text{ve direction}}} 
        + 0
    }

    \insertFig{ivp eg result}

\end{example}


\linesep
% Section %%%%%%%%%%%%%%%%%%%%%%%%%%%%%%%%%%%%%%%%%%%%%%%%%%%%
\section{Boundary Value Problem \& Standing Wave}

The boundary value problem is asking the follow: 
If we are only given the value of the function at the end points and the initial state, 
how will system evolve at later time?

\insertFig{standing wave}

In a standing wave configuraton, 
a string is usually tied at both ends (or free ends). 
It is very important to know: \cul[red]{what kinds of "vibration pattern" are allowed?}

\insertFig{question mark standing wave}

And if the standing wave starts in a shape described by the function $\Psi(x,0)$. 
After releasing, how will the waveform evolves? 




%%%%%%%%%%%%%%
\subsection{The Method of Separation of Variables}

Here introduces an alternative method to solve the wave equation 
- \bf{method of separation of variables}. 
We first assume the solution to be able to be written as a product of 2 single variable function,
one as a function position $x$ and the other as a function of time $t$.
\aleq{
    \Psi(x,t) = \tkn{Xx}{\blue{X(x)}}\tkn{Tt}{\red{T(t)}}
}
\addArrow[blue]{Xx}{(-3ex,-3ex)}
{\scriptsize This part only\\[-1ex]\scriptsize depends on $x$}
{(-1ex,-1ex)}{(0,-1ex)}
\addArrow[red]{Tt}{(3ex,-3ex)}
{\scriptsize This part only\\[-1ex]\scriptsize depends on $t$}
{(1ex,-1ex)}{(0,-1ex)}


\hfill\\[1em]
Substitute into the equation,
\aleq{
    \pdvv[2]{x}[\blue{X(x)}\red{T(t)}] &= \inv{v^2}\pdvv[2]{t}[\blue{X(x)}\red{T(t)}]\\[1ex]
    %
    \tkn{Tt2}{\red{T(t)}}\pdvv[2]{x}[\blue{X(x)}] &= \inv{v^2}\tkn{Xx2}{\blue{X(x)}}\pdvv[2]{t}[\red{T(t)}]
}
\addArrow[blue]{Xx2}{(0,-3ex)}
{\scriptsize $X(x)$ not depends on $t$.\\[-0.5ex]\scriptsize Can be taken out of $\pdv[2]{t}$}
{(0,-1ex)}{(0,-1.2ex)}
\addArrow[red]{Tt2}{(0,-3ex)}
{\scriptsize $T(t)$ not depends on $x$.\\[-0.5ex]\scriptsize Can be taken out of $\pdv[2]{x}$}
{(0,-1ex)}{(0,-1.2ex)}

\hfill\\[1ex]
Rearrange so that the L.H.S is a function of $x$ only, 
and R.H.S. is a function of $t$ only.
\aleq{
    \cub[blue]{\dinv{\blue{X(x)}}\pdvv[2]{x}[\blue{X(x)}]}{\text{only contain }x} 
    &= \cub[red]{\inv{v^2}\dinv{\red{T(t)}}\pdvv[2]{t}[\red{T(t)}]}{\text{only contain t}} 
    = \tkn{k2}{\qty(\mstack{\text{Some}\\\text{Constant}})} = \green{-k^2}
}
\addBentArrow[green]{k2}{(4ex,-8ex)}
{The only possibility for two functions\\of different variables to equal\\is when both equal to a constant}
{(0,-2ex)}{(16ex,2.5ex)}


\hfill\\[1em]
The constant is written as $-k^2$ is just for convenience.
We can now split the part with $x$ and the part with $t$,
forming two independent \nth{2} order linear ODEs. 
\aleq{
    \bcase{
        \dinv{\blue{X(x)}}\pdvv[2]{x}[\blue{X(x)}] &= \green{-k^2}
        \qquad\Rightarrow\qquad 
        \pdvv[2]{x}\blue{X(x)} + \green{k^2}\blue{X(x)} = 0\\
        %
        \inv{v^2}\dinv{\red{T(t)}}\pdvv[2]{t}[\red{T(t)}] &= \green{-k^2}
        \qquad\Rightarrow\qquad 
        \pdvv[2]{t}\red{T(t)} + v^2\green{k^2}\red{T(t)} = 0\\
    }
}

You should be very farmiliar with this kind of ODE - 
the equation of motion of SHM. Their solutions are 
\aleq{
    \bcase{
        \blue{X(x)} &= C\cos{(kx)} + D\sin{(kx)} \\
        \red{T(t)} &= A\cos{(kvt)} + B\sin{(kvt)}
    }
}

where $A,B,C,D$ are constants to be determined 
from the boundary conditions and initial conditions.


%%%%%%%%%%%%%%
\subsection{Boundary Conditions \& Solutions}

In standing wave, 
the boundary condition at the end of the string is either being fixed 
or free to move up / down. 
Here introduces two most common boundary conditions,
and the corresponding solutions.

%%%%%%%%%%%%%%
\subsubsection{Dirichlet Condition}

The \bf{Dirichlet condition} in standing wave is essentially having \bf{2 fixed ends},
i.e. requires the magnitude at both ends to be fixed at $0$ (at any time $t$).
If the string is of length $L$, then it writes:
\aleq{
    \Aboxed{
        \bcase{
            \Psi(0,t) &= 0\\
            \Psi(L,t) &= 0
        }
    }
} 

\insertFig{Dirichlet}

From the conditions, we must have $X(0)=X(L)=0$.
\begin{itemize}
    \item At $x=0$, $X(0)=C\cos(0)+D\sin(0) = 0$. 
    It holds only if $C=0$.
    
    \item At $x=L$, $X(L)= D\sin(kL) = 0$. 
    It holds only if $kL=n\pi$, with $n=\tkn{dirichlet0}{\gray{0}},1,2,3,...$.\\
    \addArrow[gray]{dirichlet0}{(3ex,3ex)}
    {\scriptsize $n=0$ is technically an answer,\\[-1ex] \scriptsize but it gives $X(x)=D\sin 0 = 0$,\\[-1ex]\scriptsize meaning the string cannot shake at all.}
    {(1ex,2ex)}{(7ex,2.5ex)}
\end{itemize}

So we require $k=\dfrac{n\pi}{L}$. 
Substitute it into $X(x)$ and $T(t)$, 
\aleq{
    \bcase{
        \blue{X(x)} &= D\sin{\qty(\dfrac{n\pi}{L}x)} \\
        \red{T(t)} &= A\cos{\qty(\dfrac{n\pi}{L}vt)} + B\sin{\qty(\dfrac{n\pi}{L}vt)}
    }
}

For each integer $n=1,2,3...$, we can construct one set of $\Psi(x,t)$:
\aleq{
    \Psi_\green{n}(x,t) &= \blue{X_\green{n}(x)}\red{T_\green{n}(t)} \\
    &= \blue{\biggl[}\sin\qty(\frac{\green{n}\pi}{L}x)\blue{\biggr]}
        \red{\biggl[}A_\green{n}\cos\qty(\frac{\green{n}\pi}{L}vt) 
            + B_\green{n}\sin\qty(\frac{\green{n}\pi}{L}vt)\red{\biggr]}
}

Then by the superposition property of linear equations, 
the general solution is the (linear) combination of all possible solutions.
\aleq{
    \Aboxed{
        \Psi(x,t) = \sum_{\green{n=1}}^{\green{\infty}}
            \blue{\biggl[}\sin\qty(\frac{\green{n}\pi}{L}x)\blue{\biggr]}
            \red{\biggl[}A_\green{n}\cos\qty(\frac{\green{n}\pi}{L}vt) 
                + B_\green{n}\sin\qty(\frac{\green{n}\pi}{L}vt)\red{\biggr]}
        %
        \qquad (\text{Dirichlet Condition})
    }
}

All the constants $A_n$, $B_n$ shall be determined only after an initial condition is given.

%%%%%%%%%%%%%%
\subsubsection{Neumann Condition}

The \bf{Neumann condition} in standing wave is essentially having \bf{2 free ends} - 
the end segments are free to move up and down following string body's motion.\\

To achieve so, the ends' holdings must not exert any vertical force at the end segment 
(otherwise the ends are not freely moving with the string body.) 
Having only horizontal force on the end segments means that the slope at both ends to be $0$ (at any time $t$).
If the string is of length $L$, then it writes:
\aleq{
    \Aboxed{
        \bcase{
            \pdvv{x}\Psi(x,t)\eval_{x=0} &= 0\\
            \pdvv{x}\Psi(x,t)\eval_{x=L} &= 0
        }
    }
} 

\insertFig{Neumann ends}

From the conditions, we must have $\eval{\dvv{X(x)}{x}}_{x=0}=\eval{\dvv{X(x)}{x}}_{x=L}=0$.
\begin{itemize}
    \item At $x=0$, $\eval{\dvv{X(x)}{x}}_{x=0}=-C\sin(0)+D\cos(0) = 0$. 
    It holds only if $D=0$.
    
    \item At $x=L$, $\eval{\dvv{X(x)}{x}}_{x=L}= -C\sin(kL) = 0$. 
    It holds only if $kL=n\pi$, with $n=\tkn{neumann0}{0},1,2,3,...$.\\
    \addArrow[gray]{neumann0}{(0,-3ex)}
    {\scriptsize This time we can keep $n=0$,\\[-1ex] \scriptsize because it gives $X(x)=C\cos 0 = C$.\\[-1ex]\scriptsize Motion is retained in $T(t)$.}
    {(0,-1ex)}{(3ex,-2.5ex)}

\end{itemize}

So we require $k=\dfrac{n\pi}{L}$. 
Notice that 
\begin{itemize}
    \item when $k\neq 0$, we can substitute it into $X(x)$ and $T(t)$, 
    \aleq{
        \bcase{
            \blue{X(x)} &= C\cos{\qty(\dfrac{n\pi}{L}x)} \\
            \red{T(t)} &= A\cos{\qty(\dfrac{n\pi}{L}vt)} + B\sin{\qty(\dfrac{n\pi}{L}vt)}
        }
    }

    \item but when $k=0$, the solution of $X(t)$ and $T(t)$ become
    \aleq{
        \blue{X(x)}= C
        \qquad\text{and}\qquad
        \barray{r@{\extracolsep{0.5ex}}l}{
            \pdvv[2]{t}\red{T(t)}&= 0\\
            \Rightarrow\quad \red{T(t)}&= At + Bs
        }
    }
\end{itemize}

So for each integer $n=0,1,2,3...$, we can construct one set of $\Psi(x,t)$:
\aleq{
    \Psi_\green{n}(x,t) &= \blue{X_\green{n}(x)}\red{T_\green{n}(t)} \\
    &= \begin{cases}
        \blue{\biggl[}1\blue{\biggr]}
            \red{\biggl[}A_\green{0}t + B_\green{0}\red{\biggr]}
        & \text{for }\green{n}=0 \\[1em]
        %
        \blue{\biggl[}\cos\qty(\frac{\green{n}\pi}{L}x)\blue{\biggr]} 
            \red{\biggl[}A_\green{n}\cos\qty(\frac{\green{n}\pi}{L}vt) + B_\green{n}\sin\qty(\frac{\green{n}\pi}{L}vt)\red{\biggr]}
        &\text{for }\green{n}>0
    \end{cases}
}

Then by the superposition property of linear equations, 
the general solution is the (linear) combination of all possible solutions.
\aleq{
    \Aboxed{
        \Psi(x,t) = \red{\biggl[}A_\green{0}t + B_\green{0}\red{\biggr]}
        + \sum_{\green{n=1}}^{\green{\infty}} 
            \blue{\biggl[}\cos\qty(\frac{\green{n}\pi}{L}x)\blue{\biggr]}
            \red{\biggl[}A_\green{n}\cos\qty(\frac{\green{n}\pi}{L}vt) 
                + B_\green{n}\sin\qty(\frac{\green{n}\pi}{L}vt)\red{\biggr]}
        %
        \qquad (\text{Neumann Condition})
    }
}

All the constants $A_n$, $B_n$ shall be determined only after an initial condition is given.


\begin{exercise}
    We can carry out similar steps to obtain the general solution in other combination of boundary conditions.
    \begin{enumerate}
        \item Use $x=0$ 's condition to elimiate one of $C$ or $D$.
        \item Use $x=L$ 's condition to determine what values of $k$ can be.
        \item Substitute $k$'s value into $\Psi(x,t)=\blue{X(x)}\red{T(t)}$.
        \item The general solution is the linear combination of $\Psi(x,t)$ of all possible $k$.
    \end{enumerate}

    As a practice, you may try to derive for the case with one fixed end and one open end.

    \insertFig{1 fix 1 open}

    You should get 
    \aleq{
        \Aboxed{
            \Psi(x,t) = \sum_{\green{n=1}}^{\green{\infty}}
                \blue{\biggl[}\sin\qty(\green{\frac{2n-1}{2}}\frac{\pi}{L}x)\blue{\biggr]}
                \red{\biggl[}A_\green{n}\cos\qty(\green{\frac{2n-1}{2}}\frac{\pi}{L}vt) 
                    + B_\green{n}\sin\qty(\green{\frac{2n-1}{2}}\frac{\pi}{L}vt)\red{\biggr]}
        }
    }

\end{exercise}


%%%%%%%%%%%%%%
\subsection{Modes of Standing Wave}

Observing that the general solution is a superposition of all simplier solutions of different $n$.
Each $n$ has its corresponding $X_n(x)$ and $T_n(t)$.
\aleq{
    \Psi(x,t) = \sum_{\green{n}}\Psi_\green{n}(x,t) 
        = \sum_{\green{n}}\qty[\blue{X_\green{n}(x)}\red{T_\green{n}(t)}]
}

\begin{itemize}
    \item $\blue{X_\green{n}(x)}$ is only about variation by position $x$ 
        - Carry info about the \blue{waveform}.
    \item $\red{T_\green{n}(t)}$ is only about variation by time $t$ 
        - Carry info about the \red{time evolution}.
\end{itemize}

Each $\Psi_\green{n}(x,t)$ is \bf{a unique set of vibration pattern} in standing wave,
and they evolve independently from each other.
Therefore we call them the \bf{normal modes} of standing wave. i.e.
\aleq{
    \qty(\mstack{\text{The n\Nth mode}\\\Psi_\green{n}(x,t)})
    = \blue{\qty(\mstack{\text{Waveform of}\\\text{the n\Nth mode}\\X_\green{n}(x)})} \times 
        \red{\qty(\mstack{\text{Time evolution}\\\text{of the n\Nth mode}\\T_\green{n}(t)})}
}

We can draw the graphs fro each $\Psi_n(x,t)$ as shown:
\begin{center}
    \begin{tabular}{c|c|c|c}
        & Dirichlet (2 fixed ends) & Neumann (2 free ends) & 1 fixed-1 free end\\
        \hline
        n=0 & & & \\
        %
        n=1 & & & \\
        %
        n=2 & & &
    \end{tabular}
    \\
    $\vdots$\\
     and so on.
\end{center}

Here I shall emphasize: 
\bf{\red{The combination of standing wave's pattern (waveform) and frequency (time evolution) is fixed}}
- It is impossible to let the string vibrate in one the above pattern but with a faster/slower frequency.\\

When we encounter a wave pattern that is not in one of the normal wave's waveform,
we must first break it down into a sum of different normal modes.
For example

\insertFig{sum of mode}

Because each mode has its own vibration frequency, 
the resulted wave will not maintain a regular shape like the initial waveform.

\begin{center}
    \begin{tabular}{c|ccc}
        & $t=0$ & $t=\dfrac{L}{2v}$ & $t=\dfrac{L}{v}$ \\
        \hline
        \makecell{\green{\nth{1} mode}\\ Period $=\dfrac{2\pi}{\frac{\pi}{L}v} = \frac{2L}{v}$}
        & & & \\
        %
        \makecell{\blue{\nth{2} mode}\\ Period $=\dfrac{2\pi}{\frac{2\pi}{L}v} = \frac{L}{v}$}
        & & & \\
        %
        \red{Sum} & & &
    \end{tabular}
\end{center}



%%%%%%%%%%%%%%
\subsection{Fourier Series}

So if we are given some arbituary pattern as the initial wave form,
how can we break it down into normal modes mathematically? 
The tool is called \bf{Fourier Series}.\\

Denote a periodic function as $f(x) = f(x+L)$, 
which has a period $L$,

\insertFig{periodic func}

it can be expanded as a series of sin and cosine terms:
\aleq{
    \Aboxed{
        f(x) = \frac{{a_0}}{2} 
            + \sum_{n=1}^{\infty} \qty[{a_n}\cos\qty(\frac{2\pi n}{L}x)
            + {b_n}\sin\qty(\frac{2\pi n}{L}x)]
    }
}

The \bf{Fourier coefficients} $a_n$ (including $a_0$) and $b_n$ can be calculated by these integrals:
\begin{empheq}[box=\fbox]{align*}
    a_n &= \frac{2}{L} \int^L_0 f(x) \cdot \cos\qty(\frac{2\pi n}{L}x) \dd{x} \\[1ex]
    b_n &= \frac{2}{L} \int^L_0 f(x) \cdot \sin\qty(\frac{2\pi n}{L}x) \dd{x}
\end{empheq}

\begin{proof}
    The above formulas work thanks to these integral properties.
    For any integer $m,n$,
    \aleq{
        \int^{2\pi}_0 \cos(mx)\cos(nx) \dd{x} &= 
        \begin{cases}
            2\pi & \text{if }m=n=0\\
            \pi & \text{if }m=n\neq 0\\
            0 & \text{if }m\neq n    
        \end{cases}\\
        %
        \int^{2\pi}_0 \sin(mx)\sin(nx) \dd{x} &= 
        \begin{cases}
            0 & \text{if }m=n=0\\
            \pi & \text{if }m=n\neq 0\\
            0 & \text{if }m\neq n    
        \end{cases}\\
        %
        \int^{2\pi}_0 \sin(mx)\cos(nx) \dd{x} &= 0
    }

    i.e. the integral $=0$ whenever $m\neq n$ or the $\sin$ / $\cos$ does not match.
    Let's have a demonstration using $\cos\qty(\dfrac{2\pi n}{L}x)$ with $n\geq 1$.
    \aleq{
        &\int^L_0 f(x)\cos\qty(\dfrac{2\pi \blue{n}}{L}x) \dd{x}\\
        %
        = &\int^L_0 \qty[\frac{a_\red{0}}{2} 
            + \sum_{m=1}^{\infty} \qty[a_\red{m}\cos\qty(\frac{2\pi \red{m}}{L}x)
            + b_\red{m}\sin\qty(\frac{2\pi \red{m}}{L}x)]] \cos\qty(\dfrac{2\pi \blue{n}}{L}x) \dd{x}\\
        %
        = &\int^L_0 \biggl[
            \cub[green]{\frac{a_\red{0}}{2} \cos\qty(\dfrac{2\pi \blue{n}}{L}x)}
                {\substack{\text{Integrate }\cos\\\text{for 1 period}\\=0}} 
            + \sum_{m=1}^{\infty} a_\red{m}\cub[red]{\cos\qty(\frac{2\pi \red{m}}{L}x)\cos\qty(\dfrac{2\pi \blue{n}}{L}x)}
                {\substack{\cos \text{ and }\cos\\ \text{Integral }\neq 0 \text{ only if }m=n}}
            + \sum_{m=1}^{\infty} b_\red{m}\cub[blue]{\sin\qty(\frac{2\pi \red{m}}{L}x)\cos\qty(\dfrac{2\pi \blue{n}}{L}x)}
                {\substack{\sin \text{ and }\cos\\ \text{Integral always gives }0}}\biggr]\dd{x}\\
        %
        = &\int^L_0 a_\blue{n}\cos\qty(\dfrac{2\pi \blue{n}}{L}x)\cos\qty(\dfrac{2\pi \blue{n}}{L}x) \dd{x} \\
        %
        = &\,a_\blue{n}\cdot \frac{L}{2}
    }
    \hfill\\[-4em]
    \aleq{
        \Rightarrow \qquad a_n = \dfrac{2}{L}\int^L_0 f(x)\cos\qty(\dfrac{2\pi \blue{n}}{L}x) \dd{x} 
    }
    As an exercise, you can also prove the same formula for $b_n$.

\end{proof}


\begin{example}
    Evolution of square wave \\

    Given the function of a periodic square wave as 
    \aleq{
        f(x) = \begin{cases}
        1 &\text{for }0<x<\frac{L}{2} \\
        -1 &\text{for }\frac{L}{2}<x<L
        \end{cases}
    }

    \insertFig{square wave}

    The Fourier coefficients can be directly computed:
    \aleq{
        a_n &= \frac{2}{L}\int_0^{\frac{L}{2}} \red{1}\cdot \cos\qty(\frac{2\pi n}{L}x) \dd{x}
            + \frac{2}{L}\int_{\frac{L}{2}}^L \red{-1}\cdot \cos\qty(\frac{2\pi n}{L}x) \dd{x}\\[1ex]
        %    
        &= 0 \\[1em]
        %
        b_n &= \frac{2}{L}\int_0^{\frac{L}{2}} \red{1}\cdot \sin\qty(\frac{2\pi n}{L}x) \dd{x}
            + \frac{2}{L}\int_{\frac{L}{2}}^L \red{-1}\cdot \sin\qty(\frac{2\pi n}{L}x) \dd{x}\\[1ex]
        %
        &= \frac{2}{n\pi}[1-(-1)^n] = 
        \begin{cases}
            0 &\text{for }n=\text{even}\\
            \dfrac{4}{n\pi} &\text{for }n=\text{odd}    
        \end{cases}
    }

    So its Fourier series write as
    \aleq{
        f(x) = \frac{4}{\pi}\qty[\cul[yellow]{
            \cul[blue]{
                \tkn{sq1}{\cul[green]{\sin(\frac{2\pi}{L}\cdot x)}} 
                + \dinv{3}\tkm{sq2}\sin(\frac{2\pi}{L}\cdot 3x)} 
            + \tkm{sq3}\dinv{5}\sin(\frac{2\pi}{L}\cdot 5x)} + \dots]
    }
    \addArrow[green]{sq1}{(0,-4ex)}{First mode}{(0,-3ex)}
    \addArrow[blue]{sq2}{(0,-7ex)}{First 2 modes}{(0,-3.2ex)}
    \addArrow[yellow]{sq3}{(0,-10ex)}{First 3 modes}{(0,-3.5ex)}

    \hfill\\[3em]

    \insertFig{square modes}

    At this point, we have already obtained each $\blue{X_\green{n}(x)}$ as
    \aleq{
        \blue{X_\green{n}(x)} =  \dfrac{4}{n\pi}\sin\qty(\frac{2\pi n}{L}x)
        \qquad {\scriptstyle(\text{odd }n\text{ only})}
    }

    To complete the n\Nth normal mode, 
    multiply $\red{T_\green{n}(t)}$ of the corresponding $n$.
    \aleq{
        \Psi_\green{n}(x,t) = \blue{X_\green{n}(x)}\red{T_\green{n}(t)} 
        = \blue{\biggl[}\dfrac{4}{\green{n}\pi}\sin\qty(\frac{2\pi \green{n}}{L}x)\blue{\biggr]}
        \red{\biggl[}A_\green{n} \cos \qty(\frac{2\pi \green{n}}{L}vt) 
            + B_\green{n}\sin \qty(\frac{2\pi \green{n}}{L}vt) \red{\biggr]}
    }
    
    Finally, the general evolution is the sum of all normal modes.
    \begin{empheq}[box=\fbox]{align*}
        \Psi(x,t) = \sum_{n=1}^{\infty}\Psi_\green{n}(x,t) 
        &= \sum_{n=1}^{\infty}\blue{X_\green{n}(x)}\red{T_\green{n}(t)} \\
        &= \sum_{\text{odd }n\text{ only}}^{\infty}\blue{\biggl[}\dfrac{4}{\green{n}\pi}\sin\qty(\frac{2\pi \green{n}}{L}x)\blue{\biggr]}
        \red{\biggl[}A_\green{n} \cos \qty(\frac{2\pi \green{n}}{L}vt) 
            + B_\green{n}\sin \qty(\frac{2\pi \green{n}}{L}vt) \red{\biggr]}
    \end{empheq}

    To exactly determine the constants $A_n,B_n$,
    an initial condition is required. 
    For example, \ul{if we specify that the string is static before released},
    \aleq{
        \Psi(x,0) = \sum_{\text{odd }n\text{ only}}^{\infty}\blue{\biggl[}\dfrac{4}{\green{n}\pi}\sin\qty(\frac{2\pi \green{n}}{L}x)\blue{\biggr]}
            \red{\biggl[}A_\green{n} \cdot 1 + B_n \cdot 0 \red{\biggr]} 
            \equiv f(x)
            = \sum_{\text{odd }n\text{ only}}^{\infty}\blue{\biggl[}\dfrac{4}{\green{n}\pi}\sin\qty(\frac{2\pi \green{n}}{L}x)\blue{\biggr]} \\
    }
    \hfill\\[-4em]
    \aleq{
        \Rightarrow\qquad \text{all } A_n &= 1 
    }
    \aleq{
        \pdvv{t}\Psi(x,t)\eval_{t=0} = \sum_{\text{odd }n\text{ only}}^{\infty}\blue{\biggl[}\dfrac{4}{\green{n}\pi}\sin\qty(\frac{2\pi \green{n}}{L}x)\blue{\biggr]}
            \red{\biggl[} -A_\green{n} \cdot 0 + B_n \cdot 1 \red{\biggr]} 
            \equiv 0 \\
    }
    \hfill\\[-4em]
    \aleq{
        \Rightarrow\qquad \text{all } B_n &= 0 
    }

\end{example}



%%%
\theend
\end{document}