\documentclass[class=article, crop=false, 12pt]{standalone}
\usepackage[subpreambles=true]{standalone}
\usepackage{../.common/common}


\author{Tony Shing}
%\pretitle{Supplementary}

\topic{T14B (Electromagnetism)}
\title{Displacement Current}

\version{2025} % leave blank for omitting

\begin{document}

\maketitle


\begin{overview}
    \begin{itemize}
        \item Definition of current, conservation of charge
        \item Fixing Ampere's law - displacement current
    \end{itemize}
\end{overview}

\vskip 1em
In electromagnetism, 
theoretically every problem can be solved through a set of PDEs
called the \bf{Maxwell Equations}.\\[-2em]
\begin{center}
    \begin{minipage}{0.3\textwidth}
        \aleq{
            \div \vvec{E} &= \frac{\rho}{\epsilon_0}\\
            \curl \vvec{E} &= -\pdvv{\vvec{B}}{t}
        }
    \end{minipage}
    \hspace{0.05\textwidth}
    \begin{minipage}{0.3\textwidth}
        \aleq{
            \div \vvec{B} &= 0\\
            \tkn{maxwell_ampere}{\curl} \vvec{B} &= \gray{\mu_0 \vvec{J} +} \cul[red]{\mu_0\epsilon_0\pdvv{\vvec{E}}{t}}
        }
    \end{minipage}
\end{center}
\addArrow[red]{maxwell_ampere}{(-5ex,0)}{}{(-3ex,1ex)}

However, a \it{system of PDEs} is too complicated to be solved.
So we need to learn different "tricks" to avoid them,
which are enough for some simple scenarios.\\

In this note, we will discuss why it is necessary to add the \cul[red]{second term} in Ampere's law.



\linesep
% content begins here
% Section %%%%%%%%%%%%%%%%%%%%%%%%%%%%%%%%%%%%%%%%%%%%%%%%%%%%
\section{Charge Conservation}

In magnetostatics, we describe everything by using current as the source. 
But in the dynamics part of E\&M, charge is the only real source, 
while current has to be treated as moving charge.

\insertFig{pool of charge + current <-> pool of static charge + moving charge}


%%%%%%%%%%%%%%
\subsection{Definition of Current}

To inter-convert the two pespectives, 
we need to apply the definition of current.
Its literal definition is simply:
\begin{center}
    \fbox{
        Current = Rate of charge's flow into/out of a volume
    }
\end{center}

Across textbooks, its definition can appear as many different formula.
\begin{enumerate}
    \item \bf{\ul{"Wire" description}} - 
    Consider the volume to flow through to be a line. 
    Current can be defined by counting the total amount of charge $Q$ through the wire. 
    \aleq{
        \Aboxed{
            I = \dvv{Q}{t}
            \qquad\text{or}\qquad 
            I = \frac{Q}{t}
            \quad \qty(\substack{\text{for people}\\\text{who can't do}\\\text{calculus}})
        }
    }
    \insertFig{line def}

    \item \bf{\ul{"Cylinder" description}} - 
    Magnify the wire to become a straight cylinder.
    Current can be defined by counting the total amount of charge 
    flowing into/out of the cylinder. 
    \aleq{
        \Aboxed{
            I = \dvv{t}\qty[\qty(\substack{\text{Charge}\\\text{per particle}})
                \qty(\substack{\text{\# of particle}\\\text{per volume}})
                \qty(\substack{\text{Volume of}\\\text{cylinder}})]
            =\dvv{t}(qnAL)
        }
    }

    By taking velocity of charge $v\sim\dvv{L}{t}$, it becomes the more common form in textbooks:
    \aleq{
        \Aboxed{
            I = nAqv
        }
    } 

    \insertFig{cyl def}

    \item \bf{\ul{Microscopic definition}} - The whole wire can be abstractly treated as a surface.
    Just like the water pipe we have learnt in Gauss's law. 

    \insertFig{water pipe, label in flux of I and out flux of I}

    This turns the flow in / out of condition becomes a flux integral.
    \aleq{
        I &= \iint \cul[blue]{nq}\vvec{v}\cdot\dd{\vvec{s}} \\[1em]
        &= \iint \tkn{rho}{\cul[blue]{\rho}} \vvec{v} \cdot \dd{\vvec{s}}\\[1em]
        &= \iint \tkn{I_rho}{\cul[red]{\rho(\vvec{r})}}\, \tkn{I_v}{\cul[red]{\vvec{v}(\vvec{r})}} \cdot \dd{\vvec{s}}\\[1em]
        &= \iint \vvec{J}\cdot\dd{\vvec{s}}
    }
    \addArrow[blue]{rho}{(0.5ex,3em)}
    {\scriptsize $nq =\qty(\substack{\text{\# of particle}\\\text{per volume}})\qty(\substack{\text{charge per}\\\text{particle}})\sim $ Charge density}
    {(0.3ex,2ex)}{(19ex,-1.7em)}
    \addArrow[red]{I_rho}{(-2ex,3em)}{}{(0,2ex)}
    \addArrow[red]{I_v}{(-6ex,3em)}
    {\scriptsize $\rho$ and $\vvec{v}$ can be different at different position\\[-1ex]\scriptsize $\therefore$ should be taken as functions of position}
    {(0,2ex)}{(22ex,-1.7em)}

    At the end, 
    we return to the microscopic definition of volume current density 
    $\vvec{J}(\vvec{r}) = \rho(\vvec{r})\vvec{v}(\vvec{r})$. 
\end{enumerate}

\insertFig{magnify the wire}


%%%%%%%%%%%%%%
\subsection{Continuity Equation}

The simplest expression of charge conservation is simply
\aleq{
    \qty(\mstack{\text{Current flow}\\\text{into the region}}) 
    - \qty{\mstack{\text{Current flow}\\\text{out of the region}}}
    \ =\  I_\text{in}(t) I_\text{out}(t)= \dvv{t}Q_\text{enc}(t)
    \ =\ \dvv{t}\qty(\mstack{\text{Charge enclosed}\\\text{in the region}})
}
\insertFig{flow into/out of region}

This relation should hold for any time $t$. 
Furthermore, we can describe this relation using the fancier density distributions:
\begin{itemize}
    \item Charge $\rightarrow$ Volume charge density
    \aleq{
        Q_\text{enc}(t) \rightarrow \iiint_\text{region} \rho(\vvec{r}, t) \dd{\tau}
    }
    \item Current $\rightarrow$ Volume current density
    \aleq{
        I_text{in/out}(t) \rightarrow \iint_\substack{\text{surface with}\\\text{in/out-flux}} \vvec{J}(\vvec{r},t)\cdot \dd{\vvec{s}}
    }
\end{itemize}

To visualize, we can think of some current flowing through a region in the wire.
The surface that wraps the region can be separated into 2 pieces - 
one that contains all the in-flux of $\vvec{J}$ 
and the other contains all the out-flux of $\vvec{J}$.

\insertFig{J through sphere}

By flux integral, we can combine the flux on two surfaces into one term:
\aleq{
    \qty(\mstack{\text{In-flux of}\\\text{current}}) - \qty(\mstack{\text{Out-flux of}\\\text{current}})
    = \qty(\mstack{\red{\text{Net Out-flux}}\\\text{of current}}) 
    &= \dvv{t}\qty(\mstack{\text{Charge enclosed}\\\text{in the region}})\\[1em]
    %
    \iint_{\substack{\text{surface with}\\\text{in-flux}}}\vvec{J}\cdot\dd{\vvec{s}}
    - \iint_{\substack{\text{surface with}\\\text{out-flux}}}\vvec{J}\cdot\dd{\vvec{s}}
    = -\oiint_{\substack{\text{surface around}\\\text{the region}}}\vvec{J}\cdot\dd{\vvec{s}}
    &= \dvv{t}\iiint_{\substack{\text{the}\\\text{region}}} \rho\dd{\tau}
}
(arrow: By convention, we take out flux as +)

And finally, we can write this into its differential form by divergent theorem.
The result is a PDE known as the \bf{continuity equation of charge}.
\aleq{
    -\oiint_{\substack{\text{surface around}\\\text{the region}}}\vvec{J}\cdot\dd{\vvec{s}}
    = -\iiint_{\substack{\text{the}\\\text{region}}}(\div \vvec{J})\dd{\tau}
    =& \iiint_{\substack{\text{the}\\\text{region}}} \pdvv{t}\rho\dd{tau}\\
    %
    \Aboxed{
        -\div \vvec{J} = \pdvv{\rho}{t}
    }
}
(arrow: divergent theorem)
(arrow: move dt into integral, become pdv t)

In practice, this PDE is more often seen in theoretical derivations
rather than to be used to convert charge/current distribution at different time.

\insertFig{framed box of moving particles to arrows? }


\linesep
% Section %%%%%%%%%%%%%%%%%%%%%%%%%%%%%%%%%%%%%%%%%%%%%%%%%%%%
\section{Displacement Current}

%%%%%%%%%%%%%%
\subsection{Problem with Ampere's Law}

So far we have never rigorously define what means by "enclosed by a loop".
Intuitively, we may think of 
\begin{itemize}
    \item Cover the Ampere loop with a surface.
    \item A line is "enclosed by the loop" if the line pokes through the surface.
\end{itemize}

If we re-visit the statement of Ampere's law in magnetostatics,
\aleq{
    \qty(\mstacke{\text{Dot product integral}\\\text{of }\vvec{B}\text{along loop}})
        = \int \vvec{B}\cdot\dd{\vvec{l}} = \mu_0 I_\text{enclosed}
        \propto \qty(\mstack{\text{Current}\\\text{enclosed}})
}

It does not involve any specification about what surface we must choose - 
which means as an universal statement, 
it should work for ALL choices of surface.\\

In magnetostatics, the choice of surface does not matter
because it requires steady current flow (independent of time) - 
which can happen only when the current is running in 1. closed loops, 
or 2. an infinitely long wire.

\insertFig{close loop or infintely long wire}

But when it comes to dynamics, 
current can discontinue in a broken wire and accumulate as charge.
Depends on which surface is chosen, 
it is arguable whether the wire is enclosed by the loop.

\insertFig{red/blue surface}

Here is a simple analogy with a basketball through the ring.



%%%%%%%%%%%%%%
\subsection{Solution: Add A New Current}

To resolve this ambiguity, 
we can make use of charge conservation:

\bf{termination of current always results in accumulation of charge,
and charge creates E-field}. 

We can construct a new type of current $I_d$,
called \bf{displacement current},
which depends on the flux of E-field produced by the charge accumulation.
In such way Ampere's law will always satisfy, indepedent on the choice of the surface.



%%%%%%%%%%%%%%
\subsection{Complication of Electro-magnetic Induction}

Combining the results of Faraday's law and Ampere's law,
we can see that time-varying E-field and B-field can induce each other.
\begin{itemize}
    \item Faraday - $\curl \vvec{E} = -\pdvv{\vvec{B}}{t}$ - 
    Time varying B-field induces (time-varying) E-field.

    \item Ampere - $\curl \vvec{B} = \gray{\mu_0\vvec{J}} + \mu_0\epsilon_0\pdvv{\vvec{E}}{t}$ -
    Time varying E-field induces (time-varying) B-field.
\end{itemize}

This means in the dynamics case, 
ideally $\vvec{E}$ and $\vvec{B}$ should always be solved together - as a system of PDE! 
Otherwise if we only look at Faraday's law or Ampere's law one at a time, 
we would end up like this mess:
\aleq{
    \text{Begin with some time-varying }\vvec{B}_0(t) \qquad
    \text{Solve an induced }\vvec{E}_1(t)\\
    %
    \mstack{\text{Solve an induced } \vvec{B}_1(t)\\\text{Total B-field becomes} \vvec{B}_0(t)+\vvec{B}_1(t)} \qquad
    \text{Solve an induced } \vvec{E}_2(t)\\
    %
    \mstack{\text{Solve an induced } \vvec{B}_2(t)}
    \qquad \cdots
} 

Solving system of PDE is just too complicated.
Basically in all practice problems you can find, 
we just compromise by \bf{assuming induced E-field do not create induced B-field}.
This is OK for small scale system because the magnitude of displacement current 
is usually small enough to ignore:
\aleq{
    \mu_0\epsilon_0 = \inv{c^2} = \inv{(3\times 10^8)^2}
    \qquad\Rightarrow\qquad
    \curl \vvec{B}_1 \sim \mu_0\epsilon_0\pdvv{\vvec{E}_1}{t} 
    \sim (10^{-16})\frac{\Delta \vvec{E}_1}{\Delta t} 
}

However, please keep in mind that 
there does exist systems where you cannot take this approximation.
For example,
\begin{itemize}
    \item Plamsa Physics (Plamsa = A mixture of high energy charges)
    \item Astronomy, e.g. structure of neutron star, surroundings around black holes
\end{itemize}

Solving PDEs is unavoidable in these cases. 
But fortunately you will never see them unless you are doing researches in the future. 




%%%
\theend
\end{document}