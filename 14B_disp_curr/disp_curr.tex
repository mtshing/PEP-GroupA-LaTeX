\documentclass[class=article, crop=false, 12pt]{standalone}
\usepackage[subpreambles=true]{standalone}
\usepackage{../.common/common}


\author{Tony Shing}
%\pretitle{Supplementary}

\topic{T14B (Electromagnetism)}
\title{Displacement Current}

\version{2025} % leave blank for omitting

\begin{document}

\maketitle


\begin{overview}
    \begin{itemize}
        \item Definition of current, conservation of charge
        \item Fixing Ampere's law - displacement current
    \end{itemize}
\end{overview}

\vskip 1em
In electromagnetism, 
theoretically every problem can be solved through a set of PDEs
called the \bf{Maxwell Equations}.\\[-2em]
\begin{center}
    \begin{minipage}{0.3\textwidth}
        \aleq{
            \div \vvec{E} &= \frac{\rho}{\epsilon_0}\\
            \curl \vvec{E} &= -\pdvv{\vvec{B}}{t}
        }
    \end{minipage}
    \hspace{0.05\textwidth}
    \begin{minipage}{0.3\textwidth}
        \aleq{
            \div \vvec{B} &= 0\\
            \tkn{maxwell_ampere}{\curl} \vvec{B} &= \gray{\mu_0 \vvec{J} +} \cul[red]{\mu_0\epsilon_0\pdvv{\vvec{E}}{t}}
        }
    \end{minipage}
\end{center}
\addArrow[red]{maxwell_ampere}{(-5ex,0)}{}{(-3ex,1ex)}

However, a \it{system of PDEs} is too complicated to be solved.
So we need to learn different "tricks" to avoid them,
which are enough for some simple scenarios.\\

In this note, we will discuss why it is necessary to add the \cul[red]{second term} in Ampere's law.



\linesep
% content begins here
% Section %%%%%%%%%%%%%%%%%%%%%%%%%%%%%%%%%%%%%%%%%%%%%%%%%%%%
\section{Charge Conservation}

In magnetostatics, we describe everything by using current as the source. 
But in the dynamics part of E\&M, charge is the only real source, 
while current has to be treated as moving charge.

\insertFig{pool of charge + current <-> pool of static charge + moving charge}


%%%%%%%%%%%%%%
\subsection{Definition of Current}

To inter-convert the two pespectives, 
we need to apply the definition of current.
Its literal definition is simply:
\begin{center}
    \fbox{
        Current = Rate of charge's flow into/out of a volume
    }
\end{center}

Across textbooks, its definition can appear as many different formula.
\begin{enumerate}
    \item \bf{\ul{"Wire" description}} - 
    Consider the volume to flow through to be a line. 
    Current can be defined by counting the total amount of charge $Q$ through the wire. 
    \aleq{
        \Aboxed{
            I = \dvv{Q}{t}
            \qquad\text{or}\qquad 
            I = \frac{Q}{t}
            \quad \qty(\substack{\text{for people}\\\text{who can't do}\\\text{calculus}})
        }
    }
    \insertFig{line def}

    \item \bf{\ul{"Cylinder" description}} - 
    Magnify the wire to become a straight cylinder.
    Current can be defined by counting the total amount of charge 
    flowing into/out of the cylinder. 
    \aleq{
        \Aboxed{
            I = \dvv{t}\qty[\qty(\substack{\text{Charge}\\\text{per particle}})
                \qty(\substack{\text{\# of particle}\\\text{per volume}})
                \qty(\substack{\text{Volume of}\\\text{cylinder}})]
            =\dvv{t}(qnAL)
        }
    }

    By taking velocity of charge $v\sim\dvv{L}{t}$, it becomes the more common form in textbooks:
    \aleq{
        \Aboxed{
            I = nAqv
        }
    } 

    \insertFig{cyl def}

    \item \bf{\ul{Microscopic definition}} - The whole wire can be abstractly treated as a surface.
    Just like the water pipe we have learnt in Gauss's law. 

    \insertFig{water pipe, label in flux of I and out flux of I}

    This turns the flow in / out of condition becomes a flux integral.
    \aleq{
        I &= \iint \cul[blue]{nq}\vvec{v}\cdot\dd{\vvec{s}} \\[1em]
        &= \iint \tkn{rho}{\cul[blue]{\rho}} \vvec{v} \cdot \dd{\vvec{s}}\\[1em]
        &= \iint \tkn{I_rho}{\cul[red]{\rho(\vvec{r})}}\, \tkn{I_v}{\cul[red]{\vvec{v}(\vvec{r})}} \cdot \dd{\vvec{s}}\\[1em]
        &= \iint \vvec{J}\cdot\dd{\vvec{s}}
    }
    \addArrow[blue]{rho}{(0.5ex,3em)}
    {\scriptsize $nq =\qty(\substack{\text{\# of particle}\\\text{per volume}})\qty(\substack{\text{charge per}\\\text{particle}})\sim $ Charge density}
    {(0.3ex,2ex)}{(19ex,-1.7em)}
    \addArrow[red]{I_rho}{(-2ex,3em)}{}{(0,2ex)}
    \addArrow[red]{I_v}{(-6ex,3em)}
    {\scriptsize $\rho$ and $\vvec{v}$ can be different at different position\\[-1ex]\scriptsize $\therefore$ should be taken as functions of position}
    {(0,2ex)}{(22ex,-1.7em)}

    At the end, 
    we return to the microscopic definition of volume current density 
    $\vvec{J}(\vvec{r}) = \rho(\vvec{r})\vvec{v}(\vvec{r})$. 
\end{enumerate}


%%%%%%%%%%%%%%
\subsection{Continuity Equation}

The simplest form of charge conservation is simply
\aleq{
    \qty(\mstack{\text{Current flow}\\\text{into the region}}) 
    \ =\  I_\text{in} = \dvv{t}Q_\text{enc}
    \ =\ \dvv{t}\qty(\mstack{\text{Charge enclosed}\\\text{in the region}})
}
\insertFig{flow into region}

We can also express this relation by the fancier density distributions:
\begin{itemize}
    \item Charge $\rightarrow$ Volume charge density
    \aleq{
        Q(t) = \iiint_\text{region} \rho(\vvec{r}, t) \dd{\tau}
    }
    \item Current $\rightarrow$ Volume current density
    \aleq{
        I(t) = \iint_\text{surface} \vvec{J}(\vvec{r},t)\cdot \dd{\vvec{s}}
    }
\end{itemize}

\linesep
% Section %%%%%%%%%%%%%%%%%%%%%%%%%%%%%%%%%%%%%%%%%%%%%%%%%%%%
\section{Displacement Current}

%%%%%%%%%%%%%%
\subsection{Derivation}



%%%%%%%%%%%%%%
\subsection{Complication of Electromagnetic Induction}




%%%
\theend
\end{document}