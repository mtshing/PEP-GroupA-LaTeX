\documentclass[class=article, crop=false, 12pt]{standalone}
\usepackage[subpreambles=true]{standalone}
\usepackage{../.common/common}


\author{Tony Shing}
%\pretitle{Supplementary}

\topic{T14B (Electromagnetism)}
\title{Displacement Current}

\version{2025} % leave blank for omitting

\begin{document}

\maketitle


\begin{overview}
    \begin{itemize}
        \item Definition of current, conservation of charge
        \item Fixing Ampere's law - displacement current
    \end{itemize}
\end{overview}

\vskip 1em
In electromagnetism, 
theoretically every problem can be solved through a set of PDEs
called the \bf{Maxwell Equations}.\\[-2em]
\begin{center}
    \begin{minipage}{0.3\textwidth}
        \aleq{
            \div \vvec{E} &= \frac{\rho}{\epsilon_0}\\
            \curl \vvec{E} &= -\pdvv{\vvec{B}}{t}
        }
    \end{minipage}
    \hspace{0.05\textwidth}
    \begin{minipage}{0.3\textwidth}
        \aleq{
            \div \vvec{B} &= 0\\
            \tkn{maxwell_ampere}{\curl} \vvec{B} &= \gray{\mu_0 \vvec{J} +} \cul[red]{\mu_0\epsilon_0\pdvv{\vvec{E}}{t}}
        }
    \end{minipage}
\end{center}
\addArrow[red]{maxwell_ampere}{(-5ex,0)}{}{(-3ex,1ex)}

However, a \it{system of PDEs} is too complicated to be solved.
So we need to learn different "tricks" to avoid them,
which are enough for some simple scenarios.\\

In this note, we will discuss why it is necessary to add the \cul[red]{second term} in Ampere's law.



\linesep
% content begins here
% Section %%%%%%%%%%%%%%%%%%%%%%%%%%%%%%%%%%%%%%%%%%%%%%%%%%%%
\section{Charge Conservation}

In magnetostatics, we describe everything by using current as the source. 
But in the dynamics part of E\&M, charge is the only real source, 
while current has to be treated as moving charge.

\insertFig{pool of charge + current <-> pool of static charge + moving charge}


%%%%%%%%%%%%%%
\subsection{Definition of Current}

To inter-convert the two pespectives, 
we now define current by its relation to charge as:
\begin{center}
    \fbox{
        Current = Rate of charge's flow into/out of a volume
    }
\end{center}

Across textbooks, this definition can appear as different formula.
\begin{enumerate}
    \item \bf{\ul{"Wire" description}} - 
    Consider the volume to flow through to be a line. 
    Current can then be defined by counting the total amount of charge $Q$ through the wire within a duration $t$. 
    \aleq{
        \Aboxed{
            I \ \defeq\ \dvv{Q}{t}
            \qquad\text{or}\qquad 
            I \ \defeq\ \frac{Q}{t}
            \quad \qty(\substack{\text{for people}\\\text{who can't do}\\\text{calculus}})
        }
    }
    \insertFig{line def}

    \item \bf{\ul{"Cylinder" description}} - 
    Magnify the wire to become a straight cylinder.
    Current can be defined by counting the total amount of charge 
    flowing into/out of the cylinder. 
    \aleq{
        I = \dvv{Q}{t} \ \Rightarrow\  
        I = \dvv{t}\qty[\qty(\substack{\text{Charge}\\\text{per particle}})
            \qty(\substack{\text{\# of particle}\\\text{per volume}})
            \qty(\substack{\text{Volume of}\\\text{cylinder}})]
        =\dvv{t}[qn(AL)]
    }

    Let the \cul[red]{average} velocity of charges be $\bar{v}\sim\dvv{L}{t}$,
    while the number density / cross-section of the wire being constant of time. 
    Then it becomes the familiar formula in textbooks:
    \aleq{
        \Aboxed{
            I = nAq\bar{v}
        }
    } 

    \insertFig{cyl def}

    \item \bf{\ul{"Microscopic" description}} - The whole wire can be abstractly treated as a surface,
    just like the water pipe we have learnt in Gauss's law. 

    \insertFig{water pipe, label in flux of I and out flux of I}

    Similarly, we can use integration over in-flux / out-flux to describe the flow of current.
    \aleq{
        I = nqvA \ \Rightarrow\ 
        I_\text{total} &= \iint \cul[blue]{nq}\vvec{v}\cdot\dd{\vvec{s}}\qquad \\[1em]
        &= \iint \tkn{rho}{\cul[blue]{\rho}} \vvec{v} \cdot \dd{\vvec{s}}\\[1em]
        &= \iint \tkn{I_rho}{\cul[red]{\rho(\vvec{r})}}\, \tkn{I_v}{\cul[red]{\vvec{v}(\vvec{r})}} \cdot \dd{\vvec{s}}
    }
    \addArrow[blue]{rho}{(0.5ex,3em)}
    {\scriptsize $nq =\qty(\substack{\text{\# of particle}\\\text{per volume}})\qty(\substack{\text{charge per}\\\text{particle}})\sim $ Charge density}
    {(0.3ex,2ex)}{(19ex,-1.7em)}
    \addArrow[red]{I_rho}{(-2ex,3em)}{}{(0,2ex)}
    \addArrow[red]{I_v}{(-6ex,3em)}
    {\scriptsize $\rho$ and $\vvec{v}$ can be different at different position\\[-1ex]\scriptsize $\therefore$ should be treated as functions of position}
    {(0,2ex)}{(22ex,-1.7em)}

    Also, recall the conversion between line current element and volume current element,
    \aleq{
        I \cdot \qty(\mstack{\text{Unit}\\\text{length}}) \ \sim\  I\dd{\vvec{l}} 
        \ &=\  \vvec{J}\dd{\tau} \ \sim\  \vvec{J} \cdot \qty(\mstack{\text{Unit}\\\text{volume}})\\[1em]
        %
        \Rightarrow\quad I \ &=\  \vvec{J}\cdot\dd{\vvec{s}} \ \sim\  \vvec{J} \cdot \qty(\mstack{\text{Unit}\\\text{area}})\\[1em]
        %
        \Rightarrow\quad I_\text{total} \ &=\ \iint\vvec{J}\cdot\dd{\vvec{s}} 
    }
    
    \insertFig{magnify wire}

    Comparing the results, we now have a new relation between current density and the flow distribution of charges.
    \aleq{
        \Aboxed{
            \vvec{J}(\vvec{r}) = \rho(\vvec{r})\vvec{v}(\vvec{r}) 
        }
    }

\end{enumerate}



%%%%%%%%%%%%%%
\subsection{Continuity Equation}

We can write the simplest description of charge conservation as
\aleq{
    \qty(\mstack{\text{Current flow}\\\text{into region}}) 
    - \qty(\mstack{\text{Current flow}\\\text{out of region}})
    \ &= \ \dvv{t}\qty(\mstack{\text{Charge enclosed}\\\text{in region}})\\[1em]
    %
    \text{i.e.}\qquad\qquad
    I_\text{in}(t) - I_\text{out}(t) \ &=\ \dvv{t}Q_\text{enc}(t)
}
\insertFig{flow into/out of region}

This relation should hold for any time $t$. 
Furthermore, we can write this relation using the fancier density distributions:
\begin{itemize}
    \item Charge $\rightarrow$ Volume charge density
    \aleq{
        Q_\text{enc}(t) \rightarrow \iiint_\text{region} \rho(\vvec{r}, t) \dd{\tau}
    }
    \item Current $\rightarrow$ Volume current density
    \aleq{
        I_\text{in/out}(t) \rightarrow \iint_{\substack{\text{surfaces}\\\text{with}\\\text{in/out-flux}}} \vvec{J}(\vvec{r},t)\cdot \dd{\vvec{s}}
    }
\end{itemize}

To visualize in the microscopic view, 
we can think of some current flowing through a region in the wire.
The surface that wraps the region can be separated into 2 pieces - 
one that contains all the in-flux of $\vvec{J}$ 
and the other contains all the out-flux of $\vvec{J}$.

\insertFig{J through sphere}

By writting as integrals, we can combine the flux on the two surfaces as one integral through a closed surface:
\aleq{
    \qty(\mstack{\text{In-flux of}\\\text{current}}) - \qty(\mstack{\text{Out-flux of}\\\text{current}})
    &= \tkn{flux_sign}{\red{-}} \qty(\mstack{\red{\text{Net Out-flux}}\\\text{of current}})\\[3em]
    %
    \text{i.e.}\qquad
    \iint_{\substack{\text{surface}\\\text{with}\\\text{in-flux}}}\vvec{J}\cdot\dd{\vvec{s}}
    - \iint_{\substack{\text{surface}\\\text{with}\\\text{out-flux}}}\vvec{J}\cdot\dd{\vvec{s}}
    &= \tkn{flux_sign2}{\red{-}}\oiint_{\substack{\text{surface}\\\text{around}\\\text{the region}}}\vvec{J}\cdot\dd{\vvec{s}}
}
\addArrow[red]{flux_sign}{(0,-3ex)}{\scriptsize By convention,\\[-1ex]\scriptsize out-flux = +ve}{(0,-1ex)}{(0,-1ex)}
\addArrow[red]{flux_sign2}{(0,4ex)}{}{(0,2ex)}

So that the conservation relation becomes:
\aleq{
    \Aboxed{
        - \qty(\mstack{\text{Net Out-flux}\\\text{of current}}) 
        \ \sim\  - \oiint_{\substack{\text{surface}\\\text{around}\\\text{the region}}}\vvec{J}\cdot\dd{\vvec{s}}
        \ =\  \dvv{t}\iiint_{\substack{\text{the}\\\text{region}}} \rho\dd{\tau}
        \ \sim\ \dvv{t}\qty(\mstack{\text{Enclosed}\\\text{charge}})
    }
}

This is the integral form of the equation of charge conservation. 
We can also also convert it into its differential form by divergent theorem.
\addArrow[red]{divJ}{(0,9ex)}{Divergent Theorem}{(1ex,3ex)}{(-11ex,-6ex)}
\addArrow[blue]{divrho}{(-6ex,5.5ex)}
{\scriptsize $\rho$ depends on position\\[-1ex]\scriptsize $\therefore$ becomes partial D when moved into integral}
{(0,4ex)}{(22.5ex,-3ex)}
\aleq{
    - \oiint\vvec{J}\cdot\dd{\vvec{s}} \quad &= \dvv{t}\iiint \rho\dd{\tau}\\[2em]
    %
    -\iiint (\tkn{divJ}{\div \vvec{J}}) \cdot \dd{\tau}\ &=\ \iiint \tkn{divrho}{\pdvv{t}}\rho\dd{\tau}\\[1em]
    %
    \Aboxed{
        \div \vvec{J} &= -\pdvv{\rho}{t}
    }
}

\vskip 1ex
This is a PDE known as the \bf{continuity equation of charge}.
For example, we can use it to predict the distribution of current given that we know the initial charge distribution.
But in practice, it is more often used in theoretical derivations.

\insertFig{relate charge distri to current dist}




\linesep
% Section %%%%%%%%%%%%%%%%%%%%%%%%%%%%%%%%%%%%%%%%%%%%%%%%%%%%
\section{Displacement Current}

%%%%%%%%%%%%%%
\subsection{Enclosed by A Loop?}

To understand the followings,
we first need to rigorously define what it means by "enclosed by a loop".
Recall the Stoke's theorem,, 
\aleq{
    \qty(\mstack{\text{Line integral}\\\text{along a loop}}) \ \sim\ 
    \oint \vvec{F}\cdot \dd{\vvec{l}} 
    = \iint (\curl \vvec{F}) \cdot \dd{\vvec{s}}
    \ \sim\ \qty(\mstack{\text{Flux integral of curl}\\\text{over the surface}\\\text{surrounded by the loop}}) 
}

Therefore, we can borrow the idea of surface integral:
\begin{itemize}
    \item Cover the Ampere loop with a surface.
    \item A line is "enclosed by the loop" if the line pokes through the surface.
\end{itemize}

\insertFig{poke through vs not poke through}

In this definition, there is no specification about the surface -
which means as an universal definition, 
it should work for ALL choices of surface.

\insertFig{work for all choice of surface}


%%%%%%%%%%%%%%
\subsection{Problem with Ampere's Law}

Now we re-visit the statement of Ampere's law in magnetostatics:
\aleq{
    \qty(\mstack{\text{Dot product integral}\\\text{of }\vvec{B}\text{ along loop}})
    \ \sim\  \oint \vvec{B}\cdot\dd{\vvec{l}} 
    \ =\ \mu_0 I_\text{enc}
    \ \sim\  \mu_0\qty(\mstack{\text{"Enclosed"}\\\text{current}})
}

In statics cases (no time-varying current), 
the statement holds for any choice of surfaces to capture the enclosed current.
This is because in order to have steady current flow - 
current must be running in 1. a closed loop, 
or 2. an infinitely long wire.

\insertFig{close loop or infintely long wire}

But when current is allowed to change with time, 
we can have current discontinues at the end of a broken wire and accumulate as charges.
Depends on which surface is chosen, 
it becomes undefined whether the current wire is enclosed by the loop.

\insertFig{red/blue surface}

Just like in basketball, the choice of surface does matter.

\insertFig{basketball}





%%%%%%%%%%%%%%
\subsection{The Fix: Another Current}

According to charge conservation:
\begin{center}
    \bf{Termination of current always results in accumulation of charge,\\
    and charge always creates E-field}. 
\end{center}

We can fix the ambiguity of Ampere's law by adding a new current term $I_d$,
called \bf{displacement current},
which is defined as 
\aleq{
    \Aboxed{
        I_d \ \defeq\ \epsilon_0 \pdvv{t}\iint_{\substack{\text{the chosen}\\\text{surface}}} \vvec{E} \cdot\dd{\vvec{s}}
    }
}

or in terms of volume current density,
\aleq{
    \Aboxed{
        \vvec{J}_d \ \defeq\ \epsilon_0 \pdvv{\vvec{E}}{t}
    }
}

So that Ampere's law is modified into: 
\begin{itemize}
    \item \ul{Integral form}:
    \aleq{
        (\vvec{B}\text{'s loop integral})\ 
        \sim\ \oint \vvec{B}\cdot\dd{\vvec{l}} 
        \ =\ \mu_0 (I_\text{enc} + I_d) \ 
        \sim\ \qty(\mstack{\text{Enclosed}\\\text{current}}) + \qty(\mstack{\text{Displacement}\\\text{current}})
    }

    \item \ul{Differential form}:
    \vskip -2.5ex
    \aleq{
        \qty(\mstack{\vvec{B} \text{'s loop integral}\\[0.5ex]\text{density}})\ \sim\ \curl \vvec{B}
        \ =\ \mu_0 (\vvec{J}_\text{enc} + \vvec{J}_d) 
        \ \sim\ \qty(\mstack{\text{Enclosed}\\[0.5ex]\text{current}\\[0.5ex]\text{density}})
        + \qty(\mstack{\text{Displacement}\\[0.5ex]\text{current}\\[0.5ex]\text{density}})
    }
\end{itemize}


With $I_d$ depending on the flux of E-field produced by the charge accumulation,
Ampere's law is always satisfied,
indepedent of the choice of the surface.
For example,

\insertFig{ red surf, blue surf}

The red surface is poked through by the real current wire.
$I_\text{enc} \neq 0$.

The blue surface is not poked through by the current wire,
but it is passed through by flux of E-field.
$I_\text{enc} = 0$ but $I_d \neq 0$.


\begin{notation}[Derivation:]

    The expression of displacement current is a direct result of charge conservation.
    We can start with the continuity equation and substitue Gauss's law:
    \aleq{
        0 &= \div \vvec{J} + \pdvv{\rho}{t}\\
        &= \div \vvec{J} + \pdvv{t}\tkn{gauss}{(\epsilon_0\div \vvec{E})}\\
        &= \div \qty(\vvec{J} + \epsilon_0\pdvv{\vvec{E}}{t})
    }
    \addBentArrow[blue]{gauss}{(-6ex,7.5ex)}
    {$\div \vvec{E} = \dfrac{\rho}{\epsilon_0}$}{(1ex,2.5ex)}{(13ex,-5ex)}

    Recall that if a vector field is divergent-less,
    it can be expressed as the curl of another vector field.
    (Just like B-field, $\div \vvec{B}=0 \ \Rightarrow \vvec{B} = \curl \vvec{A}$).
    It is the same here:
    \aleq{
        \div \qty(\vvec{J} + \epsilon_0\pdvv{\vvec{E}}{t}) = 0 
        \qquad\Rightarrow\qquad
        \vvec{J} + \epsilon_0\pdvv{\vvec{E}}{t} = \curl (\text{another field})
    }

    Comparing with the original Ampere's law,
    the "another field" has the same unit as the B-field (divided by $\mu_0$).
    So the correct modification must be 
    \aleq{
        \Aboxed{
            \curl \vvec{B} = \mu_0\qty(\vvec{J} + \epsilon_0\pdvv{\vvec{E}}{t})
        }
    }

    And its integral form can be shown by Stoke's theorem.
    \aleq{
        \Aboxed{
            \oint \vvec{B}\cdot\dd{\vvec{l}} 
            = \mu_0\qty(I + \epsilon_0\pdvv{t}\iint\vvec{E}\cdot\dd{\vvec{s}})
        }
    }

\end{notation}



%%%%%%%%%%%%%%
\subsection{Complication of Electro-Magnetic Induction}

Combining the results of Faraday's law and Ampere's law,
we can see that time-varying E-field and B-field can induce each other.
\begin{itemize}
    \item Faraday - $\curl \vvec{E} = -\pdvv{\vvec{B}}{t}$ - 
    Time varying B-field induces (time-varying) E-field.

    \item Ampere - $\curl \vvec{B} = \gray{\mu_0\vvec{J}} + \mu_0\epsilon_0\pdvv{\vvec{E}}{t}$ -
    Time varying E-field induces (time-varying) B-field.
\end{itemize}

Ideally $\vvec{E}$ and $\vvec{B}$ must be solved together - as a system of PDE! 
Because if we only look at Faraday's law or Ampere's law one at a time, 
we end up like this mess:
\aleq{
    \tkmat{
        \qty{\text{Begin with a time-varying }\vvec{B}_0(t)} 
        \& \qquad\qquad \&
        \qty{\vvec{B}_0(t) \text{ creates an induced }\vvec{E}_1(t)}\\[1ex]
        %
        \qty{\mstack{\vvec{E}_1(t)\text{ creates an induced } \vvec{B}_1(t)\\[0.5ex]\text{Total B-field }\Rightarrow \vvec{B}_0(t)+\vvec{B}_1(t)}} 
        \& \qquad\qquad \&
        \qty{\mstack{\vvec{B}_1(t)\text{ creates an induced } \vvec{E}_2(t)\\[0.5ex]\text{Total E-field }\Rightarrow \vvec{E}_1(t)+\vvec{E}_2(t)}} \\[1ex]
        %
        \qty{\mstack{\vvec{E}_2(t)\text{ creates an induced } \vvec{B}_2(t)\\[0.5ex]\text{Total B-field }\Rightarrow \vvec{B}_0(t)+\vvec{B}_1(t)+\vvec{B}_2(t)}}
        \& \qquad\qquad \&
        \cdots \\
    }{
        \draw[->,draw=blue] ($(m-1-1.east)+(0,0.5ex)$) to node[midway, blue, above] {Faraday} ($(m-1-3.west)+(0,0.5ex)$) ; 
        \draw[->,draw=red] ($(m-1-3.south west)+(0,1.5ex)$) to node[midway, red, above, sloped] {Ampere} ($(m-2-1.east)+(0,1.5ex)$) ; 
        \draw[->,draw=blue] ($(m-2-1.east)+(0,-1ex)$) to node[midway, blue, above] {Faraday} ($(m-2-3.west)+(0,-1ex)$) ; 
        \draw[->,draw=red] ($(m-2-3.south west)+(0,1.5ex)$) to node[midway, red, above, sloped] {Ampere} ($(m-3-1.east)+(0,1ex)$) ; 
        \draw[->,draw=blue] ($(m-3-1.east)+(0,-1ex)$) to node[midway, blue, above] {Faraday} ($(m-3-3.west)+(-8ex,-1ex)$) ;
    }
} 

But solving system of PDE is too complicated.
So in most practice problems, 
there is a hidden compromise - \bf{we assume induced E-field do not create induced B-field}.
This is OK for small scale system when the magnitude of displacement current 
is small enough to be ignored:
\aleq{
    \mu_0\epsilon_0 = \inv{c^2} = \inv{(3\times 10^8)^2}
    \qquad\Rightarrow\qquad
    \curl \vvec{B}_1 = \mu_0\epsilon_0\pdvv{\vvec{E}_1}{t} 
    \ \sim\ (10^{-16})\frac{\Delta \vvec{E}_1}{\Delta t} 
}

There do exist systems where you cannot take this approximation. 
Solving PDEs is unavoidable. 
For example,
\begin{itemize}
    \item Plamsa Physics (Plamsa = A mixture of high energy charges)
    \item Astronomy, e.g. structure of neutron star, surroundings around black holes
\end{itemize}

Fortunately, you will almost never see them unless you are doing researches in the future. 




%%%
\theend
\end{document}