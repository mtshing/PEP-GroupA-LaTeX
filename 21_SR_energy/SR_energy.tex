\documentclass[class=article, crop=false, 12pt]{standalone}
\usepackage[subpreambles=true]{standalone}
\usepackage{../.common/common}


\author{Tony Shing}
%\pretitle{Supplementary}

\topic{T21 (Special Relativity)}
\title{Relativistic Energy \& Momentum}

\version{2025} % leave blank for omitting

\begin{document}

\maketitle


\begin{overview}
    \begin{itemize}
        \item Constructing 4-vectors
        \item Common applications of 4-momentum
        \item Spacetime interval
        %\item Extra topic: 4-acceleration
    \end{itemize}
\end{overview}


% content begins here
% Section %%%%%%%%%%%%%%%%%%%%%%%%%%%%%%%%%%%%%%%%%%%%%%%%%%%%
\section{The 4-vectors framework}

4-vectors are 4$\times$1 vectors that
\begin{itemize}
    \item Are made of a combination of physical quantities.
    \item After multiplied by Lorentz transformation matrix, 
    its values change to what should be observed by an observer moving with speed $v$.
\end{itemize}

Plainly speaking, 
we want to "pack" physical quantities into a 4$\times$1 vectors
such that it satisfies
\aleq{
    \tkn{coorB}{\cul[red]{\bmat{P'\\ Q'\\ R' \\S'}}}
    = \tkn{lorentz}{
        \cul[gray]{\bmat{
            \gamma & -\gamma\beta & 0 & 0 \\
            -\gamma\beta & \gamma & 0 & 0 \\
            0 & 0 & 1 & 0 \\
            0 & 0 & 0 & 1
        }}
    }\, 
    \tkn{coorA}{\cul[blue]{\bmat{P\\ Q\\ R\\S}}}
}
\addArrow[red]{coorB}{(-8ex,0)}
{$P,Q,R,S$   \\ using the values seen by B}
{(-3ex,0)}{(-8ex,-2ex)}
\addArrow[blue]{coorA}{(8ex,0)}
{A vector \\ made of $P, Q, R, S$ \\[1ex] \bf{Substitute the values} \\ \bf{that are observed by A}}
{(3ex,0)}{(8ex,-2ex)}
\addArrow[gray]{lorentz}{(0,-4ex)}{The Lorentz Transformation\\A 4$\times$4 matrix}{(0,-9ex)}{(0,-1.5ex)}

\vskip 3.5em

We have already had the example of \bf{4-position vector} $\vvec{X} = \bmat{ct & x & y & z}$,
a 4$\times$1 vector that packs up the time coordinate $t$ and position coordinate $x,y,z$ of an event,
and can be used to show the values of the event's coordinate according to different observers.
\aleq{
    \tkn{coorB2}{\cul[red]{\bmat{ct'\\ x'\\ y' \\z'}}}
    = \bmat{
            \gamma & -\gamma\beta & 0 & 0 \\
            -\gamma\beta & \gamma & 0 & 0 \\
            0 & 0 & 1 & 0 \\
            0 & 0 & 0 & 1
        }\, 
    \tkn{coorA2}{\cul[blue]{\bmat{ct\\ x\\ y\\z}}}
}
\addArrow[red]{coorB2}{(-8ex,0)}
{A 4$\times$1 vector made of \\ values of $t, x, y, z$ \\ \bf{seen by B}}
{(-3ex,0)}{(-8ex,-2ex)}
\addArrow[blue]{coorA2}{(8ex,0)}
{A 4$\times$1 vector made of \\ values of $t, x, y, z$ \\ \bf{Seen by A}}
{(3ex,0)}{(8ex,-2ex)}

\vskip 0.5em
Starting from 4-position vector, 
we can derive other types of 4-vectors used in relativistics mechanics.


%%%%%%%%%%%%%%
\subsection{Time Differentiation of 4-vector}

The first thing we want to pack into a 4-vector are the observed velocity $v_x$, $v_y$, $v_z$ of an object.
\aleq{
    \vvec{U}' = \bmat{U_0' & U_1' & U_2' & U_3'}
    = 
    \bmat{U_0 & U_1 & U_2 & U_3}
    =\bmat{U}
}
%$U_0, U_1, U_2, U_3$ are functions to $v_x$, $v_y$ and $v_z$ 
% U_0', U_1' ....

In Newtonian mechanics,
velocity of an object can be computed by differentiating its position with respect to time.
\aleq{
    \vvec{v} = \dvv{\vec{r}}{t} 
    \sim \frac{\Delta \vvec{r}}{\Delta t}
    = \frac{\text{Change in position}}{\text{Change in time}}
}

The problem when involving relativity is that 
the change in time (time scale) $\Delta t$ is different between observers!
For example, 
if we naively "divide" a 4-position vector by the change in time of each observer, 
the nice Lorentz transform is broken.
\aleq{
    \qty(\mstack{\text{Coordinate Change}\\\text{Seen by B}}) \sim 
    \bmat{c(\Delta t') \\ x'} 
    = \bmat{\gamma & -\gamma\beta \\ -\gamma\beta & \gamma }
        \bmat{c(\Delta t)\\ x}
    \sim \qty(\mstack{\text{Lorentz}\\\text{Transform}})
    \qty(\mstack{\text{Coordinate Change}\\\text{Seen by A}}) 
}
But because $\Delta t' \neq \Delta t$,
it is wrong to have
\aleq{
    \dvv{\vvec{X}'}{\red{t'}} \sim \inv{\red{\Delta t'}}\bmat{c(\Delta t')\\x'}
    \red{\neq} 
    \dvv{\vvec{X}}{\red{t}} \sim 
    \inv{\red{\Delta t}}\bmat{\gamma & -\gamma\beta \\ -\gamma\beta & \gamma}\bmat{c{\Delta t}\\x}
}

In order to construct a "velocity-like" 4-vector,
we should fix to the same time scale $\Delta \tau$ when differentiating,
such that
\aleq{
    \dvv{\cul[red]{\tau}}\bmat{ct'\\x'} 
    = \bmat{\gamma & -\gamma\beta\\ -\gamma\beta &\gamma}
    \dvv{\cul[red]{\tau}}\bmat{ct \\ x} 
} 

Meanwhile by the principle of relativity, 
there should NOT be a preference to a specific observer - 
the time scale of every observer are equally "similar".
We require differentiation to 4-vectors to hold true no matter whose time scale is used.

\insertFig{no frame preference}
% I see dvv{t}
% I see dvv{t'}

\aleq{
    \dvv{\red{t'}}\bmat{ct'\\x'} 
    = \bmat{-\gamma & -\gamma\beta\\ -\gamma\beta &\gamma}\dvv{\red{t'}}\bmat{ct\\x}
}
either or are both true
\aleq{
    \dvv{\blue{t}}\bmat{ct'\\x'}
    = \bmat{-\gamma & -\gamma\beta\\ -\gamma\beta &\gamma}\dvv{\blue{t}}\bmat{ct\\x} 
}


%%%%%%%%%%%%%%%%
\subsection{4-velocity in 1D}

Every object always accompanies with one very specific observer - 
its \blue{co-moving} observer, who
\begin{itemize}
    \item moves together with the object
    \item measures the shortest time difference 
    between stationary events relative to the object.
    \item measures the largest position difference
    between two points on the object. 
\end{itemize}

\insertFig{co-moving vs moving on a ball}

Let the time scale of the co-moving observer be $\tau$.
The co-moving observer will always see the object stationary,
i.e. 4-position = $\bmat{c\tau \\ 0}$.
For an observer who \it{sees} co-moving observer at velocity $v$,
the 4-position can be found by inverse transform:
\aleq{
    \bmat{ct\\x} 
    &= \bmat{\gamma_\blue{-v} & -\gamma_\blue{-v}\beta_\blue{-v} \\ -\gamma_\blue{-v}\beta_\blue{-v} & \gamma_\blue{-v}}\bmat{c\tau\\0}\\
    &= \bmat{\gamma_v \cdot c\tau \\ \gamma_v v\tau} 
}
Thus we can get a differential relation between $t'$ and $\tau$
\aleq{
    \dvv{t}{\tau} = \dvv{\tau}(\gamma_v\tau) = \gamma_v
}

\begin{notation}[Note:]
    The conversion of time differential should be relative to the co-moving observer.
    Suppose there is a \nth{3} observer who sees the object moving at velocity $u$ and 
    his time scale is $t'$.
    
    \insertFig{3rd observer}

    The time differential of this \nth{3} observer relative to the co-moving observer is then
    \aleq{
        \dvv{t'}{\tau} = \gamma_u
    }

    The inter-conversion of time differential between \nth{2} and \nth{3} 
    


\end{notation}

The \bf{4-velocity} vector, by definition, 
is the 4-position diffentiated by the co-moving observer's time scale:
\aleq{
    \vvec{U} \ \defeq\ \dvv{\blue{tau}}\bmat{c\tau \\ 0} = \bmat{c \\ 0}
}

With inverse Lorentz transform and change in timescale between observer,
it can be written into many expressions.

% table of x, dxdtau, x', dx'dtau, x, dxd

\iffalse
\aleq{
    \vvec{U'} &= 
    \bmat{\gamma_\blue{-v} & -\gamma_\blue{-v}\beta_\blue{-v} \\ \gamma\beta & \gamma}\dvv{\blue{\tau}}\bmat{c\tau \\ 0} \\
    &= \bmat{\gamma & -\gamma\beta \\ -\gamma\beta & \gamma}\bmat{c \\ 0} \\
    &= \bmat{\gamma_v c \\ \gamma_v v}
}
    \fi





This can be easily satisfied because we already know the relation between
time scale of different observers:
\aleq{
    \bmat{c\red{t'}\\x'} 
    &= \bmat{\gamma & -\gamma\beta \\-\gamma\beta \gamma}{c\blue{t}\\x} \\
    \Rightarrow c\red{t'} &= \gamma c\blue{t} - \gamma\beta x \\
    \dvv{\red{t'}}{\blue{t}} &= \gamma \dvv{\blue{t}}{\blue{t}} - \gamma\beta\dvv{x}{\blue{t}} \\
    &= \gamma - \gamma\beta v 
}

\iffalse
However when we want to specifically describe ONE object,
the object always accompanies with a \blue{co-moving} observer (proper observer), who


\insertFig{3 guys: we agree to use tau as diferentiation when describing events on the ball}

\iffalse
I move together with the ball and my time scale is tau
you should all use tau to co
OK
OK

If you want to describe events on the ball, use my tau
If you want to describe events on the cube, use my tau
\fi

So when we want to take time differentiation to physically quantity,
we should first look at the 
\fi




\iffalse
%%%%%%%%%
\subsubsection{1D velocity 4-vector}
Let the object moves in $\hat{x}$ direction.
\begin{itemize}
    \item In a ‘static’ observer's eye (like he is standing on the object), the object is not moving. Coordinate of the object in his point of view is thus
    \[
    \vec{X} = \bmat ct\\0\emat
    \]
    His time scale is also the proper time scale, so 4-velocity is
    \[
    \vec{U} = \dd{\tau}\bmat ct\\0\emat =\dd{t}\bmat ct\\0\emat = \bmat c\\0\emat
    \]
    
    \item 4-velocity transform the same way as 4-position. For a ‘moving’ observer moving at velocity $\vec{v}$ relative to the object, the 4-velocity becomes
    \[
    \vec{U'} = \dd{t} \bmat ct'\\x'\emat 
    = 
    \dd{t} \bmat 
    \gamma_v & -\gamma_v\beta_v \\
    \-gamma_v\beta_v & \gamma_v 
    \emat
    \bmat ct\\0 \emat
    =
    \bmat 
    \gamma_v & -\gamma_v\beta_v \\
    \-gamma_v\beta_v & \gamma_v 
    \emat
    \bmat c\\0\emat 
    =
    \bmat \gamma_vc\\\gamma_v(-v)\emat
    \]
    There is a negative sign, because from ‘moving’ observer‘s view, himself is static and the object is moving at velocity $-\vec{v}$.\\
    

\end{itemize}
In general, when you see an object moving at velocity $\vec{v}$, then you can directly write down its velocity 4-vector as 
    \[
    \bmat \gamma_vc\\\gamma_vv \emat 
    \]


%%%%% subsubsection 2
\subsubsection{2D velocity 4-vector}    
On a 2D plane, suppose an object is moving at velocity $\vec{v}$ only at x direction. The velocity 4-vector is simply 
\[
\bmat \gamma_vc\\\gamma_vv\\0 \emat
\]

You can rotate your x-axis, e.g. by an angle $-\theta$. Then the object becomes moving at velocity $\vec{v}$ at a direction of $+\theta$ from your new x-axis. Recall that rotating the coordinate system for an angle $-\theta$ involve the rotation matrix
\[
\bmat
1 & 0 & 0 \\
0 & \cos{(-\theta)} & \sin{(-\theta)}\\
0 & -\sin{(-\theta)} & \cos{(-\theta)}
\emat
\]

Applying it onto the original velocity 4-vector, you can get the new velocity 4-vector.\\
\[
\bmat
1 & 0 & 0 \\
0 & \cos{(-\theta)} & \sin{(-\theta)}\\
0 & -\sin{(-\theta)} & \cos{(-\theta)}
\emat
\bmat \gamma_vc\\\gamma_vv\\0 \emat
=\bmat \gamma_vc\\\gamma_vv\cos\theta\\\gamma_vv\sin\theta \emat
\]

In general, for an object moving in velocity $\vec{v}=(v\cos\theta, v\sin\theta)$, the velocity 4-vector is in the above form.\\
\fi



%%%%%%%%%%%%%%
\subsection{4-momentum}

From 

It is defined as 
\[
\vec{P} = m\vec{U} = m\dd{\tau} \bmat ct\\x\\y\\z \emat
\]
Where m is the rest mass of the object, i.e. mass observed by the observer who is ‘static’ to the object.\\

The 4-momentum of a static object is thus
\[
\bmat mc\\0\emat
\]
And 4-momentum of a moving object of velocity $\vec{v}$ is thus
\[
\bmat\gamma_vmc\\\gamma_vmv \emat
\]

\subsubsection{Meanings of its components}
Have a close look to each component:
\begin{itemize}
    \item $\gamma_vmc$: Taylor expand it
    \[
    \gamma_vmc = \frac{mc}{\sqrt{1-\frac{v^2}{c^2}}} 
    \approx 
    mc(1+\frac{1}{2}\frac{v^2}{c^2}+\cdots)
    =
    \frac{1}{c}(mc^2+\frac{1}{2}mv^2+\cdots
    \]
    Which you can see it has some relation with classical $\text{"KE"}=\frac{1}{2}mv^2$. We call $E=\gamma_v mc^2$ as the total energy of the moving object, and $mc^2$ the rest energy, i.e. energy when v=0. Also $(\gamma_v-1)mc^2=\frac{1}{2}mv^2+\cdot$ as relativistic KE.
    
    \item $\gamma_vmv$: The classical momentum $mv$ times $\gamma_v$. Sometimes people call $\gamma_vm$ as the relativistic reduced mass, but I would avoid that. The $\gamma_v$ is introduced \textbf{NOT} because observed mass is different to different observers, but the momentum is observed not proportional to $v$ by checking the conservation of momentum. $\gamma_vmv$, named as relativistic momentum, is the conserved quantity under special relativity. 
\end{itemize}

%%%%%%%%%%%%%%
%\subsection{Extra: 4-acceleration}


\iffalse
\linesep
% Section %%%%%%%%%%%%%%%%%%%%%%%%%%%%%%%%%%%%%%%%%%%%%%%%%%%%
\section{Application: Conservation of 4-momentum}

4-momentum is the conserved quantity under special relativity, 
but not the classical energy and momentum. 
Most of the problems are about collision and decay of particles at relativistic speed.

\begin{enumerate}
    \item \textbf{Decay of particle}\\
    Particle A with rest mass $m_A$ initially at rest, decay into two particle B, C with rest mass $m_B$ and $m_C$, moving with velocity $v_B$ and $v_C$. Then their 4-momentum can be easily written down for any observer
    
    \begin{tabu}{|X[c]|X[c]|X[c]|X[c]|}
    \hline
    \quad & A & B & C 
    \\ \hline \rule{0pt}{4ex}
    \makecell{Observer at rest \\(same speed as A)} & 
    $\bmat m_Ac\\0\emat$ & 
    $\bmat \gamma_{v_B}m_Bc\\\gamma_{v_B}m_Bv_B\emat$ &
    $\bmat \gamma_{v_C}m_Cc\\\gamma_{v_C}m_Cv_C\emat$
    \\[0.5ex] \hline \rule{0pt}{4ex}
    \makecell{Moving observer with \\$\vec{v}$ relative to A} &
    $\Lambda(v)\bmat m_Ac\\0\emat$ & 
    $\Lambda(v)\bmat \gamma_{v_B}m_Bc\\\gamma_{v_B}m_Bv_B\emat$ &
    $\Lambda(v)\bmat \gamma_{v_C}m_Cc\\\gamma_{v_C}m_Cv_C\emat$
    \\[0.5ex] \hline
    \end{tabu}
    
    Conservation of 4-momentum gives
    \[
    \bmat m_Ac\\0\emat 
    = 
    \bmat \gamma_{v_B}m_Bc\\\gamma_{v_B}m_Bv_B\emat
    +
    \bmat \gamma_{v_C}m_Cc\\\gamma_{v_C}m_Cv_C\emat
    \]
    
    Notice the following:
    \begin{itemize}
        \item It is a 2-equation 2-unknown $(v_B,v_C)$ problem, which means the velocities of the decay products are always fixed. It is reasonable because conservation of 4-momentum is equivalent to conservation of both energy and momentum, i.e. always an elastic collision. 
        
        \item This conservation relation holds for any observer, i.e. it still holds no matter how many time you Lorentz transform it. This also the principle of relativity, that physics is unchanged no matter which inertial frame you choose. 
    \end{itemize}
    
    \item \textbf{4-momentum of photon}\\
    Photon has no mass. But it surely has energy and momentum. So we do not define photon’s 4-momentum from 4-velocity, but directly from its energy and momentum. Observing 4-momentum have the form
    \[
    \bmat \gamma mc\\\gamma mv \emat 
    = \bmat E/c\\p \emat
    \]
    Which $E$ is the relativistic total energy, and $p$ is the relativistic momentum of the particle. 4-momentum of photon can (and only) be written in this form. Also, we have E=pc for photon in vacuum (which will be derived in last section). So it can also be written as 
    \[
    \bmat E/c\\p\emat
    =\bmat p\\p \emat 
    =\bmat E/c\\E/c \emat
    \]
    
\end{enumerate}

%%%%%%%%%%%%%%%

From quantum mechanics, photon’s energy and momentum is related to its frequency and wavelength respectively:
\[
E=h\nu,\quad  p=\pm\frac{h}{\lambda}
\]

$+$ or $=$ sign depends on its travelling direction. Substitute into its momentum 4-vector
\[
\bmat E/c\\p\emat
= \bmat h\nu/c \\ \pm h/\lambda \emat 
= \bmat h/\lambda \\\pm h/\lambda \emat
\]

E.g. A photon moving towards a ‘static’ observer (take this as -ve direction). Transform to a ‘moving’ observer with velocity +v relative to ‘static’, the momentum 4-vector transform as
\[
\bmat
\gamma & \-\gamma\beta \\
-\gamma\beta & \gamma 
\emat 
\bmat h/\lambda \\- h/\lambda \emat
=
\bmat \gamma(1+\beta)h/\lambda \\ -\gamma(1+\beta)h/\lambda \emat
= 
\bmat h/\lambda' \\ -h/\lambda' \emat
\]

Which $\lambda'$ is the wavelength observed by ‘moving’ observer, is then
\[
\frac{h}{\lambda'} = \gamma(1+\beta) \frac{h}{\lambda} 
= \frac{1+\beta}{\sqrt{1-\beta^2}} \frac{h}{\lambda}
= \sqrt{\frac{1+\beta}{1-\beta}} \frac{h}{\lambda} 
> \frac{h}{\lambda} 
\]
\[
\lambda' < \lambda
\]

The observed wavelength become smaller, i.e. blueshift occurs. You can also spot out the stupid formula of relativistic Doppler effect which appears in many textbooks (which the sign convention is hard to remember):
\[
\lambda' = \sqrt{\frac{1\mp\beta}{1\pm\beta}} \lambda
\]
Do not use it if you feel confused. Using Lorentz matrix is enough to solve everything. 


%%%%%%%%%%%%%%%%

(It is just a 2D 4-momentum conservation problem.) A photon travelling in $+x$ direction with momentum $p$ hits a resting electron. After hitting, the photon’s momentum become $p'$, goes in the direction $\theta$ above x-axis. Electron gains some velocity $v$, moving in direction $\phi$ below x-axis. Directly write down their initial and final 4-momentum:
\[
\bmat p\\p\\0 \emat
+ \bmat m_ec\\0\\0 \emat
= 
\bmat p'\\p'\cos\theta \\ p'\sin\theta \emat
+ \bmat \gamma_vm_ec \\ \gamma_vm_ev\cos\phi \\ -\gamma_vm_ev\sin\phi \emat
\]
Which becomes 3 equations. Do some substitutions to remove $\phi$ and $v$:\\
\begin{enumerate}
    \item Take square on both sides
    \[
    \begin{cases}
    (p+m_ec-p')^2 = \gamma_v^2m_e^2c^2 \\
    (p-p'\cos\theta)^2 = \gamma_v^2m_e^2v^2\cos^2\phi \\
    (p'\sin\theta)^2 = \gamma_v^2m_e^2v^2\sin\phi \\
    \end{cases}
    \]
    
    \item Add \nth{2} and \nth{3}
    \[
    (p-p'\cos\theta)^2+(p'\sin\theta)^2 
    = p^2-2pp'\cos\theta + p'^2
    =\gamma_v^2m_e^2v^2
    =\gamma_v^2m_e^2c^2\cdot \frac{v^2}{c^2}
    \]
    
    \item Since $\gamma = \frac{1}{\sqrt{1-\frac{v^2}{c^2}}} \Rightarrow (1-\frac{v^2}{c^2})\gamma^2 = 1 $, minus the above from \nth{1}:
    \[
    (p+m_ec-p')^2-(p^2-2pp'\cos\theta+p'^2) = m_e^2c^2
    \]
\end{enumerate}

Further simplify it (do it yourself) and replace with $p=h/\lambda$, you should be able to recover the textbook formula
\[
\lambda'-\lambda = \frac{h}{m_ec}(1-\cos\theta)
\]

The problem can also be extended to the case that the photon hitting a moving electron with velocity $v_0$. But you don’t need to do direct calculation using 4-momentum $\bmat \gamma_{v_0}m_ec\\ \gamma_{v_0}m_ev_0\\0 \emat$. Do the resting electron case and then apply $\Lambda(-v_0)$ to the result is much faster.


%%%%%%%%%%%%%%%%%%%%%%

For light travelling in medium of refractive index $n, c\rightarrow \frac{c}{n}  ,\nu\rightarrow \nu,\lambda \leftarrow \frac{\lambda}{n}$ . So $E=h\nu \leftarrow h\nu, p=\frac{h}{\lambda} \leftarrow \frac{nh}{\lambda}$. The 4-momentum is now 
\[
\bmat h\nu/c \\ nh/\lambda \emat 
= \bmat h/\lambda \\ nh/\lambda \emat
\]

Assume this is what a ‘static’ observer sees. Transform to a ‘moving’ observer with velocity $-\vec{v}$ (so that the medium is moving with velocity $+\vec{v}$ in his view) :

\[
\bmat 
\gamma & \gamma\beta \\
\gamma\beta & \gamma 
\emat
\bmat h/\lambda \\ nh/\lambda \emat
=
\bmat \gamma(1+n\beta)h/\lambda \\ \gamma(n+\beta)h/\lambda \emat
=\bmat h/\lambda' \\ n'h'/\lambda' \emat
\]

\[
\Rightarrow 
\lambda' = \frac{\lambda}{\gamma(1+n\beta)}
, \quad
n' = \frac{\lambda'}{\lambda}\cdot \gamma (\beta+n) 
= \frac{n+\beta}{1+n\beta}
= n + (1-n^2)\beta -n\beta^2 + \cdots 
\]

Checking: when $v\ll c, \beta\approx 0, n'\approx n$.

\linesep
% Section %%%%%%%%%%%%%%%%%%%%%%%%%%%%%%%%%%%%%%%%%%%%%%%%%%%%
\section{Spacetime Invariants}

%%%%%%%%%%%%%%
\subsection{Spacetime Interval}

Spacetime interval is the ‘distance’ between spacetime coordinate. The ‘distance’ we commonly know is so called the Euclidean distance, defined as 
\[
(\Delta s)^2 = (\Delta x)^2 + (\Delta y)^2 + (\Delta z)^2
\]

But in spacetime coordinate, the interval (or called the Minkowski distance) is defined as 
\[
(\Delta s)^2 = -(c\Delta t)^2 + (\Delta x)^2 + (\Delta y)^2 + (\Delta z)^2
\]
Notice the minus sign before $(c\Delta t)^2$

It is defined in this way, because ‘distance’ between two points should be the same for whatever choice of coordinate system, i.e. should be invariant under coordinate transformation. For coordinate transform under Lorentz matrix
\[
\bmat 
\gamma & -\gamma\beta \\
-\gamma\beta & \gamma
\emat
\bmat c\Delta t\\\Delta x \emat
=
\bmat \gamma c\Delta t-\gamma\beta\Delta x \\ -\gamma\beta c\Delta t+\gamma\Delta x\emat
=
\bmat c\Delta t' \\\Delta x' \emat
\]

New spacetime interval is
\begin{align*}
    (\Delta s')^2 &= -(c\Delta t')^2+(\Delta x')^2\\
    &= -(\gamma c\Delta t-\gamma\beta\Delta x)^2 + (-\gamma\beta c\Delta t+\gamma\Delta x)^2\\
    &= -\gamma^2c^2\Delta t^2-\gamma^2\beta^2\Delta x^2+2\gamma^2c\beta\Delta x\Delta t + \gamma^2\beta^2c^2\Delta t^2 + \gamma^2\Delta x^2 - 2\gamma^2\beta c\Delta x\Delta t\\
    &= -\gamma^2(1-\beta^2)c^2\Delta t^2 + \gamma^2(1-\beta^2)\Delta x^2\\
    &= -(c\Delta t)^2+(\Delta x)^2\\
    &= (\Delta s)^2
\end{align*}
Showing that it is an invariant quantity.\\

We can use the interval to determine causality between events. Relations between two events A, B are classified into:
\[
(\Delta s)^2
\begin{cases}
>0 & \text{space-like}\\
=0 & \text{light-like}\\
<0 & \text{time-like}
\end{cases}
\]

\begin{itemize}
    \item \textbf{Space-like}: $(\Delta s)^2 = (x_B-x_A)^2-c^2(t_B-t_A)^2 >0$ It means when a light beam is sent out at $x_A$ right after event A happens, it cannot reach $x_B$ within the duration $(t_B-t_A)$. There cannot be any communication between the two events, thus event A cannot do any impact to event B. So no causality. Some observers may see event B happens at the same time as event A, or even earlier than A: From Lorentz transform, 
    \[
    c(t_B'-t_A') = c\Delta t' = \gamma c\Delta t-\gamma\beta\Delta x
    , \quad
    \text{with} \Delta x=x_B-x_A>c(t_B-t_A) = c\Delta t
    \]
    \[
    \text{take } \beta \left\{
    \begin{array}{l l l l}
    > \frac{c\Delta t}{\Delta x} & \quad & c\Delta t'<0 & \text{(B before A)} \\[0.3em]
    = \frac{c\Delta t}{\Delta x} & \Rightarrow & c\Delta t'=0 & \text{(same time)} \\[0.3em]
    < \frac{c\Delta t}{\Delta x} & \quad & c\Delta t'>0 & \text{(A before B)} 
    \end{array}\right.
    \]
    
    \item \textbf{Light-like}: It means when a light beam sent out at $x_A$ right after event A happens just enough to reach $x_B$ within the duration $(t_B-t_A)$. Event A can have impact on event B by light speed communication.
    
    \item \textbf{Time-like}: Event A can have impact on event B by slower-than-light-speed communication. Some observers may see event A and B happens at the same location: From Lorentz transform,
    \[
    \Delta x' = -\gamma\beta c\Delta t + \gamma\Delta x
    ' \quad
    \text{with} \Delta x = x_B-x_A<c(t_B-t_A)=c\Delta t
    \]
    \[
    \text{take } \beta \left\{
    \begin{array}{l l l l}
    > \frac{\Delta x}{c\Delta t} & \quad & c\Delta x'<0 & \quad \\[0.3em]
    = \frac{\Delta x}{c\Delta t} & \Rightarrow & c\Delta x'=0 & \text{(same location)} \\[0.3em]
    < \frac{\Delta x}{c\Delta t} & \quad & c\Delta x'>0 & \quad 
    \end{array} \right.
    \]
    
\end{itemize}



%%%%%%%%%%%%%%
\subsection{Energy-Mass Relation}

The length of a vector defines the same way as distance between coordinate. For all 4-vectors, their length can be calculated as 
\[
|\vec{V}|^2 = -(\text{Time component})^2 + \sum (\text{position components})^2
\]

\begin{enumerate}
    \item 4-velocity:
    \[
    \vec{U} = \bmat c\\0 \emat 
    \quad \text{or} \quad
    \vec{U} = \bmat \gamma_vc\\\gamma_vv \emat 
    \]
    The 'length' is always $|\vec{U}|^2 = -c^2$.
    
    \item 4-momentum:
    \[
    \vec{P} = \bmat mc\\0 \emat 
    \quad \text{or} \quad
    \vec{P} = \bmat \gamma_vmc\\\gamma_vmv \emat 
    \]
    The 'length' is always $|\vec{P}|^2 = -m^2c^2$. \\
    But recall that 4-momentum can also be written as $\vec{P} = \bmat E/c\\p\emat$. Its length is $-E^2/c^2+p^2$. So 
    \[
    -\frac{E^2}{c^2}+p^2=-m^2c^2
    \]
    \[
    E^2=m^2c^4+p^2c^2
    \]
    Which is the famoius energy-mass relation. For photon, $m=0$. So $E=pc$.
\end{enumerate}

\fi

%%%
\theend
\end{document}