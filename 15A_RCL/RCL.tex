\documentclass[class=article, crop=false, 12pt]{standalone}
\usepackage[subpreambles=true]{standalone}
\usepackage{../.common/common}


\author{Tony Shing}
%\pretitle{Supplementary}

\topic{T15A (Electromagnetism)}
\title{R-C-L Components}

\version{2025} % leave blank for omitting

\begin{document}

\maketitle


\begin{overview}
    
    This is all about finding the electrical properties of given configurations.
    In general, 
    \begin{itemize}
        \item \bf{\ul{Resistance}}: 
        Assume $V \ \xrightarrow{\text{Def.}}\ \vvec{E} \ \xrightarrow{\text{Ohm}}\ \vvec{J} \ \xrightarrow{\text{Def.}}\ I$. 
        Finally $R=\frac{V}{I}$
        \item \bf{\ul{Capacitance}}: 
        Assume $Q \ \xrightarrow{\text{Gauss}}\ \vvec{E} \ \xrightarrow{\text{Def.}}\ V$.
        Finally $C=\frac{Q}{V}$
        \item \bf{\ul{Inductance}}: 
        Assume $I \ \xrightarrow{\text{Ampere}}\ \vvec{B} \ \xrightarrow{\text{Faraday}}\ \epsilon$.
        Finally $L=\frac{\epsilon}{\dv{I}{t}}$
    \end{itemize}
\end{overview}




% content begins here
% Section %%%%%%%%%%%%%%%%%%%%%%%%%%%%%%%%%%%%%%%%%%%%%%%%%%%%
\section{Resistance}

By definition, resistance is


%%%%%%%%%%%%%%
\subsection{Ohm's Law}

%%%%%%%%%%%%%%
\subsection{Addition of Resistance}

%%%%%%%%%%%%%%
\subsection{Conductivity \& Microscopic Form of Ohm's Law}

%%%%%%%%%%%%%%
\subsection{Finding Resistance from Configurations}


\linesep
% Section %%%%%%%%%%%%%%%%%%%%%%%%%%%%%%%%%%%%%%%%%%%%%%%%%%%%
\section{Capacitance}

%%%%%%%%%%%%%%
\subsection{Potential \& Capacitance}

%%%%%%%%%%%%%%
\subsection{Addition of Capacitance}

%%%%%%%%%%%%%%
\subsection{Finding Capacitance from Configurations}

\linesep
% Section %%%%%%%%%%%%%%%%%%%%%%%%%%%%%%%%%%%%%%%%%%%%%%%%%%%%
\section{Inductance}

%%%%%%%%%%%%%%
\subsection{Self-Inducance \& Mutual Inductance}

%%%%%%%%%%%%%%
\subsection{Addition of Inductance}

%%%%%%%%%%%%%%
\subsection{Finding Inductance from Configurations}

%%%
\theend
\end{document}