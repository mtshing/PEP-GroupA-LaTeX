\documentclass[class=article, crop=false, 12pt]{standalone}
\usepackage[subpreambles=true]{standalone}
\usepackage{../.common/common}


\author{Tony Shing}
%\pretitle{Supplementary}

\topic{T15A (Electromagnetism)}
\title{R-C-L Components}

\version{2025} % leave blank for omitting

\begin{document}

\maketitle


\begin{overview}
    
    This note reviews the 3 most basic electronic components: 
    \begin{itemize}
        \item Resistors
        \item Capacitors
        \item Inductors
    \end{itemize}
    and how to find these electrical parameters from a given configuration.
    
    %%%
    The steps are very standard:
    \begin{itemize}
        \item \bf{\ul{Resistance}}: 
        Assume $V \ \xrightarrow{\text{Def.}}\ \vvec{E} \ \xrightarrow{\text{Ohm}}\ \vvec{J} \ \xrightarrow{\text{Def.}}\ I$. 
        Finally $R=\frac{V}{I}$
        \item \bf{\ul{Capacitance}}: 
        Assume $Q_f \ \xrightarrow{\text{Gauss}}\ \vvec{E} \ \xrightarrow{\text{Def.}}\ V$.
        Finally $C=\frac{Q}{V}$
        \item \bf{\ul{Inductance}}: 
        Assume $I_f \ \xrightarrow{\text{Ampere}}\ \vvec{B} \ \xrightarrow{\text{Faraday}}\ \epsilon$.
        Finally $L=\frac{\epsilon}{\dv{I}{t}}$
    \end{itemize}
\end{overview}




% content begins here
% Section %%%%%%%%%%%%%%%%%%%%%%%%%%%%%%%%%%%%%%%%%%%%%%%%%%%%
\section{Resistance}

By definition, resistance $R$ of an electronic component is defined as
\aleq{
    R\ \defeq\ \dvv{R}{I} \ \sim\ \frac{\text{Voltage across}}{\text{Current passing through}}
}

% change rundown: Ohm law {conductivity microscopic} -> resistor in circuit {symbol + real pic + addition rule + some usage} -> find from config

In circuit diagrams, components that carry resistance can be modeled as a \bf{resistor}.
There are two kinds of symbols that are used to represent a resistor.

\insertFig{resistor international vs US}


%%%%%%%%%%%%%%
\subsection{Ohm's Law}

Most daily life material has a linear response between applied voltage and current. 

\insertFig{line plot}

% non-linear response example: semi-conductor diode

In such case we will call the material as \bf{Ohmic material}. 

% physical reason for having resistance? obstruction of electron flow in a material


%%%%%%%%%%%%%%
\subsection{Addition of Resistance}

Multiple resistors can be combined to form an equivalent resistor of desired value.
% begin{center} vline series parallel

In series

\insertFig{resistors in series}

\aleq{
    R_\text{equiv} = R_1 + R_2 + \cdots R_n
}

In parallel

\insertFig{resistors in parallel}

\aleq{
    \inv{R_\text{equiv}} = \inv{R_1} + \inv{R_2} + \cdots + \inv{R_n} 
}

%The addition rules can be proven by Kirchoff's rules


%%%%%%%%%%%%%%
\subsubsection{Resistors in practice}



%%%%%%%%%%%%%%
\subsection{Conductivity \& Microscopic Ohm's Law}

Resistance of an object is a property not only depends on the type of material, 
but also depends on the object's shape. 
To eliminate the shape's effect, 


By the definition of voltage and current in terms of field-like quantities,

\aleq{
    \norm{V} = \int \vvec{E}\cdot \dd{\vvec{l}}
    \qquad\text{and}\qquad
    \norm{I} = \iint \vvec{J}\cdot \dd{\vvec{s}}
}

\subsubsection{Conductivity}




%%%%%%%%%%%%%%
\subsection{Finding Resistance from Configurations}


\linesep
% Section %%%%%%%%%%%%%%%%%%%%%%%%%%%%%%%%%%%%%%%%%%%%%%%%%%%%
\section{Capacitance}

%%%%%%%%%%%%%%
\subsection{Potential \& Capacitance}

%%%%%%%%%%%%%%
\subsection{Addition of Capacitance}

%%%%%%%%%%%%%%
\subsection{Finding Capacitance from Configurations}

\linesep
% Section %%%%%%%%%%%%%%%%%%%%%%%%%%%%%%%%%%%%%%%%%%%%%%%%%%%%
\section{Inductance}

%%%%%%%%%%%%%%
\subsection{Self-Inducance \& Mutual Inductance}

%%%%%%%%%%%%%%
\subsection{Addition of Inductance}

%%%%%%%%%%%%%%
\subsection{Finding Inductance from Configurations}

%%%
\theend
\end{document}