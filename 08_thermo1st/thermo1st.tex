\documentclass[class=article, crop=false, 12pt]{standalone}
\usepackage[subpreambles=true]{standalone}
\usepackage{../.common/common}


\author{Tony Shing}
%\pretitle{Supplementary}

\topic{Note 08 (Thermodynamics)}
\title{Thermodynamics \nth{1} Law}

\version{2025} % leave blank for omitting

\begin{document}

\maketitle


\begin{overview}
    \begin{itemize}
        \item Terminologies: State functions \& Processes
        \item Review the 4 most common thermal processes 
        $\bcase{
            &\text{Formula true for everything} \\ 
            &\text{Formula for ideal gas only}
        }$
        \item Calculation for thermal cycle, efficiency and COP
    \end{itemize}
\end{overview}


% content begins here
% Section %%%%%%%%%%%%%%%%%%%%%%%%%%%%%%%%%%%%%%%%%%%%%%%%%%%%
\section{Thermodynamics \nth{1} Law}

The thermodynamics \nth{1} law is essentially energy conservation.
\aleq{
    &&&&\var{Q} = \dd{U} + \var{W}  &&&&(\text{Physics convention})
}
\begin{itemize}
    \item $\var{Q} = $ Heat \cul[red]{input to} the system.
    \item $\dd{U} = $ Change in internal energy of the system.
    \item $\var{W} = $ Work done \cul[red]{by} the system.
\end{itemize}

\ul{Note:} The convention in Chemistry books are different from Physics books.
\aleq{
    &&&&\dd{U} = \var{Q} + \var{W}  &&&&(\text{Chemistry convention})
}
This is due to chemists refer $\var{W}$ as the work done \cul[red]{to} system,
so work done \cul[blue]{by} system is $\cul[blue]{-}\var{W}$.


%%%%%%%%%%%%%%
\subsection{Terminologies in Thermodynamics}

Before we deep dive into the world of thermodynamics, 
here are some terminologies you should understand.

\begin{enumerate}
    \item \bf{\ul{State}}\\
    We describe a system in a "specific state" 
    if the system can be "well-distinguished" by a set of parameters.
    \begin{center}
        \red{Different parameters \quad$\Leftrightarrow$\quad Different states}
    \end{center}

    For example,
    \begin{itemize}
        \item \ul{Ideal gas system}\\
        Every possible state of a box of gas can be described solely by 3 parameters $(P,V,T)$.\\
        \gray{But only 2 of them are independent, because we have ideal gas law $PV=nRT$.}\\

        \item \ul{Mechanical system}\\
        Mechanical system must satisfy Newton's \nth{2} law - a \nth{2} order ODE
        - which the particle's motion is completely determined after knowing its initial position $(x,y,z)$ and velocity $(v_x,v_y,v_z)$, 
        6 parameters in total.\\
        \gray{To describe the state of a mechanical system with $N$ bodies, it takes $6N$ parameters.}

    \end{itemize}

    \vskip 1ex
    \item \bf{\ul{State Space}}\\
    The state space is an abstract representation of \cul[red]{the set of all possible states} using state parameters.
    Because ideal gas system only takes 3 parameters, 
    we can draw it out as a 3D space, 
    with each coordinate representing a possible state of the gas

    \insertFig{state space}

    \vskip 1ex
    \item \bf{\ul{Process}}\\
    It is the transition between states.
    If the state space can plotted out, 
    processes are the paths connecting different states.

    \insertFig{process}

    For ideal gas, because only 2 of $(P,V,T)$ are independent,
    we can project the processes' path onto a 2D plane 
    creating the P-V diagram (or P-T/V-T diagram).

    \vskip 1ex
    \item \bf{\ul{State Function}}\\
    State functions are functions that only take state parameters as inputs.
    \begin{center}
        \red{A state function's values are well-defined for every state.}
    \end{center}

    For example, for a scalar state function,
    we can draw it as a smooth height map:

    \insertFig{height map -> PV diagram}

    You can think of a state function like some potential function.
    \aleq{
        \mstack{\text{Total change}\\\text{of }F(P,V)} 
        = \int_{\substack{\yellow{\text{Any process}}\\\blue{(P_1,V_1)}\to\red{(P_2,V_2)}}} \dd{F}
        = F(\red{P_2,V_2}) - F(\blue{P_1,V_1})
    }



\end{enumerate}

%%%%%%%%%%%%%%
\subsection{State Functions v.s. Non-State Functions}

Here are some examples of state function in thermodynamics:
\begin{itemize}
    \item \ul{The state parameters themselves} - you can always write something like $F(P,V,T)=P$.
    
    \item \ul{Internal energy} - such as potential energy and kinetic energy. 
    
    \insertFig{PE}

    \item \ul{Entropy} \gray{(Only in some situation)}
    
\end{itemize}

And here are some examples that are NOT state function in thermodynamics:
\begin{itemize}
    \item \ul{Work Done} \\
    By definition,
    \aleq{
        \Delta \text{(W.D.)} = \tkn{WD_P}{\cul[red]{P}}\cdot \tkn{WD_V}{\cul[blue]{\Delta V}}
    }
    \addArrow[red]{WD_P}{(-3ex,-3ex)}
    {\scriptsize $P$ is a state parameter\\[-0.8ex]\scriptsize No problem}
    {(0,-1ex)}{(-5ex,-1ex)}
    \addArrow[blue]{WD_V}{(3ex,-3ex)}
    {\scriptsize $\Delta V$ = \cul[blue]{\blue{Change}} of a state parameter\\[-0.8ex]\scriptsize NOT well-defined on a given state}
    {(0,-1ex)}{(8ex,-1ex)}

    \vskip 1.5em
    Because you cannot define work done with only one state, 
    it cannot be a state function. 
    In fact, we always visualize W.D. as the area under curve (= process!) in the P-V diagram,
    indicating that it is a property of a process.
    \aleq{
        \sum_{\text{segments }i} P_iV_i \sim \int_\text{process} P\dd{V} = \text{Area under curve}
    }

    \insertFig{PV work done}

    \item \ul{Heat}\\
    According to the \nth{1} law,
    \aleq{
        \var{Q} = \tkn{H_U}{\cul[red]{\dd{U}}} + \tkn{H_W}{\cul[blue]{\var{W}}}
    }
    \addArrow[red]{H_U}{(-3ex,-3ex)}{\scriptsize State function}{(0,-1ex)}{(-2ex,0)}
    \addArrow[blue]{H_W}{(3ex,-3ex)}{\scriptsize Process dependent}{(0,-1ex)}{(2ex,0)}

    So heat must also be dependent of a process.

\end{itemize}

\begin{notation}[Side note:]
    Note the notation difference between $\dd{U}$ v.s. $\var{W},\var{Q}$:
    \begin{itemize}
        \item Change in state function $\Rightarrow$ Use \red{$\dd$} 
        \item Change in path dependent function $\Rightarrow$ Use \red{$\var$}
    \end{itemize}

    This is significant when doing (line) integral:
    \begin{itemize}
        \item \red{$\dd$} $\Rightarrow$ Independent of path. 
        Simply substract the initial value from final value.
        \aleq{
            \int \dd{U} = U_f - U_i
        }

        \item \red{$\var$} $\Rightarrow$ Path dependent.
        Must do the integral explicitly.
        \begin{itemize}
            \item Without knowing the given path, we must write $\int \red{\var}{W}, \int \red{\var}{Q}$
            \item After the path is known, we can write $\int_\green{C} \red{\dd}{W}, \int_\green{C} \red{\dd}{Q}$
        \end{itemize}
    \end{itemize}
\end{notation}


\linesep
% Section %%%%%%%%%%%%%%%%%%%%%%%%%%%%%%%%%%%%%%%%%%%%%%%%%%%%
\section{The 4 Most Common Processes}

\begin{notation}
    In the following section, I will stick to the colour scheme:
    \begin{itemize}
        \item \fcbox[red]{Red boxed} - True for any processes.
        \item \fcbox[blue]{Blue boxed} - Only true for ideal gas. Derivation requires ideal gas' properties.
    \end{itemize}
\end{notation}

\vskip 1em
Normally, thermodynamics texts only concern these 4 processes:
\begin{center}
    \begin{tabular}{c|c}
        Isovolumetric (Isochoric) & $V=$ const. \\
        \hline
        Isobaric & $P =$ const. \\
        \hline
        Isothermal & $T=$ const. \\
        \hline
        Adiabetic & $\var{Q} = 0$
    \end{tabular}
\end{center}

It is essential to find the $\dd{U}, \var{W}$ and $\var{Q}$ for each of the 4 process.
We can even promote the derivation to find them for arbituary processes.

%%%%%%%%%%%%%%
\subsection{Internal Energy}

As mentioned, internal energy is a state function - 
\red{change in internal energy is independent of process.}
So if the function form of $U$ is not known, we can only write
\aleq{
    \Acboxed[red]{
        \Delta U = U(P_2,V_2,T_2) - U(P_1,V_1,T_1)
    }
}

In case of ideal gas, 
the function form of $U$ is derived using kinetic theory:
\vskip 1em
\aleq{
    U(P,V,T) = \tkn{dof}{\frac{\cbox[blue]{i}}{2}}\tkn{U_ideal1}{\cul[green]{PV}} = \tkn{gas_law}{\frac{i}{2}\tkn{U_ideal2}{\cul[green]{NkT}}}
}
\addArrow[blue]{dof}{(0,3ex)}
{\scriptsize $i=$ degree of freedom}{(0,4ex)}
\addBelowArrow[green]{U_ideal1}{U_ideal2}{\scriptsize By ideal gas law}{-2ex}

Therefore,
\aleq{
    \Acboxed[blue]{
        \Delta U = \frac{i}{2}(P_2V_2 - P_1V_1) = \frac{i}{2}Nk\tkn{U_Tonly}{(\cul[green]{T_2-T_1})}
    }
}
\addArrow[green]{U_Tonly}{(0,-5ex)}{$U$ of ideal gas can be written as a function of only $T$!}{(0,-1ex)}



%%%%%%%%%%%%%%
\subsection{Isovolumetric Process (const. $V$)}

\insertFig{iso V}

\begin{enumerate}
    \item \ul{Work Done} \\
    By definition of the process, 
    \aleq{
        \Acboxed[red]{
            \int \var{W} = \int_{\yellow{\text{iso. }V}} \dd{W} 
            = \int_{\yellow{\text{iso. }V}} P\tkn{isoV_W}{\cul[red]{\dd{V}}} = 0
        }
    }
    \addArrow[red]{isoV_W}{(0,-5ex)}{$V$ cannot change}{(0,-0.8ex)}

    \item \ul{Heat} \\
    By \nth{1} law,
    \aleq{
        \var{Q} = \dd{U} + \ccancelto[red]{0}{\var{W}} = \dd{U}
    }
    So we can write 
    \aleq{
        \Acboxed[red]{
            \int \var{Q} = \int_{\yellow{\text{iso. }V}} \dd{Q} = \int \dd{U} = U_2 - U_1
        }
    }

    We can also define $C_V$, the \bf{heat capacity under constant volume}:
    \aleq{
        \Acboxed[red]{
            \int_{\yellow{\text{iso. }V}} \dd{Q} = \int C_\yellow{V} \dd{T}
            \qquad\Leftrightarrow\qquad
            C_\yellow{V}(T) = \qty(\dvv{Q}{T})_\yellow{\text{iso. }V} = \qty(\dvv{U}{T})
        }
    }

    When the internal energy of ideal gas law is given,
    \aleq{
        U_2-U_1 &= \frac{i}{2}(P_2V_2 - P_1V_1) = \frac{i}{2}Nk(T_2-T_1) \\
        &= \frac{i}{2}V(P_2-P_1) 
    }
    so we have
    \aleq{
        \Acboxed[blue]{
            \int_{\yellow{\text{iso. }V}} \dd{Q} \equiv U_2-U_1 = \frac{i}{2}V(P_2-P_1) = \frac{i}{2}Nk(T_2-T_1)
        }
    }
    and
    \aleq{
        \Acboxed[blue]{
            C_\yellow{V}(T) \equiv \qty(\dvv{U}{T}) = \frac{i}{2}Nk = (\text{A constant})
        }
    }

\end{enumerate}

%%%%%%%%%%%%%%
\subsection{Isobaric Process (const. $P$)}

\insertFig{iso P}

\begin{enumerate}
    \item \ul{Work Done}\\
    By definition of work done,
    \aleq{
        \Acboxed[red]{
            \int \var{W} = \int_\yellow{\text{iso. }P} \dd{W}
            = \int P\dd{V} = \tkn{isoP_W}{\cul[red]{P}}\int \dd{V} = P(V_2-V_1)
        }
    }
    \addArrow[red]{isoP_W}{(0,-4ex)}{$P$=constant}{(0,-0.8ex)}

    Substituting ideal gas law, we can also write
    \aleq{
        \Acboxed[blue]{
            \int_\yellow{\text{iso. }P} \dd{W} = Nk(T_2-T_1)
        }
    }

    \item \ul{Heat}\\
    By \nth{1} law,
    \aleq{
        \var{Q} &= \dd{U} + \var{W}\\
        \Acboxed[red]{
            \int_\yellow{\text{iso. }P} \dd{Q} &= (U_2-U_1) + P(V_2-V_1)
        }
    }

    We can also define $C_P$, the \bf{heat capacity under constant pressure}:
    \aleq{
        \Acboxed[red]{
            \int_{\yellow{\text{iso. }P}} \dd{Q} = \int C_\yellow{P} \dd{T}
            \qquad\Leftrightarrow\qquad
            C_\yellow{P}(T) = \qty(\dvv{Q}{T})_\yellow{\text{iso. }P} = \qty(\dvv{U}{T}) + P\qty(\dvv{V}{T})
        }
    }

    Then for ideal gas,
    \aleq{
        \int_\yellow{\text{iso. }P} \dd{Q} &\equiv (U_2-U_1) + P(V_2-V_1) \\
        &= \frac{i}{2}P(V_2-V_1) + P(V_2-V_1) \\
        \Acboxed[blue]{
            \int_\yellow{\text{iso. }P} \dd{Q} &= \frac{i+2}{2}P(V_2-V_1) = \frac{i+2}{2}Nk(T_2-T_1)
        }
    }

    and 
    \aleq{
        \Acboxed[blue]{
            C_\yellow{P}(T) \equiv \qty(\dvv{U}{T}) + P\qty(\dvv{P}{T})= \frac{i+2}{2}Nk = (\text{A constant})
        }
    }
\end{enumerate}

\vskip 1em
%%%%%%%%%%%%%%
\subsection{Isothermal Process (const. $T$)}

\insertFig{iso T}

\begin{enumerate}
    \item \ul{Work Done}\\
    We are unable to reduce the form $\cbox[red]{\displaystyle \int_\yellow{\text{iso. }T}\var{W} = \int P\dd{V}}$ 
    unless we know what is the relation between $P$ and $V$ of the material.\\

    For ideal gas, this relation is already known: $\displaystyle P = \frac{NkT}{V}$.
    Then the integral becomes
    \aleq{
        \Acboxed[blue]{
            \int_\yellow{\text{iso. }T} \var{W} 
            = \int \frac{NkT}{V}\dd{V} 
            = Nk\tkn{isoT}{\cul[blue]{T}}\int \frac{\dd{V}}{V} 
            = NkT\ln\qty(\frac{V_2}{V_1})
        }
    }
    \addArrow[blue]{isoT}{(0,-5ex)}{$T=$ constant}{(0,-1ex)}

    \item \ul{Heat}\\
    Even with \nth{1} law, again, we cannot reduce the form $\cbox[red]{\displaystyle \int_\yellow{\text{iso. }T}\var{Q} = (U_2 - U_1) + \int P\dd{V}}$
    unless we know what is the relation between $P$ and $V$ of the material.

    For ideal gas, because its internal energy can be written in a form that only has $T$,
    \aleq{
        \int \dd{U} = \frac{i}{2}Nk(T_2-T_1) = 0
    }

    Then 
    \aleq{
        \Acboxed[blue]{
            \int_\yellow{\text{iso. }T} \var{Q} 
            = 0 + \int_\yellow{\text{iso. }T} \var{W} 
            = NkT\ln\qty(\frac{V_2}{V_1})
        }
    }

\end{enumerate}

%%%%%%%%%%%%%%
\subsection{Adiabetic Process ($\var{Q}=0$)}

\insertFig{adiabetic}

\begin{enumerate}
    \item \ul{Heat}\\
    By definition of the process, 
    there must be no energy exchange in the form of heat.
    \aleq{
        \Acboxed[red]{
            \var{Q} = 0
        }
    }


    \item \ul{Work Done}\\
    By \nth{1} law, 
    \aleq{
        \var{Q} &= \dd{U} + \var{W} \\
        \Acboxed[red]{
            \var{W} &= - \dd{U}
        }
    }


    \item \ul{Adiabetic Relation}\\
    Because the $P,V,T$ of initial and final states in an adiabetic process are different, 
    we also need to derive the equation of the line connecting the two states.
    The set of equations of the lines is generally called \bf{adiabetic relation}.\\

    The relation depends on the function form of $U(P,V)$.
    Although we can always start the derivation from
    \aleq{
        \Acboxed[red]{
            -\dd{U} = \var{W} = P\dd{V}
        }
    }

    For ideal gas, $U = \frac{i}{2}PV$, 
    so $\dd{U} = \frac{i}{2}(P\dd{V} + V\dd{P})$.
    Substitute to above,
    \aleq{
        -\frac{i}{2}(P\dd{V}+V\dd{P}) &= P\dd{V} \\
        -\frac{i}{2} V\dd{P} &= \frac{i+2}{2}P\dd{V}\\
        \inv{P}\dd{P} &= -\frac{i+2}{i}\inv{V}\dd{V}\\
        \int \inv{P}\dd{P} &= -\frac{i+2}{i} \int \inv{V}\dd{V}\\
        \ln{P} &= -\frac{i+2}{i} \ln{V} + (\text{Constant})\\
        \ln\qty(PV^{\frac{i+2}{i}}) &= (\text{Constant}) \\
        \Acboxed[blue]{
            PV^{\frac{i+2}{i}} &= (\text{Constant})
        }
    }

    In general, adiabetic relation of common materials would have the form $\cbox[red]{PV^\gamma = (\text{constant})}$,
    where the index $\gamma$ is called \bf{adiabetic constant}. \\

    For ideal gas, $\cbox[blue]{\gamma = \frac{i+2}{i} = \frac{C_P}{C_V}}$.
    However this value \cul[red]{does NOT apply to every material}. 
    To some material, its adiabetic "constant" may not even be a constant.  


\end{enumerate}

\newpage
%%%%%%%%%%%%%%
\subsection{Summary}

Summarizing under a table:

\begin{center}
    \begin{tabular}{c|c|c|c}
        & \makecell{$\dd{U}$ \\ \green{is a state function}} 
        & \makecell{$\var{W}$ \\ \green{derive from definition:}\\\green{$\var{W} = \int P\dd{V}$}} 
        & \makecell{$\var{Q}$ \\ \green{derive by \nth{1} law:}\\\green{$\var{Q} = \dd{U} + \var{W}$}}
        \\[1em]
        \hline
        Iso. V 
        &

        $\begin{aligned} 
            & \,\cbox[red]{U(P_2,V,T_2) - U(P_1,V,T_1)}\\
            =& \,\cbox[blue]{\frac{i}{2}V(P_2-P_1)}\\
            =& \,\cbox[blue]{\frac{i}{2}Nk(T_2-T_1)}
        \end{aligned}$
        & 0
        & 
        \makecell{
            \phantom{\scriptsize abc}\\
            $\begin{aligned}
                &= \cbox[red]{\dd{U} + 0}\\
                &= \cbox[red]{\int C_V \dd{T}} \\
                &= \cbox[blue]{\cub[blue]{\frac{i}{2}Nk}{C_V}(T_2-T_1)}
            \end{aligned}$\\
            \phantom{\scriptsize abc}
        }
        \\[1em]
        \hline
        %
        Iso. P
        & 
        $\begin{aligned} 
            & \,\cbox[red]{U(P,V_2,T_2) - U(P,V_2,T_1)}\\
            =& \,\cbox[blue]{\frac{i}{2}P(V_2-V_1)}\\
            =& \,\cbox[blue]{\frac{i}{2}Nk(T_2-T_1)}
        \end{aligned}$
        & 
        $\begin{aligned}
            & \,\cbox[red]{P(V_2-V_1)} \\
            =& \,\cbox[blue]{Nk(T_2-T_1)}
        \end{aligned}$
        & 
        \makecell{
            \phantom{\scriptsize abc}\\
            $\begin{aligned}
                &= \cbox[red]{\dd{U} + \var{W}}\\
                &= \cbox[red]{\int C_P \dd{T}} \\
                &= \cbox[blue]{\cub[blue]{\frac{i+2}{2}Nk}{C_P}(T_2-T_1)}
            \end{aligned}$\\
            \phantom{\scriptsize abc}\\
        }
        \\[2em]
        \hline
        %
        Iso. T
        & 
        $\begin{aligned} 
            & \,\cbox[red]{U(P_1,V_2,T) - U(P_1,V_2,T)}\\
            =& \,\cbox[blue]{0}\\
        \end{aligned}$
        & 
        \makecell{
            \phantom{\scriptsize abc}\\
            $\begin{aligned}
                & \,\cbox[red]{\int P\dd{V}} \\
                =& \,\cbox[blue]{NkT\ln\qty(\frac{V_2}{V_1})}
            \end{aligned}$\\
            \phantom{\scriptsize abc}
        }
        & 
        $\begin{aligned}
            &= \cbox[red]{\dd{U} + \var{W}}\\
            &= \cbox[blue]{NkT\ln\qty(\frac{V_2}{V_1})}
        \end{aligned}$
        \\[1em]
        \hline
        Adiabetic
        & 
        \makecell{
            \phantom{\scriptsize abc}\\
            $\begin{aligned} 
                & \,\cbox[red]{U(P_2,V_2,T_2) - U(P_1,V_1,T_1)}\\
                =& \,\cbox[blue]{\frac{i}{2}(P_2V_2-P_1V_2)}\\
                =& \,\cbox[blue]{\frac{i}{2}Nk(T_2-T_1)}
            \end{aligned}$\\
            \phantom{\scriptsize abc}
        }
        
        & 
        $\begin{aligned} 
            &= \cbox[red]{-\dd{U}}\\
            &= \cbox[blue]{-\frac{i}{2}(P_2V_2-P_1V_2)}\\
            &= \cbox[blue]{-\frac{i}{2}Nk(T_2-T_1)}
        \end{aligned}$
        & 0
    \end{tabular}
\end{center}




\linesep
\newpage
% Section %%%%%%%%%%%%%%%%%%%%%%%%%%%%%%%%%%%%%%%%%%%%%%%%%%%%
\section{Solving Thermal Cycle}

In this section, 
we will deal with one of the extremely common problem in thermodynamics - 
given an arbituary thermal cycle, 
derive the formula of effciency / coefficient of performance (COP). 

\insertFig{random cycle}

In general, you can follow these steps:
\begin{notation}[]
    \begin{enumerate}
        \item Write down the relation of $P,V,T$ between initial/final states of each process.
        \item Write down the $\dd{U}, \var{W}, \var{Q}$ for each process. 
        \item Identify if the $\var{Q}$ of each process is an input or output.
        \item Calculate efficiency / COP according to the signs of $\var{Q}$.
    \end{enumerate}
\end{notation}

\begin{example}
    A thermal cycle of \cul[blue]{ideal gas} with all 4 kinds processes.
    \begin{enumerate}
        \item Iso. T (expansion)
        \item Adiabetic (expansion)
        \item Iso. P (contraction)
        \item Iso. V
    \end{enumerate}

    \insertFig{cycle of 4 process}

    \begin{enumerate}
        \item Write down the relations of $P,V,T$ between initial/final states of each process.
        \begin{center}
            \begin{minipage}{0.65\linewidth}
                \centering
                \begin{tabular}{>{\centering\arraybackslash}m{3cm} >{\centering\arraybackslash}m{3cm} >{\centering\arraybackslash}m{3cm}}
                    & Process & Relation
                    \\
                    \hline
                    \cbox[blue]{1} $\rightarrow$ \cbox[blue]{2}
                    & iso. T
                    & $T_1=T_2$
                    \\
                    \hline
                    \cbox[blue]{2} $\rightarrow$ \cbox[blue]{3}
                    & Adiabetic
                    & $P_2V_2^\gamma=P_3V_3^\gamma$
                    \\
                    \hline
                    \cbox[blue]{3} $\rightarrow$ \cbox[blue]{4}
                    & iso. P
                    & $P_3=T_4$
                    \\
                    \hline
                    \cbox[blue]{4} $\rightarrow$ \cbox[blue]{1}
                    & iso. V
                    & $V_4=V_1$
                \end{tabular}
            \end{minipage}
            \hspace{1ex}
            \begin{minipage}{0.3\linewidth}
                \centering
                \red{
                You can use ideal gas law to get more relations.\\
                But may do it later.}
            \end{minipage}
        \end{center}
        
    \item Write down the $\dd{U}, \var{W}, \var{Q}$ for each process.
    \begin{center}
        \begin{tabular}{>{\centering\arraybackslash}m{2cm} 
            >{\centering\arraybackslash}m{2.5cm} 
            c}
            & Process & $\dd{U},\var{Q},\var{W}$
            \\
            \hline
            \cbox[blue]{1} $\rightarrow$ \cbox[blue]{2}
            & iso. T
            & \makecell[l]{
                \phantom{\scriptsize abc}\\
                $\bcase{
                    \Delta U &= 0\\
                    \int \var{W} &= NkT_1\ln\qty(\frac{V_2}{V_1})\\
                    \int \var{Q} &= NkT_1\ln\qty(\frac{V_2}{V_1}) \quad \red{>0}\\
                }$\\
                \phantom{\scriptsize abc}
            }
            \\
            \hline
            \cbox[blue]{2} $\rightarrow$ \cbox[blue]{3}
            & Adiabetic
            & \makecell[l]{
                \phantom{\scriptsize abc}\\
                $\bcase{
                    \Delta U &= \frac{i}{2}(P_3V_3-P_2V_2) = \frac{i}{2}Nk(T_3-T_2)\\
                    \int \var{W} &= -\frac{i}{2}(P_3V_3-P_2V_2) = -\frac{i}{2}Nk(T_3-T_2)\\
                    \int \var{Q} &= 0\\
                }$\\
                \phantom{\scriptsize abc}
            }
            \\
            \hline
            \cbox[blue]{3} $\rightarrow$ \cbox[blue]{4}
            & iso. P
            & \makecell[l]{
                \phantom{\scriptsize abc}\\
                $\bcase{
                    \Delta U &= \frac{i}{2}P_3(V_4-V_3) = \frac{i}{2}Nk(T_4-T_3)\\
                    \int \var{W} &= P_3(V_4-V_3) = Nk(T_4-T_3)\\
                    \int \var{Q} &= \frac{i+2}{2}P_3(V_4-V_3) = \frac{i+2}{2}Nk(T_4-T_3)\quad \red{<0}\\
                }$\\
                \phantom{\scriptsize abc}
            }
            \\
            \hline
            \cbox[blue]{4} $\rightarrow$ \cbox[blue]{1}
            & iso. V
            & \makecell[l]{
                \phantom{\scriptsize abc}\\
                $\bcase{
                    \Delta U &= \frac{i}{2}V_4(P_1-P_4) = \frac{i}{2}(T_1-T_4)\\
                    \int \var{W} &= 0\\
                    \int \var{Q} &= \frac{i}{2}V_4(P_1-P_4) = \frac{i}{2}(T_1-T_4) \quad \red{>0}\\
                }$\\
                \phantom{\scriptsize abc}
            }
        \end{tabular}
    \end{center}
    
    
    \item Identify if the $\var{Q}$ of each process is an input or output. 
    Recall in the \nth{1} law's convention,
    \aleq{
        \tkn{eg_Q}{\var{Q}} = \tkn{eg_U}{\dd{U}} + \tkn{eg_W}{\var{W}}
    }
    \addArrow[red]{eg_Q}{(-5ex,-3ex)}{Heat \ul{input}\\to the system}{(0,-1ex)}{(-5ex,-1ex)}
    \addArrow[red]{eg_U}{(0,-3ex)}{Increase in $U$\\of the system}{(0,-1ex)}{(0,-2ex)}
    \addArrow[red]{eg_W}{(5ex,-3ex)}{W.D. \ul{by}\\the system}{(0,-1ex)}{(4ex,-1ex)}

    \vskip 1em
    If $\var{Q}> 0$, it is a heat input, otherwise it is a heat output. 
    We can check for each process,
    \begin{itemize}
        \item With heat input: \cbox[blue]{1} $\rightarrow$ \cbox[blue]{2}, \cbox[blue]{4} $\rightarrow$ \cbox[blue]{1}
        \item With heat output: \cbox[blue]{3} $\rightarrow$ \cbox[blue]{4}
    \end{itemize}

    \vskip 1em
    \item Calculate efficiency / COP according to the signs of $\var{Q}$.
    By definition, 
    \aleq{
        \Aboxed{
            \text{Efficiency} = \eta \ \defeq\  \frac{\text{W.D.}}{\text{Heat input}} 
            = 1- \qty|\frac{\text{Heat output}}{\text{Heat intput}}|
        }
    }
    
    \insertFig{engine eff}

    and
    \aleq{
        \Aboxed{
            \mstack{\text{Coefficienty of}\\\text{Performance}\\(\text{COP})} 
            \ \defeq\  
            \frac{\text{Heat remove}}{\text{Work done}} 
                = \qty|\frac{\text{Heat output}}{\text{Heat intput - Heat output}}|
                = \frac{1-\eta}{\eta}
        }
    }

    \insertFig{engine cop}

    For example in the cycle with 4 processes, we can compute the efficiency as
    \aleq{
        \eta = 1 - \qty|\frac{\dfrac{i+2}{2}Nk(T_4-T_3)}{\dfrac{i}{2}Nk(T_1-T_4) + NT_1\ln\qty(\dfrac{V_2}{V_1})}|
    }
    
    \end{enumerate}

\end{example}

\vskip 1em
\begin{example}
    Thermal processes of photon gas \\

    This time we are dealing with a non-ideal gas material. 
    Photon gas' internal energy and pressure relation are given by
    \aleq{
        \bcase{
            U &= 3PV \blue{= \tkn{photon_U}{aVT^4}}\\
            P &= \frac{1}{3}aT^4
        }
        \quad\quad\quad
        \text{with }a = \text{some constant} 
    }
    \addBentArrow[blue]{photon_U}{(-5.5ex,-4.5ex)}
    {\scriptsize Substitute\\[-1ex]\scriptsize to get}{(0,-1ex)}{(10ex,1ex)}

    \gray{
        In comparison, we cannot use any properties of ideal gas, i.e. 
        $\bcase{
            U &= \frac{i}{2}PV = \frac{i}{2}NkT\\
            P &= \frac{NkT}{V}
        }$
    }

    We have to re-derive $\dd{U}, \var{W}$ and $\var{Q}$ before we can proceed to solve any thermal cycle.

    \begin{enumerate}
        \item \ul{Internal Energy}\\
        Remember that $U$ is always a state function. 
        Its change is independent of process.
        \aleq{
            \Delta U &= U(P_2,V_2,T_2) - U(P_1,V_1,T_1)\\
            \Acboxed[green]{\Delta U &= 3(P_2V_2 - P_1V_1) = a(V_2T_2^4 - V_1T_1^4)}
        }

        \item \ul{Iso V.}\\
        By definition of the process, $\int \var{W}$ is always $0$. 
        Then for heat,
        \aleq{
            \Acboxed[green]{
                \int_\yellow{\text{iso. }V} \dd{Q} = \Delta U
                &= 3V(P_2 - P_1) = aV(T_2^4 - T_1^4)
            }
        } 

        The heat capacity under constant volume is then
        \aleq{
            \Acboxed[green]{
                \int C_\yellow{V} \dd{T} = \dvv{U}{T} = 4aVT^3 = (\text{NOT a constant})
            }
        }

        \item \ul{Iso P. and Iso T.}\\
        Note that in the pressure relation, $P = \inv{3}aT^4$, 
        \cul[red]{if $P$ is fixed, then $T$ is also fixed!}
        \aleq{
            \Acboxed[green]{\Delta U &= 3P(V_2-V_1) = aT^4(V_2-V_1)}\\
            \Acboxed[green]{\int_\yellow{\text{iso. }P} \var{W} &= P(V_2-V_1) = \frac{1}{3}aT^4(V_2-V_1)}\\
            \Acboxed[green]{\int_\yellow{\text{iso. }P} \var{Q} &= \Delta U + \int \var{W} = 4P(V_2-V_1) = \frac{4}{3}aT^4(V_2-V_1)}
        }

        However, it is impossible to define the heat capacity under constant pressure, 
        because you cannot change temperature under constant pressure.
        \aleq{
                \Acboxed[green]{
                \int_\yellow{\text{iso. }P} \var{Q} 
                = \int C_\yellow{P} \dd{T} 
                = \int (\text{undefined})\cdot (0)
            }
        }

        \item \ul{Adiabetic}\\
        By definition of the process, $\int \var{Q}$ is always $0$.
        Then we can derive the adiabetic relation:
        \aleq{
            P\dd{V} &= -\dd{U}\\
            &= -\dd{(3PV)} = -3(P\dd{V}+V\dd{P})\\
            4P\dd{V} &= -3V\dd{P}\\
            \frac{4}{3} \int \frac{\dd{V}}{V} &= -\int\frac{\dd{P}}{P}\\
            \ln(V^{\frac{4}{3}}) &= -\ln P + (\text{constant})\\
            PV^{\frac{4}{3}} &= (\text{constant})
        }

        The adiabetic constant for photon gas is $\gamma =\frac{4}{3} \neq \frac{C_V}{C_P}$, obviously.

    \end{enumerate}

\end{example}

\begin{example}

    Carnot cycle by photon gas \\

    By definitoin, a Carnot cycle is made of 4 processes:

    \begin{enumerate}
        \item Iso. T (expansion)
        \item Adiabetic (expansion)
        \item Iso. T (contraction)
        \item Adiabetic (contraction)
    \end{enumerate}

    \insertFig{cycle of photon carnot}

    \begin{enumerate}
        \item Write down the relations of $P,V,T$ between initial/final states of each process.
        \begin{center}
            \begin{minipage}{0.8\linewidth}
                \centering
                \begin{tabular}{>{\centering\arraybackslash}m{3cm} >{\centering\arraybackslash}m{3cm} >{\centering\arraybackslash}m{3cm}}
                    & Process & Relation
                    \\
                    \hline
                    \cbox[blue]{1} $\rightarrow$ \cbox[blue]{2}
                    & iso. P \& T
                    & $P_1=P_2, T_1=T_2$
                    \\
                    \hline
                    \cbox[blue]{2} $\rightarrow$ \cbox[blue]{3}
                    & Adiabetic
                    & $P_2V_2^\gamma=P_3V_3^\gamma$
                    \\
                    \hline
                    \cbox[blue]{3} $\rightarrow$ \cbox[blue]{4}
                    & iso. P \& T
                    & $P_3=P_4, T_3=T_4$
                    \\
                    \hline
                    \cbox[blue]{4} $\rightarrow$ \cbox[blue]{1}
                    & Adiabetic
                    & $P_4V_4^\gamma=P_1V_1^\gamma$
                \end{tabular}
            \end{minipage}
        \end{center}
        
    \item Write down the $\dd{U}, \var{W}, \var{Q}$ for each process.
    \begin{center}
        \begin{tabular}{>{\centering\arraybackslash}m{2cm} 
            >{\centering\arraybackslash}m{2.5cm} 
            c}
            & Process & $\dd{U},\var{Q},\var{W}$
            \\
            \hline
            \cbox[blue]{1} $\rightarrow$ \cbox[blue]{2}
            & iso. P \& T
            & \makecell[l]{
                \phantom{\scriptsize abc}\\
                $\bcase{
                    \Delta U &= aT_1^4(V_2-V_1)\\
                    \int \var{W} &= \inv{3} aT_1^4(V_2-V_1)\\
                    \int \var{Q} &= \frac{4}{3}aT_1^4(V_2-V_1) \quad \red{>0}\\
                }$\\
                \phantom{\scriptsize abc}
            }
            \\
            \hline
            \cbox[blue]{2} $\rightarrow$ \cbox[blue]{3}
            & Adiabetic
            & \makecell[l]{
                \phantom{\scriptsize abc}\\
                $\bcase{
                    \Delta U &= a(V_3T_3^4-V_2T_2^4)\\
                    \int \var{W} &= -a(V_3T_3^4-V_2T_2^4)\\
                    \int \var{Q} &= 0\\
                }$\\
                \phantom{\scriptsize abc}
            }
            \\
            \hline
            \cbox[blue]{3} $\rightarrow$ \cbox[blue]{4}
            & iso. P
            & \makecell[l]{
                \phantom{\scriptsize abc}\\
                $\bcase{
                    \Delta U &= aT_3^4(V_4-V_3)\\
                    \int \var{W} &= \inv{3} aT_3^4(V_4-V_3)\\
                    \int \var{Q} &= \frac{4}{3}aT_3^4(V_4-V_3) \quad \red{<0}\\
                }$\\
                \phantom{\scriptsize abc}
            }
            \\
            \hline
            \cbox[blue]{4} $\rightarrow$ \cbox[blue]{1}
            & iso. V
            & \makecell[l]{
                \phantom{\scriptsize abc}\\
                $\bcase{
                    \Delta U &= a(V_1T_1^4-V_4T_4^4)\\
                    \int \var{W} &= -a(V_1T_1^4-V_4T_4^4)\\
                    \int \var{Q} &= 0\\
                }$\\
                \phantom{\scriptsize abc}
            }
        \end{tabular}
    \end{center}

    \item Identify if the $\var{Q}$ of each process is an input or output.
    We can check for each process,
    \begin{itemize}
        \item With heat input: \cbox[blue]{1} $\rightarrow$ \cbox[blue]{2}
        \item With heat output: \cbox[blue]{3} $\rightarrow$ \cbox[blue]{4}
    \end{itemize}


    \item Calculate efficiency according to the signs of $\var{Q}$.
    \aleq{
        \Aboxed{
            \eta
            = 1- \qty|\frac{\text{Heat output}}{\text{Heat intput}}|
            = 1-\qty|\frac{\ccancel[red]{\dfrac{4}{3}a}T_3^4(V_4-V_3)}{\ccancel[red]{\dfrac{4}{3}a}T_1^4(V_2-V_1)}|
        }
    }

    To simplify, we can use the adiabetic relation:
    \aleq{
        PV^{\frac{4}{3}} &= (\text{const.})\\
        \inv{3}aT^4V^{\frac{4}{3}} &= (\text{const.})\\
        T^3V &= (\text{const.})
    }

    And the relations between state parameters:
    \aleq{
        \bcase{
            & T_1=T_2,\  T_3=T_4\\
            & T_2^3V_2 = T_3^3V_3,\  T_4^3V_4=T_1^3V_1
        }
    }

    The efficiency is now
    \aleq{
        \eta
        &= 1- \qty|\frac{T_3^4(V_4-V_3)}{T_1^4(V_2-V_1)}| \\[1ex]
        &= 1- \frac{T_3}{T_1}\qty|\frac{T_3^3V_4 - T_3^3V_3}{T_1^3V_2 - T_1^3V_1}|\\[1ex]
        &= 1- \frac{T_3}{T_1}\qty|\frac{T_\red{4}^3V_4 - T_3^3V_3}{T_\red{2}^3V_2 - T_1^3V_1}|\\[1ex]
        &= 1- \frac{T_3}{T_1}
    }

    which is exactly the same as Carnot cycle with ideal gas!

    \end{enumerate}

\end{example}




%%%
\theend
\end{document}