\documentclass[class=article, crop=false, 12pt]{standalone}
\usepackage[subpreambles=true]{standalone}
\usepackage{../.common/common}


\author{Tony Shing}
%\pretitle{Supplementary}

\topic{T12 (Electromagnetism)}
\title{Magnetostatics}

\version{2025} % leave blank for omitting

\begin{document}

\maketitle


\begin{overview}
    \begin{itemize}
        \item Basic problems: Find $\vvec{B}$ by Biot-Savat law with integration
        \item Divergent-less of B-field
        \item Ampere's law, line integral, curl \& Stokes' theorem
        \item Magnetic vector potential \& Poisson equation
    \end{itemize}


\end{overview}

\vskip 1em
In electromagnetism, 
theoretically every problem can be solved through a set of PDEs
called the \bf{Maxwell Equations}.\\[-2em]
\begin{center}
    \begin{minipage}{0.3\textwidth}
        \aleq{
            {\div} \vvec{E} &= \frac{\rho}{\epsilon_0}\\
            \curl \vvec{E} &= -\pdvv{\vvec{B}}{t}
        }
    \end{minipage}
    \hspace{0.05\textwidth}
    \begin{minipage}{0.3\textwidth}
        \aleq{
            \tkn{maxwell_div}{\div} \vvec{B} &= 0\\
            \tkn{maxwell_ampere}{\curl} \vvec{B} &= \mu_0 \vvec{J} + \mu_0\epsilon_0\pdvv{\vvec{E}}{t}
        }
    \end{minipage}
\end{center}
\addArrow[blue]{maxwell_div}{(-5ex,0)}{}{(-3ex,1ex)}
\addArrow[red]{maxwell_ampere}{(-5ex,0)}{}{(-3ex,1ex)}

However, a \it{system of PDEs} is just too complicate to be solved.
So we need to learn different "tricks" to avoid them,
which are enough for some simple scenarios.\\

Magnetostatics only concerns the \nth{2} and \nth{4} equation of the set 
- \cul[blue]{Gauss's law on B-field} and \cul[red]{Ampere's law}. 




\linesep
% content begins here
% Section %%%%%%%%%%%%%%%%%%%%%%%%%%%%%%%%%%%%%%%%%%%%%%%%%%%%
\section{Basic Skill: Biot-Savat Law with Integration}

Physically, currents are just moving charges. 
There is no such thing called "point current".
However we can imagine a line of current being divided into many infinitestimally small segments
such that each current segment "looks like" a vector point source.
\aleq{
    \ \ \qty(\mstack{\text{Current}\\\text{line}}) \ =\ I\vvec{L} 
    \quad\ \xRightarrow{\phantom{000}}\quad\ \ 
    \int I \dd{\vvec{l}} \ \sim\ \sum \qty(\mstack{\text{Current}})\qty(\mstack{\text{Unit}\\\text{length}})
}

\insertFig{current segment source}

This is why you will never find Biot-Savat law in your high school textbook.
\aleq{
    \cus[red]{\vvec{B} = \inv{4\pi\epsilon_0} \frac{I\vvec{L}}{r^2}\cross \hhat{r}}
        {\substack{\text{Always remember that}\\\text{You should not write this}}}
    \qquad\xRightarrow{\phantom{000}}\quad
    \cus[blue]{\int \dd{\vvec{B}} 
    = \int_{\substack{\text{whole}\\\text{line}}} \frac{\mu_0}{4\pi} \frac{I\dd{\vvec{l}}}{r^2}\cross\hhat{r}}
        {\substack{\text{Biot-Savat law can only be}\\\text{written as an integral}}}
}

Furthermore, because we are living in a 3D world,
current does not always travel along a line segment, 
but may flow on a surface or through an object such that the current is position dependent. 
In these cases, we should describe current as a distribution of flow (i.e. vector field).

\insertFig{current volume density + surface density}

When current is travelling on a surface 
(so that each wire is like a ribbon strip),
we can describe current as \bf{surface current density} $\vvec{K}$. 
\aleq{
    I\dd{\vvec{l}} 
    \qquad\xRightarrow{\phantom{000}}\quad 
    \vvec{K}\dd{s}
}

When current is travelling in an object 
(so that each wire is like a continuous volume),
we can describe current as \bf{volume current density} $\vvec{J}$.
\aleq{
    I\dd{\vvec{l}} 
    \qquad\xRightarrow{\phantom{000}}\quad 
    \vvec{J}\dd{\tau}
}


\begin{notation}[Caution:]
    In magnetostatics problem, we require the current to be a constant flow.
    Although you may have learnt that a moving point charge $q$ traveling at velocity $v$ acts like a point current source $qv$,
    this current is only temporary.

    \insertFig{point charge not equal continuous flow}

    The true formula of B-field by point charge needs to consider the travelling time of B-field.
    This is completely out of our scope.

\end{notation}

\vskip 1em
\begin{example}
    Suppose there is a wire lying on the x-axis, 
    with its ends at $x=a$ and $x=b$.
    Let there be uniform current $I$ flowing along it.
    What is the B-field on an arbituary point on the z axis?\\

    \insertFig{rod}

    We can analyze by dividing the rod into infinitestimal pieces:
    \begin{itemize}
        \item Each segment has a length $\dd{x}$
        \item Each unit of current segment is thus $I\dd{\vvec{x}} = I\dd{x}\hhat{x}$.
        \item For the segment at position $x$, 
        its distance from the targeted point is $\sqrt{z^2+x^2}$.
    \end{itemize}

    \insertFig{infinte element}

    Thus we can calculate $\vvec{B}$ by Biot-Savat law.
    \aleq{
        \vvec{B} = \frac{\mu_0}{4\pi}\int_a^b \frac{I\dd{x}\hhat{x}}{z^2+x^2}\cross \qty(\substack{\text{directon to}\\\text{target point}})
    }
    To do the cross product, 
    we need to resolve the direction's component from the segment to the target point. 
    By the triangle:

    \insertFig{component}

    \aleq{
        \hhat{r} &= \frac{x}{\sqrt{z^2+x^2}}\hhat{x} + \frac{z}{\sqrt{z^2+x^2}}\hhat{z}\\
        \Rightarrow\quad \hhat{x}\cross\hhat{r} &= \frac{z}{\sqrt{z^2+x^2}}(-\hhat{y})
    }

    In this situation, $-\hhat{y}$ is the out of paper direction. 

    \insertFig{directions notation}

    The B-field should be integrated by
    \aleq{
        B_y = - \frac{\mu_0}{4\pi}\int_a^b \frac{I\dd{x}}{z^2+x^2}\frac{z}{\sqrt{z^2+x^2}}
    }


\end{example}


\linesep
% Section %%%%%%%%%%%%%%%%%%%%%%%%%%%%%%%%%%%%%%%%%%%%%%%%%%%%
\section{Divergent-less of B-field}

Similar to E-field, there is the \bf{Gauss's law for B-field},
which has two different expressions: 
\aleq{
    \oiint \vvec{B}\cdot \dd{\vvec{s}} &= 0 &(\text{Integral form})\\[1ex]
    %
    \div \vvec{B} &= 0 &(\text{Differential form})
}

This law is purely an observation to B-field, claiming that 
\aleq{
    \Aboxed{
        \mstack{\text{Magnetic point source}\\[0.5ex]\text{does not exist}}
        \qquad\Leftrightarrow\qquad
        \mstack{\text{Total flux of B-field}\\[0.5ex]\text{on a closed surface always}= 0}
    }
}

Because so far no one has found any magnetic monopoles,
we determine that B-field lines must exist as closed loops,
and can never form diverging/converging patterns like E-field does.\\

This law is not as important as the other 3 in the Maxwell equation because 
it does not involve any source terms.
It is only sometimes useful when we need to make symmetry claims or simplify derivations.



\linesep
% Section %%%%%%%%%%%%%%%%%%%%%%%%%%%%%%%%%%%%%%%%%%%%%%%%%%%%
\section{Ampere's Law}

The Ampere's Law (in magnetostatics) has two different expressions:
\aleq{
    \oint \vvec{B}\cdot \dd{\vvec{l}} &= \mu_0 I &(\text{Integral form})\\[1ex]
    %
    \curl \vvec{B} &= \mu_0 \vvec{J} &(\text{Differential form})
}

It is easier to study the physical meaning and visualize by the integral form.
After that we can generalize to the differential form by introducing an operator called \bf{curl}.

%%%%%%%%%%%%%%
\subsection{Revisit: Dot Product Line Integral}

The literal description in Ampere's law integral form is
\aleq{
    \qty(\mstack{\text{Dot product line integral}\\\text{of magnetic field along a loop}})
    \ =\ \oint \vvec{B}\cdot \dd{\vvec{l}}
    \ =\ \mu_0 I
    \ =\ (\text{Constant})(\text{Current enclosed})
}

Recall that we can use the sign of a dot product between 2 vectors to 
determine if the vectors are in similar / opposite directions. 

\insertFig{dot product sign}

Now consider that we are travelling in a vector field along some path.
At each step, 
we can take note of field vector there and our travelling direction,
then compute their dot product.\\

If the sum of all the dot products $>0$, 
we are travelling more or less the same direction relative to the field's flow.\\

If the sum of all the dot products $<0$, 
we are travelling more or less the opposite direction relative to the field's flow.\\

\insertFig{flow along + flow opposite}

If we divide our path into infinitestimal small segments,
then the sum become line integral.
\aleq{
    \int_\text{path} \vvec{F}\cdot \dd{l}
    \quad
    \begin{cases}
        > 0 &\Rightarrow\quad \sim \text{Our path is along the flow}\\
        < 0 &\Rightarrow\quad \sim \text{Our path is opposite to the flow}
    \end{cases}
}


%%%%%%%%%%%%%%
\subsection{Detection of Rotating Flow}

The situation becomes interesting if we choose our path to be a closed loop - 
this loop integral becomes an indicator whether there are rotation trends around the loop.\\

By convention, we \cul[red]{always use an anti-clockwise loop} (Just like right hand rule). 
\begin{itemize}
    \item If $\displaystyle \oint_{\substack{\text{our}\\\text{loop}}} \vvec{F}\cdot \dd{l} > 0$,
    our loop may have enclosed some rotation centers of anti-clockwise flow.

    \insertFig{High degree of anti clock flow}

    \insertFig{loop overlay -> +ve}

    \item If $\displaystyle \oint_{\substack{\text{our}\\\text{loop}}} \vvec{F}\cdot \dd{l} < 0$,
    our loop may have enclosed some rotation centers of clockwise flow.

    \insertFig{High degree of clock flow}

    \insertFig{loop overlay -> -ve}

    \item If $\displaystyle \oint_{\substack{\text{our}\\\text{loop}}} \vvec{F}\cdot \dd{l} \approx 0$,
    our loop probably does not enclose any rotation centers.

    \insertFig{No rotation flow}

    \insertFig{loop overlay -> 0}
\end{itemize}



%%%%%%%%%%%%%%
\subsection{Curl}

However there is a problem in using loop integral of dot product - 
if we choose the loop too arbituarily, 
the calculated dot product is ambiguous to tell where the rotation centers are.

\insertFig{arbituary surface}

E.g. If we choose an irregular loop, it may not catch the rotation centers.\\

To tackle this problem,
we need to introduce the \bf{curl} operator:
\aleq{
    \tkn{curl}{\cul[red]{\curl}}\ \bullet 
    \ \defeq\ \curlRec{\bullet_x}[\bullet_y][\bullet_z]
    \ \defeq\ \tkn{curl2}{\cul[blue]{\text{curl}}}\ \bullet
}
\addArrow[red]{curl}{(0,-3ex)}
{\scriptsize Like gradient operator\\[-1ex]\scriptsize but with a cross}
{(0,-1ex)}{(0,-1ex)}
\addArrow[blue]{curl2}{(0,-3ex)}
{\scriptsize Sometimes we\\[-1ex]\scriptsize just write "curl"}
{(0,-1ex)}{(0,-1ex)}


\vskip 1em
The curl operator can be applied on a vector function,
and will return another vector function.
\aleq{
    \curl{F} 
    &= \bmat{\pdvv{x} & \pdvv{y} & \pdvv{z}}\cross \bmat{F_x & F_y & F_z}\\[1ex]
    &= \curlRecDet{F_x}[F_y][F_z]\\[1ex]
    &= \curlRec{F_x}[F_y][F_z]\\[1ex]
    &= (\text{A vector})
}

Each component of the curl of a vector field is related to its
\bf{loop integral along an infinitestimal small loop in each direction.}


%%%%%%%%%%%%%%
\subsubsection{Geometrical Interpretation}

To visualize, we can draw 3 infinitestimal small loop around a point $(x,y,z)$.
Because of symmetry in all 3 directions,
it suffices to just analyze the loop that is parallel to the $x$-$y$ plane.

\insertFig{infinitestmal loops + floating loop}

The dot product on each side of the loop are calculated as follow:

\insertFig{magnify loop}

\begin{itemize}
    \item Edge 1 :
    \begin{itemize}
        \item Vector field on the edge center = \red{$\vvec{F}\qty(x,y+\frac{\Delta y}{2},z)$}
        \item Edge length = \green{$\Delta x$}, in $-x$ direction
    \end{itemize}
    $\Rightarrow \text{ Dot product } = \vvec{F}\qty(x,y+\frac{\Delta y}{2},z)\cdot (\Delta x)(-\hhat{x})
    = -F_{\cul[blue]{\cul[blue]{x}}}\qty(x,y+\frac{\Delta y}{2},z)\Delta x$

    \insertFig{edge 1}

    \item Edge 2 :
    \begin{itemize}
        \item Vector field on the edge center = \red{$\vvec{F}\qty(x-\frac{\Delta x}{2},y,z)$}
        \item Edge length = \green{$\Delta y$}, in $-y$ direction
    \end{itemize}
    $\Rightarrow \text{ Dot product } = \vvec{F}\qty(x-\frac{\Delta x}{2},y,z)\cdot (\Delta y)(-\hhat{y})
    = -F_{\cul[blue]{\cul[blue]{y}}}\qty(x-\frac{\Delta x}{2},y,z)\Delta y$

    \insertFig{edge 2}

    \item Edge 3 :
    \begin{itemize}
        \item Vector field on the edge center = \red{$\vvec{F}\qty(x,y-\frac{\Delta y}{2},z)$}
        \item Edge length = \green{$\Delta x$}, in $+x$ direction
    \end{itemize}
    $\Rightarrow \text{ Dot product } = \vvec{F}\qty(x,y-\frac{\Delta}{2},z)\cdot (\Delta x)(+\hhat{x})
    = F_{\cul[blue]{\cul[blue]{x}}}\qty(x,y-\frac{\Delta y}{2},z)\Delta x$

    \insertFig{edge 3}

    \item Edge 4 :
    \begin{itemize}
        \item Vector field on the edge center = \red{$\vvec{F}\qty(x+\frac{\Delta x}{2},y,z)$}
        \item Edge length = \green{$\Delta y$}, in $+y$ direction
    \end{itemize}
    $\Rightarrow \text{ Dot product } = \vvec{F}\qty(x+\frac{\Delta x}{2},y,z)\cdot (\Delta y)(+\hhat{y})
    = F_{\cul[blue]{\cul[blue]{y}}}\qty(x+\frac{\Delta x}{2},y,z)\Delta y$

    \insertFig{edge 4}

\end{itemize}

Therefore the total dot product along the loop is
\aleq{
    &\textstyle \cus[gray]{F_y\qty(x+\frac{\Delta x}{2},y,z)\Delta y}{\text{Edge 4}} 
        - \cus[gray]{F_y\qty(x-\frac{\Delta x}{2},y,z)\Delta y}{\text{Edge 2}}
        - \cus[gray]{F_x\qty(x,y+\frac{\Delta y}{2},z)\Delta x}{\text{Edge 1}} 
        + \cus[gray]{F_x\qty(x,y-\frac{\Delta y}{2},z)\Delta x}{\text{Edge 3}} \\[1em]
    %
    = &\qty(\frac{F_x\qty(x+\frac{\Delta x}{2},y,z) - F_x\qty(x-\frac{\Delta x}{2},y,z)}{\blue{\Delta x}}
        - \frac{F_x\qty(x,y+\frac{\Delta y}{2},z) - F_x\qty(x,y-\frac{\Delta y}{2},z)}{\blue{\Delta y}}) (\Delta x\Delta y)\\[1em]
    %
    = &\qty(\cut[red]{\cbox[red]{\frac{F_x\qty(\cbox[red]{\textstyle x+\frac{\Delta x}{2}},y,z) 
            - F_x\qty(\cbox[red]{\textstyle x-\frac{\Delta x}{2}},y,z)}{\cbox[red]{\Delta x}}}}{\text{This is exactly partial x}}
        -\cut[green]{\cbox[green]{\frac{F_x\qty(x,\cbox[green]{\textstyle y+\frac{\Delta y}{2}},z) 
            - F_x\qty(x,\cbox[green]{\textstyle y-\frac{\Delta y}{2}},z)}{\cbox[green]{\Delta y}}}}{\text{This is exactly partial y}}) (\Delta x\Delta y)\\[1em] 
    %
    = &\qty(\pdvv{F_y}{x} - \pdvv{F_x}{y})(\Delta x\Delta y)\\[1em]
    %
    =& \qty(\mstack{\text{Curl's }\\[0.5ex] z\text{ component}})
        \qty(\mstack{\text{Unit area}\\[0.5ex]\text{parallel to xy plane}})\\[1em]
    %
    =& \qty(\mstack{\text{Curl's }\\[0.5ex] z\text{ component}})
        \qty(\mstack{\text{Unit area}\\[0.5ex]\text{\blue{normal to z direction}}})
}

We can expect the similar results in the other 2 directions.
Gather them together:
\aleq{
    \bcase{
        \qty(\mstack{\text{Loop integral}\\\text{normal to}\\[0.2ex]x \text{ direction}})
        = \qty(\pdvv{F_z}{y} - \pdvv{F_y}{z})(\dd{y}\dd{z})
        = (\curl \vvec{F})_x (\dd{y}\dd{z})
        = \qty(\mstack{\text{Curl's }\\[0.5ex] x\text{ component}})
            \qty(\mstack{\text{Unit area}\\[0.2ex]\text{normal to}\\[0.2ex]x \text{ direction}})\\[1em]
        %
        \qty(\mstack{\text{Loop integral}\\\text{normal to}\\[0.2ex]y \text{ direction}})
        = \qty(\pdvv{F_z}{x} - \pdvv{F_x}{z})(\dd{z}\dd{x})
        = (\curl \vvec{F})_y (\dd{z}\dd{x})
        = \qty(\mstack{\text{Curl's }\\[0.5ex] y\text{ component}})
            \qty(\mstack{\text{Unit area}\\[0.2ex]\text{normal to}\\[0.2ex]y \text{ direction}})\\[1em]
        %
        \qty(\mstack{\text{Loop integral}\\\\\text{normal to}\\[0.2ex]z \text{ direction}})
        = \qty(\pdvv{F_y}{x} - \pdvv{F_x}{y})(\dd{x}\dd{y})
        = (\curl \vvec{F})_z (\dd{x}\dd{y})
        = \qty(\mstack{\text{Curl's }\\[0.5ex] z\text{ component}})
            \qty(\mstack{\text{Unit area}\\[0.2ex]\text{normal to}\\[0.2ex]z \text{ direction}})\\[1em]
    }
}

\vskip 1ex
Therefore we can geometrically interpret curl as
\aleq{
    \Aboxed{
        \qty(\mstack{\text{\bf{Curl's}}\\\text{\bf{i\Nth\ component}}}) 
        = (\curl \vvec{F})_i 
        = \frac{(\text{Loop integral normal to i\Nth\ direction})}{(\text{Area enclosed by the loop})}
        \sim \qty(\mstack{\text{\bf{loop integral}}\\\tkn{curl_density}{\text{\bf{\cul[blue]{density}}}}})
    }
}
\addArrow[blue]{curl_density}{(0,-4ex)}{\scriptsize This density\\[-1ex]\scriptsize is by area}{(0,-1ex)}{(0,-1ex)}


\vskip 1em
%%%%%%%%%%%%%%
\subsubsection{Stokes' Theorem}

With the geometrical interpretation,
we can directly state (without proof) a convenient formula related to divergence - 
the \bf{Stokes' theorem}:
\aleq{
    \Aboxed{
        \oint \vvec{F}\cdot \dd{\vvec{l}} &= \iint (\curl \vvec{F}) \cdot \dd{\vvec{s}}
    }
}

which is basically 
\aleq{
    \qty(\mstack{\text{Total }\\[0.5ex]\text{Loop integral}})
    \quad \sim \sum_{\text{All area}} \qty(\mstack{\text{Loop integral}\\[0.5ex]\text{per area}})\times \qty(\text{Area})
}


%%%%%%%%%%%%%%
\subsection{Ampere's Law - Explanation}


The Ampere's law is purely an \cul[red]{observation} about the relation between B-field and currents:
\aleq{
    \Aboxed{
        \mstack{\text{Total line integral of B-field}\\[0.5ex]\text{along a closed loop}\neq 0}
        \qquad\Leftrightarrow\qquad
        \mstack{\text{There are currents}\\[0.5ex]\text{circled by the loop}}
    }
}

The two forms of Ampere's law are describing this same observation:
\begin{itemize}
    \item \ul{Integral form}:
    \aleq{
        (\vvec{B}\text{'s loop integral})\ \sim\ \oint \vvec{B}\cdot\dd{\vvec{l}} 
        \ =\ \mu_0 I \ \sim\ (\text{Current})
    }

    \item \ul{Differential form}:
    \aleq{
        \qty(\mstack{\vvec{B} \text{'s loop integral}\\[0.5ex]\text{density}})\ \sim\ \curl \vvec{B}
        \ =\ \mu_0 \vvec{J}\ \sim\ \qty(\mstack{\text{Current}\\[0.5ex]\text{density}})
    }
\end{itemize}

And the two form can be inter-converted by Stokes' theorem.
\aleq{
    \oint \vvec{B}\cdot\dd{\vvec{l}} \quad &=\quad \mu_0 I\\[2em]
    \tkn{curlB}{\iiint (\curl \vvec{B}) \cdot \dd{\vvec{s}}}\ &=\ \mu_0\tkn{curlI}{\iint \vvec{J} \cdot \dd{\vvec{s}}}
}
\addArrow[red]{curlB}{(0,9ex)}{Stokes' Theorem}{(1ex,3ex)}{(-12ex,-6ex)}
\addBentArrow[blue]{curlI}{(-4ex,13ex)}{Current to current density}{(1ex,3ex)}{(18ex,-8ex)}


%%%%%%%%%%%%%%
\subsection{Applying Ampere's Law Integral Form}

In beginner electromagnetism, 
there is only one type of Ampere's law related problems: 
\quoting{
    Given the current distribution, 
    find the B-field everywhere by Ampere's law integral form\\
    in some \cul[red]{very symmetrical} scenarios.
}
which is basically asking you to \it{revert} the loop integral calculation:
\aleq{
    \oint \vvec{B}\cdot\dd{\vvec{l}} = \mu_0 I
    \qquad\Rightarrow\qquad
    \vvec{B} = \text{ ??? } \text{in terms of }I
}

If $I$ has a very ugly distribution, 
there is nothing we can do except solving some partial differential equations.
But \red{if $I$ distributes very symmetrically, 
$\vvec{B}$ should also be symmetrical}, 
such that the loop integral can be broken into multiplications.\\

In these cases, we can choose an "Ampere" loop to to be integrated where
\begin{enumerate}
    \item $\vvec{B}$ has constant magnitude everywhere along the loop.
    \item $\vvec{B}$ forms the same angle with the tangent vector everywhere along the loop.
\end{enumerate}

\vskip 1ex
Only then, the loop integral can be broken down as
\aleq{
    \oint \vvec{B}\cdot \dd{\vvec{l}} 
    &= \oint \tkn{ampere_dot}{\cul[green]{\norm{\vvec{B}}\norm{\dd{\vvec{l}}}\cos\theta}}\\[1ex]
    &= \tkn{ampere_B}{\cul[red]{\norm{\vvec{B}}}}\ \tkn{ampere_theta}{\cul[blue]{\cos\theta}}\ \oiint \norm{\dd{\vvec{l}}}\\[2.5em]
    &= \norm{\vvec{B}}\ \cos\theta\ (\text{Perimeter of loop})
}
\addArrow[green]{ampere_dot}{(5ex,0)}
{\scriptsize Just dot product\\[-1ex]\scriptsize $\vvec{a}\cdot\vvec{b}=\norm{\vvec{a}}\norm{\vvec{b}}\cos\theta$}
{(8ex,0)}{(6ex,0)}
\addBentArrow[red]{ampere_B}{(-8ex,-3.5ex)}
{Same magnitude everywhere\\[-0.5ex]Can move out of integral!}
{(0,-1.5ex)}{(-12.5ex,1ex)}
\addBentArrow[blue]{ampere_theta}{(8ex,-3.5ex)}
{Form same angle everywhere\\[-0.5ex]Can move out of integral!}
{(0,-1.5ex)}{(12.5ex,1ex)}

such that we can find the magnitude of $\vvec{B}$ with simple division
\aleq{
    \Aboxed{
        \norm{\vvec{B}} 
        = \frac{(\text{Total loop integral})}{(\text{Perimeter of loop})\cos\theta}
        = \frac{\mu_0 I}{(\text{Perimeter of loop})\cos\theta}
    }
}

In fact, there are not many of these "very symmetrical" cases.
These examples below, with their respective Ampere loop,
are basically all the variations you can find in textbooks.

\insertFig{different ampere loop}

\begin{example}
    Given an infinitely long wire with current $I$ flowing along.
    By cylindrical symmetry, the B-field must satisfy:
    \begin{itemize}
        \item Unchange when translate along the wire (z-axis).
        \item Unchange when rotate about the wire (z-axis).
    \end{itemize}
    
    Note that cylindrical symmetry does not restrict the B-field to be circle loops.
    For example, all of these configurations satisfy cylindrical symmetry:

    \insertFig{loop (purely theta) vs spiral up (has z) vs spiral out (has r)}

    We need to rule out the 2 latter cases before we can use Ampere's law to find $\vvec{B}$:

    \begin{enumerate}
        \item \ul{$\vvec{B}$ is divergent-less} : \\
        If the B-field has radial component,
        we can draw a cylindrical Gaussian box around the wire.
        Then we would get the flux of $\vvec{B} \neq 0$.

        \insertFig{cyl gauss surface}

        Therefore the B-field must not have $r$ component.

        \item \ul{Another Ampere's loop} : \\
        If the B-field has z component, 
        we can draw an Ampere loop that is parallel to the wire.
        By Ampere's law, there is no current enclosed by this loop,
        so the loop integral should be $0$.

        \insertFig{alternative ampere loop}

        \begin{itemize}
            \item By translational symmetry, 
            the dot product at the top (edge 1) and bottom (edge 3) should cancel each other. 

            \item We can choose a loop which stretches to infinity far away.
            Because there should be no B-field at infinity, dot product is $0$.
            So the dot product on the wire (edge 4) should also be $0$.

            \item Then if we choose a loop that is not infinitly far,
            we can conclude that the dot product on edge 3 should be $0$.
            This implies $\vvec{B}$ has no $z$ component.
        \end{itemize}
    \end{enumerate}

    After all these symmetry arguments,
    we can finally conclude that B-field around a wire must be circular loops. 
    We can choose a circular Ampere loop of radius $r$ to find the magnitude of B-field
    at distance $r$ from the wire.
    \aleq{
        \norm{\vvec{B}} 
        &= \frac{\mu_0 I}{(\text{Perimeter of loop})\cos\theta}\\
        &= \mu_0 I \cdot \inv{(2\pi r)} \cdot \inv{\cos \tkn{ampere_cylth}{\cul[green]{0^\circ}}}\\
        &= \frac{\mu_0 I}{2\pi r}\\[1ex]
        \Rightarrow\quad \vvec{B} &= \frac{\mu_0 I}{2\pi r} \tkn{ampere_cylph}{\cul[blue]{\hhat{\theta}}}
    }
    \addArrow[green]{ampere_cylth}{(7ex,0)}
    {\scriptsize B-field = angular \\[-1ex]\scriptsize $\therefore$ Tangent to loop}{(2ex,0.5ex)}{(5.5ex,1ex)}
    \addArrow[blue]{ampere_cylph}{(3ex,0)}
    {\scriptsize You have to manually add the unit vector}{(2ex,0.5ex)}{(15ex,0)}


\end{example}


\begin{example}
    Givne an infinitely long solenoid made of a uniform density of coils $\frac{N}{L}$,
    with a current $I$ passiing through.
    This is another cylindrical symmetry case.
    \begin{itemize}
        \item Unchange when translate along the wire (z-axis).
        \item Unchange when rotate about the wire (z-axis).
    \end{itemize}
    
    Again, cylindrical symmetry does not restrict the B-field to be purely along z direction. 
    We need to first argue that the B-field does not have radial or angular components 
    before we can apply Ampere's law.

    \begin{enumerate}
        \item \ul{$\vvec{B}$ is divergent-less} : \\
        If the B-field has radial component,
        we can draw a cylindrical Gaussian box around the solenoid.
        Then we would get the flux of $\vvec{B}\neq 0$.

        \insertFig{cylinder around solenoid}

        Therefore the B-field must not have $\hhat{r}$ component.

        \item \ul{Alternative Ampere loop} : \\


    \end{enumerate}
    

\end{example}


\begin{example}
    Given an infinitely long hollow cylinder with inner radius $=a$ and outer radius $=b$,
    and its charge density is proportional to distance from center $r$, i.e. $\rho(\vvec{r}) =kr$
    
    For cylinder, we can claim by rotational symmetry around the axis and translation symmetry along the axis,
    that the E-field must satisfy:
    \begin{itemize}
        \item Only point in $r$ direction.
        \item Magnitude does not depend on $\theta$ or $z$.
    \end{itemize}

    \insertFig{cylinder symm}

    Therefore we can choose the Gaussian surface to be a cylindrical sheet radius $r$ 
    and an arbituary length $L$ (which will be cancelled later)
    to find the magnitude of E-field at distance $r$ from the rotation axis.
    \aleq{
        \norm{\vvec{E}} 
        &= \frac{Q/\epsilon_0}{(\text{Total surface area})\cos\theta}\\
        &= \frac{Q}{\epsilon_0} \cdot \tkn{gauss_cylcurve}{\cul[red]{\inv{(2\pi r L)}}} \cdot \inv{\cos \tkn{gauss_cylth}{\cul[green]{0^\circ}}}\\[1em]
        &= \frac{Q}{2\pi\epsilon_0 r L}\\[1ex]
        \Rightarrow\quad \vvec{E} &= \frac{Q}{2\pi\epsilon_0 r L} \tkn{gauss_cylr}{\cul[blue]{\hhat{r}}}
    }
    \addBentArrow[red]{gauss_cylcurve}{(7ex,-4ex)}
    {\scriptsize E-field = radial \\[-1ex]\scriptsize $\therefore$ Only go through the curved surface \\[-1ex]\scriptsize Top/bottom surface has no flux}
    {(0,-3ex)}{(12ex,1ex)}
    \addArrow[green]{gauss_cylth}{(7ex,0)}
    {\scriptsize E-field = radial \\[-1ex]\scriptsize $\therefore$ Normal to curved surface}{(2ex,0.5ex)}{(8.5ex,1ex)}
    \addArrow[blue]{gauss_cylr}{(3ex,0)}
    {\scriptsize You have to manually add the unit vector}{(2ex,0.5ex)}{(15ex,0)}

    \insertFig{not side surface cuz flux = 0}

    This time the charge density depends on position,
    so the total charge enclosed by the surface needs to be computed by integration.

    \begin{enumerate}
        \item For radial distance $r<a$, there is no charge enclosed because the cylinder is hollow.
        So $Q=0$ implying $\vvec{E}=0$.

        \insertFig{inner}

        \item For radial distance $a<r<b$, 
        total enclosed charge are distributed from radius $=a$ to radius $=r$,
        which calculates as
        \aleq{
            Q = \int^r_a \rho \cdot 2\pi rL\dd{r} 
            =  2\pi kL \int^r_a r^2 \dd{r} 
            =  \frac{2\pi kL}{3} (r^3-a^3)
        }
        So the E-field is
        \aleq{
            \vvec{E} = \frac{Q}{2\pi\epsilon_0 r L}\hhat{r} 
            = \frac{k}{3\epsilon_0}\qty(r^2 - \frac{a^3}{r}) \hhat{r}
        }
        
        \insertFig{middle}

        \item For radial distance $r>b$,
        total enclosed charge are distributed from radius $=a$ to radius $=b$,
        which calculates as
        \aleq{
            Q = \int^b_a \rho \cdot 2\pi rL\dd{r} 
            =  2\pi kL \int^r_a r^2 \dd{r} 
            =  \frac{2\pi kL}{3} (b^3-a^3)
        }
        So the E-field is
        \aleq{
            \vvec{E} = \frac{Q}{2\pi\epsilon_0 r L}\hhat{r} 
            = \frac{k}{3\epsilon_0r}\qty(b^3 - a^3) \hhat{r}
        }

        \insertFig{outer}

    \end{enumerate}

\end{example}



\linesep
% Section %%%%%%%%%%%%%%%%%%%%%%%%%%%%%%%%%%%%%%%%%%%%%%%%%%%%
\section{Magnetic Vector Potential}

%%%%%%%%%%%%%%
\subsection{Mathematical Origin}

Magnetic field also has a corresponding potential function $\vvec{A}(\vvec{r})$.
But unlike electric potential,
the magnetic potential is a vector function. 
The reason to create it is purely mathematical:
\begin{itemize}
    \item \ul{Observation}: 
    B-field always form loops because we have never observed any magnetic point source.
    $\Rightarrow$ B-field is divergent-less.

    \item \ul{Mathematical fact}: 
    Any divergent-less field can be expressed as the curl of some vector function (i.e. vector potential).
\end{itemize}

Therefore we can define a vector function $\vvec{A}(\vvec{r})$,
called \bf{magnetic vector potential}, such that 
\aleq{
    \Aboxed{
        \vvec{B}(\vvec{r}) = \curl \vvec{A}(\vvec{r})
    }
}

(Procedures for computing the reverse exists, however basically no one would use it.)

%%%%%%%%%%%%%%
\subsection{Poisson Equation}

If we substitute $\vvec{B} = \curl \vvec{A}$ into the Ampere's law differential form,
we arrive at a new equation: 
%\addBelowArrow[blue]{laplacian1}{laplacian2}{\scriptsize This is called\\[-1ex]\scriptsize \bf{Laplacian Operator}}{-1ex}{(0,-2ex)}
\aleq{
    \mu_0 \vvec{J} &= \curl \vvec{B}\\
    &= \curl (\curl \vvec{A})\\
    &= \grad(\div \vvec{A}) - \laplacian \vvec{A}\\
    &= -\laplacian \vvec{A}\\
    \Aboxed{
        \laplacian \vvec{A}(\vvec{r}) &= -\mu_0\vvec{J}(\vvec{r})
    }
}
% $\curl \curl \vvec{A} \equiv \grad(\div \vvec{A}) - \laplacian \vvec{A}$ This is an identity relating these operators
% In magnetostatics, we can CHOOSE $\div \vvec{A}=0$. The mathematical reason is called "gauge fixing".
This is again the \bf{Poisson equation}, 
the same PDE that we have encountered in electric potential 
($\laplacian V(\vvec{r}) = - \frac{\rho(\vvec{r})}{\epsilon_0}$).
But for magnetic potential,
because it is a vector function, 
it is actually 3 equations, one for each direction:
\aleq{
    &\laplacian \bmat{A_x \\ A_y \\ A_z} = -\mu_0 \bmat{J_x\\J_y\\J_z}\\ 
    \Rightarrow 
    &\bcase{
        \laplacian A_\red{x}(\vvec{r}) &= -\mu_0 J_\red{x}(\vvec{r})\\
        \laplacian A_\red{y}(\vvec{r}) &= -\mu_0 J_\red{y}(\vvec{r})\\
        \laplacian A_\red{z}(\vvec{r}) &= -\mu_0 J_\red{z}(\vvec{r})
    }
} 

The Poisson equation of $\vvec{A}$ is 
\cul[red]{the general method to relate magnetic vector potential with current distribution}.
that works for any configurations of potential or current.

\insertFig{find one and other, add I = lap A and A = ??? I with arrow }

However, being general does not mean it is always easy to solve: 
\begin{itemize}
    \item \ul{$\vvec{A}(\vvec{r})$ to $\vvec{J}(\vvec{r})$} : 
    The Laplacian operator is just a sum of \nth{2} order derivatives. 
    (And we just need to do it 3 times). Relatively easy.

    \item \ul{$\vvec{J}(\vvec{r})$ to $\vvec{A}(\vvec{r})$} : 
    Need to solve the Poisson equation (for 3 times),
    which are \nth{2} order non-homogeneous
    \\ \phantom{\ul{$\vvec{J}(\vvec{r})$ to $\vvec{A}(\vvec{r})$} :} linear PDEs. Awful!
\end{itemize} 

Unfortunately in realistic problems,
it is more frequent to ask for $\vvec{A}(\vvec{r})$ from $\vvec{J}(\vvec{r})$,
because we can usually confine the current distribution in a small region 
by using a very small test object;
but for potential, it is always everywhere. 

\insertFig{charge in small object -> can treat like point charge. but V spread everywhere}

Again we are not going to discuss its general solution here. 
Here I only provide you the solution in one very special case - 
When the \cul[red]{region of interest is infinitely large} + \cul[red]{$\vvec{A}$ is chosen to be divergent-less}, 
i.e. $\vvec{A}(\vvec{r}=\infty)=0$, 
the solution is the Biot-Savat law for magnetic vector potential.
\aleq{
    \Aboxed{
        \vvec{A}(\red{\vvec{r}}) 
        &= \frac{\mu_0}{4\pi}\underset{\substack{\text{infinitely large}\\\text{space}}}{\iiint}\ 
            \frac{\vvec{J}(\blue{\vvec{r}'})}{\norm{\red{\vvec{r}}-\blue{\vvec{r}'}}}\dd[3]{\blue{\vvec{r}'}}
    }\\[1ex]
    %
    &\sim \frac{\mu_0}{4\pi} \sum_\text{everywhere} \frac{(\text{current})}{(\text{distance})}\\[1em]
    %
    &\equiv\ \mstack{\text{Biot-Savat law for magnetic vector potential}\\\scriptsize \text{(written in a fancy vector form)}}
}

\insertFig{infinitely large region + potential element plot}




%%%%%%%%%%%%%%
\subsection{Finding $\vvec{B}$ from $\vvec{J}$}

On the other hand, Poisson equation provides an alternative to calculate B-field distribution from current distribution.
If we compare the Ampere's law and Poisson equation of $\vvec{A}$:
\begin{itemize}
    \item \ul{Poisson equation} $\scriptstyle \qty(\laplacian \vvec{A} = -{\mu_0 \vvec{J}})$ : 
    Although $\vvec{A}(\vvec{r})$ is a vector function,
    the Poisson equations for each compoenent $B_x(\vvec{r})$, $B_y(\vvec{r})$, $B_z(\vvec{r})$ 
    are \cul[red]{independent}.
    \aleq{
        \laplacian \vvec{A} = -\mu_0 \vvec{J}
        \quad\Rightarrow\quad
        \bcase{
            \lapRec{A_\red{x}} &= -\mu_0 J_\red{x}\\[1ex]
            \lapRec{A_\red{y}} &= -\mu_0 J_\red{y}\\[1ex]
            \lapRec{A_\red{z}} &= -\mu_0 J_\red{z}
        }
    }
    % only involve x components
    % only involve y components
    % only involve z components

    \item \ul{Ampere's law} $\scriptstyle \qty(\curl \vvec{B} = \mu_0 \vvec{J})$ : 
    $\vvec{B}(\vvec{r})$ is a vector function with 3 components $B_x(\vvec{r})$, $B_y(\vvec{r})$, $B_z(\vvec{r})$,\\
    \phantom{Ampere's law $\scriptstyle \qty(\curl \vvec{B} = \mu_0 \vvec{J})$ :} 
    Because of the curl operation, 
    the PDE for each component of $\vvec{B}$ all mix together.
    \aleq{
        \curl \vvec{B} = \mu_0 \vvec{J}
        \quad\Rightarrow\quad
        \bcase{
            \curlPla[y][z]{B_z}[B_y] &= J_x\\[1ex]
            \curlPla[z][x]{B_x}[B_z] &= J_y\\[1ex]
            \curlPla[x][y]{B_y}[B_z] &= J_z\\[1ex]
        }
    }

\end{itemize}

In practice, there is no reason to try to solve the more difficult PDE of $\vvec{B}$,
if alternatively we can solve the easier PDE of $\vvec{J}$,
and then take curl to get $\vvec{B}$ (i.e. via $\vvec{B} = -\curl \vvec{A}$). 

\insertFig{triangle}

In this way, we can tell the solution of Ampere's law as a PDE, 
which is as expected, the Biot-Savat law for $\vvec{B}$.

\aleq{
    \Aboxed{
        \vvec{B}(\red{\vvec{r}}) = \curl \vvec{A}(\red{\vvec{r}}) 
        &= \frac{\mu_0}{4\pi}\underset{\substack{\text{infinitely large}\\\text{space}}}{\iiint}\ 
            \frac{\vvec{J}(\blue{\vvec{r}'})}{\norm{\red{\vvec{r}}-\blue{\vvec{r}'}}^2}
            \cross\qty[\frac{\red{\vvec{r}}-\blue{\vvec{r}'}}{\norm{\red{\vvec{r}}-\blue{\vvec{r}'}}}]\dd[3]{\blue{\vvec{r}'}}
    }\\[1ex]
    %
    &\sim \frac{\mu_0}{4\pi} \sum_\text{everywhere} \frac{(\text{curent})}{(\text{distance})^2} \cross \qty(\substack{\text{unit}\\\text{vector}})\\[1em]
    %
    &\equiv\ \mstack{\text{Biot-Savat law for magnetic field}\\\scriptsize \text{(But written in a fancier vector form)}}
}







\linesep
%%%%%%%%%%%%%%
Here we shall summarize the methods of solving magnetostatics problems:
\begin{enumerate}
    \item Very symmetric configurations 
    $\Rightarrow$ Ampere's law integral form. No calculus required.

    \item Not so symmetric but satisfies $\vvec{A}(\vvec{r}=\infty)=0$ 
    $\Rightarrow$ Multiple integral with Biot-Savat law.

    \item All the above do not apply 
    $\Rightarrow$ Solve Poisson equation explicitly. PDE hell %\emoji{skull}.
\end{enumerate}

%%%
\theend
\end{document}