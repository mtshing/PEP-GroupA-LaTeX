\documentclass[class=article, crop=false, 12pt]{standalone}
\usepackage[subpreambles=true]{standalone}
\usepackage{../.common/common}


\author{Tony Shing}
%\pretitle{Supplementary}

\topic{T13B (Electromagnetism)}
\title{Magnetic Material \& Magnetization}

\version{2025} % leave blank for omitting

\begin{document}

\maketitle


\begin{overview}
    \begin{itemize}
        \item Model of magnetic dipole
        \item How to describe dipole arrangement: Magnetization field, bound currents
        \item Material under external $\vvec{B}$ field: 
        \begin{itemize}
            \item A special case of material model: \cul[red]{Linear magnetic material} 
            \item Describe external field: Free currents, auxiliary field $\vvec{H}$
        \end{itemize}
    \end{itemize}
    \begin{center}
        \red{\bf{\ul{If linear magnetic material is all you need, you can skip most of the maths.}}}
    \end{center}
\end{overview}



% content begins here
% Section %%%%%%%%%%%%%%%%%%%%%%%%%%%%%%%%%%%%%%%%%%%%%%%%%%%%
\section{Magnetic Dipole}

%%%%%%%%%%%%%%
\subsection{Magnetic Response in Materials}

In modern theory, magnetic response in atoms is purely a quantum mechanical effect.
But before quantum mechanics being turned into a complete mathematics theory,
scientists had been explaining magnetic response with a few semi-classical theories.

%%%%%%%%%%%%%%
\subsubsection{Electron as Magnetic Source}

In an atom, the majority of magnetic response comes from the electrons.
An electron behaves as a magnetic dipole in 2 ways:
\begin{itemize}
    \item \bf{\ul{Orbital motion}} - Electron orbiting around the nucleus, 
    equivalent to a tiny current loop.

    \item \bf{\ul{Spin}} - Electron carries a fundamental quantum property called \bf{spin},
    which determines how it interacts with B-field.     
    It cannot be described fully by any classical picture.
\end{itemize}

\insertFig{orbital motion + spin}

%Textbooks usually depict spin as a rotating charged sphere, although electron is a point particle.

Magnetic response of an atom is the combined effect from both orbital motion and spin. 
Depends on which effect dominates, 
magnetic reponses can be classified into 2 types:
\begin{itemize}
    \item \bf{\ul{Diamagnetism}} - When orbital motion dominates. Exists on most non-metals.
    \item \bf{\ul{Paramagnetism}} - When spin dominates. Exists on many metals, 
    and some non-metals that contain chemical bonds with unpaired electrons.
\end{itemize}

In general, effects from spin is much stronger than that from orbital motion.
However since chemical bonds are usually made of pairs of electrons with opposite spins (Pauli exclusion principle),
such that the effect from spin cancels out, 
making orbital motion the dominant effect.

\insertFig{normal bond -> dia mat; unusual bond with unpair electron -> para mat}


%%%%%%%%%%%%%%
\subsubsection{Diamagnetism}

In a diamagnetic material, 
its chemical bonds do not contain any unpaired electrons 
such that all effects from spin cancel. 

(Pauli exclusion principle)

 and the magnetic response is only due to orbital motion of electrons.


although the effect from spin being much stronger.
However when atoms form chemical bonds, 
effects of spin can be cancelled due to paring of electrons .


When an external B-field is applied, 
\red{electron orbital motion reacts by weakening the external B-field}.
This is equivalent to creating magnetic dipoles in the material that point opposite to the external B-field.

\insertFig{original -> weaken = create opposite dipole}

In the semi-classical theory of diamagnetism, 
the orbital motion of electron is characterized by:
\begin{itemize}
    \item Electron (charge $-q$) orbits around a nucleus with charge $+Zq$ 
    ($Z$ = atomic number).

    \item Orbit of electron is a circle of \cul[red]{fixed} radius $r$. 
    (Bohr model of atom)

\end{itemize}

Diamagnetism is due to the \cul[red]{decrease in orbital speed under external B-field}.

\begin{enumerate}
    \item By Coulomb's law, first calculate the electon's original orbital speed $v$:
    \aleq{
        \frac{m v^2}{r} = \inv{4\pi\epsilon_0}\frac{Zq^2}{r^2}
        \qquad\Rightarrow\qquad
        v = \sqrt{\frac{Zq^2}{4\pi\epsilon_0 m r}}
    }

    \item When an external B-field $\vvec{B}$ is applied perpendicular to the orbital plane, 
    the moving electron experiences a magnetic force $\vvec{F} = (-q)\vvec{v}\cross\vvec{B}$.

    \insertFig{magnetic force on orbiting electron}

    So the electron's orbital speed is changed to a new value $v'$.
    Newton's law writes:
    \aleq{
        \frac{m v'^2}{r} &= \inv{4\pi\epsilon_0}\frac{Zq^2}{r^2} - qv'B \\
        \Rightarrow\quad
        v' &= \frac{-qB \ \tkn{orbit1}{\cul[red]{+}}\ \sqrt{(qB)^2+4\qty(\frac{m}{r})\qty(\frac{Zq^2}{4\pi\epsilon_0 r^2})}}{2\qty(\frac{m}{r})}\\
        &=\, \sqrt{\frac{Zq^2}{4\pi\epsilon_0 r^2} +\qty(\frac{qBr}{2m})^2} \tkm{orbit2}- \frac{qB r}{2m} \\
        & <\, \sqrt{\frac{Zq^2}{4\pi\epsilon_0 m r}}
    }
    
\end{enumerate}





%%%%%%%%%%%%%%
\subsubsection{Paramagnetism}





\begin{notation}[Side Note:]

    Textbooks usually describe \bf{ferromagnetism} as the \nth{3} kind of magnetic response.
    However its origin is very different from dia-/paramagnetism:
    \begin{itemize}
        \item Dia-/paramagnetism comes from individual atoms' response to B-field.
        They can exist even if there is only one atom.

        \item Ferromagnetism is due to the collective behavior of many atoms. 
        It is actually a big topic in statistical mechanics.
    \end{itemize}

    On a ferromagnetic material, 
    
\end{notation}


%%%%%%%%%%%%%%
\subsection{Potential from Magnetic Dipole}

This is the classical model of magnetic dipole.
Same calculations can be found in many textbooks.

\begin{itemize}
    \item A circular loop of radius $R$, carrying current $I$.
    \item We are only interested in the potential/field far away from the dipole, 
    i.e. $\norm{\vvec{r}} \gg R$.
    The position vector $\vvec{r}$ is referenced from the center of the dipole.
\end{itemize}

\insertFig{magnetic dipole model}

The derivation of magnetic potential from a dipole is straightforward by Biot-Savart law,
while applying Taylor approximation in the middle step. 

\begin{enumerate}
    \item The total potential contributions towards position $\vvec{r}$ by two current elements at opposite position on the loop is
    \aleq{
        \dd{\vvec{A}}(\vvec{r}) 
        &= \frac{\mu_0}{4\pi} \qty( \frac{I\dd{\vvec{l}}_+}{\norm{\vvec{r}_+}} 
            + \frac{I\dd{\vvec{l}}_-}{\norm{\vvec{r}_-}})\\[2ex]
        &= \frac{\mu_0 I}{4\pi} \qty(\frac{\tkn{currSeg1}{R\dd{\theta}}}{\norm{\vvec{r}_+}}\tkn{currSeg3}{(\hhat{\theta})} 
            + \frac{\tkn{currSeg2}{R\dd{\theta}} }{\norm{\vvec{r}_-}}\tkn{currSeg4}{(-\hhat{\theta})})\\[2ex]
        &= \frac{\mu_0 IR\dd{\theta}}{4\pi} 
            \qty(\frac{1}{\tkn{cosLaw1}{\cul[blue]{\sqrt{\norm{\vvec{r}}^2 + R^2 - 2\norm{\vvec{r}}R\cos\phi}}}}
            - \frac{1}{\tkn{cosLaw2}{\cul[blue]{\sqrt{\norm{\vvec{r}}^2 + R^2 + 2\norm{\vvec{r}}R\cos\phi}}}}) \hhat{\theta}\\[2ex]
            %
        &= \frac{\mu_0 IR\dd{\theta}}{4\pi \norm{\vvec{r}}} 
            \qty(\inv{\sqrt{1 + \frac{R^2}{\norm{\vvec{r}}^2} - \frac{2R}{\norm{\vvec{r}}}\cos\phi}}
            - \inv{\sqrt{1 + \frac{R^2}{\norm{\vvec{r}}^2} + \frac{2R}{\norm{\vvec{r}}}\cos\phi}}) \hhat{\theta}
    }
    \addBentArrow[red]{currSeg1}{(20ex,2ex)}{}{(0,2ex)}
    \addBentArrow[red]{currSeg2}{(5ex,2ex)}{\scriptsize By geometry, $\norm{\dd{\vvec{l}}} = R\dd{\theta}$}{(0,2ex)}{(8ex,-0.5ex)}
    \addBentArrow[green]{currSeg3}{(20ex,-3ex)}{}{(0,-1ex)}
    \addBentArrow[green]{currSeg4}{(5ex,-3ex)}{\scriptsize And they point in opposite direction}{(0,-1ex)}{(12ex,1ex)}
    \addBentArrow[blue]{cosLaw1}{(7ex,-2ex)}{\scriptsize These are just cosine law}{(0,-1.5ex)}{(7.1ex,0.5ex)}
    \addBentArrow[blue]{cosLaw2}{(-7ex,-2ex)}{}{(0,-1.5ex)}

    \insertFig{show geometry, opposite current elements and position vectors}

    \item Taylor expansion ``\cul[red]{$(1+x)^n \approx 1+nx$ for $x\ll 1$}" is always applied at this step. 
    Because we are only interested in the potential far from the dipole, $\frac{R}{\norm{\vvec{r}}} \ll 1$, 
    \begin{itemize}
        \item $\frac{R^2}{\norm{\vvec{r}}^2}\ \approx 0$
        \item $\inv{\sqrt{1 - \frac{R}{\norm{\vvec{r}}}\cos\phi}} 
            = \qty(1 - 2\frac{R}{\norm{\vvec{r}}}\cos\phi)^{-\half}
            \approx 1 + \frac{R}{\norm{\vvec{r}}}\cos\phi$ 
    \end{itemize}

    So the potential becomes
    \aleq{
        \dd{\vvec{A}}(\vvec{r}) &= \frac{\mu_0 IR\dd{\theta}}{4\pi \norm{\vvec{r}}}
            \qty[\qty(1 + \frac{R}{\norm{\vvec{r}}}\cos\phi)
            - \qty(1 - \frac{R}{\norm{\vvec{r}}}\cos\phi)] \hhat{\theta}\\[1ex]
        &= \frac{\mu_0 IR\dd{\theta}}{4\pi \norm{\vvec{r}}}
            \cdot 2\frac{R}{\norm{\vvec{r}}}\cos\phi \cdot \hhat{\theta}
    }
    
    \item Before doing any integration, we first need to express every non-constant in terms of $\theta$: 
    \begin{itemize}
        \item Recall what we learnt about polar coordinate,
        the angular unit vector $\hhat{\theta}_{\{r,\theta\}}$ is a function of angle $\theta$.
        \aleq{
            \hhat{\theta} = -(\sin\theta) \hhat{x} + (\cos\theta) \hhat{y}
        }

        \item $\phi$ is the angle between the target's position $\vvec{r}$ and position vector of $I\dd{\vvec{l}_+}$. 
        Expressing them in x-y-z coordinate:
        \begin{itemize}
            \item Target's position: $\vvec{r} = x\hhat{x}+y\hhat{y}+z\hhat{z}$
            \item Position vector of $I\dd{\vvec{l}_+}$: $\vvec{R} = (R\cos\theta) \hhat{x} + (R\sin\theta)\hhat{y}$ 
        \end{itemize}
        So that $\cos\phi$ can be calculated as their dot product:
        \aleq{
            \cos\phi = \frac{\vvec{r}\cdot\vvec{R}}{\norm{\vvec{r}}\norm{\vvec{R}}}
            = \frac{x\cos\theta + y\sin\theta}{\norm{\vvec{r}}}
        }
    \end{itemize}
    
    \item Finally integrate over $\theta$ from $0$ to $\pi$ to find the total contribution of the loop.
    \aleq{
        \int \dd{\vvec{A}}(\vvec{r}) 
        &= \int_0^{\pi} \frac{\mu_0 IR\dd{\theta}}{4\pi \norm{\vvec{r}}}
            \cdot 2\frac{R}{\norm{\vvec{r}}}\cos\phi \cdot \hhat{\theta}\\[1ex]
        &= \frac{\mu_0 IR^2}{2\pi \norm{\vvec{r}}^2} 
            \int_0^{\pi} \qty(\frac{x\cos\theta + y\sin\theta}{\norm{\vvec{r}}})
            \qty(-(\sin\theta) \hhat{x} + (\cos\theta) \hhat{y}) \dd{\theta}\\[1ex]
        &= \frac{\mu_0 IR^2}{2\pi \norm{\vvec{r}}^3} 
            \int_0^{\pi} (-x\hhat{x}+y\hhat{y})\cus[red]{\sin\theta\cos\theta}{\substack{\int_0^{\pi} \sin\theta\cos\theta\, \dd{\theta}\\ =\ 0}}
            + (x\hhat{y})\cus[red]{\cos^2\theta}{\substack{\int_0^{\pi} \cos^2\theta\, \dd{\theta}\\ =\ \frac{\pi}{2}}} 
            - (y\hhat{x})\cus[red]{\sin^2\theta}{\substack{\int_0^{\pi} \sin^2\theta\, \dd{\theta}\\ =\ \frac{\pi}{2}}}  \dd{\theta}\\
        &= \frac{\mu_0 IR^2}{2\pi \norm{\vvec{r}}^3} \cdot\frac{\pi}{2}
           \cdot \qty(x\hhat{y} - y\hhat{x})
    }

    It happens that the term in the bracket is the cross product of $\hhat{z}$ and $\vvec{r}$, since
    \aleq{
        \hhat{z}\cross\vvec{r}
        =\bdet{
            \hhat{x} & \hhat{y} & \hhat{z}\\0 & 0 & 1 \\ x & y & z
        }
        = -y\hhat{x} + x\hhat{y}
    }

    So that the potential becomes
    \aleq{
        \vvec{A}(\vvec{r}) 
        &= \frac{\mu_0 IR^2}{4\norm{\vvec{r}}^2} \qty(\frac{x\hhat{y} - y\hhat{x}}{\norm{\vvec{r}}})\\[1ex]
        &= \frac{\mu_0 IR^2}{4\norm{\vvec{r}}^2} \frac{\hhat{z}\cross\vvec{r}}{\norm{\vvec{r}}}\\[1ex]
        &= \frac{\mu_0}{4\pi}\,\frac{I\pi R^2}{\norm{\vvec{r}}^2}\, (\hhat{z}\cross\vvec{r})
    }

    \item To emphasize that we now treat a magnetic dipole as ``one unit of source", 
    We define the \bf{magnetic dipole moment} $\vvec{m}$,
    \aleq{
        \Aboxed{
            \vvec{m}\ \defeq\ I\cdot\qty(\substack{\text{Loop's}\\\text{Area}})\cdot\qty(\substack{\text{Unit vector}\\\text{normal to loop}})
        }
    }

    In our case of a loop in x-y plane, $\vvec{m} = I(\pi R^2) \hhat{z}$.
    Finally we reach the standard formula of magnetic potential from a magnetic dipole:
    \aleq{
        \Aboxed{
            \vvec{A}(\vvec{r}) = \frac{\mu_0}{4\pi}\,\frac{\vvec{m}\cross\hhat{r}}{\norm{\vvec{r}}^2}
        }
    }
\end{enumerate}

As a comparison with Biot-Savart law formula from current element,

\aleq{
    \dd{\vvec{A}}(\vvec{r}) = \frac{\mu_0}{4\pi}\frac{I\dd{\vvec{l}}}{\norm{\vvec{r}}}
}

\insertFig{V from point charge source}

\begin{itemize}
    \item $\vvec{A}$ is always along the same direction as the current element.
    \item Dependance on distance is $\inv{r}$.
\end{itemize}

\aleq{
    \vvec{A}(\vvec{r}) = \frac{\mu_0}{4\pi}\frac{\vvec{m}\cross\hhat{r}}{\norm{\vvec{r}}^2}
}

\insertFig{V from dipole source}

\begin{itemize}
    \item Magnitude of potential depends on the current's magnitude AND the angle (cross product) between $\vvec{m}$ and $\hhat{r}$.
    \item Dependance on distance is $\inv{r^2}$.
\end{itemize}



%%%%%%%%%%%%%%
\subsection{B-field from Magnetic Dipole}

The standard derivation of B-field from dipole is through the relation $\vvec{B} = \curl \vvec{A}$.
It is nothing more than some boring vector calculus.  
Here I quote the final result:
\aleq{
    \Aboxed{
        \vvec{B}(\vvec{r}) = \frac{\mu_0}{4\pi}\frac{3(\vvec{m}\cdot\hhat{r})\hhat{r}-\vvec{m}}{\norm{\vvec{r}}^3}
    }
}

And this formula is rarely used because vector calculation is annoying. 

\begin{notation}[The boring derivation:]
    \begin{enumerate}
        \item Derivation is possible with high school calculus if we turn all vectors into x-y-z form: 
        $\vvec{r}= x\hhat{x} + y\hhat{y} + z\hhat{z}$, $\norm{\vvec{r}} = \sqrt{x^2 + y^2 + z^2}$
        and $\vvec{m} = m_x \hhat{x} + m_y \hhat{y} + m_z \hhat{z}$:
        \aleq{
            \frac{\vvec{m}\cross\hhat{r}}{\norm{\vvec{r}}^2}
            \, &=\, \frac{\vvec{m}}{\norm{\vvec{r}}^2}\cross\frac{\vvec{r}}{\norm{\vvec{r}}}
            \, =\, \frac{1}{\norm{\vvec{r}}^3}(\vvec{m} \cross \vvec{r})
            \, =\, \frac{1}{\norm{\vvec{r}}^3}\bdet{\hhat{x} & \hhat{y} & \hhat{z}\\m_x&m_y&m_z\\x&y&z} \\
            \, &= \, \frac{1}{(x^2 + y^2 + z^2)^{\frac{3}{2}}}[(m_yz-m_zy)\hhat{x} + (m_zx-m_xz)\hhat{y} + (m_xy - m_yx)\hhat{z}]
        }

        \item First the $x$ term in curl, $\qty(\pdvv{\bullet_z}{y}-\pdvv{\bullet_y}{z})\hhat{x}$, can be calculated by:
        \addArrow[blue]{dipoleB1}{(-2ex,5.5ex)}{}{(0,2.2ex)}
        \addArrow[blue]{dipoleB1}{(10ex,5.5ex)}{}{(1ex,2.2ex)}
        \addArrow[blue]{dipoleB2}{(5.5ex,5.5ex)}{}{(2.5ex,2.2ex)}
        \aleq{
            \pdvv{\bullet_z}{y} &= \pdvv{y}\qty[\frac{(m_xy - m_yx)}{(x^2 + y^2 + z^2)^{\frac{3}{2}}}] \\[1ex]
            &= \frac{m_x (x^2 + y^2 + z^2)^{\frac{3}{2}} - (m_xy - m_yx)\cdot 3y (x^2 + y^2 + z^2)^{\half}}{(x^2 + y^2 + z^2)^3}\\[1ex]
            &= \frac{m_x\norm{\vvec{r}}^2 - 3(m_xy - m_yx)y}{\norm{\vvec{r}}^{5}}\\[1ex]
            %
            \pdvv{\bullet_y}{z} &= \pdvv{z}\qty[\frac{(m_zx-m_xz)}{(x^2 + y^2 + z^2)^{\frac{3}{2}}}] \\[1ex]
            &= \frac{-m_x (x^2 + y^2 + z^2)^{\frac{3}{2}} - (m_zx - m_xz)\cdot 3z (x^2 + y^2 + z^2)^{\half}}{(x^2 + y^2 + z^2)^3}\\[1ex]
            &= \frac{-m_x\norm{\vvec{r}}^2 - 3(m_zx - m_xz)z}{\norm{\vvec{r}}^{5}}\\[1ex]
            %
            \Rightarrow\ \qty(\pdvv{\bullet_z}{y}-\pdvv{\bullet_y}{z})\hhat{x}
            &= \frac{2m_x\norm{\vvec{r}}^2 - 3(m_xy - m_yx)y + 3(m_zx - m_xz)z}{\norm{\vvec{r}}^{5}} \hhat{x}\\[1ex]
            &= \frac{2m_x\norm{\vvec{r}}^2 - 3m_xy^2 + 3m_yxy + 3m_zxz - 3m_xz^2 \red{+3m_xx^2 -3m_xx^2}}{\norm{\vvec{r}}^{5}} \hhat{x}\\[1ex]
            &= \frac{\cul[blue]{2m_x\norm{\vvec{r}}^2} - \cul[blue]{3m_x(x^2+y^2+z^2)} + 3m_xx^2 + 3m_yxy + 3m_zxz}{\norm{\vvec{r}}^{5}} \hhat{x}\\[1ex]
            &= \frac{\tkn{dipoleB1}{-m_x\norm{\vvec{r}}^2} + 3\cul[blue]{(m_x x + m_y y + m_z z)}x}{\norm{\vvec{r}}^{5}} \hhat{x}\\[1ex]
            &= \frac{-m_x\norm{\vvec{r}}^2 + 3\tkn{dipoleB2}{\cul[blue]{(\vvec{m}\cdot\vvec{r})}}x}{\norm{\vvec{r}}^{5}} \hhat{x}
        }
        
    \end{enumerate}
\end{notation}

\begin{notation}[]
    \begin{enumerate}

        \item[3.] Similar for $y$ and $z$ terms:
        \aleq{
            \qty(\pdvv{\bullet_x}{z}-\pdvv{A_z}{x})\hhat{y} 
            &= \frac{-m_y\norm{\vvec{r}}^2 + 3(\vvec{m}\cdot\vvec{r})y}{\norm{\vvec{r}}^{5}} \hhat{y}\\
            %
            \qty(\pdvv{\bullet_y}{x}-\pdvv{A_x}{y})\hhat{z} 
            &= \frac{-m_z\norm{\vvec{r}}^2 + 3(\vvec{m}\cdot\vvec{r})z}{\norm{\vvec{r}}^{5}} \hhat{z}
        }

        \item[4.] Summing all 3 components gives:
        \addArrow[blue]{dipoleB3}{(5ex,5.5ex)}{}{(0.7ex,2ex)}
        \addArrow[blue]{dipoleB4}{(30ex,6ex)}{}{(1ex,1.5ex)}
        \addBentArrow[red]{dipoleB5}{(7ex,1.5ex)}{\scriptsize Take out magnitude to become unit vector}{(0,2ex)}{(14ex,-0.4ex)}
        \aleq{
            \curl \vvec{A}(\vvec{r}) &= \curlRec{A_x}[A_y][A_z] \\[1ex]
            &= \frac{\mu_0}{4\pi}\,\frac{3(\vvec{m}\cdot\vvec{r})\cul[blue]{(x\hhat{x} + y\hhat{y} + z\hhat{z})}
                -\cul[blue]{(m_x\hhat{x} + m_y\hhat{y} + m_z\hhat{z})}\norm{\vvec{r}}^2}{\norm{\vvec{r}}^{5}} \\[1ex]
            &= \frac{\mu_0}{4\pi}\,\frac{3(\vvec{m}\cdot\vvec{r})\tkn{dipoleB3}{\cul[blue]{\vvec{r}}} 
                - \tkn{dipoleB4}{\cul[blue]{\vvec{m}}}\norm{\vvec{r}}^2}{\norm{\vvec{r}}^5}\\[1.2ex]
            &= \frac{\mu_0}{4\pi} \frac{3(\vvec{m}\cdot\red{\hhat{r}})\tkm{dipoleB5}\red{\hhat{r}} \, \red{\norm{\vvec{r}}^2} - \vvec{m}\norm{\vvec{r}}^2}{\norm{\vvec{r}}^5}\\
            \Aboxed{
                \vvec{B}(\vvec{r}) &= \frac{\mu_0}{4\pi}\frac{3(\vvec{m}\cdot\hhat{r})\hhat{r}-\vvec{m}}{\norm{\vvec{r}}^3}
            }
        }
    \end{enumerate}
        
\end{notation}



\linesep
% Section %%%%%%%%%%%%%%%%%%%%%%%%%%%%%%%%%%%%%%%%%%%%%%%%%%%%
\section{Describing Dipole Arrangement}

In general, all materials are made of many tiny magnetic dipoles.
To analyze the material's magnetic properties, 
we can begin with the property of a single dipole,
then sum the contributions of all dipoles according to the dipole arrangement in the material.
\aleq{
    \cub[blue]{\vvec{A}(\vvec{r}) = \frac{\mu_0}{4\pi} \frac{\vvec{m}_i }{\norm{\vvec{r}}^2} \cross \qty(\frac{\vvec{r}}{\norm{\vvec{r}}})}
        {\text{By a single dipole from origin}}
    \qquad\Rightarrow\qquad
    \cub[red]{\vvec{A}(\vvec{r}) = \frac{\mu_0}{4\pi} \sum_{\substack{\text{All dipoles } i}} 
        \frac{\vvec{m}_i }{\norm{\vvec{r}-\vvec{r}_i}^2} \cross \qty(\frac{\vvec{r}-\vvec{r}_i}{\norm{\vvec{r}-\vvec{r}_i}})}
        {\text{By many dipoles at different positions }\vvec{r}_i}
}


\insertFig{single dipole point to r vs many dipoles at different ri to r}

Here we introduce two quantities that describe magnetic dipole arrangements in materials.
\begin{itemize}
    \item \bf{Magnetization field} - $\vvec{M}(\vvec{r})$
    \item \bf{Bound current distributions} - $\vvec{J}_b(\vvec{r})$ and $\vvec{K}_b(\vvec{r})$
\end{itemize} 


%%%%%%%%%%%%%%
\subsection{Magnetization Field}

When the dipoles in the material are so dense such that we can treat them as a continuous distribution,
we can replace 
\begin{itemize}
    \item Summmation of all dipoles \ $\xRightarrow{\text{become}}$\  Volume integral over the whole object.
    \item Discrete dipoles source \ $\xRightarrow{\text{become}}$\ A vector distribution called \bf{magnetization field} $\vvec{M}(\vvec{r})$.
\end{itemize}
\aleq{
    \Aboxed{
        \vvec{A}(\vvec{r}) = \frac{\mu_0}{4\pi} \underset{\substack{\substack{\text{Whole}\\\text{material}}}}{\iiint}
        \frac{\vvec{M}(\vvec{r}')}{\norm{\vvec{r}-\vvec{r}'}^2} \cross \qty(\frac{\vvec{r}-\vvec{r}'}{\norm{\vvec{r}-\vvec{r}'}}) \dd[3]{\vvec{r}'}
    }
}

\insertFig{dispole arrangement is similar to some vector field in the object}

The magnetization field $\vvec{M}(\vvec{r})$ can also be interpreted as \bf{magnetic dipole density} 
because its usage is similar to charge density $\vvec{J}(\vvec{r})$. 
Comparing with the Biot-Savart law for magnetic potential:
\begin{itemize}
    \item When the source is made of current elements:
    \aleq{
        \vvec{A}(\red{\vvec{r}}) = \frac{\mu_0}{4\pi}{\iiint}
            \frac{\vvec{J}(\blue{\vvec{r}'})}{\norm{\red{\vvec{r}}-\blue{\vvec{r}'}}} \dd[3]{\blue{\vvec{r}'}}
        %
        \quad\sim\quad \frac{\mu_0}{4\pi} \sum \frac{(\text{current density})}{(\text{distance})}
    }

    \item When the source is made of dipoles:
    \aleq{
        \vvec{A}(\red{\vvec{r}}) = \frac{\mu_0}{4\pi}{\iiint}
            \frac{\vvec{M}(\blue{\vvec{r}'})}{\norm{\red{\vvec{r}}-\blue{\vvec{r}'}}^2} \cross \qty(\frac{\red{\vvec{r}}-\blue{\vvec{r}'}}{\norm{\red{\vvec{r}}-\blue{\vvec{r}'}}})\dd[3]{\blue{\vvec{r}'}}
        %
        \quad\sim\quad \frac{\mu_0}{4\pi} \sum \frac{(\text{dipole density})}{(\text{distance})^2}\cross \qty(\substack{\text{unit}\\\text{vector}})
    }
\end{itemize}

\insertFig{charge density vs dipole density in the same material}


%%%%%%%%%%%%%%
\subsection{Bound Current Distribution}

Ultimately, magnetic dipoles are just loops of currents.
If the dipoles are aligned non-uniformly in a material,
some regions may appear to have a higher flow of current in one direction than the other way.

\insertFig{dipole loops non uniform}

These regions with extra current flow are described as \bf{bound current} distribution in the material -
they are always ``bounded" to regions where the current flow is not cancelled.
So bound current distribution can be used to describe magnetic dipole arrangement in the material.

%%%%%%%%%%%%%%
\subsubsection{Mathematical Origin}

With vector calculus, 
the potential formula can be rewritten into a ``current densities form".
\aleq{
    \vvec{A}(\vvec{r}) &= \frac{\mu_0}{4\pi} \underset{\substack{\substack{\text{Whole}\\\text{material}}}}{\iiint}
        \frac{\vvec{M}(\vvec{r}')}{\norm{\vvec{r}-\vvec{r}'}^2} \cross \qty(\frac{\vvec{r}-\vvec{r}'}{\norm{\vvec{r}-\vvec{r}'}}) \dd[3]{\vvec{r}'}\\[1ex]
    %
    &= (\cdots \text{ \it{After more boring vector calculus} } \cdots)\\[1ex]
    %
    &= \frac{\mu_0}{4\pi} \underset{\substack{\substack{\text{Whole}\\\text{material}}}}{\iiint}
        \frac{\curl \vvec{M}(\vvec{r}')}{\norm{\vvec{r}-\vvec{r}'}} \dd[3]{\vvec{r}'}
        + \frac{\mu_0}{4\pi} \underset{\substack{\substack{\text{Surface}\\\text{of material}}}}{\oiint}
        \frac{\vvec{M}(\vvec{r}')}{\norm{\vvec{r}-\vvec{r}'}} \cross \dd[2]{\vvec{r}'}
}

By comparing with the Biot-Savart law for electric potential 
$\vvec{A}\sim \frac{\mu_0}{4\pi} \sum \frac{(\text{current density})}{(\text{distance})}$,
we identify the 2 source terms as the \bf{bound current densities}:
\begin{itemize}
    \item \bf{\ul{The \nth{1} term:}} \quad
    \aleq{
        \frac{\mu_0}{4\pi} \underset{\substack{\substack{\text{Whole}\\\text{material}}}}{\iiint}
            \frac{\curl \vvec{M}(\vvec{r}')}{\norm{\vvec{r}-\vvec{r}'}} \dd[3]{\vvec{r}'}
        \quad\sim\quad
        \frac{\mu_0}{4\pi} \sum_{\substack{\text{Inside}\\\text{material}}} \frac{(\curl \vvec{M})}{(\text{distance})}
    }

    \cul[red]{$\curl \vvec{M}$} appears as some current density distributed \cul[red]{inside} the material.
    Therefore it is defined as the \bf{volume bound current density} $\vvec{J}_b$.
    \aleq{
        \Aboxed{
            \vvec{J}_b(\vvec{r}) \ \defeq\ \curl \vvec{M}(\vvec{r})
        }
    }

    \vskip 1ex
    \item \bf{\ul{The \nth{2} term:}} \quad
    \aleq{    
        \frac{\mu_0}{4\pi} \underset{\substack{\substack{\text{Surface}\\\text{of material}}}}{\oiint}
        \frac{\vvec{M}(\vvec{r}')}{\norm{\vvec{r}-\vvec{r}'}} \cross \dd[2]{\vvec{r}'}
        \quad\sim\quad
        \frac{\mu_0}{4\pi} \sum_{\substack{\text{On material}\\\text{surface}}} \frac{(\text{Surface cross product of }\vvec{M})}{(\text{distance})}
    }

    The \cul[red]{surface cross product of $\vvec{M}$} means that we are taking cross product between $\vvec{M}$ and normal vector of the surface.
    It appears as some current density distributed \cul[red]{on the surface} of the material. 
    So it is defined as the \bf{surface bound current density} $\vvec{K}_b$.
    \aleq{
        \Aboxed{
            \vvec{K}_b(\vvec{r}) \ \defeq\ \vvec{M}(\vvec{r})\cross \hhat{n}
        }
    }
    where $\hhat{n}$ is the (outward) unit normal vector on the material surface.

\end{itemize}


Finally, the expression of the ``current densities" form is nothing more than saying that
the magnetic potential from a material is the result of the two kinds of current distributions.
\aleq{
    \vvec{A}(\vvec{r})
    &= \frac{\mu_0}{4\pi} \underset{\substack{\substack{\text{Whole}\\\text{material}}}}{\iiint}
        \frac{\red{\vvec{J}_b(\vvec{r}')}}{\norm{\vvec{r}-\vvec{r}'}} \dd[3]{\vvec{r}'}
        \ +\ \frac{\mu_0}{4\pi} \underset{\substack{\substack{\text{Surface}\\\text{of material}}}}{\oiint}
        \frac{\red{\vvec{K}_b(\vvec{r}')}}{\norm{\vvec{r}-\vvec{r}'}} \dd[2]{\vvec{r}'}\\[1ex]
        %
    &= \ \ \,\qty(\mstack{\text{Contribution}\\\text{by currents inside}\\\text{the material}})
        \ \ +\ \  \qty(\mstack{\text{Contribution}\\\text{by currents on}\\\text{material's surface}})
}

\begin{notation}[The boring derivation:]
    \begin{enumerate}
        \item Deriving the ``current densities form" begins with a vector calculus identity:
        \addArrow[blue]{potential1}{(-20ex,-4ex)}{\scriptsize Note: Differentiation\\[-1ex]\scriptsize is w.r.t. $x',y',z'$}
        {(0,-1ex)}{(3ex,-1ex)}
        \addBentArrow[blue]{potential2}{(5ex,2ex)}{\scriptsize This part is just a unit vector}
        {(0,4ex)}{(9.5ex,-0.7ex)}
        \vskip -3em
        \aleq{
            \grad \qty(\inv{\norm{\vvec{r}-\vvec{r}'}}) 
            &= \qty(\hhat{x}\pdvv{\tkm{potential1}\cul[blue]{x'}}+\hhat{y}\pdvv{\cul[blue]{y'}}+\hhat{z}\pdvv{\cul[blue]{z'}}) 
                \qty(\inv{\sqrt{(x - x')^2 + (y - y')^2 + (z - z')^2}})\\[0.5ex]
            &= \frac{\hhat{x}(x-x')+\hhat{y}(y-y')+\hhat{z}(z-z')}{\qty[(x - x')^2 + (y - y')^2 + (z - z')^2]^{\frac{3}{2}}}\\[0.5ex]
            &= \frac{\vvec{r}-\vvec{r}'}{\norm{\vvec{r}-\vvec{r}'}^3}\\
            &= \inv{\norm{\vvec{r}-\vvec{r}'}^2}\,\tkn{potential2}{\cul[blue]{\qty(\frac{\vvec{r}-\vvec{r}'}{\norm{\vvec{r}-\vvec{r}'}})}}
            \qquad\quad \red{\qty(\substack{\text{This is basically }\dv{r'}\qty(\inv{r-r'}) = \inv{(r-r')^2}\\ \text{ but in vector version}})}
        }

        \item So the potential formula can be rewritten as:
        \aleq{
            \vvec{A}(\vvec{r})
            &= \frac{\mu_0}{4\pi} {\iiint}
            \frac{\vvec{M}(\vvec{r}')}{\norm{\vvec{r}-\vvec{r}'}^2} \cross \qty(\frac{\vvec{r}-\vvec{r}'}{\norm{\vvec{r}-\vvec{r}'}}) \dd[3]{\vvec{r}'} \\[1ex]
            &= \frac{\mu_0}{4\pi} {\iiint}
                \cus[blue]{\vvec{M}(\vvec{r}') \cross \grad \qty(\inv{\norm{\vvec{r}-\vvec{r}'}})}
                {\scriptsize \vvec{G}\,\cross\,\grad f \ \ (\text{Note the order of }f\text{ and }\vvec{G})} \dd[3]{\vvec{r}'}\\[0.5ex]
            &= \frac{\mu_0}{4\pi} {\iiint}
                \cus[blue]{-\curl \qty(\frac{\vvec{M}(\vvec{r}')}{\norm{\vvec{r}-\vvec{r}'}})}{\scriptsize -\curl(f\vvec{G})} 
                + \cus[blue]{\frac{\curl \vvec{M}(\vvec{r}')}{\norm{\vvec{r}-\vvec{r}'}}}{\scriptsize f\curl \vvec{G}} \dd[3]{\vvec{r}'}
        }
        \vskip -1em
        Here we used the product rule of curl \fbox{$\curl{(f\vvec{G})} = f\curl\vvec{G} + (\grad f)\cross\vvec{G}$}, 
        where $f$ is a scalar function (like $\inv{\norm{\vvec{r}-\vvec{r}'}}$) and $\vvec{G}$ is a vector function (like $\vvec{M}$).

        \item Finally use the ``divergence theorem for curl" to convert the \nth{1} term's volume integral into a surface integral over the volume's surface:
        \addArrow[blue]{potential3}{(1ex,5.5ex)}{\scriptsize Divergence theorem for curl}
        {(0,2.5ex)}{(-11ex,-2.5ex)}
        \aleq{
            \vvec{A}(\vvec{r})
            &= \frac{\mu_0}{4\pi} \cul[blue]{{\iiint}
                \curl \qty(\frac{\vvec{M}(\vvec{r}')}{\norm{\vvec{r}-\vvec{r}'}}) \dd[3]{\vvec{r}'}}
                \ +\  \frac{\mu_0}{4\pi} {\iiint}\frac{\curl \vvec{M}(\vvec{r}')}{\norm{\vvec{r}-\vvec{r}'}} \dd[3]{\vvec{r}'}\\[2.5ex]
            &= \frac{\mu_0}{4\pi} \cul[blue]{{\oiint}\,
                \frac{\vvec{M}(\vvec{r}')}{\norm{\vvec{r}-\vvec{r}'}} \tkn{potential3}{\cross} \dd[2]{\vvec{r}'}}
                \ +\  \frac{\mu_0}{4\pi} {\iiint}\frac{\curl \vvec{M}(\vvec{r}')}{\norm{\vvec{r}-\vvec{r}'}} \dd[3]{\vvec{r}'}\\[1.5ex]
            &=\ \qty(\mstack{\text{Contribution}\\\text{from material's}\\\text{surface}})
                \ \ +\ \  \qty(\mstack{\text{Contribution}\\\text{from inside}\\\text{the material}})
        }

    \end{enumerate}
    Remind that we are integrating regions where dipoles exist. 
    So this volume integral corresponds to the whole material and surface integral corresponds to only the surface of the material.

    
    
\end{notation}



%%%%%%%%%%%%%%
\subsubsection{Visualization}

The bound current densities are related to $\vvec{M}$ pretty much like normal current density are related to $\vvec{B}$ in Ampere's law.

\begin{itemize}
    \item \bf{\ul{Volume bound current}}: 
    By drawing an Ampere's loop and check how many dipoles are pointing in the same direction as we travel along the loop,
    \aleq{
        \Aboxed{
            \mstack{\text{More dipole}\\\text{align with the circle} }
            \qquad\Leftrightarrow\qquad 
            \mstack{\text{Stronger current}\\\text{at the center}}
        }
    }
    
    \insertFig{dipole 's current in toroid shape}

    Recall that we can use sign of dot product line integral to identify positions with vector field in rotation pattern,
    and curl operator $\curl$ is equivalent to finding dot product line integral per area. 
    \aleq{
        \qty(\mstack{\text{Bound current}\\\text{volume density}}) 
        \ \sim\ \vvec{J}_b = \curl \vvec{M} \ \sim\  
        \qty(\mstack{\text{\# of dipoles per area}\\\text{around a rotation center}} )
    }

    \vskip 2ex
    \item \bf{\ul{Surface bound current}}:
    On the material surface, we can see that
    \aleq{
        \Aboxed{
            \mstack{\text{More dipole}\\\text{align in the same direction} }
            \qquad\Leftrightarrow\qquad 
            \mstack{\text{More current flow}\\\text{on the material surface}}
        }
    }
    \insertFig{surface out of paper dipole form current to left}

    As we have defined the normal vector $\hhat{n}$ of the material surface to be pointing outward, 
    the direction of current is exacly along $\vvec{M}\cross\hhat{n}$.
    \aleq{
        \qty(\mstack{\text{Bound current}\\\text{surface density}}) 
        \ \sim\ \vvec{K}_b = \vvec{M}\cross \hhat{n} \ \sim\  
        \qty(\mstack{\text{Same direction dipole}\\\text{per surface area}} )
    }

\end{itemize}

\begin{notation}[Side note:]
    For \cul[red]{paramagnetic and diamagnetic materials},  
    we almost never observe their $\vvec{M}$ field or bound current because 
    \begin{itemize}
        \item Magnitude of a dipole is too small.
        Typical magnetic dipole moment are of order $10^{-24}$ A$\cdot$m$^2$.

        \item The dipoles are usually randomly arranged, 
        or in patterns such that their effects cancel out.
        Especially when the material is large.
    \end{itemize}

    Their \red{magnetization effect is neligible unless it is placed under an external B-field},
    which forces its dipoles to align and change their dipole moment's magnitude.

    \insertFig{para, dia vs ferro}

    But we see magnetization in ferromagnetic material because their dipoles are uniformly aligned at room temperature,
    making the material behaves like a giant dipole.
\end{notation}



\linesep
% Section %%%%%%%%%%%%%%%%%%%%%%%%%%%%%%%%%%%%%%%%%%%%%%%%%%%%
\section{Material under External B-field}

Now we study what will happen when a material is magnetized by an external B-field:
\begin{enumerate}
    \item External B-field acts on material, 
    aligns the dipoles and changes their dipole moments.
    
    \item New dipoles alignment create a new B-field, which also affects surrounding dipoles.
    
    \item Alignment continues, until it reaches an equilibrium between external field and interactions between dipoles.
    
\end{enumerate}

Theoretically, how the dipoles reponse to the external B-field is a complicated process 
when all the bondings and interactions are involved - it highly depends on the material's structure.
So the final dipole alignment might not always align with the external B-field! 

\insertFig{random spherical dipole + regular external B-field = irregular align ellipse dipole}

In a general theory about material's B-field response, 
one may treat the magnetization field $\vvec{M}$ (i.e. new dipole alignment) as some function to the total B-field $\vec{B}_\text{total}$,
\aleq{
    \vvec{M} \ =\ f(\vvec{B}_\text{total}) 
    \ \sim\  \cub[blue]{\vvec{a}^{(1)}_iB_i + \vvec{a}^{(2)}_{ij}B_iB_j + \vvec{a}^{(3)}_{ijk}B_iB_jB_k + \cdots}{\text{\scriptsize Like a Taylor expansion}}
}

This $f$ function \cul[red]{denotes a theoretical model that we have choosen} to investigate the material.
And the \cul[red]{model parameters $\vvec{a}^{(1)}, \vvec{a}^{(2)}, \vvec{a}^{(3)}$... shall be determined from experiment}.

%%%%%%%%%%%%%%
\subsection{Special Case: Linear Magnetic Material}

Linear magnetic material is the simplest model of B-field response - where the dipoles are completely free to rotate,
such that new dipole alignment is directly proportional to the external B-field's magnitude and direction.
\cul[red]{This assumption is applicable to most daily life materials}.

\insertFig{dipoles uniform under external B-field}

The linear model only has 1 parameter to be determined from experiments -  
the proportionality constant between $\vvec{M}$ and $\vvec{B}_\text{total}$.
\aleq{
    \vvec{M} = \qty(\substack{\text{Some}\\\text{constant}}) \cdot \vvec{B}_\text{total}
}

Unlike using $\vvec{P}$ v.s. $\vvec{E}$ in linear dielectric model, 
linear magnetization model is almost always written by $\vvec{M}$ v.s. a new kind of field $\vvec{H}$.
We shall explain what $\vvec{H}$ right after.
\aleq{
    \Aboxed{
        \vvec{M} = \chi_m \vvec{H}
    }
}

Here the $\chi_m$ is called \bf{magnetic susceptibility},
a pure number (no unit) whose value depends on the type of material.
\begin{itemize}
    \item $\chi_m = 0$ for vacuum (by defintion), because there are no dipoles in a vacuum.
    \item $\chi_m<0$ for diamagnetism, 
    since their dipoles are opposite to the B-field's direction.
    \item $\chi_m>0$ for paramagnetism, 
    since their dipoles follow the B-field's direction.
\end{itemize}



%%%%%%%%%%%%%%
\subsection{Describing External Field}

The total B-field is a result of the external B-field plus the field induced by dipole alignment.
After we have chosen the model about dipole alignment,
now we look at the external B-field.


%%%%%%%%%%%%%%
\subsubsection{Free Current Distribution}

In order to create an external B-field around the material, 
we need a ``setup" to build an external source of currents. 
For example, apply current through a long solenoid:

\insertFig{free charge vs bound charge in dielectric setup}

The currents that flow in the setup are given the name \bf{free charges},
to distinguish from the bound currents (dipole alignment) in the material. 
The are "free" because we can always control them by varying the setup,
making them known quantities in calculation.

\insertFig{adjust solenoid current -> adjust free current amount}

\aleq{
    \mu_0 n I_\text{free} = \norm{\vvec{B}}
}

For calculation, free current densities are denoted like bound currents:

\begin{itemize}
    \item \bf{\ul{Volume free current density}} $\vvec{J}_f(\vvec{r})$ - 
    Current distribution \cul[red]{inside} the setup.

    \item \bf{\ul{Surface free charge density}} $\vvec{K}_f(\vvec{r})$ - 
    Current \cul[red]{on any surfaces} of the setup.
\end{itemize}

\vskip 1ex
In real practices, solenoid or a large U-shape electromagnet are the most common setup to create external B-field.
The B-field can be directly control by controlling current.
So line or surface free current are mostly all you need. No reason to make things complicated.



%%%%%%%%%%%%%%
\subsubsection{Auxiliary Field $\vvec{H}$}

Previously, we have relate bound currents to a vector field quantity - the magnetization field $\vvec{M}$.
Similarly, free current can be related with another vector field.
From Ampere's law, 
\aleq{
    \inv{\mu_0}\curl \vvec{B}_\text{total}\ 
    &=\ \ \ \,\vvec{J}_\text{total}\ \ = \ \qty(\mstack{\text{All}\\\text{currents}})\\[0.5ex]
    &=\ \vvec{J}_f + \vvec{J}_b \ = \ \qty(\mstack{\text{Free currents}\\\text{on setup}}) + \qty(\mstack{\text{Bound currents}\\\text{on material}})\\[0.5ex]
    &=\ \vvec{J}_f + \curl \vvec{M}\\[0.5ex]
    \curl \qty(\inv{\mu_0}\vvec{B} - \vvec{M})\ &=\ \vvec{J}_f
}

Here we define the \bf{Auxiliary field} $\vvec{H}$:
\aleq{
    \Acboxed[red]{
        \vvec{H} \ \defeq\ \frac{1}{\mu_0} \vvec{B} - \vvec{M}\tkm{H_field}
    }
}
\addArrow[red]{H_field}{(5ex,0)}{\scriptsize This relation connects\\[-1ex]\scriptsize all 3 field quantities}{(2ex,0.7ex)}{(7ex,0)}

\vskip -1em
Such that it is related to the free charge density by:
\aleq{
    \Aboxed{
        \bcase{
            \vvec{J}_f(\vvec{r}) \ &\defeq\ \curl \vvec{H}(\vvec{r}) \\[0.5ex]
            \vvec{K}_f(\vvec{r}) \ &\defeq\ \vvec{H}(\vvec{r})\cross \hhat{n}
        }
    }
}
where $\hhat{n}$ is the (outward) unit normal vector on the equipment surface.
Cross product with it shows the direction of the surface current relative to the $\vvec{H}$ field.\\

Notice the relations of $\vvec{H}$ with free currents are almost identical to $\vvec{B}$ with total currents in Ampere's law.
\red{In the special case of linear magnetic material}, 
we can solve for $\vvec{H}$ from the given free currents like the Ampere's law integral form or with Biot-Savart law.
\aleq{
    I_f &= \oint \vvec{H}(\vvec{r})\cdot \dd{\vvec{l}}\\[1ex]
    &= \oint \tkn{ampere_dot}{\cul[green]{\norm{\vvec{H}}\norm{\dd{\vvec{l}}}\cos\theta}}\\[1ex]
    &= \tkn{ampere_B}{\cul[red]{\norm{\vvec{H}}}}\ \tkn{ampere_theta}{\cul[blue]{\cos\theta}}\ \oint \norm{\dd{\vvec{l}}}\\[2em]
    &= \norm{\vvec{H}}\ \cos\theta\ (\text{Perimeter of loop})\\[2ex]
    \Aboxed{
        \norm{\vvec{H}} &= \frac{I_f}{(\text{Perimeter of loop})\cos\theta}
    }
}
\addArrow[green]{ampere_dot}{(5ex,0)}
{\scriptsize Just dot product\\[-1ex]\scriptsize $\vvec{a}\cdot\vvec{b}=\norm{\vvec{a}}\norm{\vvec{b}}\cos\theta$}
{(8ex,0)}{(6ex,0)}
\addBentArrow[red]{ampere_B}{(-8ex,-3.5ex)}
{\scriptsize Same magnitude everywhere\\[-1ex]\scriptsize Can move out of integral}
{(0,-1.5ex)}{(-8.5ex,1ex)}
\addBentArrow[blue]{ampere_theta}{(8ex,-3.5ex)}
{\scriptsize Form same angle everywhere\\[-1ex]\scriptsize Can move out of integral}
{(0,-1.5ex)}{(8.5ex,1ex)}

But note that \red{this can be wrong for other models of magnetic material}.


\begin{notation}[Side Note:]
    Although $\vvec{H}$ forms a PDE relationship to free current just like $\vvec{B}_\text{total}$ with total current,
    we cannot solve for $\vvec{H}$ exactly by free current if we don't know the material's model. 
    This is because:
    \begin{itemize}
        \item $\vvec{B}_\text{total}$ is guarenteed to be ``divergent-less",
        such that we can define a vector potential function $\vvec{A}$ and solve for $\vvec{B}_\text{total}$ uniquely.

        \item $\vvec{M}$ may not be ``divergent-less" since the dipole arrangement can be arbituary, 
        e.g. if they arrange into some vortex-like pattern.
        Then by $\vvec{H} = \frac{1}{\mu_0}\vvec{B}_\text{total} - \vvec{M}$,
        there is no guarenteed that $\vvec{H}$ is ``divergent-less" either.
    \end{itemize} 

    \insertFig{vortex dipole arrangement -> M field not curl 0}
  
    But in the \cul[red]{special case of linear magnetic material},
    \aleq{
        \cbox{\mstack{\vvec{M} \text{ aligns}\\[0.2ex]\text{with }\vvec{B}_\text{total}}}
        \quad\Rightarrow\quad
        \cbox{\mstack{\vvec{M} \text{ is}\\[1ex]\text{``divergent-less"}}}
        \quad\Rightarrow\quad
        \cbox{\mstack{\vvec{H} = \frac{1}{\mu_0}\vvec{B}_\text{total} - \vvec{M}\\[0.5ex]\text{is also ``divergent-less"}}}
    }

    So it is fine to solve for $\vvec{H}$ like normal Ampere's law problems for linear magnetic material.

    
\end{notation}


%%%%%%%%%%%%%%
\subsection{Measurement in Practice}

In problems that involve magnetic material, 
\begin{itemize}
    \item The material's magnetization (dipole alignment) is the cause of all troubles.
    But usually we are only told which material model we need to apply. 
    \it{(E.g. Given a linear magnetic material ...)}

    \item Actual calculation requires measurements from equipment setup. 
    In magnetostatic experiments, usually the control/measure-able parameters are
    \begin{itemize}
        \item \ul{Applied current} - Directly connect with a power source. All you need is an ammeter.
        \item \ul{Total B-field} - Can be measured through magnetic force.
    \end{itemize}
\end{itemize}

\insertFig{control set up with voltage or free charge procedure}


Then once we have chosen a model of the material, we can connect the two measurables by:
\aleq{
    \vvec{H} = \inv{\mu_0}\vvec{B}_\text{total} - f(\vvec{B}_\text{total})
}

Then the model parameters in $f(\cdots)$ can be determined by experiments. 
But in the special case of linear magnetic material, 
since there is only 1 constant parameter $\chi_m$ for each material, 
\aleq{
    \vvec{H} &= \inv{\mu_0}\vvec{B}_\text{total} - \chi_m\vvec{H}\\
    (1+\chi_m)\vec{H} &= \inv{\mu_0}  \vvec{B}_\text{total}\\
    \vvec{H} &= \inv{\mu_0(1+\chi_m)}  \vvec{B}_\text{total}\\
    &= \inv{(\text{A constant})}\cdot \vvec{B}_\text{total}
}

We usually use this new constant to represent the material's response to B-field,
by defining
\aleq{
    \boxed{
        \mu \ \defeq\ \mu_0(1+\chi_m)   \ \defeq\ \mu_0\mu_r 
    }
    \qquad\text{such that}\qquad
    \vvec{H} = \inv{\mu}\vvec{B}_\text{total}
}
where
\begin{itemize}
    \item $\mu$ = \bf{Absolute permeability}, or \bf{magnetic permeability}. Has the same unit as $\mu_0$.

    \item $\mu_r$ = \bf{Relative permeability}. A pure number. $<1$ for diamagetism and $>1$ for paramagnetism. 

\end{itemize}

Note that \red{$\vvec{H}$ only depends on applied current, i.e. can be easily controlled in experiments.}
\begin{itemize}
    \item For convenience, experimentalists always prefer using $\vvec{H}$ to describe material's magnetic response.
    This is why the linear model is written as $\vvec{M} \propto \vvec{H}$ 
    instead of $\vvec{M} \propto \vvec{B}_\text{total}$.

    \item $\chi_m$ is more oftenly tabulated in material handbooks and used in theories.
    This is because most material has very small $\chi_m$ (e.g. $10^{-5}$ for aluminum).
    Tabulating $\mu_r = 1+\chi_m$ only makes the values harder to read.
\end{itemize}



\insertFig{do you prefer reading table of chi m or mu r?}



%%%%%%%%%%%%%%
\subsection{Calculation Example}

Here are some examples of questions you may see in the chapter of linear magnetic material.
In principle, one of the information about $\vvec{H}$ (free current) or $\vvec{B}_\text{total}$ (force) will be given,
then you are asked to find the other, and so as $\vvec{M}$ and bound currents.










%%%
\theend
\end{document}