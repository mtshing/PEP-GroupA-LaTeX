\documentclass[class=article, crop=false, 12pt]{standalone}
\usepackage[subpreambles=true]{standalone}
\usepackage{../.common/common}


\author{Tony Shing}
%\pretitle{Supplementary}

\topic{Supplementary Note - Mechanics}
\title{Central Force Motion}

\version{2025} % leave blank for omitting

\begin{document}

\maketitle

%\heading{Lecture}{Tony}

\begin{overview}
    \begin{itemize}
        \item Equation of motion for central force motion, and its standard solution.
        \item Derivation of Kelper's Law. 
        \item Two-body motion is reducible to one-body motion $\Rightarrow$ solution only takes simple modification.
    \end{itemize}
\end{overview}


\begin{notation}
    \begin{enumerate}
        %
        \item For clear viewing, all vector (with arrow $\vec{}$ or hat $\hat{}$ ) will also be in \textbf{bold font}.
        %
        \item Spherical coordinate $(r,\theta, \phi)$ takes the convention in most textbooks,  
        where $\theta \in [0,2\pi]$ is the \underline{azimuthal angle} and $\phi \in [0,\pi]$ is the \underline{polar angle}.
        % 
        \item Coordinates are represented by $(x,y,z)$ and $(r,\theta,\phi)$. Their conversion is by the position vector's component: 
        \aleq{
            \vvec{r} = x\vvec{x}+y\vvec{y}+\vvec{z} 
            = (r\sin{\phi}\cos{\theta})\hhat{x}+(r\sin{\phi}\sin{\theta})\hhat{y} + (r\cos{\phi})\hhat{z} 
            = r\hhat{r}_\green{\text{at }(r,\theta,\phi)}
        }
        Be careful that $\vvec{r}$ is the position \red{\textbf{vector}} and $r$ is the \red{\textbf{radial component}}. 
        But in polar/ spherical coordinate, length of position vector $\abs{\vvec{r}} = r$ exactly.
        %
        \item All derivatives are written in full form $\dv{}{t}$ for clarity (Dot notation is hard to read). 
        %
    \end{enumerate}
\end{notation}


% content begins here
% Section %%%%%%%%%%%%%%%%%%%%%%%%%%%%%%%%%%%%%%%%%%%%%%%%%%%%
\section{The General Solution}

An object undergoes a central force motion if it is moving in a potential that depends purely on the radial coordinate of the object. 
i.e. $V(r,\theta, \phi) = V(r)$. The Newton's \nth{2} law simply writes:

\aleq{
    \Aboxed{
        m\dv[2]{\vvec{r}}{t} = -\dv{V(r)}{r}
    }
}


\red{Such potential is guarenteed to be conservative}, by showing that the curl on the force is $0$.

\begin{enumerate}
    \item The force only has radial component and also solely depends on the radial coordinate.
    \aleq{
        \vvec{F}(r,\theta, \phi) &= -\grad V(r) = -\gradSph{V(r)} \\
        %
        &= -\pdv{V(r)}{r}\hhat{r} + 0 + 0 \\
        &= f(r)\hhat{r}
    }
    %
    \item Then apply curl in spherical coordinate. Observe that all the terms vanish given that $F_\theta = F_\phi = 0$ and $F_r$ is a function on $r$ only.
    \aleq{
        \curl \vvec{F} = \curlSph{F_r}[F_\theta][F_\phi]
    }
\end{enumerate} 

%%%%%%%%%%%%
\vskip 1em
\subsection{Derivation} \label{standard proof}

\begin{itemize}
    \item The force from potential is the only force and is conservative $\Rightarrow$ Energy conserves.
    \aleq{
        E = \half m\abs{\hhat{v}}^2 + V(r) = \half m \qty[\qty(\dv{r}{t})^2+\qty(r\dv{\theta}{t})^2] + V(r)= \text{const}
    }
    \item Force is along radial direction $\Rightarrow$ Torque = $\hhat{r}\cross f(r)\hhat{r} = 0$ $\Rightarrow$ Angular momentum conserves.
    \aleq{
        L = m \vvec{r}\cross\vvec{v} = m r^2 \dv{\theta}{t} = \text{const}
    }
\end{itemize}

The standard rundown (every textbook does the same) is to treat $E$ and $L$ as adjustable parameters and derive the function of the object's trajectory $r(\theta)$.

\begin{enumerate}
    \item Substitute the equation of $L$ into equation of $E$ and rearange terms.
    \aleq{
        E &= \half m \qty[\qty(\dv{r}{t})^2+\qty(\frac{L}{mr})^2] +V(r)\\[0.5em]
        \Rightarrow\ \dv{r}{t} &= \sqrt{\frac{2}{m}\qty(E-V(r) - \frac{L^2}{2mr^2})}
    }
    \item By $L = mr^2 \dv{\theta}{t} \Rightarrow \dd{t} = \frac{mr^2}{L}\dd{\theta}$, then move all terms that contain $r$ to one side.
    \aleq{
        \dv{r}{t} &= \frac{L}{mr^2}\dv{r}{\theta} = \sqrt{\frac{2}{m}\qty(E-V(r) - \frac{L^2}{2mr^2})} \\[0.5em]
        \dd{\theta} &= \inv{r^2}\frac{\dd{r}}{\sqrt{\frac{2mE}{L^2} - \frac{2mV(r)}{L^2} - \inv{r^2}}}
    }
    \item This expression is ready to be integrated. To make it easier, substitute $u=\inv{r}$.
    \aleq{
        \int\dd{\theta} &= -\int\frac{\dd{u}}{\sqrt{\frac{2mE}{L^2} - \frac{2mV(u^{-1})}{L^2} - u^2}}
    }
    Now it cannot be further simplified until we know $V(r)$.
\end{enumerate}

%%%%%%%%%%%%
\vskip 1em
\subsection{Power Law Potential}

A function satisfies \textbf{power law} if it is in the form $f(x)\sim x^\alpha$ for any real number $\alpha$. 
Most of the time we deal with potential function belongs in such form, i.e. $V(r) = kr^\alpha$. The most common ones are:
\begin{itemize}
    \item $\alpha=-1$ : Columb's Law / Newtonian gravity -- $V(r) = \pm \frac{k}{r}$
    \item $\alpha=2$ : Simple harmonic motion -- $V(r) = \half k r^2$
\end{itemize}

Substituting the power law potential, mathematicians found that the integral can be resolved into simple functions only when $\alpha = 2, -1$ or $-2$.  


% Section %%%%%%%%%%%%%%%%%%%%%%%%%%%%%%%%%%%%%%%%%%%%%%%%%%%%
\vskip 1em
\rule{\textwidth}{1pt}
\section{Case $\alpha=-1$: Inverse square force}

The RHS integral can be simplified by trigonometric substitution.
\aleq{
    -\int\frac{\dd{u}}{\sqrt{\frac{2mE}{L^2} \red{\pm} \frac{2m\red{ku}}{L^2} - u^2}} 
    &= -\int\frac{\dd{u}}{\sqrt{\frac{2mE}{L^2} + \frac{m^2k^2}{L^4} - \qty(u \mp \frac{mk}{L^2})^2}} 
    & \qty(\substack{\text{completing}\\\text{square}}) \\
    %
    &= -\int \frac{\dd{\qty(u\mp\frac{mk}{L^2})}}{\sqrt{A^2 - \qty(u\mp\frac{mk}{L^2})^2}}
    & {\scriptstyle \qty(\text{take }A = \sqrt{\frac{2mE}{L^2} + \frac{m^2k^2}{L^4}})} \\
    %
    &= -\int \frac{\dd\qty(A\cos{\phi})}{\sqrt{A^2-A^2\cos^2\phi}}
    & {\scriptstyle \qty(\text{take } u\mp\frac{mk}{L^2} = A\cos{\phi})} \\
    %
    &= \int \dd{\phi} \\
    %
    &= \cos^{-1}\qty(\frac{u\mp\frac{mk}{L^2}}{\sqrt{\frac{2mE}{L^2} + \frac{m^2k^2}{L^4}}}) + C\\
    %
    &= \cos^{-1}\qty(\frac{\frac{L^2u}{mk}\mp 1}{\sqrt{\frac{2EL^2}{mk^2} + 1}}) + C \\
}

Finally putting back the LHS's $\theta$ and $u=\inv{r}$, we arrive
\aleq{
    \theta &= \cos^{-1}\qty(\frac{\frac{L^2}{mk\red{r}}\mp 1}{\sqrt{\frac{2EL^2}{mk^2} + 1}}) + C \\
    \Aboxed{\inv{r} &= \frac{mk}{L^2}\qty(\sqrt{1+\frac{2EL^2}{mk^2}}\cos{\qty(\theta -C)} \pm 1)}
}
where the $\pm$ sign is for: \red{$+$ve = attractive force, $-$ve = repulsive force.} 
This formula is usually written in a less complicated form by taking 
\begin{itemize}
    \item $\sqrt{1+\frac{2EL^2}{mk^2}} \,\rightarrow\, \epsilon =$ encentricity
    \item $C \,\rightarrow\, \theta_0 =$ some initial angle
    \item $\frac{mk}{L^2} \,\rightarrow\, \inv{r_0} =$ some initial radial distance 
\end{itemize} 

And becomes 
\aleq{
    \Aboxed{r &= \frac{r_0}{\epsilon\cos{\qty(\theta -\theta_0)} \pm 1}}
}


%%%%%%%%%%%%%%%%%
\vskip 1em
\subsection{Kepler's \nth{1} Law -- Equation of Trajectory}

The original statement from Kepler only concerns elliptical/circular orbit, because they were the only types of orbits observed.

\quoting{The orbit of every planet is an ellipse with the sun at one of the two foci.}

With modern mathematics, we can show that every conic sections are possible trajectories. 
Substituting cartesian coordinate $r=\sqrt{x^2+y^2}$, $\cos{\theta}=\frac{x}{r}$ and $\sin{\theta}=\frac{y}{r}$ to the above results,
we arrive at the quadratic form of conic sections, i.e. $Ax^2+Bxy+Cy^2+Dx+Ey+F=0$, regardness of the force being attractive or repulsive.
\aleq{
    \qty(\epsilon^2\cos^2\theta_0-1)x^2 + \qty(\epsilon^2\sin^2\theta_0-1)y^2 + \qty(2\sin{\theta_0}\cos{\theta_0})xy -(2\epsilon r_0\cos\theta_0) x - (2\epsilon r_0 \sin\theta_0) y + r_0^2 = 0
} 

By choosing $\theta_0 = 0$, the conic section "lies flat". We can further simplify it in to the standard form:
\aleq{
    (1-\epsilon^2)x^2 + y^2 + (2\epsilon r_0)x &= r_0^2 \\
    %
    \qty(x + \frac{r_0\epsilon }{1-\epsilon^2})^2 + \frac{y^2}{1-\epsilon^2} &= \frac{r_0^2}{1-\epsilon^2} + \frac{r_0^2\epsilon^2}{(1-\epsilon^2)^2} \\
    &= \qty(\frac{r_0}{1-\epsilon^2})^2 \\
    %
    \frac{\qty(x + \frac{r_0\epsilon}{1-\epsilon^2})^2}{\qty(\frac{r_0}{1-\epsilon^2})^2} + \frac{y^2}{\qty(\frac{r_0}{\sqrt{1-\epsilon^2}})^2} &= 1 \\[0.5em]
    %
    \Rightarrow \quad 
    \Aboxed{
        \frac{(x-c)^2}{a^2} + \frac{y^2}{b^2} &= 1
    }
} 

Geometrically, no matter what kinds of conic section it is, we may identify these letters with the following names.
Although they are mostly used in circular/elliptical orbits because they are positive real number only if $\epsilon<1$.

\begin{itemize}
    \item \textbf{Semi-latus rectum} --- $l = r_0$ \qquad ($l$ is another commonly used symbol)
    \item \textbf{Semi-major axis} --- $\boxed{a = \frac{r_0}{1-\epsilon^2} = -\frac{k}{2E}}$
    \item \textbf{Semi-minor axis} --- $\boxed{b = \frac{r_0}{\sqrt{1-\epsilon^2}}}$ 
    \item \textbf{Focal distance} --- $\boxed{c = \frac{r_0\epsilon}{1-\epsilon^2} = a\epsilon}$ \qquad\qquad  (Another name for $c$ = linear encentricity)
\end{itemize}



The value of encentricity controlled the type of conic section. 

\begin{itemize}
    \item \textbf{$\epsilon = 0$ \,---\, Circle}
    \begin{itemize}
        \item $E = \frac{-mk^2}{2L^2} = -\frac{k}{2r_0}$.
        %
        \item Polar form: $r=\frac{r_0}{0+1} = r_0$. i.e. Radial distance = constant.
    \end{itemize}
    %
    \item \textbf{$0<\epsilon<1 $ \,---\, Ellipse}
    \begin{itemize}
        \item $ \frac{-mk^2}{2L^2} < E < 0$.
        %
        \item By taking $\theta-\theta_0 = 0$ or $\pm\pi$ in the polar form, the range of radial distance is between \red{$\frac{r_0}{1-\epsilon}$ (aphelion)} and \red{$\frac{r_0}{1+\epsilon}$ (perihelion)}. \\
        (mnemonic(?): alphabetically, "a" is superior over "p")

        \insertFig{illustrate geometry of ellispe}

        \gray{Need more geometry description}

    \end{itemize}
    %
    \item \textbf{$\epsilon=1$ \,---\, Parabola}
    \begin{itemize}
        \item $E=0$
        %
        \item From the polar form, $\theta=\theta_0 \Rightarrow r=\frac{r_0}{2}$. $r$ increases to $\infty$ when $\theta$ changes to $\pm \pi$.
        %
        \item Semi-major axis, semi-minor axis and focal distance are undefined (infinite). 
        Writing the standard form needs to start with
        \aleq{ 
            (1-1^2)x^2 + y^2 + (2r_0)x &= r_0^2 \\
            y^2 &= -2r_0\qty(x-\frac{r_0}{2})
        }

        \insertFig{illustrate geometry of parabola}

        \gray{Need more geometry description}

    \end{itemize}
    %
    \item \textbf{$\epsilon>1$ \,---\, Hyperbola}
    \begin{itemize}
        \item $E>0$
        %
        \item From the polar form, $\theta=\theta_0 \Rightarrow r=\frac{r_0}{1+\epsilon}$ (perihelion). 
        However unlike parabola, radial distance already reaches infinity when $\cos(\theta-\theta_0)=-\inv{\epsilon}$.
        This defines the inclination angle of the asymptotes.

        \insertFig{illustrate geometry of hyperbola}

        \gray{Need more geometry description}

    \end{itemize}
\end{itemize}


\textbf{Note: } In the case of repulsive force, parabolar and hyperbola are the only possible trajectoryies. 
This is because the object always has $E\geq 0$.


%%%%%%%%%%%%%%%%%
\vskip 1em
\subsection{Kepler's \nth{2} Law -- Conservation of Angular Momentum}

The original statement for Kelper's \nth{2} Law says:

\quoting{A line joining a planet and the Sun sweeps out equal areas during equal intervals of time.}

It is written in such way because the concept of angular momentum did not exist during Kelper's life (Newton was birth after Kepler's death).
With mathematics, we can show that:

\begin{itemize}
    \item The angle $\theta$ of an arc is related to sweeping time by $\dd{t} = \frac{mr^2}{L}\dd{\theta}$
    \item The arc's area is $Area = \iint r\dd{\theta}\dd{r} = \int \half r^2 \dd{\theta}$
\end{itemize}

So combining,
\aleq{
    Area_{(t_2-t_1)} = \int_{t_1}^{t_2} \half \cdot \frac{L}{m} \dd{t} \tikzmark{kepler2}= \frac{L}{2m}{(t_2-t_1)}
}
\addArrow{kepler2}{(8pt, 2ex)}{(8pt, 4.5ex)}{(8pt, 6ex)}{\scriptsize $\substack{\text{True only if}\\L \text{ is not a function of }t}$}

Note that this is true even for parabola and hyperbola orbits.

\insertFig{kepler2 law illustration}


%%%%%%%%%%%%%%%%%
\vskip 1em
\subsection{Kepler's \nth{3} Law -- Period of Elliptical Orbit}

The original statement for Kepler's \nth{3} Law only applies to circular and elliptical orbit, 
because the concept of "period" only applies to closed orbits. 

\quoting{The ratio of the square of an object's orbital period with the cube of the semi-major axis of its orbit is the same for all objects orbiting the same primary.}

If we go back to the standard proof in section \ref{standard proof} and stop at step 1, 
without substituting $\dd{t} = \frac{mr^2}{L}\dd{\theta}$, we can integrate and obtain a relation between $t$ and $r$.
Note that we can simplify this integral without removing the $\pm$ sign, so it applies to both attractive and repulsive cases.
\aleq{
    \dd{t} &= \frac{\dd{r}}{\sqrt{\frac{2}{m}\qty(E-V(r) - \frac{L^2}{2mr^2})}} \\
    %
    &= \frac{\dd{r}}{\sqrt{\frac{2}{m}\qty(E \red{\pm\frac{k}{r}} - \frac{L^2}{2mr^2})}} \\[0.5em]
    %
    \int \dd{t} &= \sqrt{-\frac{m}{2E}} \int \frac{r\dd{r}}{\sqrt{r^2 \mp \frac{k}{E}r + \frac{L^2}{2mE}}}
    & \qty(\substack{\text{Denominator} \\\text{multiply }-\frac{r^2}{E}}) \\
    %
    &= \sqrt{-\frac{m}{2E}} \int \frac{r\dd{r}}{\sqrt{-\qty(r\mp\frac{k}{2E})^2+\frac{k^2}{4E^2}+\frac{L^2}{2mE}}}
    & \qty(\substack{\text{completing}\\\text{square}})
}

We can substitute $a=-\frac{k}{2E}$ and $a^2\epsilon^2 = \frac{k^2}{4E^2}\qty(1+\frac{2EL^2}{mk^2})$ to simplify the notations. 
\aleq{
    \Rightarrow \int \dd{t} &= \sqrt{-\frac{m}{2E}} \int \frac{r\dd{r}}{\sqrt{-(r\pm a)^2+a^2\epsilon^2}}\\
    %
    &= \sqrt{-\frac{m}{2E}} \int \frac{(a\epsilon\cos{\varphi}\mp a) \dd{(a\epsilon\cos{\varphi}\mp a)}}{\sqrt{a^2\epsilon^2-(a\epsilon\cos{\varphi})^2}}
    & {\scriptstyle \qty(\text{take } r\pm a = a\epsilon\cos{\varphi})} \\
    %
    &= \sqrt{-\frac{m}{2E}} \int -(a\epsilon\cos{\varphi}\mp a) \dd{\varphi}\\
    %
    &= \sqrt{-\frac{m}{2E}} (-a\epsilon\sin{\varphi} \pm a\varphi)  + C\\
    %
    &= \sqrt{\frac{a}{k}}(-a\epsilon\sin{\varphi} \pm a\varphi)  + C
    & {\scriptstyle \qty(\text{by } a = -\frac{k}{2E})}
}

The symbol $\varphi$ is commonly denoted as the \textbf{eccentric anomaly}.
Its geometrical relation with the rotation angle $\theta$ is as depicted:

\insertFig{eccentric anomaly}

$\varphi$ and $\theta$ can be inter-converted through $r$. Although this is usually done numerically. 
\aleq{
    \frac{r_0}{\epsilon\cos{\theta} \pm 1} = r = a\epsilon\cos{\varphi} \mp a \\[0.5em]
    %
    \Rightarrow \quad \frac{r_0}{a} = 1-\epsilon^2 = (1\mp \epsilon \cos{\varphi})(1\pm \epsilon\cos{\theta})
}
We can immediately check that in elliptical orbit, $\theta = 0 \Leftrightarrow \varphi = 0$ and $\theta=\pi \Leftrightarrow \varphi=\pi$. 
So taking $\varphi$ from $0$ to $2\pi$ is the same as revolving one cycle of $\theta$ from $0$ to $2\pi$. 
Substitute these into the $t$ v.s. $\varphi$ relation, we arrive at the familar Kepler's \nth{3} law formula.
\aleq{
    \text{1 Period} = \int^T_0 \dd{t} 
    &= \eval{\sqrt{\frac{a}{k}}(-a\epsilon\sin{\varphi} \pm a\varphi)}_{\varphi =0}^{\varphi =2\pi} \\
    \Aboxed{
        T &= \frac{2\pi}{\sqrt{k}}a^{\frac{3}{2}}
    }
}

\textbf{Note:} Trajectory are more frequently expressed in terms of $\varphi$ in space engineering texts because of the $t$ v.s. $\varphi$ relation --
It is a lot easier for computing transit time along any segment of any orbit.








% Section %%%%%%%%%%%%%%%%%%%%%%%%%%%%%%%%%%%%%%%%%%%%%%%%%%%%
\vskip 1em
\rule{\textwidth}{1pt}
\section{Two-body Problem Reduction}

In a two body system, where their interaction forces satisfy:
\begin{itemize}
    \item its magnitude that only depends on the distance between the two objects.
    \item its direction is along the line that connects the two objects.
\end{itemize} 
Then it is easy to show that the equation of motions can be rewritten into the form of one body subjecting to the same force. 
Thus we can solve their trajectories by solving only 1 ODE.

\insertFig{two body config}

The pairs of Newton \nth{2} Law should start in this form:
\aleq{
    \begin{cases}
        m_1\dvv[2]{\vvec{r}_1}{t} &= \vvec{F}(\abs{\vvec{r}_2-\vvec{r}_1}) \\[1em]
        m_2\dvv[2]{\vvec{r}_2}{t} &= -\vvec{F}(\abs{\vvec{r}_{2}-\vvec{r}_1})
    \end{cases}
}

We can immediately see that, after dividing the masses and subtracting one of them by the other, we arrive at
\aleq{
    \dv[2]{\vvec{r}_2}{t} - \dvv[2]{\vvec{r}_1}{t} &= -\qty(\inv{m_2}+\inv{m_1})\vvec{F}(\abs{\vvec{r}_{2}-\vvec{r}_1})\\
    \Rightarrow \qquad \dv[2]{\vvec{u}}{t} &= -\qty(\inv{m_2}+\inv{m_1})\vvec{F}(\abs{\vvec{u}}) 
}

by taking $\vvec{u} = \vvec{r}_2 - \vvec{r}_1$. It is common to denote $\mu = \inv{\inv{m_1}+\inv{m_2}}$ as the \textbf{reduced mass}, 
so that it is equivalent to the Newton \nth{2} Law describing an object of mass $\mu$ moving along its trajectory $\vvec{u}$ subjecting to a force $-\vvec{F}(\abs{\vvec{u}})$. 
On the other hand, adding the two equation gives the Newton \nth{2} Law about the center of mass, $\vvec{r}_{CM} = \frac{m_1\vvec{r}_1+m_2\vvec{r}_2}{m_1+m_2}$:
\aleq{
    m_2\dv[2]{\vvec{r}_2}{t} + m_1\dvv[2]{\vvec{r}_1}{t} &= 0 \\
    \Rightarrow \qquad \dv[2]{t}\qty(m_1\vvec{r}_1+m_2\vvec{r}_2) &= \dv[2]{t}\qty(\frac{m_1\vvec{r}_1+m_2\vvec{r}_2}{m_1+m_2}) = 0
}

\insertFig{equiv: two obj orbit each other == 1 obj, redeuced mass, orbit around origin, r=r2-r1. finally retrieve back r1,r2 from u}

Now the problem is reduced to two independent ODEs --- a still annoying one for $\vvec{u}$ and a trivial one for $\vvec{r}_{CM}$ respectively.
\aleq{
    \begin{cases}
        \mu\dvv[2]{\vvec{u}}{t} &= -\vvec{F}(\abs{\vvec{u}}) \\[1em]
        \dvv[2]{\vvec{r}_{CM}}{t} &= 0
    \end{cases}
}

which is comparatively easier than solving a system of 2 ODEs. 
Finally with $\vvec{u}$ and $\vvec{r}_{CM}$, we can retrieve back $\vvec{r}_1$ and $\vvec{r}_2$ by the geometric relation.
\aleq{
    \begin{cases}
        \vvec{r}_1 = \vvec{r}_{CM} - \frac{m_2}{m_1+m_2}\vvec{u} \\
        \vvec{r}_2 = \vvec{r}_{CM} + \frac{m_1}{m_1+m_2}\vvec{u}
    \end{cases}
}


\end{document}