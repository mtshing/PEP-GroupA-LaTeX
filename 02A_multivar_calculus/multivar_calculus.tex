\documentclass[class=article, crop=false, 12pt]{standalone}
\usepackage[subpreambles=true]{standalone}
\usepackage{../.common/common}


\author{Tony Shing}
%\pretitle{Supplementary}

\topic{Note 2A (Math for Physics)}
\title{Multivariable Calculus}

\version{2025} % leave blank for omitting

\begin{document}

\maketitle

%\heading{Lecture}{Tony}

\begin{overview}
    \begin{itemize}
        \item Comparison between single variable functions \& multivariable functions
        \item Partial differentiation (on scalar function)
        \item Multiple integral (on scalar function)
    \end{itemize}
\end{overview}



% content begins here
% Section %%%%%%%%%%%%%%%%%%%%%%%%%%%%%%%%%%%%%%%%%%%%%%%%%%%%
\section{Functions with Multiple Variables}

To well-define a function $f(x)$ in advanced mathematics,
we actually need to specify the function's \bf{domain} and \bf{image}.\\

\begin{itemize}
    \item Domain = The set of values that be substitute into $x$.
    \item Image = The set of all possible output of $f(x)$. 
\end{itemize}

E.g. Formal notation in math text to define $f(x) = \inv{|x|}$:
\aleq{
    f\colon \tkn{domain}{\cul[red]{\mathbb{R}}} & \longrightarrow \tkn{image}{\cul[blue]{\mathbb{R}^+}} \\
  x & \longmapsto \inv{|x|}
}
\addArrow[red]{domain}{(-10ex,-4ex)}{Domain}{(0,-1ex)}
\addArrow[blue]{image}{(10ex,-4ex)}{Image}{(0,-1ex)}

We can classify functions by whether their domain/image are made of single number / tuple of numbers.



%%%%%%%%%%%%%%
\subsection{Single Variable Scalar Function}

They are the functions that you have already learnt.

\begin{itemize}
    \item Domain = A set of single number
    \item Image = A set of single number 
\end{itemize}

For example, 
\aleq{
    f(x) = \sqrt{x-1} \qquad \Rightarrow \qquad 
    \bcase{\text{Domain } &= \text{ Any real number }\geq 1 \\
    \text{Image } &= \text{ Any real number }\geq 0}
}


\insertFig{number line map}


%%%%%%%%%%%%%%
\subsection{Multivariable Scalar Function}

\begin{itemize}
    \item Domain = A set of tuples of number, like $x = (1,2,3)$
    \item Image = A set of single number, like $f(x) = 5$
\end{itemize}

For example,
\aleq{
    f(x,y) = \sqrt{1-xy} \qquad \Rightarrow \qquad 
    \bcase{\text{Domain } &= \text{Any pair of values }x,y\text{ where } xy\leq 1 \\
    \text{Image } &= \text{ Any real number }\geq 0}
}

\insertFig{1-sqrt(xy)}

\ul{Example in Physics:}
\begin{itemize}
    \item Gravitational potential energy
    \aleq{
        U(x,y,z) = -\frac{GMm}{r} = -\frac{GMm}{\sqrt{x^2+y^2+z^2}}
    }

    \item Density distribution in object $\rho(x,y,z)$
    \insertFig{density distribution}
\end{itemize}


%%%%%%%%%%%%%%
\subsection{Single Variable Vector Function}

\begin{itemize}
    \item Domain = A set of single number
    \item Image = A set of tuple of numbers
\end{itemize}

For example,
\aleq{
    \vvec{r}(t) = (x(t), y(t)) = (t^2, 3t^3) \qquad \Rightarrow \qquad 
    \bcase{\text{Domain } &= \text{Any real number } t \\
    \text{Image } &= (x,y) \text{ pairs restricting on } x=\qty(\frac{y}{3})^{\frac{2}{3}}}
}

\insertFig{t2, 3t3}

\ul{Example in Physics:}
\begin{itemize}
    \item Displacement, velocity, acceleration
    \aleq{
        \bcase{
            \vvec{s}(t) &= (x(t), y(t), z(t)) \\
            \vvec{v}(t) &= (v_x(t), v_y(t), v_z(t)) \\
            \vvec{a}(t) &= (a_x(t), a_y(t), a_z(t))
        }
    }
\end{itemize}

%%%%%%%%%%%%%%
\subsection{Multivariable Vector Function}

\begin{itemize}
    \item Domain = A set of tuple of numbers
    \item Image = A set of tuple of numbers
\end{itemize}

For example,
\aleq{
    \vvec{r}(u,v) = (r_1(u,v), r_2(u,v)) = (u^2+v^2, u-1-v^2) \qquad \Rightarrow \qquad 
    \bcase{\text{Domain } &= \text{The whole u-v plane} \\
    \text{Image } &= \text{ Region depicted below}}
}

\insertFig{u2v2}

\ul{Example in Physics:}
\begin{itemize}
    \item Gravitational force
    \aleq{
        \vvec{F}(\vvec{r}) &= \vvec{F}(x,y,z)\\
        &= -\frac{GMm}{|\vvec{r}|^2}\cdot \red{\qty(\frac{\vvec{r}}{|\vvec{r}|})} \tkm{runit}\\
        &= -\frac{GMm}{x^2+y^2+z^2} \cdot \qty(
        \frac{\green{x}\blue{\hhat{x}} + \green{y}\blue{\hhat{y}} + \green{z}\blue{\hhat{z}}
        }{\sqrt{x^2+y^2+z^2}}) \tkm{sep_component}\\
        &= \qty[\frac{-GMm\green{x}}{(x^2+y^2+z^2)^{\frac{3}{2}}}]\blue{\hhat{x}}
        + \qty[\frac{-GMm\green{y}}{(x^2+y^2+z^2)^{\frac{3}{2}}}]\blue{\hhat{y}}
        + \qty[\frac{-GMm\green{z}}{(x^2+y^2+z^2)^{\frac{3}{2}}}]\blue{\hhat{z}}
    }
    \addArrow[red]{runit}{(5ex, 0)}{Unit vector of $\vvec{r}$}{(2ex,0)}{(8ex,0)}
    \addArrow[blue]{sep_component}{(5ex, 0)}{Separate into components of $\hhat{x}/\hhat{y}/\hhat{z}$}{(2ex,0)}{(18ex,0)}
\end{itemize}

%%%%%%%%%%%%%%
\subsection{Function Composition for multivariable functions}

For single variable scalar function, you should be familiar with the what function composition is. 
For example, if $f(x) = \sin{x}, g(x) = e^{x}$, we can have these compositions:

\aleq{
    f(f(x)) = \sin{(\sin{x})} \quad,\quad 
    f(g(x)) = \sin{(e^x)} \quad, \quad 
    g(f(x)) = e^{sin{x}} \quad, \quad
    g(g(x)) = e^{e^x}
}

Note that function composition is the key in chain rule. 

\aleq{
    \dvv{x}f(g(x)) = \dvv{f(g(x))}{g(x)}\cdot \dvv{g(x)}{x}
}

However for multivariable functions, we can construct function composition only if 
the number of output matches the next function's number of input. For example, let's have
\aleq{
    \bcase{
        f(p,q) &= \sqrt{p+q} \qquad \blue{\text{2 inputs, 1 output. Denote as }(2 \xrightarrow{f} 1)} \\
        \vvec{g}(t) &= (t-1, t^2) \qquad \blue{\text{1 input, 2 outputs. Denote as }(1 \xrightarrow{g} 2)} \\
        \vvec{h}(u,v) &= (u^2+v, u-v) \qquad \blue{\text{2 inputs, 2 outputs. Denote as }(2 \xrightarrow{h} 2)}
    }
}

We can have the following composition:
\aleq{
    f(\vvec{g}(t)) &= \sqrt{(t-1) + (t^2)} \qquad \blue{(1 \xrightarrow{g} 2 \xrightarrow{f} 1)} \\
    f(\vvec{h}(u,v)) &= \sqrt{(u^2+v) + (u-v)} \qquad \blue{(2 \xrightarrow{h} 2 \xrightarrow{f} 1)} \\
    \vvec{g}(f(p,q)) &= (\sqrt{p+q}-1, p+q) \qquad \blue{(2 \xrightarrow{f} 1 \xrightarrow{g} 2)} \\
    \vvec{h}(\vvec{g}(t)) &= ((t-1)^2+(t^2), (t-1)-(t^2)) \qquad \blue{(1 \xrightarrow{g} 2 \xrightarrow{h} 2)} \\
    \vvec{h}(\vvec{h}(t)) &= ((u^2+v)^2+(u-v), (u^2+v)-(u-v)) \qquad \blue{(2 \xrightarrow{h} 2 \xrightarrow{h} 2)}
}

But these are NOT allowed:
\aleq{
    g(h(u,v)) \quad &\colon \quad \blue{(2 \xrightarrow{h} 2 \ \red{\Rightarrow}\ 1 \xrightarrow{g} 2)} \\
    h(f(p,q)) \quad &\colon \quad \blue{(2 \xrightarrow{f} 1 \ \red{\Rightarrow}\ 2 \xrightarrow{h} 2)} \\
}

\linesep
% Section %%%%%%%%%%%%%%%%%%%%%%%%%%%%%%%%%%%%%%%%%%%%%%%%%%%%
\section{Limits on Multivariable Scalar Function}

In single variable functions, $\lim_{x\to a} f(x) = L$ means 
when the input $x$ is "close enough" to a value $a$, output of $f(x)$ must be "close" to some $L$.
This idea can be extended to multivariable function, i.e.
\aleq{
    \lim_{(x_1,x_2,...,x_n)\to(a_1,a_2,...,a_n)} f(x_1, x_2,...,x_n) = L
}

The idea requires all the inputs $x_1, x_2, ...$ to be "close enough" to some corresponding values $a_1, a_2,...$,
only after then the output of $f(...)$ will be "close" enough to some $L$. 
We can visually compare it with single variable function as follow:

\insertFig{limit sing var vs mul var}

However, the "existance" of limit in multivariable functions has a much stricter requrement.
\begin{itemize}
    \item Single variable function: 
    \begin{itemize}
        \item Input $x$ must approach $a$ from either left ($x^-$) or right ($x^+$).
        \item "Existance" of limit only require showing both left/right limit approach to the same output $L$.
    \end{itemize}
    
    \item Multivariable function, (e.g. functions with 2 inputs):
    \begin{itemize}
        \item Inputs $(x_1, x_2)$ can approach the point $(a_1, a_2)$ along any trajectories on the plane. 
        \item "Existance" of limit require showing that along ALL trajectories
    \end{itemize}
\end{itemize}

\insertFig{2 dir vs inf dir approach}

 Proving a limit exist rigorously is a lot harder in multivariable function.
 But in physics, we almost never need to deal with any strange functions that has limit only along certain trajectories.
 We may assume that every function we encounter is well-behaved, and then calculation can be done like in single variable functions. E.g.
\aleq{
    \lim_{(x,y)\to \qty(\frac{\pi}{2},\frac{\pi}{2})} \sin{x}\cos{y} = \sin\qty(\frac{\pi}{2})\cos\qty(\frac{\pi}{2})
}
 





\linesep
% Section %%%%%%%%%%%%%%%%%%%%%%%%%%%%%%%%%%%%%%%%%%%%%%%%%%%%
\section{Partial Differentiation}

\begin{itemize}
    \item Notation: $\pdvv{x}, \pdvv{y}, ...$
    \item Usually pronouced as "partial x", "partial y", etc.
\end{itemize}

Comparing with ordinary differentiation to single variable function, 
the notation difference is to emphasize that the \bf{differentiation is only about 1 of the inputs}. 

%%%%%%%%%%%%%%
\subsection{Definition \& Geometrical Interpretation}

The limit definition of partial differentiation of $f(x_1, x_2,..., x_n)$ at $(a_1, a_2,...,a_n)$ in the i$^{th}$ input (\red{$x_i$}) 's direction is defined as:
\aleq{
    &\pdvv{\red{x_i}}f(x_1, x_2,...,\red{x_i},...,x_n) \\
    = &\lim_{\red{\Delta x_i \to 0}} \qty[\frac{f(x_1,x_2,...,\red{x_i+\Delta x_i},...,x_n) - f(x_1, x_2,...,\red{x_i},...,x_n)}{\red{\Delta x_i}}]
}

Note that \red{the limit only acts on the i$^{th}$ input.} Other inputs remains untouched.\\

Therefore in calculation, when doing partial differentiation over $x_i$,
only $x_i$ is differentiated (the same way we do in single variable differentiation), while the other $x$ are treated as constants.\\

E.g. $f(x,y,z) = x^2y\sin{z}$
\aleq{
    \pdvv{f}{\red{x}} &= \red{2x}\cdot y\sin{z} &\qty(\dvv{x}x^2 = 2x, \text{don't touch }y,z)\\
    \pdvv{f}{\red{y}} &= x^2 \cdot \red{1} \cdot \sin{z} &\qty(\dvv{y}y = 1, \text{don't touch }x,z)\\
    \pdvv{f}{\red{z}} &= x^2y\cdot \red{\cos{z}} &\qty(\dvv{z}\sin{z} \cos{z}, \text{don't touch }x,y)
}

The visualization to partial differentiation is straightforward. 
Take a 2-inputs function $f(x,y)$ as example, we can draw the followings:

\insertFig{partial d geo interp}

$\pdvv{x}$ = On the plane of constant $y$, find slope along $x$ direction.

$\pdvv{y}$ = On the plane of constant $x$, find slope along $y$ direction.\\


We can conclude: 
\begin{center}
    $\pdvv{x_i}$ = Find slope / rate of change of function with respect to $x_i$
\end{center}


%%%%%%%%%%%%%%
\subsection{Evaluation}

Calculation rules for partial differentation are the same as you have learnt in single variable differentiation. 
\red{The only exception is chain rule}, which is the\bf{sum of chain rule with respect to each of the input.}

\aleq{
    \pdvv{x_i}f(\vvec{g}(x_1,x_2,...,x_n)) &= \sum_j \pdvv{g_j}f(\vvec{g})\pdvv{g_j}{x_i} \\
    &= \pdvv{g_1}f(\vvec{g})\pdvv{x_i}g_1(x_1,x_2,...,x_n) + \pdvv{g_2}f(\vvec{g})\pdvv{x_i}g_2(x_1,x_2,...,x_n) + ...
}

\it{As for now you do not need to remember this formula. We will be able to write it in a more compact (and easier to remember) form after learning matrix.}\\

As an example of calculation, suppose we start with two functions without knowing their exact expression: 
\aleq{
    f(p,q) & &\qquad \blue{(2 \xrightarrow{f} 1)} \\
    \vvec{h}(u,v) &= (h_1(u,v), h_2(u,v)) = (h_1,h_2)  &\qquad \blue{(2 \xrightarrow{h} 2)}
}
And construct the following composition:
\aleq{
    f(\vec{h}(u,v)) &= f((h_1, h_2))= f((h_1(u,v), h_2(u,v))) &\qquad \blue{(2 \xrightarrow{h} 2 \xrightarrow{h} 1)}
}

Because $f(\vec{h}(u,v))$ takes 2 inputs $u,v$, there must be 2 partial differentiations (one for $u$ and one for $v$).
With chain rule, the partial differentiations write as \\

With respect to $u$:
\aleq{
    \pdvv{u}f(\vvec{h}(u,v)) &= \pdvv{\tkn{uu}{\cul[red]{\cul[blue]{u}}}} f
        (\tkn{h11}{\cul[red]{h_1}}, \tkn{h12}{\cul[blue]{h_2}}) \\[2.5em]
        %
    &= \cub[red]{{\qty(\pdvv{\cbox[red]{h_1}}f(\cbox[red]{h_1},h_2) \cdot \cbox[red]{\pdvv{u}h_1})}}{\text{Chain rule over }h_1 \text{ only}} 
        + \cub[blue]{\qty(\pdvv{\cbox[blue]{h_2}}f(h_1,\cbox[blue]{h_2})\cbox[blue]{\pdvv{u}h_2})}{\text{Chain rule over } h_2 \text{ only}}
}
\addUnderArrow[red]{uu}{h11}{$u$ on $h_1$ }{-2ex}{(0,0)}{(0,-0.5ex)}
\addAboveArrow[blue]{uu}{h12}{$u$ on $h_2$}{5ex}{(0.5ex,-0.5ex)}

With respect to $v$:
\aleq{
    \pdvv{v}f(\vvec{h}(u,v)) &= \pdvv{\tkn{vv}{\cul[red]{\cul[blue]{v}}}} f
        (\tkn{h21}{\cul[red]{h_1}}, \tkn{h22}{\cul[blue]{h_2}}) \\[2.5em]
        %
    &= \cub[red]{{\qty(\pdvv{\cbox[red]{h_1}}f(\cbox[red]{h_1},h_2) \cdot \cbox[red]{\pdvv{v}h_1})}}{\text{Chain rule over }h_1 \text{ only}} 
        + \cub[blue]{\qty(\pdvv{\cbox[blue]{h_2}}f(h_1,\cbox[blue]{h_2})\cbox[blue]{\pdvv{v}h_2})}{\text{Chain rule over } h_2 \text{ only}}
}
\addUnderArrow[red]{vv}{h21}{$v$ on $h_1$ }{-2ex}{(0,0)}{(0,-0.5ex)}
\addAboveArrow[blue]{vv}{h22}{$v$ on $h_2$}{5ex}{(0.5ex,-0.5ex)}

We may do straightforward substitution, if the functions' expressions are given. Let's say,   
\aleq{
    f(p,q) &= \sqrt{p+q} \qquad \text{and}\qquad \vvec{h}(u,v) = (u^2+v, u-v) = (h_1, h_2)
}

Then 
\aleq{
    \pdvv{u}f(\vvec{h}(u,v)) 
    &= \cul[red]{\qty(\pdvv{h_1}f(h_1, h_2)\pdvv{u}h_1)} + \cul[blue]{\qty(\pdvv{h_2}f(h_1, h_2)\pdvv{u}h_2)} \\[1em]
    &= \cul[red]{\qty(\pdvv{h_1}\sqrt{h_1+h_2}\eval_{\substack{h_1=u^2+v \\ h_2=u-v}} \cdot \pdvv{u}(u^2+v))}  
        + \cul[blue]{\qty(\pdvv{h_2}\sqrt{h_1+h_2} \eval_{\substack{h_1=u^2+v \\ h_2=u-v}}\cdot \pdvv{u}(u-v))}  \\[1em]
    &= \cul[red]{\qty(\inv{2\sqrt{h_1+h_2}}\eval_{\substack{h_1=u^2+v \\ h_2=u-v}}\cdot 2u)}
         + \cul[blue]{\qty(\inv{2\sqrt{h_1+h_2}}\eval_{\substack{h_1=u^2+v \\ h_2=u-v}}\cdot (-1))} \\[1em]
    &= \frac{2u}{2\sqrt{u^2+u}} + \frac{-1}{2\sqrt{u^2+u}} \\[1em]
    &= \frac{2u-1}{2\sqrt{u^2+u}}\\[1em]
    %
    \pdvv{v}f(\vvec{h}(u,v)) 
    &= \cul[red]{\qty(\pdvv{h_1}f(h_1, h_2)\pdvv{v}h_1)} + \cul[blue]{\qty(\pdvv{h_2}f(h_1, h_2)\pdvv{v}h_2)} \\[1em]
    &= \cul[red]{\qty(\pdvv{h_1}\sqrt{h_1+h_2}\eval_{\substack{h_1=u^2+v \\ h_2=u-v}} \cdot \pdvv{v}(u^2+v))}  
        + \cul[blue]{\qty(\pdvv{h_2}\sqrt{h_1+h_2} \eval_{\substack{h_1=u^2+v \\ h_2=u-v}}\cdot \pdvv{v}(u-v))}  \\[1em]
    &= \cul[red]{\qty(\inv{2\sqrt{h_1+h_2}}\eval_{\substack{h_1=u^2+v \\ h_2=u-v}}\cdot (1))}
         + \cul[blue]{\qty(\inv{2\sqrt{h_1+h_2}}\eval_{\substack{h_1=u^2+v \\ h_2=u-v}}\cdot (-1))} \\[1em]
    &= \frac{1}{2\sqrt{u^2+u}} + \frac{-1}{2\sqrt{u^2+u}} \\[1em]
    &= 0
}

We can also compute the composition directly for result checking:
\aleq{
    f(\vvec{h}(u,v)) = \sqrt{u^2+v + u-v} = \sqrt{u^2+u}
}
\aleq{
    \Rightarrow\quad \pdvv{f}{u} = \frac{2u+1}{2\sqrt{u^2+u}} \qquad \text{and} \qquad \pdvv{f{v}} = 0
}

\begin{exercise}
    Given the functions and their composition: 
    \aleq{
        \begin{cases}
            f(p,q)=\sqrt{p,q} \\ \vvec{g}(t) = (t-1,t^2)
        \end{cases}
        \qquad \Rightarrow \qquad
        f(\vvec{g}(t)) = \sqrt{t^2+t-1}
    }
    Compute the derivative $\dvv{f(\vvec{g}(t))}{t}$, by
    \begin{enumerate}
        \item directly differentiating against $t$
        \item first differentiate via chain rule over $(p,q)$
    \end{enumerate}
    (You should get equal results.)
\end{exercise}



\linesep
% Section %%%%%%%%%%%%%%%%%%%%%%%%%%%%%%%%%%%%%%%%%%%%%%%%%%%%
\section{Multiple Integral}

The limit definition of multiple integral can be written as

\aleq{
    &\idotsint_\qty(\substack{\text{Some}\\\text{region}})
    f(x_1, x_2,...,x_n) \dd{x_1}\dd{x_2}...\dd{x_n} \\
    = & 
    \lim_{\Delta x_1, \Delta x_2,...,\Delta x_n to 0} \sum_{\substack{\text{all divisions}\\\text{in the region}}}
    f(\xi_1, \xi_2, ..., \xi_n) \Delta x_1 \Delta x_2...\Delta x_n 
}

Recall that we have introduced 2 geometrical interpretations of integration.
Here we can demonstrate them on the two most frequently used multiple integral.

%%%%%%%%%%%%%%
\subsection{Double Integral}

For functions with 2 inputs.\\

\ul{Interpretation 1: Volume under surface, bounded by base area}

\aleq{
    \iint_{\red{\substack{\text{Some}\\\text{Area}}}} f(x,y) \dd{x}\dd{y} 
    = \lim_{\Delta x, \Delta y \to 0} \sum_{\red{\substack{\text{All divisions}\\\text{in the area}}}} 
    \tkn{pillar_h}{\cul[blue]{f(\xi_x, \xi_y)}} \tkn{pillar_a}{\cul[yellow]{\Delta x\Delta y}}
}\\
\addArrow[blue]{pillar_h}{(0,-4ex)}{Pillar's height}{(0,-2ex)}
\addArrow[yellow]{pillar_a}{(4ex,-4ex)}{Pillar's base area}{(0,-2ex)}{(7ex,0)}

\insertFig{vol under surface}


\ul{Interpretation 2: Weighted sum over an area}

\aleq{
    \iint_{\red{\substack{\text{Some}\\\text{Area}}}} f(x,y) \dd{x}\dd{y} 
    = \lim_{\Delta x, \Delta y \to 0} \sum_{\red{\substack{\text{All divisions}\\\text{in the area}}}} 
    \tkn{weight2}{\cul[blue]{f(\xi_x, \xi_y)}} \tkn{area2}{\cul[yellow]{\Delta x\Delta y}}
}
\addArrow[blue]{weight2}{(0,-6ex)}{$\substack{\displaystyle \text{Weight for the}\\\displaystyle \text{grid at }(\xi_x,\xi_y)}$}{(0,-2ex)}
\addArrow[yellow]{area2}{(4ex,-4ex)}{Area of each grid}{(0,-2ex)}{(7ex,0)}

\insertFig{area grid divi}

For a physics example, the \it{area mass density distribution} $\sigma(x,y)$ may depends on position coordinate $(x,y)$.
\begin{itemize}
    \item Each small grid has an area $\dd{x} \dd{y}$
    \item At position $(\xi_x, \xi_y)$, the grid has a density $\sigma(x,y)$
\end{itemize}

Thus, 
\aleq{
    \text{Total mass} &= \text{Sum of mass of all small grids}\\[1em]
    &= \sum_{\substack{\text{all small grids}\\\text{in the area}}} \qty(\substack{\text{density}\\\text{of each grid}})\qty(\substack{\text{area}\\\text{of each grid}}) \\[1em]
    &= \sum_{\substack{\text{all small grids}\\\text{in the area}}} \sigma(x,y) \cdot (\Delta x \Delta y) \\[1em]
    &\approx \iint_{\text{the area}} \sigma(x,y) \dd{x}\dd{y}
}


%%%%%%%%%%%%%%
\subsection{Triple Integral}

\ul{Interpretation 1: ??? under volume}\\

Sorry, we live in a 3D space. No idea how to draw 4D objects.\\


\ul{Interpretation 2: Weighted sum over a volume}

\aleq{
    \iiint_{\red{\substack{\text{Some}\\\text{Volume}}}} f(x,y,z) \dd{x}\dd{y}\dd{z}
    = \lim_{\Delta x, \Delta y, \Delta z \to 0} \sum_{\red{\substack{\text{All divisions}\\\text{in the volume}}}} 
    \tkn{weight3}{\cul[blue]{f(\xi_x, \xi_y, \xi_z)}} \tkn{vol3}{\cul[yellow]{\Delta x\Delta y \Delta z}}
}
\addArrow[blue]{weight3}{(0,-6ex)}{$\substack{\displaystyle \text{Weight for the}\\\displaystyle \text{cube at }(\xi_x,\xi_y, \xi_z)}$}{(0,-2ex)}
\addArrow[yellow]{vol3}{(4ex,-4ex)}{Volume of each cube}{(0,-2ex)}{(7ex,0)}

Similar to double integral, if $\rho(x,y,z)$ is the \it{volume mass density distribution}, 
Thus, 
\aleq{
    \text{Total mass} &= \text{Sum of mass of all small cubes}\\[1em]
    &= \sum_{\substack{\text{all small cubes}\\\text{in the volume}}} \qty(\substack{\text{density}\\\text{of each cube}})\qty(\substack{\text{volume}\\\text{of each cube}}) \\[1em]
    &= \sum_{\substack{\text{all small cubes}\\\text{in the volume}}} \sigma(x,y) \cdot (\Delta x \Delta y) \\[1em]
    &\approx \iiint_{\text{the volume}} \sigma(x,y) \dd{x}\dd{y}
}

%%%%%%%%%%%%%%
\subsection{Evaluating Multiple Integral}

The difficulty in calculation mostly comes from determining the region to be integrated.
Here are the main steps in your calculation:

\begin{enumerate}
    \item Decide the integration order, i.e. how to divide a region.
    \begin{itemize}
        
        \item The integration order decide the expression. Follow the expression to integrate "from inside to outside".
        \aleq{
            \iiint f(x,y,z) \dd{x}\dd{y}\dd{z} = \yellow{\tkn{out}{\int}\qty(\blue{\tkn{mid}{\int}\qty(\tkn{in}{\red{\int f(x,y,z)\dd{x}}})\dd{y}})\dd{z}}
        }
        \addArrow[red]{in}{(4ex,-3ex)}{inner = \nth{1}}{(0,-2ex)}{(2ex,0)}
        \addArrow[blue]{mid}{(0,-3ex)}{middle = \nth{2}}{(0,-2ex)}
        \addArrow[yellow]{out}{(-8ex,-3ex)}{outer = \nth{3}}{(0,-2ex)}{(-4ex,0)}

        \item Calculation is just like how you do to single variable functions, but do it multiple times. 
        \item While integrating one variable, treat the others as constants.
    \end{itemize}

    \item Derive the corresponding upper/lower bounds
    \begin{itemize}
        \item It would be easier if you can draw out the region.
        \item Note that if you switch the integration order, the bounds must change.
    \end{itemize}

\end{enumerate}


\begin{example}
    Integrate $f(x,y) = x^2y-xy^3$ over the region bounded by 
    $\begin{cases}x&=1\\x&=4\end{cases}$ and $\begin{cases}y&=2\\y&=3\end{cases}$

    \insertFig{rect int}

    \ul{Integration order 1:} First $x$, then $y$.
    \begin{enumerate}
        \item Integrate $x$ = Sum all grid with the same $y$ coordinate to form horizontal strips.
        \insertFig{hoz strip}
        \aleq{
            \int_{x=1}^{x=4}f(x,y)\dd{x}
        }
        
        \item Integrate $y$ = Sum all horizontal strips to from the integration region.
        \insertFig{hoz stip to region}
        \aleq{
            \int_{y=2}^{y=3}\qty[\int_{x=1}^{y=4}f(x,y)\dd{x}]\dd{y}
        }

        \item In the calculation, follow the expression's order: Integrate $x$ first, then $y$. 
        Note that before integrating $y$, you need to clear all $x$ by substituting the given upper/lower bounds.
        \aleq{
            &\int_{y=2}^{y=3}\red{\qty[\int_{x=1}^{y=4}x^2y-xy^3\dd{x}]}\dd{y}\\[1em]
            = &\int_{y=2}^{y=3}\red{\qty[\frac{x^3}{3}y - \frac{x^2}{2}y^3]\eval_{x=1}^{x=4}}\dd{y}\\[1em]
            = &\int_{y=2}^{y=3} \cub[red]{\qty(\frac{64}{3}y-\frac{16}{2}y^3)}{\text{Subst. }x=4} - \cub[red]{\qty(\frac{1}{3}y - \half y^3)}{\text{Subst. }x=1} \dd{y} \\[1em]
            = &\int_{y=2}^{y=3} 21y - \frac{15}{2}y^3 \dd{y} \\[1em]
            = &\qty[\frac{21}{2}y^2-\frac{15}{8}y^4]\eval_{y=2}^{y=3} \\[1em]
            = &\frac{-555}{8}
        }
    \end{enumerate}

    \ul{Integration order 2:} First $y$, then $x$.
    \begin{enumerate}
        \item Integrate $y$ = Sum all grid with the same $x$ coordinate to form vertical strips.
        \insertFig{vert strip}
        \aleq{
            \int_{y=3}^{y=2}f(x,y)\dd{y}
        }
        
        \item Integrate $y$ = Sum all horizontal strips to from the integration region.
        \insertFig{hoz stip to region}
        \aleq{
            \int_{x=1}^{x=4}\qty[\int_{y=2}^{y=3}f(x,y)\dd{y}]\dd{x}
        }

        \item In the calculation, follow the expression's order: Integrate $y$ first, then $x$. 
        Note that before integrating $x$, you need to clear all $y$ by substituting the given upper/lower bounds.
        \aleq{
            &\int_{x=1}^{x=4}\red{\qty[\int_{y=2}^{y=3}x^2y-xy^3\dd{y}]}\dd{x}\\[1em]
            = &\int_{x=1}^{x=4}\red{\qty[\half x^2y^2 - \inv{4}xy^4]\eval_{y=2}^{y=3}}\dd{x}\\[1em]
            = &\int_{x=1}^{x=4} \cub[red]{\qty(\frac{9}{2}x^2 - \frac{81}{4}x)}{\text{Subst. }y=3} - \cub[red]{\qty(2x^2-4x)}{\text{Subst. }y=2} \dd{x} \\[1em]
            = &\int_{x=1}^{x=4} \frac{5}{2}x^2 - \frac{65}{4}x \dd{x} \\[1em]
            = &\qty[\frac{5}{6}x^3-\frac{65}{8}x^2]\eval_{x=1}^{x=4} \\[1em]
            = &\frac{-555}{8}
        }
    \end{enumerate}

\end{example}

However, if the boundaries of the region is ugly, 
some integration order make your life easier than the others.
As a demonstration, consider integrating over the below region (with an arbituary $f(x,y)$):

\insertFig{ugly region}

\ul{Integration order 1:} First $y$, then $x$.

\insertFig{vert bar than add }

This approach is easy because all vertical strips have the same bounds:
\begin{itemize}
    \item Upper bound: The curve $y=x^2+1$
    \item Lower bound: The curve $y=x-1$
\end{itemize}

We can write the integral expression as a single term.
\aleq{
    \int_{x=0}^{x=1}\int_{y=x-1}^{y=x^2+1} f(x,y) \dd{y}\dd{x}
}

\ul{Integration order 2:} First $x$, then $y$.\\

Note that the bounds of horizontal strips are different for different $y$:

\insertFig{different bound}

So we need to integrate each region individually.

\aleq{
    I_1 = \int_{y=1}^{y=2}\int_{x=\sqrt{y+1}}^{x=1} f(x,y) \dd{x}\dd{y}
}
\aleq{
    I_2 = \int_{y=0}^{y=1}\int_{x=0}^{x=1} f(x,y) \dd{x}\dd{y}
}
\aleq{
    I_3 = \int_{y=0}^{y=-1}\int_{x=0}^{x=y+1} f(x,y) \dd{x}\dd{y}
}

And the final answer would be the sum to all 3 regions $I_1+I_2+I_3$. 
Although we should get the same value as we integrate $y$ first then $x$,
integrating $x$ first then $y$ takes a lot more effort.


%%%
\theend
\end{document}