\documentclass[class=article, crop=false, 12pt]{standalone}
\usepackage[subpreambles=true]{standalone}
\usepackage{../common/packages}
\usepackage{../common/command}
\usepackage{../common/format}


\author{Tony Shing}
%\pretitle{Supplementary}

\topic{Note 6A - Mechanics}
\title{Vectors in Polar Coordinates}

\version{2025} % leave blank for omitting

\begin{document}

\maketitle

%\heading{Lecture}{Tony}

\begin{overview}
    \begin{itemize}
        \item The unit vectors in polar coordinate are not "constant" of position and time.
        \item Angular quantities ($\theta/\omega/\alpha$) $\sim$ Angular component of ($s/v/a$). 
        \item Relative angular velocity does not exists. You need to consider the radial part too.
    \end{itemize}
\end{overview}


\begin{notation}
    \begin{enumerate}
        %
        \item For clear viewing, all vector (with arrow $\vec{}$ or hat $\hat{}$ ) will also be in \textbf{bold font}.
        %
        \item Notations for physics quantities: Position vector = $\vvec{r}$, velocity vector = $\vvec{v}$, acceleration vector = $\vvec{a}$.
        % 
        \item Coordinates are represented by $(x,y)$ and $(r,\theta)$. Their conversion is according to the position vector's component: 
        \aleq{
            \vvec{r} = x\hhat{x}+y\hhat{y} = (r\cos{\theta})\hhat{x}+(r\sin{\theta})\hhat{y} = r\hhat{r}
        }
        Note that $\vvec{r}$ is the \red{\textbf{position vector}} and $r$ is the \red{\textbf{radial component}}. 
        But in polar coordinate, \red{\textbf{length of position vector $\abs{\vvec{r}} = r$}} exactly.
        %
        \item All derivatives will be written in full form $\dv{}{t}$ for clarity, because dot notation is hard to read. E.g. 
        \aleq{
            \vvec{v} &= \dv{x}{t}\hhat{x} + \dv{y}{t}\hhat{y} \quad \Rightarrow \quad v_x = \dv{x}{t} \, ,\, v_y = \dv{y}{t}
        }
        %
    \end{enumerate}
\end{notation}


% content begins here
% Section %%%%%%%%%%%%%%%%%%%%%%%%%%%%%%%%%%%%%%%%%%%%%%%%%%%%
\section{The Vector Expressions}

\subsection{Unit Vectors as function of coordinate}

In x-y coordinate, every vector can be expressed in terms of the two unit vectors $\{\hhat{x}, \hhat{y}\}$ and their components.
\aleq{
    \vvec{s} = s_x \hhat{x} + s_y \hhat{y}
}

When switching into polar coordinate, we wish to do the same, but with two different unit vectors $\{\hhat{r}, \hhat{\theta}\}$.
\aleq{
    \vvec{s} = s_r \hhat{r} + s_\theta \hhat{y}
}

The problem about $\{\hhat{r}, \hhat{\theta}\}$ is that their directions depends on the coordinate $(r, \theta)$. We require:
\begin{itemize}
    \item $\hhat{r}$ should always be radially outward, i.e. extend from the origin to the current point.
    \item $\hhat{\theta}$ should always be perpendicular to $\hhat{r}$, like a anti-clockwise torque. 
\end{itemize}

\insertFig{Compare two pair of unit vector in xy and polar respectively -> xy same, polar different}

From the figure, we can see that the pairs of unit vectors at different positions are pointing in different directions. 
Notation-wise we should write them as:
\aleq{
    \hhat{r}_\red{\text{at }(r_1, \theta_1)} \neq \hhat{r}_\red{\text{at }(r_2, \theta_2)} 
    \quad, \quad
    \hhat{\theta}_\red{\text{at }(r_1, \theta_1)} \neq \hhat{\theta}_\red{\text{at }(r_2, \theta_2)} 
}

This makes a big difference in differentiation, because the unit vectors are essentially functions of the coordinates. Compare
\begin{itemize}
    \item Vector in terms of $\{\hhat{x}, \hhat{y}\}$
    \aleq{
        \vvec{s} &= s_x \hhat{x} + s_y \hhat{y} \\
        \dv{\vvec{s}}{t} &= \dv{s_x}{t}\hhat{x} + s_x\tikzmark{xh}\red{\underline{\dv{\hhat{x}}{t}}} 
                            + \dv{s_y}{t}\hhat{y} + s_y\tikzmark{yh}\red{\underline{\dv{\hhat{y}}{t}}}
    }\\
    \begin{tikzpicture}[remember picture,overlay]
        \draw[arrows=->, draw=red]    ($ (pic cs:xh) + (8pt, -4ex) $) -- ( $ (pic cs:xh) +(8pt,-2.5ex) $ );
        \node[text=red] at ( $ (pic cs:xh) + (10pt,-4.8ex) $ )  {$\scriptstyle =0$};
        %
        \draw[arrows=->, draw=red]    ($ (pic cs:yh) + (8pt, -4ex) $) -- ( $ (pic cs:yh) +(8pt,-2.5ex) $ );
        \node[text=red] at ( $ (pic cs:yh) + (10pt,-4.8ex) $ )  {$\scriptstyle =0$};
    \end{tikzpicture}
    %
    \red{$\hhat{x},\hhat{y}$ always point in the same direction. Never change by position $\Rightarrow$ Differentiation = 0.}
    %
    \item Vector in terms of $\{\hhat{r}, \hhat{\theta}\}$
    \aleq{
        \vvec{s} &= s_r\hhat{r} + s_\theta\hhat{\theta} \\
        \dv{\vvec{s}}{t} &= \dv{s_r}{t}\hhat{r} + s_r \tikzmark{rh}\blue{\underline{\dv{\hhat{r}}{t}}}
                            + \dv{s_\theta}{t}\hhat{\theta} + s_\theta \tikzmark{th}\blue{\underline{\dv{\hhat{\theta}}{t}}}
    }\\[1em]
    \begin{tikzpicture}[remember picture,overlay]
        \draw[arrows=->, draw=RoyalBlue]    ($ (pic cs:rh) + (8pt, -4ex) $) -- ( $ (pic cs:rh) +(8pt,-2.5ex) $ );
        \node[text=RoyalBlue] at ( $ (pic cs:rh) + (10pt,-4.8ex) $ )  {$\substack{\hhat{r}\text{ contains } \\ r\text{ \& }\theta}$};
        %
        \draw[arrows=->, draw=RoyalBlue]    ($ (pic cs:th) + (8pt, -4ex) $) -- ( $ (pic cs:th) +(8pt,-2.5ex) $ );
        \node[text=RoyalBlue] at ( $ (pic cs:th) + (10pt,-4.8ex) $ )  {$\substack{\hhat{\theta}\text{ contains } \\ r\text{ \& }\theta}$};
    \end{tikzpicture}
    %
    \blue{$\hhat{r}, \hhat{\theta}$ are functions of $r$ and $\theta$, and $(r, \theta)$ are functions of $t$. So differentiating against $t$ is non-zero in general.}
\end{itemize}


\textbf{Note 1}: The above are just product rules, but applied on (component)$\times$(unit vector).

\textbf{Note 2}: Unit vectors are functions of coordinate for any non-rectangular coordinate.

\insertFig{curve space unit vectors}



%%%%%%%%%%%%%%%%%%%
\subsection{Differentiation on Polar Unit Vectors}

Because $\{\hhat{x}, \hhat{y}\}$ never changes by position, we usually call a rectangular coordinate as an \textbf{"ambient coordinate"},
and use them as a reference to express other unit vectors. 
To tell how to differentiate the polar unit vectors, we can first express them in the $\bigcirc\hhat{x} + \square\hhat{y}$ form. Then differentiation is solely on the components.

\insertFig{polar coor -> xy coor}

The relations between $\{\hhat{r}, \hhat{\theta}\}$ and $\{\hhat{x}, \hhat{y}\}$ are purely trigonometric:
\aleq{
    \begin{cases}
        \hhat{r}_\green{\text{at }(r,\theta)} = \green{\cos\theta} \cdot\hhat{x} + \green{\sin\theta}\cdot\hhat{y} \\
        \hhat{\theta}_\green{\text{at }(r,\theta)} = -\green{\sin\theta}\cdot\hhat{x} + \green{\cos\theta}\cdot\hhat{y}
    \end{cases}
}

Obviously $\hhat{r}$ and $\hhat{\theta}$ are function to $\theta$ only. So
\aleq{
    \Aboxed{\pdv{\hhat{r}}{r} = \pdv{\hhat{\theta}}{r} = 0}
}

and differentiating on $\theta$ gives
\aleq{
    \Aboxed{
    \begin{cases}
        \red{\pdvv{\hhat{r}}{\theta}} = -\sin\theta \hhat{x} + \cos\theta \hhat{y} = \red{\hhat{\theta}_{\text{at }(r,\theta)}} \\[1em]
        \red{\pdvv{\hhat{\theta}}{\theta}} = -\cos\theta \hhat{x} - \sin\theta \hhat{y} = \red{-\hhat{r}_{\text{at }(r,\theta)}}
    \end{cases}
    }
}

The two unit vectors give each other upon differentiation by $\theta$ is purely a numerical coincident. 
(This is related to rotational symmetry.) Also worth mentioning, the time differentiation is simply applying the (partial-D) chain rule:
\aleq{
    \Aboxed{
    \begin{cases}
        \red{\dvv{\hhat{r}}{t}} = \cancelto{0}{\pdvv{\hhat{r}}{r}}\dvv{r}{t} + \pdvv{\hhat{r}}{\theta}\dvv{\theta}{t} = \red{\dvv{\theta}{t}\cdot\hhat{\theta}_{\text{at }(r,\theta)}} \\[1em]
        \red{\dvv{\hhat{\theta}}{t}} = \cancelto{0}{\pdvv{\hhat{\theta}}{r}}\dvv{r}{t} + \pdvv{\hhat{\theta}}{\theta}\dvv{\theta}{t} = \red{-\dvv{\theta}{t}\cdot\hhat{r}_{\text{at }(r,\theta)}}
    \end{cases}
    }
}





\linesep
% Section %%%%%%%%%%%%%%%%%%%%%%%%%%%%%%%%%%%%%%%%%%%%%%%%%%%%
\section{Kinematic Quantities in terms of $(r,\theta)$}

\subsection{Position Vector}


A position vector in the $\bigcirc\hhat{x} + \square\hhat{y}$ form has its components equal to the coordinate it is pointing to, i.e.
\aleq{
    \text{Pointing at coordinate }(X,Y) \,\Leftrightarrow\, \text{ Expression }=X\hhat{x} + Y\hhat{y} 
}

This is obviously not true for vector expressed in other coordinates, e.g. \red{You cannot write a vector pointing at polar coordinate $(R,\Theta)$ as $R\hhat{r}+\Theta\hhat{\theta}$.}
For proper conversion, we must first use the conversion between the unit vectors. Observed their relations can be written as a matrix:
\aleq{
    \bmat{
        \hhat{r} \\ \hhat{\theta}
    }
    =
    \bmat{
        \cos\theta & \sin\theta \\ -\sin\theta & \cos\theta
    }
    \bmat{
        \hhat{x} \\ \hhat{y}
    }
}

This square matrix $\bmat{\cos\theta & \sin\theta \\ -\sin\theta & \cos\theta}$ is known as the "rotation matrix", which when multiplied to a vector, will geometrically rotate the vector about the origin by an angle $\theta$. 
Its inverse is trivial enough - by replacing $\theta$ to $-\theta$. One can easily check that:
\aleq{
    \bmat{
        \cos\theta & -\sin\theta \\ \sin\theta & \cos\theta
    }
    \bmat{
        \cos\theta & \sin\theta \\ -\sin\theta & \cos\theta
    }
    =\bmat{
        1 & 0 \\ 0 & 1
    }
}

Substitute this into a position vector tells us how to write a position vector properly by its polar coordinate $(r, \theta)$ and unit vectors $\{\hhat{r}, \hhat{\theta}\}$.
\aleq{
    \vvec{r} &= x\hhat{x} + y\hhat{y} \\
    &=
    \bmat{
        x & y
    }
    \bmat{
        \hhat{x} \\ \hhat{y}
    }\\
    &=
    \bmat{
        x & y
    }
    \bmat{
        \cos\theta & -\sin\theta \\ \sin\theta & \cos\theta
    }
    \bmat{
        \cos\theta & \sin\theta \\ -\sin\theta & \cos\theta
    }
    \bmat{
        \hhat{x} \\ \hhat{y}
    }\\
    &= 
    \bmat{
        x & y
    }
    \bmat{
        \cos\theta & \sin\theta \\ \sin\theta & \cos\theta
    }
    \bmat{
        \hhat{r} \\ \hhat{\theta}
    }\\
    &=
    \bmat{
        x\cos\theta+y\sin\theta & -x\sin\theta + y\cos\theta
    }
    \bmat{
        \hhat{r} \\ \hhat{\theta}
    }\\
    &=
    \bmat{
        (r\cos\theta)\cos\theta + (r\sin\theta)\sin\theta & -(r\cos\theta)\sin\theta + (r\sin\theta)\cos\theta
    }
    \bmat{
        \hhat{r} \\ \hhat{\theta}
    }\\
    &=
    \bmat{
        r & 0
    }
    \bmat{
        \hhat{r} \\ \hhat{\theta}
    }\\
    \Aboxed{
        &= r\hhat{r}
    }
}

So \red{the proper way to write a vector in terms of its polar coordinate is ... just by its radial component.} Isn't that weird?
\\
\\
Normally when we describe a point on a 2D plane, we need two information, e.g. its $x$ and $y$ components.
Why is there only 1 information ($r$ component) in the polar form? This is because the second piece of information is hidden in the unit vector $\hhat{r}$:
\aleq{
    \vvec{r} = r \tikzmark{rhat}\underline{\hhat{r}} 
    \quad\overset{\substack{\text{more} \\ \text{accurately}}}{\Rightarrow}\quad 
    \vvec{r} = r\cdot \hhat{r}_{\tikzmark{rhat2}\green{\text{at }(r,\theta)}}
}\\
\begin{tikzpicture}[remember picture,overlay]
    \draw[arrows=->]    ($ (pic cs:rhat) + (4pt, -2.5ex) $) -- ( $ (pic cs:rhat) +(4pt,-1ex) $ );
    \node[anchor=east] at ( $ (pic cs:rhat) + (15pt,-4ex) $ )  {\scriptsize which direction?};
    %
    \draw[arrows=->, draw=YellowGreen]    ($ (pic cs:rhat2) + (22pt, -2.5ex) $) -- ( $ (pic cs:rhat2) +(22pt,-1ex) $ );
    \node[anchor=west, text=YellowGreen] at ( $ (pic cs:rhat2) + (-5pt,-4ex) $ )  {\scriptsize in $\theta$ direction!};
\end{tikzpicture}

We cannot tell which direction $\hhat{r}$ is pointing to, if we do not know the "$\green{\text{at }(r,\theta)}$" part. (Sadly, most text will omit writing this part.)
Unlike x-y coordinate, we do not care the two unit vectors much because we always know that $\hhat{x}$ is the one pointing horizontally and $\hhat{y}$ is the one pointing vertically.
\aleq{
    \vvec{r} = x\cdot \tikzmark{xhat}\underline{\hhat{x}} + y\cdot \tikzmark{yhat}\underline{\hhat{y}}
}
\begin{tikzpicture}[remember picture,overlay]
    \draw[arrows=->]    ($ (pic cs:xhat) + (4pt, -2.5ex) $) -- ( $ (pic cs:xhat) +(4pt,-1ex) $ );
    \node[anchor=east] at ( $ (pic cs:xhat) + (15pt,-4ex) $ )  {\scriptsize always horiztonal};
    %
    \draw[arrows=->]    ($ (pic cs:yhat) + (4pt, -2.5ex) $) -- ( $ (pic cs:yhat) +(4pt,-1ex) $ );
    \node[anchor=west] at ( $ (pic cs:yhat) + (-5pt,-4ex) $ )  {\scriptsize always vertical};
\end{tikzpicture}

\insertFig{same expression $r\hhat{r}$, all in different direction, because $\theta$ is not labelled}
\insertFig{x, y always horizontal/vertical, no matter anywhere}


%%%%%%%%%%%%%%%%%%%%%%%%%%%%%%%%%%%%%%%

\subsection{Displacement Vector}

Displacement vector is the subtraction between two position vector. In x-y coordinate, the subtraction is simply done within the components:
\aleq{
    \vvec{r}_2 - \vvec{r}_1 = (x_2 \hhat{x} + y_2\hhat{y}) - (x_1\hhat{x} + y_1\hhat{y}) = [x_2 - x_1]\hhat{x} + [y_2 - y_1]\hhat{y} 
}

But in polar form, you cannot!
\aleq{
    \vvec{r}_2 - \vvec{r}_1 = (r_2\hhat{r}) - (r_1\hhat{r}) \neq [r_2-r_1]\hhat{r}
}

This is the fault of omitting the "$\green{\text{at }(r,\theta)}$" part. Remember, $\hhat{r}$ at different position are differnt vectors. 
The true component subtraction will require a lot of trigonometry. 
\aleq{
    \vvec{r}_2 - \vvec{r}_1 = (r_2\cdot \hhat{r}_\green{\text{at }(r_2,\theta_2)}) - (r_1 \cdot \hhat{r}_\green{\text{at }(r_1,\theta_1)}) = (\text{something nasty!})
}

\insertFig{polar vector subtraction require nasty geometry}

Furthermore, if we really try to subtract by component, the result "vector" on the polar grid is not even a straightline. (This happens in every curved coordinate system.)

\insertFig{curved vector}

The only valid definition is the \textbf{infinitestimal displacement vector}, i.e. when we subtract two very close vector, 
their difference is approximately a straight line and we can take limit to its length to $0$. 
\aleq{
    \dd{\vvec{r}} &= \dd{\qty(r\hhat{r}_\green{\text{at }(r,\theta)})} \\
    &= \dd{(r)}\cdot \hhat{r}_\green{\text{at }(r,\theta)} + r\cdot \dd\qty(\hhat{r}_\green{\text{at }(r,\theta)})\\
    \Aboxed{
        &= \dd{(r)}\cdot \hhat{r}_\green{\text{at }(r,\theta)} + r\cdot \dd{(\theta)} \cdot \hhat{\theta}_\green{\text{at }(r,\theta)}
    }
}

\insertFig{components in displacement vector}

The differential $\dd{(\hhat{r}_\green{\text{at }(r,\theta)})}$ comes from $\dv{\hhat{r}}{\theta} = \hhat{\theta}$ which is derived previously. 
In textbooks you will usually find the form without "$\green{\text{at }(r,\theta)}$", which looks like
\aleq{
    \dd{\vvec{r}} = (\dd{r})\hhat{r} + (r\dd{\theta})\hhat{\theta}
}

It is normal if you find reading disability and are confused with all the $r, \vvec{r}$ and $\hhat{r}$. 


%%%%%%%%%%%%%%%%%%%%%%%%%%%%%%%%%%
\subsection{Velocity Vector}

Velocity is defined by displacement divided by period of time $\Delta t$, and then taking $\Delta t \to 0$.
From the infinitestimal displacement, we immediately get
\aleq{
    \vvec{v} &= \lim_{\Delta t \to 0} \frac{\vvec{r}(t)}{\Delta t} \\
    &= \dv{\vvec{r}(t)}{t} \\
    \Aboxed{
        &= \dv{r}{t}\cdot \hhat{r}_\green{\text{at }(r,\theta)} + r\cdot \dv{\theta}{t} \cdot \hhat{\theta}_\green{\text{at }(r,\theta)}
    }
}

We can identify the two components of a velocity vector. Notation-wise they may also be written as
\aleq{
    \vvec{v} = \qty(\dv{r}{t}) \hhat{r} + \qty(r\dv{\theta}{t}) \hhat{\theta} = v_r\hhat{r} + r\omega \hhat{\theta} = v_r \hhat{r} + v_\theta \hhat{\theta}
}
\begin{itemize}
    \item Radial component - $\displaystyle v_r = \dv{r}{t}$ 
    \item Angular component - $\displaystyle v_\theta = r\omega = r\dv{\theta}{t}$
\end{itemize}

\insertFig{components in velocity vector}

%%%%%%%%%%%%%%%%%%%%%%%%%%%%%%%%%%
\subsection{Acceleration Vector}

Acceleration is defined by differentiating the velocity once again. 
But with $\hhat{r}$ and $\hhat{\theta}$ being function of $t$ as well, the full expansion requires lengthy product rule.
\aleq{
    \vvec{a}(t) &= \dv{\vvec{v}(t)}{t} \\[0.5em]
    &= \dv{t}\qty(\dv{r}{t}\cdot \hhat{r}_\green{\text{at }(r,\theta)} + r\cdot \dv{\theta}{t} \cdot \hhat{\theta}_\green{\text{at }(r,\theta)}) \\[0.5em]
    %
    &= \dv[2]{r}{t}\cdot\hhat{r}_\green{\text{at }(r,\theta)} + \dv{r}{t}\cdot \dv{t}\qty(\hhat{r}_\green{\text{at }(r,\theta)}) 
    + \dv{r}{t}\cdot\dv{\theta}{t} \cdot \hhat{\theta}_\green{\text{at }(r,\theta)} + r\cdot \dv[2]{\theta}{t}\cdot \hhat{\theta}_\green{\text{at }(r,\theta)} 
    + r\cdot \dv{\theta}{t} \cdot\dv{t}\qty(\hhat{\theta}_\green{\text{at }(r,\theta)}) \\[0.5em]
    %
    &= \dv[2]{r}{t}\cdot\hhat{r}_\green{\text{at }(r,\theta)} + \dv{r}{t}\cdot\dv{\theta}{t}\cdot\hhat{\theta}_\green{\text{at }(r,\theta)}
    + \dv{r}{t}\cdot\dv{\theta}{t} \cdot \hhat{\theta}_\green{\text{at }(r,\theta)} + r\cdot \dv[2]{\theta}{t}\cdot \hhat{\theta}_\green{\text{at }(r,\theta)} 
    + r\cdot \dv{\theta}{t} \cdot\dv{\theta}{t}\cdot \hhat{r}_\green{\text{at }(r,\theta)} \\[0.5em]
    %
    &= \qty[\dv[2]{r}{t} - r\cdot \qty(\dv{\theta}{t})^2]\hhat{r}_\green{\text{at }(r,\theta)}
    + \qty[2\dv{r}{t}\cdot\dv{\theta}{t} + r\cdot \dv[2]{\theta}{t}]\hhat{\theta}_\green{\text{at }(r,\theta)}
} 

There are 4 terms in total. Notation-wise we commonly identify them as:
\begin{itemize}
    \item (Along $\hhat{r}$) \textbf{Radial acceleration} - $\displaystyle \dv[2]{r}{t}$
    %
    \item (Along $\hhat{r}$) \textbf{Centripetal acceleration} - $\displaystyle - r\cdot \qty(\dv{\theta}{t})^2 = -r\cdot \omega^2$\\
    \blue{Minus sign for pointing toward origin.}
    %
    \item (Along $\hhat{\theta}$) \textbf{Coriolis acceleration} - $\displaystyle 2\dv{r}{t}\cdot\dv{\theta}{t} = 2 v_r\cdot \omega$\\
    \blue{This term appears only if radial distance is changing, i.e. $\dv{r}{t}\neq 0$.}
    %
    \item (Along $\hhat{\theta}$) \textbf{Euler acceleration} - $\displaystyle r\cdot \dv[2]{\theta}{t} = r\alpha$\\
    \blue{This term appears only if the angular component is accelerating, i.e. $\dv[2]{\theta}{t} \neq 0$.}
\end{itemize}

They are also the 4 acceleration terms corresponding to the pseudo-forces in a rotating reference frame. 



%%%%%%%%%%%%%%%%%%%%%%%%%%%%%%%%%%
\subsection{The Angular Quantities}

Observe that if we perform cross product to a vector with its position vector, we essentially \red{remove its radial component and only the angular compoenet is left}. Because
\aleq{
    \hhat{r}_\green{\text{at }(r,\theta)} \cross \hhat{r}_\green{\text{at }(r,\theta)} &= 0 \\
    \hhat{r}_\green{\text{at }(r,\theta)} \cross \hhat{\theta}_\green{\text{at }(r,\theta)} & = \hhat{z} \,\leftarrow\, \substack{\scriptstyle \hhat{z} \text{ is independent}\\ \text{of position too}}
}
So for a general vector $\vvec{s}$,
\aleq{
    \hhat{r}_\green{\text{at }(r,\theta)} \cross \vvec{s} 
    &= \hhat{r}_\green{\text{at }(r,\theta)} \cross (s_r \hhat{r}_\green{\text{at }(r,\theta)} + s_\theta \hhat{\theta}_\green{\text{at }(r,\theta)}) \\
    &= s_r(\hhat{r}_\green{\text{at }(r,\theta)} \cross \hhat{r}_\green{\text{at }(r,\theta)}) + s_\theta(\hhat{r}_\green{\text{at }(r,\theta)} \cross \hhat{\theta}_\green{\text{at }(r,\theta)})\\
    &= s_\theta \hhat{z}
}

Apply this on our displacement / velocity / acceleration vector will give the familiar definitions of angular displacement / angular velocity / angular acceleration.

\begin{enumerate}
    \item \textbf{Position vector:} It has no angular component. $\hhat{r}_\green{\text{at }(r,\theta)}\cross\vvec{r} = r(\hhat{r}_\green{\text{at }(r,\theta)}\cross\hhat{r}_\green{\text{at }(r,\theta)})=0$.
    %
    \item \textbf{Infinitestimal displacement vector:}
    \aleq{
        \hhat{r}_\green{\text{at }(r,\theta)}\cross\dd{\vvec{r}} &= \cancelto{0}{(\dd{r})\hhat{r}_\green{\text{at }(r,\theta)}\cross\hhat{r}_\green{\text{at }(r,\theta)}} 
        + (r\dd{\theta})\hhat{r}_\green{\text{at }(r,\theta)}\cross\hhat{\theta}_\green{\text{at }(r,\theta)} \\[0.5em]
        &= r\dd{\theta}\hhat{z}
    }
    The \red{infinitestimal angular displacement vector} is formally defined in $\hhat{z}$ direction:
    \aleq{
        \Aboxed{
            \dd{\vvec{\theta}} = (\dd{\theta})\hhat{z} = \frac{\hhat{r}_\green{\text{at }(r,\theta)}\cross\dd{\vvec{r}}}{r} = \frac{\text{Angular component of }\dd{\vvec{r}}}{r}\hhat{z}
        }
    }
    %
    %Its magnitude is simply $\displaystyle \dd{\theta} = \frac{\abs{\dd{\vvec{r}}}}{r} = \frac{\text{arc length}}{\text{radius}}$.
    %
    \item \textbf{Velocity vector:}
    \aleq{
        \hhat{r}_\green{\text{at }(r,\theta)}\cross\vvec{v} &= \cancelto{0}{\dv{r}{t}\cdot \hhat{r}_\green{\text{at }(r,\theta)}\cross\hhat{r}_\green{\text{at }(r,\theta)}} 
        + r\cdot \dv{\theta}{t} \hhat{r}_\green{\text{at }(r,\theta)}\cross \hhat{\theta}_\green{\text{at }(r,\theta)}\\[0.5em]
        &= r\dv{\theta}{t}\hhat{z}
    }
    The \red{angular velocity vector} is formally defined in $\hhat{z}$ direction:
    \aleq{
        \Aboxed{
            \vvec{\omega} = \omega \hhat{z} = \dv{\theta}{t}\hhat{z} = \frac{\hhat{r}_\green{\text{at }(r,\theta)}\cross\vvec{v}}{r} =\frac{\text{Angular component of }\vvec{v}}{r}\hhat{z}
        }
    }
    %Its magnitude is our familiar definition $\displaystyle \omega = \frac{\abs{\vvec{v}}}{r}$.
    %
    \item \textbf{Acceleration vector:}
    \aleq{
        \hhat{r}_\green{\text{at }(r,\theta)}\cross\vvec{a} &= 
        \cancelto{0}{\qty[\dv[2]{r}{t} - r\cdot \qty(\dv{\theta}{t})^2]\hhat{r}_\green{\text{at }(r,\theta)}\cross\hhat{r}_\green{\text{at }(r,\theta)}}
        + \qty[2\dv{r}{t}\cdot\dv{\theta}{t} + r\cdot \dv[2]{\theta}{t}]\hhat{r}_\green{\text{at }(r,\theta)}\cross\hhat{\theta}_\green{\text{at }(r,\theta)}\\[0.5em]
        &= \qty[2\dv{r}{t}\cdot\dv{\theta}{t} + r\cdot \dv[2]{\theta}{t}]\hhat{z}
    }
    Different from the above, the angular acceleration is defined as the \nth{2} derivative to the angular displacement and ignore the Coriolis term. Therefore its definition is uglier.
    \aleq{
        \Aboxed{
            \vvec{\alpha} = \alpha\hhat{z} = \dv[2]{\theta}{t}\hhat{z} = \inv{r}\qty(\hhat{r}_\green{\text{at }(r,\theta)}\cross\vvec{a} - 2\dv{r}{t}\cdot\dv{\theta}{t}\hhat{z})
        }
    }
    %Its magnitude is $\displaystyle \alpha = \inv{r}\qty(\abs{\vvec{a}} - 2v_r\cdot\omega)$. 
    %When the radial distance is fixed, the Coriolis term vanishes and it recovers a prettier form $\displaystyle \alpha = \frac{\abs{\vvec{a}}}{r}$.
\end{enumerate}



%%%%%%%%%%%%%%%%%%%%%%%%%%%%%
\linesep
\section{Relative Angular Velocity?}

\begin{center}
    \red{Relative angular velocity makes sense only when two observers share the same origin.}
\end{center}

As in the above derivations, the angular velocity is only about one of the two components in a velocity vector. 
It is forbidden to do vector addition just by one component. 
The only exception is that when the coordinate systems have the same origin.
To do it appropriately, we must also consider the radial component and the unit vectors. 
But then the vector addition will become a "mess" of trigonometry.

\insertFig{relative angular velocity addition is mostly useless}


%%%
\end{document}