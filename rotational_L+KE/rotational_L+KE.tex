\documentclass[class=article, crop=false, 12pt]{standalone}
\usepackage[subpreambles=true]{standalone}
\usepackage{../common/packages}
\usepackage{../common/command}
\usepackage{../common/format}


\author{Tony Shing}
%\pretitle{Supplementary}

\topic{Note 6B - Mechanics}
\title{Angular Momentum and Rotational KE}

\version{2025} % leave blank for omitting

\begin{document}

\maketitle

%\heading{Lecture}{Tony}

\begin{overview}
    \begin{itemize}
        \item Review: Center of Mass
        \item Mathematical Origin of Angular Momemtum and Moment of Inertia, Parallel Axis Theorem
        \item Rotational KE
    \end{itemize}
\end{overview}


\begin{notation}
    \begin{enumerate}
        %
        \item For clear viewing, all vector (with arrow $\vec{}$ or hat $\hat{}$ ) will also be in \textbf{bold font}.
        %
        \item Notations for physics quantities: Position vector = $\vvec{r}$, velocity vector = $\vvec{v}$, acceleration vector = $\vvec{a}$.
        % 
        \item Coordinates are represented by $(x,y)$ and $(r,\theta)$. Their conversion is according to the position vector's component: 
        \aleq{
            \vvec{r} = x\hhat{x}+y\hhat{y} = (r\cos{\theta})\hhat{x}+(r\sin{\theta})\hhat{y} = r\hhat{r}
        }
        Note that $\vvec{r}$ is the \red{\textbf{position vector}} and $r$ is the \red{\textbf{radial component}}. 
        But in polar coordinate, \red{\textbf{length of position vector $\abs{\vvec{r}} = r$}} exactly.
        %
        \item All derivatives will be written in full form $\dv{}{t}$ for clarity, because dot notation is hard to read. E.g. 
        \aleq{
            \vvec{v} &= \dv{x}{t}\hhat{x} + \dv{y}{t}\hhat{y} \quad \Rightarrow \quad v_x = \dv{x}{t} \, ,\, v_y = \dv{y}{t}
        }
        %
    \end{enumerate}
\end{notation}


% content begins here
% Section %%%%%%%%%%%%%%%%%%%%%%%%%%%%%%%%%%%%%%%%%%%%%%%%%%%%
\setcounter{section}{-1}
\section{Review: Center of Mass}

Given a set of point objects with individual masses $\{m_1, m_2, ..., m_n\}$ at positions $\{\vvec{r}_1, \vvec{r}_2, ..., \vvec{r}_n\}$,
the coordinate of center of mass of these objects can be calculate with the formula:
\aleq{
    \Aboxed{
        \vvec{r}_{CM} = \frac{m_1\vvec{r}_1 + m_2\vvec{r}_2 + ... + m_n\vvec{r}_n}{m_1+m_2+...+m_n} = \frac{\sum_i^n m_i \vvec{r}_i}{\sum_i^n m_i}
    }
}

In case of a continuous distribution of masses, it becomes an integration:
\aleq{
    \Aboxed{
        \vvec{r}_{CM} = \frac{\int \vvec{r}\dd{m}}{\int \dd{m}} \sim \frac{\int \vvec{r}\rho \dd{x}\dd{y}\dd{z}}{M_\text{total}}
    }
}
with $\rho$ being the density as a function of position.

\begin{proof}
The formula can be proven by mathematical induction. Starting with 2 masses $m_1$ and $m_2$ at positions $\vvec{r}_1$ and $\vvec{r}_2$ respectively,
the center of mass is "naïvely" defined as \red{the position where the net torque under gravity = 0}. 
Assume both masses are on the x axis to simplify our visualization:

\insertFig{two mass on x axis}

From the figure, the x coordinates of the masses are $x_1$ and $x_2$. Let the coordinate of the center of mass be $x_{CM}$.
Consider the torques with pivot at $x_{CM}$:
\aleq{
    \anticlockwise :\quad  m_1g(x_{CM}-x_1) &= m_2g(x_2-x_{CM}) \\
    (m_1+m_2)x_{CM} &= m_1x_1 + m_2x_2 \\
    x_{CM} &= \frac{m_1x_1+m_2x_2}{m_1+m_2}
}

Follow the logic of induction, assume the formula works for $k$ masses, i.e. 
\aleq{
    x_{CM}^{(k)} = \frac{m_1x_1+m_2x_2 +...+ m_kx_k}{m_1+m_2+...+m_k} 
}

The center of mass is where the net torque under gravity = 0 by the two objects \{Group of the first $k$ masses\} and the $(k+1)^\text{th}$ mass
\aleq{
    \anticlockwise :\quad  (m_1+m_2+...+m_k)g(x_{CM}^{(k+1)} - x_{CM}^{(k)}) &= m_{k+1}g(x_{k+1}-x_{CM}^{(k+1)}) \\
    (m_1+m_2+...+m_k+m_{k+1})x_{CM}^{(k+1)} &= (m_1+m_2+...+m_k)x_{CM}^{(k)} + m_{k+1}x_{k+1} \\
    &= (m_1x_1+m_2x_2+...+m_kx_k) + m_{k+1}x_{k+1} \\
    x_{CM}^{(k+1)} &= \frac{m_1x_1+m_2x_2+...+m_{k+1}x_{k+1}}{m_1+m_2+...+m_{k+1}}
}

So the formula also works for $k+1$ masses. 
Because the same operation can be done on $y$ and $z$ coordinates, we can promote this formula to the vector expression
\aleq{
    \vvec{r}_{CM} = \frac{m_1\vvec{r}_1+m_2\vvec{r}_2+...+m_n\vvec{r}_n}{m_1+m_2+...+m_n} 
}

\end{proof}

%%%%%%%%%%%%%%%%%%%%%%%%%%%%%%%%%%%%%%%%%%%%%%%%%%%%%%%%%%%%%%%%%%%%%%%%%%%%%%%%%%%
\linesep
\section{Torque, Angular Momentum \& Moment of Inertia}

%%%%%%
\subsection{Mathematical Origin}

Recall that if we take cross product to $\dd{\vvec{r}}/\vvec{v}/\vvec{a}$ with position vector, 
we can isolate their angular components $\dd{\theta}/\omega/\alpha$. Do the same to force $\vvec{F}$, 
we arrive at the definition of torque and angular momentum.
\aleq{
    \Aboxed{\vvec{\tau} &\defeq \vvec{r}\cross \vvec{F}} \\
    &= r\hhat{r} \cross m\vvec{a} \\
    &= rma_\theta \hhat{z} \\
    &=rm\qty[2\dv{r}{t}\cdot\dv{\theta}{t} + r\cdot \dv[2]{\theta}{t}]\hhat{z} &(\text{\scriptsize Corriolis and Euler terms})\\
    &= m\qty[\dv{r^2}{t}\cdot\dv{\theta}{t} + r^2\cdot \dv[2]{\theta}{t}]\hhat{z} &({\scriptstyle \text{ \nth{1} term: }2r\dd{r} \ \to\  \dd{(r^2)} })\\
    &= \dv{t}\qty(mr^2\dv{\theta}{t})\hhat{z} &(\text{\scriptsize product rule})
}

Comparing with relation between force and momentum, we can define a similar quantity that represents the angular component of momentum,
i.e. the angular momentum $L$,
\aleq{
    \vvec{F} = \dv{\vvec{p}}{t} = \dv{t}\qty(\text{momentum}) \qquad\text{v.s.}\qquad \vvec{\tau} = \dv{t}\qty(mr^2\dv{\theta}{t})\hhat{z} = \dv{t}\qty(\substack{\text{Something}\\\text{momentum-like}})
}
\aleq{
    \Rightarrow \qquad \Aboxed{\vvec{L} \defeq mr^2\dv{\theta}{t}\hhat{z}}
}

And since momentum is the product of an inertia quantity (mass) and velocity, 
we can propose that the angular momentum is also a product of some new inertial quantity and the velocity in rotation (angular velocity).
So we arrive at the definition of moment of inertia $I$.
\aleq{
    \vvec{p}=m\vvec{v} = (\text{inertia})\cdot(\text{velocity}) \qquad\text{v.s.}\qquad \vvec{L} = (mr^2)\cdot \dv{\theta}{t} = \qty(\substack{\text{something}\\\text{related to}\\\text{inertia}})\cdot (\substack{\text{Angular}\\\text{velocity}})
}
\aleq{
    \Rightarrow \qquad \Aboxed{I \defeq mr^2}
}

%%%%%%%%%%%%%%
\subsection{Rules of using Moment of Inertia}

Unlike mass, the inertia quantity in linear motion, moment of inertia is not a universal quantity.

\begin{enumerate}
    \item \textbf{Dependent on the choice of coordinate system}
    \insertFig{moment of inertia coordinate dependent}
    As illustrated, for observers from different coordinate system, because they may describe the same object with different position vectors,
    their observed angular momentum, angular velocity and moment of inertia can all be different. 
    \begin{table}[h!]
    \centering
        \begin{tabular}{ c c c }
        & Observed by A & Observed by B \\ \hline \hline
        $\vvec{\omega}$ & $\vvec{r}_A\cross\vvec{v}$ & $\vvec{r}_B\cross\vvec{v}$ \\ \hline
        $\vvec{L}$ & $m\vvec{r}_A\cross\vvec{v}$ & $m\vvec{r}_B\cross\vvec{v}$ \\ \hline
        $I$ & $m\abs{\vvec{r}_A}^2$ & $m\abs{\vvec{r}_B}^2$ 
        \end{tabular}
    \end{table}
    The only special case is when the observers share the same origin - then they will observe the same moment of inertia. 
    Thereforer \red{you should always fix an origin when doing rotation problems}.
    %
    \item \textbf{Not always well-defined in multi-body system}
    \insertFig{many body for I}
    Suppose we have many objects moving with their individual velocities.
    As we know that the net torque can be found by summing each object's contribution, so does the angular momentum. 
    \aleq{
        \vvec{\tau}_\text{total} = \sum_i \vvec{\tau}_i 
        \quad\Rightarrow\quad 
        \vvec{L}_\text{total} = \sum_i \qty(m_i \vvec{r}_i\cross\vvec{v}_i) = \sum_i \qty(m_i \abs{\vvec{r}_i}^2\vvec{\omega}_i)
    }
    In general, there is no way to take out the "$mr^2$" terms out of the summation sign, and thus the moment of inertial is not well-defined. 
    \red{The only exception is that when all objects move in the same angular velocity}. Then we can write it as 
    \aleq{
        \vvec{L}_\text{total} = \sum_i \qty(m_i \abs{\vvec{r}_i}^2\red{\vvec{\omega}}) = \qty(\sum_i m_i\abs{\vvec{r}_i}^2)\red{\vvec{\omega}} = I_\text{total}\ \red{\vvec{\omega}}
    }
    %
\end{enumerate}


%%%%%%%%%%%%%%%%%%%%%%%%%%%%%%
\subsection{Breaking down by Center of Mass}

In many scenerios, the objects are not rotating about a fixed point. E.g. 
\begin{itemize}
    \item Planetary motion: The Moon is rotating around the Earth, but the Earth is also moving around the sun.
    \item Rolling without slipping: The cylinder/sphere is rotating about its center, while its center is also moving.
\end{itemize}

In these cases, we should not choose the rotation center (the sun, the cylinder's CM, etc.) as the origin, 
because the values of angular momentum and moment of inertia depends on the choices of origin -- 
\red{Moving the origin will change the total angular momentum, making it difficult to apply angular momentum conservation.}
\\\\
A proper treatment is to separate the position vectors of the objects into two parts - 
the position vector of the center of mass $\vvec{r}_{CM}$, and the displacement vectors of each object relative to the center of mass $\vvec{R}_{i}$.
\aleq{
    \begin{cases}
        \vvec{r}_i &= \vvec{r}_{CM} + \vvec{R}_{i} \\
        \vvec{v}_i &= \vvec{v}_{CM} + \vvec{V}_{i} \qquad\qquad \qty(\substack{\text{\scriptsize so as the velocity}\\\text{\scriptsize by differentiating }\vvec{r}} )
    \end{cases}
}

\insertFig{r cm and R i}

Substitute them into the definition of angular momentum, we get
\aleq{
    \vvec{L} &= \sum_i (m_i \vvec{r}_i\cross \vvec{v}_i) \\
    &= \sum_i\qty[m_i(\vvec{r}_{CM}+\vvec{R}_i)\cross(\vvec{v}_{CM}+\vvec{V}_i)] \\
    &= \sum_i m_i\qty[\vvec{r}_{CM}\cross\vvec{v}_{CM} + \vvec{r}_{CM}\cross\vvec{V}_i + \vvec{R}_i\cross\vvec{v}_{CM} + \vvec{R}_i\cross\vvec{V}_i] \\
    &= \qty(\sum_i m_i)\qty[\vvec{r}_{CM}\cross\vvec{v}_{CM}] + \vvec{r}_{CM}\cross\tikzmark{L1}\cancelto{0}{\qty[\sum_i m_i\vvec{V}_i]}
        + \tikzmark{L2}\cancelto{0}{\qty[\sum_i m_i\vvec{R}_i]}\cross\vvec{v}_{CM} + \sum_i \qty[m_i \vvec{R}_i \cross\vvec{V}_i] \\[3em]
    &= \quad\tikzmark{L3}\red{\underline{\qty(\sum_i m_i)\qty[\vvec{r}_{CM}\cross\vvec{v}_{CM}]}} \quad + \quad \tikzmark{L4}\blue{\underline{\sum_i \qty[m_i \vvec{R}_i \cross\vvec{V}_i]}}\\[3.2em]
    &= \quad\cunderline[YellowGreen]{\tikzmark{L5}M_{total}\abs{\vvec{r}_{CM}}^2}\vvec{\omega_{CM}} \quad+\quad \sum_i \qty[\cunderline[orange]{\tikzmark{L6}m_i\abs{\vvec{R}_i}^2}\vvec{\Omega}_i] \\[3.2em]
    &= \quad \cunderline[YellowGreen]{\tikzmark{L7}I_{CM,O}}\vvec{\omega}_{CM} \quad+\quad \sum_i\qty[\cunderline[orange]{\tikzmark{L8}I_{i,CM}}\ \vvec{\Omega}_i]
}
\addArrow{L1}{(30pt,-3ex)}{(30pt, -4.8ex)}{(20pt, -7ex)}{$\substack{\text{Total velocity}\\\text{relative to CM = 0}}$}
\addArrow{L2}{(32pt,-3ex)}{(32pt, -4.8ex)}{(45pt, -7ex)}{$\substack{\text{Total displacement}\\\text{different from CM = 0}}$}
\addArrow[red]{L3}{(60pt,-4ex)}{(60pt, -5.8ex)}{(60pt, -8ex)}{$\substack{\text{Angular momentum only about}\\\text{CM's motion around origin}}$}
\addArrow[RoyalBlue]{L4}{(45pt,-4ex)}{(45pt, -5.8ex)}{(45pt, -8ex)}{$\substack{\text{Angular momentum only about}\\\text{object's motion around CM}}$}
\addArrow[YellowGreen]{L5}{(30pt,-2ex)}{(30pt, -3.5ex)}{(35pt, -6ex)}{$\substack{\text{Moment of inertia of}\\\text{an object with mass }M_\text{total}\\\text{relative to origin}}$}
\addArrow[orange]{L6}{(22pt,-3ex)}{(22pt, -4.8ex)}{(27pt, -6ex)}{$\substack{\text{Moment of inertia of}\\\text{each object relative to CM}}$}
\addArrow[YellowGreen]{L7}{(15pt,2ex)}{(25pt, 5ex)}{(0, 0)}{}
\addArrow[orange]{L8}{(15pt,2ex)}{(45pt, 6ex)}{(0, 0)}{}


This tells us that we can write the system's angular momentum by checking the \underline{center of mass's motion} and \underline{individual objects's motion around the center of mass} individually,
which simplifies the calculation to angular momentum in many problems. 

\begin{description}
    \item[\underline{Note 1:}]~\\
    \red{We can only use the center of mass for such breakdown.} $\sum_i\qty(m_i\vvec{R}_i)$ and $\sum_i\qty(m_i\vvec{V}_i)$ vanish only because of the CM's property. 
    All 4 terms must be kept if an arbituary point is chosen (and the calculation cannot be simplified).
    %
    \item[\underline{Note 2:}]~\\
    We can prove $\sum_i\qty(m_i\vvec{R}_i) = 0$ using the CM formula. Differentiate to get $\sum_i\qty(m_i\vvec{V}_i) = 0$.
    \aleq{
        \sum_i\qty(m_i\vvec{R}_i) &= \sum_i\qty[m_i(\vvec{r}_i - \vvec{r}_{CM})] \\
        &= \sum_i\qty(m_i\vvec{r}_i) - \qty(\sum_i m_i)\vvec{r}_{CM} \\
        &= \sum_i\qty(m_i\vvec{r}_i) - \cancel{\qty(\sum_i m_i)}\qty(\frac{\sum_i m_i\vvec{r}_i}{\cancel{\sum_i m_i}})\\
        &= 0
    }
\end{description}


%%%%%%%%%%%%%%%%%%%%%%%%%%%%%
\subsection{Parallel Axis Theorem}

Another common scenerio is when all masses rotate at the same angular velocity $\vvec{\omega}_O$ about an arbituary but fixed point (not center of mass). E.g. Rigid body rotation.
In such case, we can always choose the rotation center as the origin. 

\insertFig{rigid body motion}

The angular momentum writes as:
\aleq{
    \vvec{L} &= \sum_i (m_i \vvec{r}_i\cross \vvec{v}_i) \\
    &= \sum_i(m_i\abs{\vvec{r}_i}^2\red{\vvec{\omega}_O}) \qquad \red{\leftarrow \substack{\text{All objects have}\\\text{the same angular velocity}\\\text{relative to }O}}\\
    &= \sum_i(m_i\abs{\vvec{r}_{CM}+\vvec{R}_i}^2\vvec{\omega}_O) \\
    &= \qty[\sum_i \qty(m_i \abs{\vvec{r}_{CM}}^2) + \sum_i\qty(m_i \abs{\vvec{R}_i}^2) + 2\cunderline[YellowGreen]{\tikzmark{L9}\cancel{\sum_i\qty(m_i \vvec{r}_{CM}\cdot\vvec{R}_i)}}]\vvec{\omega}_O \\
    &= \qty[\cunderline[red]{\tikzmark{L10}M_\text{total}\abs{\vvec{r}_{CM}}^2} + \cunderline[RoyalBlue]{\tikzmark{L11}\sum_i\qty(m_i \abs{\vvec{R}_i}^2)}]\vvec{\omega}_O\\[3.2em]
    &= \qty[\cunderline[red]{\tikzmark{L12}I_{CM,O}} + \cunderline[RoyalBlue]{\tikzmark{L13}I_{i,CM}}]\vvec{\omega}_O
}
\addArrow[YellowGreen]{L9}{(50pt,-4ex)}{(50pt, -5.8ex)}{(90pt, -8ex)}{$\textstyle =\vvec{r}_{CM}\cdot\qty(\sum_i m_i\vvec{R}_i) = \vvec{r}_{CM}\cdot 0$}
\addArrow[red]{L10}{(30pt,-2ex)}{(30pt, -4ex)}{(30pt, -6.5ex)}{$\substack{\text{Moment of inertia of}\\\text{an object with mass }M_\text{total}\\\text{relative to origin}}$}
\addArrow[RoyalBlue]{L11}{(45pt,-4ex)}{(45pt, -5.8ex)}{(45pt, -7.5ex)}{$\substack{\text{Moment of inertia}\\\text{relative to CM}}$}
\addArrow[red]{L12}{(15pt,2ex)}{(25pt, 5ex)}{(0, 0)}{}
\addArrow[RoyalBlue]{L13}{(15pt,2ex)}{(45pt, 6ex)}{(0, 0)}{}

This tells us a special case where two moment of inertia relative to different points can be directly - 
One of them must be chosen as the rotation center and the other must be the center of mass. 
\aleq{
    \Aboxed{I_\text{equiv} = I_{CM,O} + I_{i,CM}}
}

Similar to the previous, \red{we can only use the center of mass for such breakdown.}
The dot product term cannot be canceled if an arbituary point is chosen (and the calculation cannot be simplified).





%%%%%%%%%%%%%%%%%%%%%%%%%%%%%%%%%%%%%%%%%%%%%%%%%%%%%%%%%%%%%%%%%%%%%%%%%%%%%%%%%%%
\linesep
\section{Rotational KE}

%%%%%%
\subsection{Mathematical Origin}
There is always only 1 expression to kinetic energy --- $\half m\abs{\vvec{v}}^2$. 
It is only the matter of which coordinate system we used to expand $\vvec{v}$. 
With $\{x,y\}$ coordinate, it is in the familar form:
\aleq{
    \vvec{v} = \dv{x}{t}\hhat{x} + \dv{y}{t}\hhat{y} 
    \quad\Rightarrow\quad 
    \half m\abs{\vvec{v}}^2 = \half m\qty[\qty(\dv{x}{t})^2 + \qty(\dv{y}{t})^2]
    =\half m v_x^2 + \half m v_y^2
}
But with $\{r,\theta\}$ coordinate, it becomes:
\aleq{
    \vvec{v} = \dv{r}{t}\hhat{r} + r\dv{\theta}{t}\hhat{\theta} 
    \quad\Rightarrow\quad 
    \half m\abs{\vvec{v}}^2 = \half m\qty[\qty(\dv{r}{t})^2 + r^2\qty(\dv{\theta}{t})^2]
    = \half m v_r^2 + \half mr^2\omega^2
}

We can spot the familiar term $ \half mr^2\omega^2 = \half I\omega^2$ in textbook, which is in fact only 1 of the components in the total KE. 
We can write $\text{KE} =\half I\omega^2$ only if we are sure that the object is in pure rotation. 
Otherwise you should always write the KE from radial velocity too.

%%%%%%%
\subsection{Breaking down by Center of Mass}

Similar to angular momentum, we can do similar treatment on KE to isolate the center of mass's movement and individual objects' motions relative to center of mass.

\insertFig{same pic as in angular momentum?}

\aleq{
    \text{KE} &= \sum_i \qty(\half m_i \abs{\vvec{v}_i}^2) \\
    &= \sum_i \qty(\half m_i \abs{\vvec{v}_{CM}+\vvec{V}_i}^2)\\
    &= \sum_i \qty(\half m_i \abs{\vvec{v}_{CM}}^2) + \sum_i \qty(\half m_i \abs{\vvec{V}_i}^2) + \cunderline[YellowGreen]{\cancel{\tikzmark{K1}\sum_i\qty(m_i \vvec{v}_{CM}\cdot\vvec{V}_i)}} \\
    &= \cunderline[red]{\tikzmark{K2}\half M_\text{total}\abs{\vvec{v}_{CM}}^2} + \cunderline[RoyalBlue]{\tikzmark{K3}\sum_i \qty(\half m_i \abs{\vvec{V}_i}^2)}\\[3.2em]
    &= \half M_\text{total}\abs{\vvec{v}_{CM}}^2 + \sum_i \qty[\cunderline[red]{\tikzmark{K4}\half m_i\qty(\dv{R_i}{t})^2} + \cunderline[RoyalBlue]{\tikzmark{K5}\half m_i R_i^2\qty(\dv{\Theta_i}{t})^2}]\\
}
\addArrow[YellowGreen]{K1}{(50pt,-4ex)}{(50pt, -5.8ex)}{(90pt, -8ex)}{$\textstyle =\vvec{v}_{CM}\cdot\qty(\sum_i m_i\vvec{V}_i) = \vvec{v}_{CM}\cdot 0$}
\addArrow[red]{K2}{(30pt,-2.5ex)}{(30pt, -4.5ex)}{(30pt, -7ex)}{$\substack{\text{KE of a point object}\\\text{with mass }M_\text{total}}$}
\addArrow[RoyalBlue]{K3}{(45pt,-4ex)}{(45pt, -5.8ex)}{(45pt, -7.5ex)}{$\substack{\text{KE only about}\\\text{motion relative to CM}}$}
\addArrow[red]{K4}{(30pt,-2.8ex)}{(30pt, -4.2ex)}{(20pt, -5.5ex)}{\scriptsize Radial KE relative to CM}
\addArrow[RoyalBlue]{K5}{(30pt,-2.8ex)}{(30pt, -4.2ex)}{(50pt, -5.5ex)}{\scriptsize Angular KE relative to CM}

Similar to the previous, \red{we can only use the center of mass for such breakdown.}
The dot product term cannot be canceled if an arbituary point is chosen (and the calculation cannot be simplified).

%%%%%%%%%%%%%%%%%%%%%%%%%%%%%%%%%%%%%%%%%%%%%%%%%%%%%%%%%%%%%%%
\linesep
\section{Usage on angular momentum and rotation KE}

In practice, most of the problems will be either one of the two cases:
\begin{enumerate}
    \item Objects are rotating around a fixed center at the same angular velocity $\vvec{\omega}$.
    \insertFig{fix origin cases}
    Then we should:
    \begin{itemize}
        \item Choose the rotation center as the origin.
        \item Calculate moment of inertia $I = I_O = \sum_i m_i\abs{\vvec{r}_O}^2$ relative to the origin. 
        Parallel axis theorem is likely useful.
        \item Directly write 
        \aleq{
            \vvec{L} &= I_O\vvec{\omega} \\
            \text{KE} &=\half I_O\abs{\vvec{\omega}}^2
        }
    \end{itemize}
    %
    \item Objects are rotating around their CM, and the CM may or may not be moving.
    \insertFig{move around CM cases}
    Then we should:
    \begin{itemize}
        \item Choose any static point as the origin. 
        Furthermore, if we choose an origin such that the position vector of center of mass is perpendicular to its velocity vector, 
        the subsequent calculations may simplify more.
        \item Depends on how the objects are moving relative to CM:
        \begin{description}
            \item[Pure rotation around CM (e.g. Rolling):] No radial terms.            
            \aleq{
                \vvec{L} &= M_\text{total}\vvec{r}_{CM} \cross \vvec{v}_{CM} + I_{i,CM}\vvec{\omega}\\
                \text{KE} &= \half M_\text{total}\abs{\vvec{v}_{CM}}^2 + \half I_{i,CM}\abs{\vvec{\omega}}^2
            }
            \item[Otherwise:] Must individually write out the terms for each object.
            \aleq{
                \vvec{L} &= M_\text{total}\vvec{r}_{CM} \cross \vvec{v}_{CM} + \sum_i\qty(m_i \vvec{R}_i\cross\vvec{V}_i)\\
                \text{KE} &= \half M_\text{total}\abs{\vvec{v}_{CM}}^2 + \sum_i\qty(\half m_i \abs{\vvec{V}_i}^2) + \sum_i\qty(\half m_i \abs{\vvec{R}_i}^2\abs{\vvec{\omega}_i}^2)
            }
        \end{description}
    \end{itemize}
\end{enumerate}

\begin{center}
    \red{Final reminder: Once you choose the origin, never change it again.}
\end{center}

\end{document}