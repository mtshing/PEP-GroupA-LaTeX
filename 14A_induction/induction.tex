\documentclass[class=article, crop=false, 12pt]{standalone}
\usepackage[subpreambles=true]{standalone}
\usepackage{../.common/common}


\author{Tony Shing}
%\pretitle{Supplementary}

\topic{T12 (Electromagnetism)}
\title{Magnetic Induction}

\version{2025} % leave blank for omitting

\begin{document}

\maketitle


\begin{overview}
    \begin{itemize}
        \item Lorentz force
        \item Faraday's law, motional/transformer EMF \& Lenz's law 
    \end{itemize}
\end{overview}

\vskip 1em
In electromagnetism, 
theoretically every problem can be solved through a set of PDEs
called the \bf{Maxwell Equations}.\\[-2em]
\begin{center}
    \begin{minipage}{0.3\textwidth}
        \aleq{
            \div \vvec{E} &= \frac{\rho}{\epsilon_0}\\
            \tkn{maxwell_faraday}{\curl} \vvec{E} &= -\pdvv{\vvec{B}}{t}
        }
    \end{minipage}
    \hspace{0.05\textwidth}
    \begin{minipage}{0.3\textwidth}
        \aleq{
            \div \vvec{B} &= 0\\
            \curl \vvec{B} &= \mu_0 \vvec{J} + \mu_0\epsilon_0\pdvv{\vvec{E}}{t}
        }
    \end{minipage}
\end{center}
\addArrow[blue]{maxwell_faraday}{(-5ex,0)}{}{(-3ex,1ex)}

However, a \it{system of PDEs} is too complicated to be solved.
So we need to learn different "tricks" to avoid them,
which are enough for some simple scenarios.\\

Electrostatics only concerns the \nth{3} equation of the set - \cul[blue]{Faraday's law}. 



\linesep
% content begins here
% Section %%%%%%%%%%%%%%%%%%%%%%%%%%%%%%%%%%%%%%%%%%%%%%%%%%%%
\setcounter{section}{-1}
\section{A Brief History of Electromagnetism}

Electromagnetic induction is likely the most confusing topic in beginner E\&M.
In my opinion, taking reference of the history is helpful to unify the concepts you have learnt.

\begin{center}
    \begin{tabularx}{\textwidth}{
        >{\centering\arraybackslash}m{0.15\textwidth} 
        p{0.8\textwidth}
        }

        Year & \makecell[c]{Advancement} \\ 
        \hline
        Before 1500s & 
        Different electrostatics phenomena were known.
        But they were not unified or explained at all. \\
        %
        1600 &
        \makecell[tl]{
            \href{https://en.wikipedia.org/wiki/William_Gilbert_(physicist)}{William Gilbert}
            was the first person to use the word "electrical" to describe\\ electrostatics phenomena. 
            Also the first to propose that electrical effect is\\ due to flows of particles.
        }\\[1.5em]
        %
        1750 &
        \makecell[tl]{
            \href{https://en.wikipedia.org/wiki/Benjamin_Franklin}{Benjamin Franklin} 
            developed a one "fluid" theory of electricity, 
            and called\\ this fluid "charge".
        }\\[1.5em]
        %
        1784 &
        \makecell[tl]{
            \href{https://en.wikipedia.org/wiki/Charles-Augustin_de_Coulomb}{Charles-Augustin de Coulomb}
            experimentally showed that force between\\ charged objects $\propto \inv{r^2}$.
            \gray{(Coulomb's law $F = \inv{4\pi\epsilon_0}\frac{Qq}{r^2}$)}
        }\\[1.5em]
        %
        1800 &
        \makecell[tl]{
            \href{https://en.wikipedia.org/wiki/Alessandro_Volta}{Alessandro Volta}
            Made the first battery from electro-chemistry.\\
            \gray{(First time to have steady current.)}
        }\\[1.5em]
        %
    \end{tabularx}
\end{center}


\begin{center}
    \begin{tabularx}{\textwidth}{
        >{\centering\arraybackslash}m{0.15\textwidth} 
        p{0.8\textwidth}
        }
        
        1820 &
        \makecell[tl]{
            \href{https://en.wikipedia.org/wiki/Hans_Christian_\%C3\%98rsted}{Hans Christian Ørsted}
            discovered that current wire can deflect compress.\\
            \gray{(First time to relate electric and magnetic phenomena.)}
        }\\[2em]
        %
        1820 &
        \makecell[tl]{
            \href{https://en.wikipedia.org/wiki/Andr\%C3\%A9-Marie_Amp\%C3\%A8re}{André-Marie Ampère}
            formulated and verified the force between current wires.\\
            \gray{$(F = I\vvec{l}_1\cross \frac{\mu_0 I}{2\pi r}\vvec{l}_2)$}
        }\\[2em]
        %
        1831 &
        \makecell[tl]{
            \href{https://en.wikipedia.org/wiki/Michael_Faraday}{Michael Faraday}
            discovered magnetic induction. Experiments include:\\
            - Across iron core: Reading appears at the instant when switch is on/off.\\
            - Moving frame: Reading appears when wireframe moves, changes shape \\
            or when magnetic field change.\\
            \red{insertFig}
        }\\[1.5em]
        %
        1834 &
        \makecell[tl]{
            \href{https://en.wikipedia.org/wiki/Emil_Lenz}{Emil Lenz}
            Explained direction of induced current by energy conservation.\\
            \gray{(Lenz's Law)}
        }\\[1.5em]
        %
        1860 &
        \makecell[tl]{
            \href{https://en.wikipedia.org/wiki/James_Clerk_Maxwell}{James Clerk Maxwell}
            unified past discoveries into 20 equations, 
            and used\\ field description for the first time.\\
            \red{This was the first time $\vb{E}$ and $\vb{B}$ appeared in Physics.}\\
            \red{Before Maxwell, everything was described in terms of force.}\\ 
        }\\[4em]
        %
        1893 &
        \makecell[tl]{
            \href{https://en.wikipedia.org/wiki/Oliver_Heaviside}{Oliver Heaviside}
            combined Maxwell's 20 equations into 4, by vector calculus.\\
            \gray{(This is the version of Maxwell's equation we now know.)}
        }\\[1em]
        %
        1895 &
        \makecell[tl]{
            \href{https://en.wikipedia.org/wiki/Hendrik_Lorentz}{Hendrik Lorentz}
            derive the correct force on charges under both $\vvec{E}$ and $\vvec{B}$.\\
            \gray{(Lorentz force formula)}
        }\\[1em]

    \end{tabularx}
\end{center}


\linesep
% Section %%%%%%%%%%%%%%%%%%%%%%%%%%%%%%%%%%%%%%%%%%%%%%%%%%%%
\section{Motion under Lorentz Force}

Nowadays we know that Lorentz force is the fundamental explanation to magnetic induction.
But interestingly, it was formulated correctly only in the very late history of classical E\&M.
\aleq{
    \Aboxed{
        m\vvec{a} = \vvec{F} = q\vvec{E} + q\vvec{v}\cross\vvec{B}
    }
}

In general, $\vvec{E}$ and $\vvec{B}$ may be functions of time and position. 
For simplicity, we will only discuss \cul[red]{the special case when the fields are constant}.
By symmetry, we can assume that $\vvec{B}$ is only in $\hhat{z}$ direction, 
i.e. $\vvec{B} = B_z\hhat{z}$.
Then the Newton's \nth{2} law then writes as
\aleq{
    m\dvv{t}\bmat{v_x\\v_y\\v_z} = q\bmat{E_x\\E_y\\E_z} 
        + q\bdet{\hhat{x} & \hhat{y} & \hhat{z}\\ v_x & v_y & v_z \\ 0 & 0 & B_z}
}

which turns into a system of 3 \nth{1} order linear ODEs.
\aleq{
    \bcase{
        m\dvv{v_x}{t} &= qE_x + qv_yB_z\\
        m\dvv{v_y}{t} &= qE_y - qv_xB_z\\
        m\dvv{v_z}{t} &= qE_z
    }
}

This system is simple enough that we do not need to use any matrix methods.
\begin{itemize}
    \item \bf{\ul{z component}}\\
    Motion in $z$ direction is independent of $x$/$y$.
    Can be solved alone.
    \aleq{
        m\dvv{v_z}{t} = qE_z 
        \qquad\Rightarrow\qquad
        &v_z(t) = \frac{qE_z}{m}t + v_{z0}\\[1ex]
        &z(t) = \half\frac{qE_z}{m}t^2 
            + \tkn{init_vz}{\cul[blue]{v_{z0}}}t + \tkn{init_z}{\cul[blue]{z_0}}
    }
    \addArrow[blue]{init_vz}{(0,-4ex)}{\scriptsize Initial\\[-1ex]\scriptsize z velocity}{(0,-1ex)}{(-3ex,0.5ex)}
    \addArrow[blue]{init_z}{(0,-4ex)}{\scriptsize Initial\\[-1ex]\scriptsize z coordinate}{(0,-1ex)}{(3ex,0.5ex)}
    
    which is just a constant acceleration motion.

    \item \bf{\ul{x/y components}}\\
    They can be solved by differentiating one of them,
    then substitute into the other.
    \aleq{
        \dvv[2]{v_x}{t} &= \frac{qB_z}{m}\dvv{v_y}{t} \\[1ex]
        &= \frac{qB_z}{m}\qty(\frac{qE_y}{m}-\frac{qB_z}{m}v_x)\\[1ex]
        &= - \frac{q^2B_z^2}{m}\qty(v_x - \frac{E_y}{B_z})
    }

    which is once again the familiar ODE of SHM.
    \aleq{
        v_x(t) &= -C' \sin\qty(\frac{qB_z}{m}t + \phi) + \frac{E_y}{B_z}\\[1ex]
        \Rightarrow\quad x(t) &= \cus[red]{C'\qty(\frac{m}{qB_z})}{\text{constant}}\cos \qty(\frac{qB_z}{m}t + \phi) 
            + \frac{E_y}{B_z}t + x_0\\[1ex]
        &= \tkn{init_r}{\cul[blue]{C}} \cos \qty(\frac{qB_z}{m}t + \tkn{phase}{\cul[blue]{\phi}}) 
            + \frac{E_y}{B_z}t + \tkn{init_x}{\cul[blue]{x_0}}
    }
    \addArrow[red]{init_r}{(5ex,4ex)}{}{(1ex,2ex)}
    \addArrow[blue]{init_r}{(0,-4ex)}{\scriptsize Radius}{(0,-1ex)}
    \addArrow[blue]{phase}{(0,-4ex)}{\scriptsize Phase}{(0,-1ex)}
    \addArrow[blue]{init_x}{(0,-3ex)}{\scriptsize Initial\\[-1ex]\scriptsize x coordinate}{(0,-1ex)}{(0,-1ex)}

    \vskip 1em
    And subsitute back to equation of $v_y$ gives
    \aleq{
        v_y(t) &= \frac{m}{qB_z}\dvv{v_x}{t} - \frac{E_x}{B_z} \\[1ex]
        &= -C' \cos\qty(\frac{qB_z}{m}t + \phi) - \frac{E_x}{B_z}\\[1ex]
        \Rightarrow\quad y(t) &= -\cus[red]{C' \qty(\frac{m}{qB_z})}{\text{constant}} \sin \qty(\frac{qB_z}{m}t + \phi) 
            - \frac{E_x}{B_z}t + y_0 \\[1ex]
        &= -\tkn{init_r2}{\cul[blue]{C}} \sin \qty(\frac{qB_z}{m}t + \tkn{phase2}{\cul[blue]{\phi}}) 
            - \frac{E_y}{B_z}t + \tkn{init_y}{\cul[blue]{x_0}}
    }
    \addArrow[red]{init_r2}{(5ex,4ex)}{}{(1ex,2ex)}
    \addArrow[blue]{init_r2}{(0,-4ex)}{\scriptsize Radius}{(0,-1ex)}
    \addArrow[blue]{phase2}{(0,-4ex)}{\scriptsize Phase}{(0,-1ex)}
    \addArrow[blue]{init_y}{(0,-3ex)}{\scriptsize Initial\\[-1ex]\scriptsize y coordinate}{(0,-1ex)}{(0,-1ex)}

    \vskip 1em
    The result $x$-$y$ motion is a combination of circular motion + drifting.

    \insertFig{lorentz drift}

    \aleq{
        \bmat{x(t)\\y(t)} 
        = 
        \cus[red]{C\bmat{\cos \qty(\frac{qB_z}{m}t+\phi) \\ -\sin \qty(\frac{qB_z}{m}t+\phi)}}{\text{Circular motion}}
        \ +\ \cus[blue]{\inv{B_z}\bmat{E_y\\-E_x}t}{\text{Drifting}} \ +\ \bmat{x_0\\y_0}
    }

    \begin{enumerate}
        \item Note that the drifting direction is not intuitive: 
        \begin{itemize}
            \item With $E_x$ only, drifting direction = $-\hhat{y}$
            \item With $E_y$ only, drifting direction = $\hhat{x}$
        \end{itemize}
        In general, the drifting direction is along direction of $\vvec{E}\cross\vvec{B}$ 
        with a speed $\dfrac{\norm{\vvec{E}}}{\norm{\vvec{B}}}$
    
        \vskip 1em
        \item The rotation radius and speed maintain a constant ratio:
        \aleq{
            \text{Radius}=R \qquad\Rightarrow\qquad \text{Speed}=\frac{qB_z}{m}R
        }
        i.e. Angular velocity is fixed at $\omega = \dfrac{qB_z}{m}$,
        independent of the charge's initial velocity.\\ 
        \gray{(One can also derive the same result by $\dfrac{mv^2}{r} = qvB$)}\\

    \end{enumerate}
\end{itemize}






\linesep
% Section %%%%%%%%%%%%%%%%%%%%%%%%%%%%%%%%%%%%%%%%%%%%%%%%%%%%
\section{Magnetic Induction}


%%%%%%%%%%%%%%
\subsection{EMF - Force or Voltage?}

The term \bf{electromotive force} (EMF) was invented by Alessandro Volta in 1801,
for explaining observations in electrochemistry.\\

Metal electrodes in electrolyte 
\begin{itemize}
    \item[$\Rightarrow$] Current generates spontaneously.
    \item[$\Rightarrow$] There must be some kind of "force" pushing the current! 
\end{itemize}

\insertFig{electrolyte}

In early 1800s, scientists tended to describe things like mechanical system,
i.e. any motions of objects must be driven by some kind of "force".
So when there is a steady current, 
there must be a "force" that keep pushing the charges forward.
namely the "electromotive force". 

\begin{itemize}
    \item However, EMF took the unit of volt because 
    voltage measurement was the only way to quantify the magnitude of EMF.
    \item Later when Michael Faraday discovered the phenomena of magnetic induction,
    he used the same word to refer to the source of the induced current.
\end{itemize}


Today we still keep the name as a "force", 
even though we already know much better how current is generated in different sources.
Maybe because people are already used to refer the same term \red{"EMF" 
as the general name for ANY kind of current source}, 
rather than discriminating them by their origin of energy.

\insertFig{i dun care}

Below we only focus on the two kinds of EMF that relate to magnetic induction phenomena.
\begin{itemize}
    \item \bf{\ul{Motional EMF}} : Source of energy = Motion of the charges' "container".
    Can be explained via Lorentz force.

    \item \bf{\ul{Transformer EMF}} : Source of energy = Change in magnetic field.
    Explanation requires relativity.
\end{itemize}


%%%%%%%%%%%%%%
\subsection{Motional EMF}

We have already learnt that under Lorentz force, 
charges moves in a circle in a constant B-field.

\begin{enumerate}
    \item We can think one step forward -
    if a charge is contained inside a \it{massless} pipe.
    This massless pipe will be dragged along by the charge.
    If the pipe is restricted to move horizonally,
    it will undergo SHM!

    \insertFig{pip SHM}

    
    \item However if the pipe contains both kinds of charges
    (such that it becomes electrically neutral),
    opposite charges repel each other.
    


\end{enumerate}

%%%%%%%%%%%%%%
\subsubsection{A Moving Battery}

%%%%%%%%%%%%%%
\subsubsection{Generalization to Flux}

%%%%%%%%%%%%%%
\subsection{Transformer EMF}

%%%%%%%%%%%%%%
\subsection{Maxwell-Faraday Equation}

%%%%%%%%%%%%%%
\subsection{Lenz's Law}


\linesep
% Section %%%%%%%%%%%%%%%%%%%%%%%%%%%%%%%%%%%%%%%%%%%%%%%%%%%%
\section{Basic Problems in Magnetic Induction}


%%%
\theend
\end{document}