\documentclass[class=article, crop=false, 12pt]{standalone}
\usepackage[subpreambles=true]{standalone}
\usepackage{../.common/common}


\author{Tony Shing}
%\pretitle{Supplementary}

\topic{T14A (Electromagnetism)}
\title{Magnetic Induction}

\version{2025} % leave blank for omitting

\begin{document}

\maketitle


\begin{overview}
    \begin{itemize}
        \item Motion under Lorentz force
        \item Faraday's law: Motional \& transformer EMF
        \item Lenz's law: Direction of induced EMF
        \item Appendix: A brief history of electromagnetism
    \end{itemize}
\end{overview}

\vskip 1em
In electromagnetism, 
theoretically every problem can be solved through a set of PDEs
called the \bf{Maxwell Equations}.\\[-2em]
\begin{center}
    \begin{minipage}{0.3\textwidth}
        \aleq{
            \div \vvec{E} &= \frac{\rho}{\epsilon_0}\\
            \tkn{maxwell_faraday}{\curl} \vvec{E} &= -\pdvv{\vvec{B}}{t}
        }
    \end{minipage}
    \hspace{0.05\textwidth}
    \begin{minipage}{0.3\textwidth}
        \aleq{
            \div \vvec{B} &= 0\\
            \curl \vvec{B} &= \mu_0 \vvec{J} + \mu_0\epsilon_0\pdvv{\vvec{E}}{t}
        }
    \end{minipage}
\end{center}
\addArrow[blue]{maxwell_faraday}{(-5ex,0)}{}{(-3ex,1ex)}

However, a \it{system of PDEs} is too complicated to be solved.
So we need to learn different "tricks" to avoid them,
which are enough for some simple scenarios.\\

Magnetic induction concerns the \nth{3} equation of the set - \cul[blue]{Faraday's law}. 





\linesep
% content begins here
% Section %%%%%%%%%%%%%%%%%%%%%%%%%%%%%%%%%%%%%%%%%%%%%%%%%%%%
\section{Motion under Lorentz Force}

Nowadays we know that Lorentz force is the fundamental explanation to magnetic induction.
But interestingly, it was formulated correctly only in the very late history of classical E\&M.
\aleq{
    \vvec{F} &= \vvec{F}_E + \vvec{F}_B \\
    \Aboxed{
        &= q\vvec{E} + q\vvec{v}\cross\vvec{B}
    }
}

In general, $\vvec{E}$ and $\vvec{B}$ can be functions of time and position,
but here we will only discuss \cul[red]{the special case when the fields are constant}.\\

By symmetry, we can assume that $\vvec{B}$ is only in $\hhat{z}$ direction, 
i.e. $\vvec{B} = B_z\hhat{z}$.
Then the Newton's \nth{2} law can be expanded as
\aleq{
    m\vvec{a} = m\dvv{t}\bmat{v_x\\v_y\\v_z} = q\bmat{E_x\\E_y\\E_z} 
        + q\bdet{\hhat{x} & \hhat{y} & \hhat{z}\\ v_x & v_y & v_z \\ 0 & 0 & B_z}
}

which is a system of 3 first order linear ODEs.
\aleq{
    \bcase{
        m\dvv{v_x}{t} &= qE_x + qv_yB_z\\
        m\dvv{v_y}{t} &= qE_y - qv_xB_z\\
        m\dvv{v_z}{t} &= qE_z
    }
}

This system is simple enough that we do not need to use any matrix methods.
\begin{itemize}
    \item \bf{\ul{Equation for z component}}\\[0.5ex]
    Motion in $z$ direction is independent of $x$/$y$.
    It can be solved alone.
    \aleq{
        m\dvv{v_z}{t} = qE_z 
        \qquad\Rightarrow\qquad
        v_z(t) &= \frac{qE_z}{m}t + v_{z0}\\[1em]
        \qquad\Rightarrow\qquad
        z(t) &= \half\frac{qE_z}{m}t^2 
            + \tkn{init_vz}{\cul[blue]{v_{z0}}}t + \tkn{init_z}{\cul[blue]{z_0}}
    }
    \addArrow[blue]{init_vz}{(-1ex,-2ex)}{\scriptsize Initial\\[-1ex]\scriptsize z velocity}{(0,-1ex)}{(-2ex,-1ex)}
    \addArrow[blue]{init_z}{(1ex,-2ex)}{\scriptsize Initial\\[-1ex]\scriptsize z coordinate}{(0,-1ex)}{(2ex,-1ex)}
    
    \vskip 1ex
    which is just a constant acceleration motion with an acceleration $\dfrac{qE_z}{m}$.

    \item \bf{\ul{Equation for x/y components}}\\[0.5ex]
    They can be solved by first differentiating one of them,
    then substitute into the other.
    \aleq{
        \dvv[2]{v_x}{t} &= \frac{qB_z}{m}\dvv{v_y}{t} \\[1ex]
        &= \frac{qB_z}{m}\qty(\frac{qE_y}{m}-\frac{qB_z}{m}v_x)\\[1ex]
        &= - \frac{q^2B_z^2}{m}\qty(v_x - \frac{E_y}{B_z})
    }

    which is the familiar ODE of SHM.
    \aleq{
        v_x(t) &= -C \sin\qty(\frac{qB_z}{m}t + \phi) + \frac{E_y}{B_z}\\[1ex]
        \Rightarrow\quad x(t) &= \cub[red]{C\qty(\frac{m}{qB_z})}{\text{A constant}}\cos \qty(\frac{qB_z}{m}t + \phi) 
            + \frac{E_y}{B_z}t + x_0\\[1ex]
        &= \tkn{init_r}{\cul[blue]{R}} \cos \qty(\frac{qB_z}{m}t + \tkn{phase}{\cul[blue]{\phi}}) 
            + \frac{E_y}{B_z}t + \tkn{init_x}{\cul[blue]{x_0}}
    }
    \addArrow[red]{init_r}{(5ex,4ex)}{}{(1ex,2ex)}
    \addArrow[blue]{init_r}{(0,-4ex)}{\scriptsize Radius}{(0,-1ex)}
    \addArrow[blue]{phase}{(0,-4ex)}{\scriptsize Phase}{(0,-1ex)}
    \addArrow[blue]{init_x}{(0,-3ex)}{\scriptsize Initial\\[-1ex]\scriptsize x coordinate}{(0,-1ex)}{(0,-1ex)}

    \vskip 1em
    And subsitute back to equation of $v_y$ gives
    \aleq{
        v_y(t) &= \frac{m}{qB_z}\dvv{v_x}{t} - \frac{E_x}{B_z}
        = -C \cos\qty(\frac{qB_z}{m}t + \phi) - \frac{E_x}{B_z}\\[1ex]
        \Rightarrow\quad y(t) &= -\cub[red]{C \qty(\frac{m}{qB_z})}{\text{A constant}} \sin \qty(\frac{qB_z}{m}t + \phi) 
            - \frac{E_x}{B_z}t + y_0 \\[1ex]
        &= -\tkn{init_r2}{\cul[blue]{R}} \sin \qty(\frac{qB_z}{m}t + \tkn{phase2}{\cul[blue]{\phi}}) 
            - \frac{E_y}{B_z}t + \tkn{init_y}{\cul[blue]{x_0}}
    }
    \addArrow[red]{init_r2}{(5ex,4ex)}{}{(1ex,2ex)}
    \addArrow[blue]{init_r2}{(0,-4ex)}{\scriptsize Radius}{(0,-1ex)}
    \addArrow[blue]{phase2}{(0,-4ex)}{\scriptsize Phase}{(0,-1ex)}
    \addArrow[blue]{init_y}{(0,-3ex)}{\scriptsize Initial\\[-1ex]\scriptsize y coordinate}{(0,-1ex)}{(0,-1ex)}

    \vskip 1em
    The result $x$/$y$ motion is a combination of circular motion + drifting.

    \insertFig{lorentz drift}

    \aleq{
        \bmat{x(t)\\y(t)} 
        = 
        \cus[red]{R\bmat{\cos \qty(\frac{qB_z}{m}t+\phi) \\ -\sin \qty(\frac{qB_z}{m}t+\phi)}}{\text{Circular motion}}
        \ +\ \cus[blue]{\inv{B_z}\bmat{E_y\\-E_x}t}{\text{Drifting}} \ +\ \bmat{x_0\\y_0}
    }

    \begin{enumerate}
        \item Be careful that the drifting direction is not intuitive: 
        \begin{itemize}
            \item With $E_x$ only, drifting direction = $-\hhat{y}$
            \item With $E_y$ only, drifting direction = $\hhat{x}$
        \end{itemize}
        In general, the drifting is along the direction of $\vvec{E}\cross\vvec{B}$ 
        with a speed $\dfrac{\norm{\vvec{E}}}{\norm{\vvec{B}}}$.
    
        \vskip 1em
        \item The rotation radius and speed maintain a constant ratio:
        \aleq{
            \text{Radius}=R \qquad\Rightarrow\qquad \text{Speed}=\frac{qB_z}{m}R
        }
        i.e. Angular velocity is always $\omega = \dfrac{qB_z}{m}$,
        independent of the charge's initial velocity.
        You can also arrive at the same result from equation of circular motion:
        \aleq{
            \frac{mv^2}{R} &= qvB \\
            v &= \frac{qB}{m}R
        }
        (But how do you argue that it is a circular motion without solving ODEs?)

    \end{enumerate}
\end{itemize}






\linesep
% Section %%%%%%%%%%%%%%%%%%%%%%%%%%%%%%%%%%%%%%%%%%%%%%%%%%%%
\section{Magnetic Induction}


%%%%%%%%%%%%%%
\subsection{EMF - Force or Voltage?}

The term \bf{electromotive force} (EMF) was invented by 
\href{https://en.wikipedia.org/wiki/Alessandro_Volta}{Alessandro Volta} in 1801,
for explaining observations in electrochemistry.\\

Metal electrodes in electrolyte 
\begin{itemize}
    \item[$\Rightarrow$] Current generates spontaneously.
    \item[$\Rightarrow$] There must be some kind of "force" pushing the current! 
\end{itemize}

\insertFig{electrolyte}

In early 1800s, scientists tended to describe things like a mechanical system,
i.e. any motions of objects must be driven by some kind of "force" - 
When there is a flow current, 
there must be some "force" that keeps pushing the charges forward,
namely the "electromotive force". 

\begin{itemize}
    \item However, EMF took the unit of volt because 
    voltage measurement was the only way to quantify the magnitude of EMF.

    \item Later when \href{https://en.wikipedia.org/wiki/Michael_Faraday}{Michael Faraday}
    discovered the phenomena of magnetic induction,
    he used the same word to refer to the source of the induced current.
\end{itemize}


Today we still keep its name as a "force", 
even though we already know much better how current is generated in different sources.
Maybe because people are already used to refer the same term \red{"EMF" 
as the general name for ANY kind of current source}, 
rather than differentiating them by the origin of energy.

\insertFig{i dun care}

In discussions of magnetic induction, 
there are two kinds of EMF relate to this phenomenon.
\begin{itemize}
    \item \bf{\ul{Motional EMF}} : Source of energy = Motion of the charges' "container".
    Can be explained by Lorentz force.

    \item \bf{\ul{Transformer EMF}} : Source of energy = Change in magnetic field.
    Explanation requires relativity.
\end{itemize}


%%%%%%%%%%%%%%
\subsection{Motional EMF}

Under Lorentz force, 
free charges move in circles in a constant B-field.
But if the charges' movements are restricted, 
charge distribution becomes uneven by building up potential difference. 

%%%%%%%%%%%%%%
\subsubsection{A Moving Battery}

Suppose there is only ONE charge in the middle of a conducting rod.
When the charge is given some initial velocity perpendicular to the rod
(e.g. we give the rod a push),
the charge will start to move in a circlar trajectory, 
and the rod will be dragged along by the charge. \\

(If the rod is restricted to move horizonally, it becomes SHM.)

\insertFig{pipe SHM}

But for charges that are at the ends of the rod,
they will be pushed against the end walls and cannot move.
Very soon, charge will accumulate at the end of the rod, 
build up a E-field (i.e. potential difference) between the ends of the rod. 
The central charges stop moving once the force from the built-up E-field balance the magnetic force, 
i.e. when $q\vvec{E} = q\vvec{v}\cross \vvec{B}$.

\insertFig{pipe potential}

Note that if the rod suddenly stop moving,  
the built-up potential difference,
and thus the stored energy will be released and lost.
\begin{center}
    \red{The rod can be used to drive current. \\
    It is a battery - but only when it keeps moving!}
\end{center}


We can analyze how the EMF arise from its energy conservation process.
Let the B-field be constant and in $\hhat{z}$ direction, i.e. $\vvec{B}=B_z\hhat{z}$.
\begin{enumerate}

    \item Note that the work done on a charge by magnetic force is always $0$, 
    because $\vvec{F}_B$ always perpendicular to the travelling direction $\dd{\vvec{r}}$,
    (i.e. the same direction as $\vvec{v} =\dv{\vvec{r}}{t}$).
    \aleq{
        \text{W.D.} &= \vvec{F}_B \cdot \dd{\vvec{r}} = q(\vvec{v}\cross\vvec{B})\cdot\dd{\vvec{r}} = 0
    }
    
    \item Meanwhile we can separate the W.D. into contributions along $x$ and $y$ direction:
    \aleq{
        \text{W.D.} &= q\qty[(\vvec{v}_x+\vvec{v}_y)\cross \vvec{B}]\cdot \qty(\dd{\vvec{x}}+\dd{\vvec{y}})\\
        %
        &= \cus[blue]{\ccancelto[blue]{0}{q(\vvec{v}_x\cross\vvec{B})\cdot\dd{\vvec{x}}}}
            {\substack{\vvec{v}_x\cross \vvec{B}= v_xB_z(\hhat{x}\cross \hhat{z})
                \\\phantom{\vvec{v}_{xi} \vvec{B}}=v_xB_z(-\hhat{y})\\[0.5ex]
                \text{Dotting with }\hhat{x}\text{ gives }0}}
            %
            + q(\vvec{v}_x\cross\vvec{B})\cdot\dd{\vvec{y}} 
            + q(\vvec{v}_y\cross\vvec{B})\cdot\dd{\vvec{x}}
            %
            + \cus[blue]{\ccancelto[blue]{0}{q(\vvec{v}_y\cross\vvec{B})\cdot\dd{\vvec{y}}}}
            {\substack{\vvec{v}_y\cross \vvec{B}= v_yB_z(\hhat{y}\cross \hhat{z})
                \\\phantom{\vvec{v}_{xi} \vvec{B}}=v_yB_z(\hhat{x})\\[0.5ex]
                \text{Dotting with }\hhat{y}\text{ gives }0}}\\[1ex]
        %
        &= (-qv_xB_z\hhat{y})\cdot \dd{\vvec{y}} + (qv_yB_z\hhat{x}) \cdot \dd{\vvec{x}}\\[1ex]
        %
        &= \cus[red]{(-qv_xB_z)\dd{y}}
                {\substack{\text{W.D. in }\hhat{y}\text{ direction}\\[0.5ex]\sim F_y \dd{y}}}
            +\cus[red]{(qv_yB_z)\dd{x}}
                {\substack{\text{W.D. in }\hhat{x}\text{ direction}\\[0.5ex]\sim F_x \dd{x}}}\\[1ex]
        %
        &\equiv 0 \quad\text{\scriptsize (Required by magnetic force)}
    }

    To always hold true, these two terms must always have equal magnitude but opposite sign.

    \item When charges accumulate at the ends of the rod,
    equilibrium happens when $q\vvec{E} = q\vvec{v}\cross\vvec{B}$.
    So when a charge climbs up the E-field for a small distance $\dd{\vvec{y}}$,
    it will gain a PE of:
    \addArrow[blue]{d_emf}{(0,-2ex)}{\scriptsize The potential gain\\[-1ex]\scriptsize i.e. EMF}{(0,-1ex)}{(2ex,-0.5ex)}
    \aleq{
        q\dd{\tkn{d_emf}{\cul[blue]{\epsilon}}} 
            \ \equiv\ (q\vvec{E})\cdot\dd{\hhat{y}}
            = q(\vvec{v}\cross\vvec{B})\cdot\dd{\hhat{y}}
            = -(qv_xB_z)\dd{y}
    }
    
    \vskip 1em
    This implies an energy conservation between:
    \begin{itemize}
        \item W.D. in $x$ direction = Any energy that cause the charge moving horizontally.
        Must be indirectly applied on the charge through the movement of the rod. 
        
        \item W.D. in $y$ direction = Any energy that cause the charge moving vertically.
        Can be independent of the rod's movement because charge can flow freely along the rod.
        
    \end{itemize} 
    And this conversion is possible all because of the "weird" direction of magnetic force (always perpendicular to $v$).

    \insertFig{WD by B = 0}

    $F_\text{Lorentz}$ is in the same direction as $\dd{y}$, opposite 


    
    
\end{enumerate}

To summarize, motional EMF is the result of energy conservation between
\aleq{
    \cus[red]{(qv_yB_z)\dd{x}}
    {\substack{\text{W.D. along }\hhat{x}\text{ direction}\\[0.5ex]\sim F_x \dd{x}}} 
}



%%%%%%%%%%%%%%
\subsubsection{Generalization to Arbituary Wire}



%%%%%%%%%%%%%%
\subsubsection{Magnetic Flux - A Geometrical Relation}

Consider a wire with a changing shape in under a static B-field. 
Each segment $i$ of the wire has a length + orientation $\dd{l_i}$ and 
is moving in velocity $v_i$.
With some algebra, the EMF generated in each segment can be re-written as 
\aleq{
        \dd{\epsilon} &= (\vvec{v}_i\cross\vvec{B})\cdot\dd{\vvec{l}_i}\\
        &= (\dd{\vvec{l}_i}\cross\vvec{v}_i)\cdot \vvec{B}\\
        &= -(\vvec{v}_i\cdot\dd{\vvec{l}_i})\cdot \vvec{B}
    }

Meanwhile, within a short duration $\dd{t}$,
the displacement of the segment is $\vvec{v}_i\dd{t}$.
Thus we can approximate the swept area by the segment as a parallelogram:
\aleq{
    \qty(\mstack{\text{Swept}\\\text{Area}})
    &= \qty(\mstack{\text{Area of parallelogram}\\\text{made by } v_i\dd{t} \text{ \& }\dd{l}_i})\\
    &\approx \norm{\vvec{v}_i}\norm{\dd{\vvec{l}_i}}\sin \qty(\mstack{\text{Angle between}\\\vvec{v}_i\dd{t} \text{ \& }\dd{l}_i})\\
    &= (\vvec{v}_i \dd{t})\cross(\dd{\vvec{l}_i})
    %
    \dvv{t} v_i\dd{t} \text{ \& }\dd{l}_i &= \vvec{v}_i\cross \dd{l}_i
} 

\insertFig{swept area}

We can now relate EMF generated in the whole wire with swept area by the wire as:
\aleq{
    \sum \epsilon = -\sum_{\substack{\text{All}\\\text{segments}}}
        \qty[(\vvec{v}_i\cross\dd{\vvec{l}_i})\cdot \vvec{B}]
    &= -\sum_{\substack{\text{All}\\\text{segments}}} 
        \qty[\dvv{t}\qty(\mstack{\text{Swept}\\\text{area}})_i\cdot \vvec{B}]\\
    &= -\dvv{t}\sum_{\substack{\text{All}\\\text{segments}}}
        \qty[\qty(\mstack{\text{Swept}\\\text{area}})\cdot \vvec{B}]
} 

When the segments are infinitestimally short, the sum becomes integral.
\aleq{
    \epsilon = -\int (\vvec{v}\cross\dd{\vvec{l}_i})\cdot\vvec{B}
    &= -\dvv{t}\qty[\iint_{\substack{\text{Swept}\\\text{area}}}\vvec{B}\cdot\dd{\vvec{s}}]\\
    &= -\dvv{t}\qty(\mstack{\text{Magnetic flux through}\\\text{the swept area}})
}

Pay attention to the depedendence to $t$ on the RHS:
\begin{itemize}
    \item The B-field $\vvec{B}$ is static. It is not a function of $t$.
    
    \item $\dd{\vvec{s}}$ is just a notation saying that this is a flux integral. 

    \item The only thing that depends on $t$ is the \cul[red]{integration range}.
\end{itemize}

So to be more accurate, the equation of motional EMF may be written as 
\aleq{
    \epsilon = - \int_{\substack{\text{along a wire}\\\text{of shape of }l(t)}} (\vvec{v}\cross\dd{\vvec{l}})\cdot \vvec{B}
    = -\dvv{t} \iint_{\substack{\text{Area }S(t)\\\text{swept by wire}}} \vvec{B}\cdot\dd{\vvec{s}}
}

to emphasize that it is the wire's shape / swept area varying with time.



%%%%%%%%%%%%%%
\subsection{Transformer EMF}

The modern explanation to transformer EMF is by relativity - 
motional EMF and transformer EMF are the same phenomenon being observed in different reference frame. 

\insertFig{different ref frame}

EMF generated due to motion of the ring
EMF generated due to changing magnetic field strength

Consider a reference frame where the charge is static to the observer. 
Because the charge is not moving, 
Lorentz force by magnetic field $q\vvec{v}\cross\vvec{B}$ must be $0$.
But if a time-varying B-field is applied, 
experiments show that a transformer EMF will create, 
i.e. some mysterious forces appear on the charges, 
forcing them to form a potential difference.

How to explain this mysterious force? 
Add an induced E-field, and claim it to be generated by time-varying B-field.

\insertFig{fix newton 2nd law}






%%%%%%%%%%%%%%
\subsection{Maxwell-Faraday Equation}




%%%%%%%%%%%%%%
\subsection{Lenz's Law}

The formula of Faraday's law, in particular the integral form,
can only be used to tell the magnitude of the induced EMF, 
but not its direction.
\aleq{
    \epsilon = -\dvv{t}\iint \vvec{B}\cdot\dd{\vvec{s}} 
}

This minus sign is not useful for determining the direction of EMF at all.

To determine the EMF's direction, we can apply \bf{Lenz's law},
which is in principle,
\begin{center}
    Induced EMF always want to "oppose" change in magnetic flux
\end{center}

And right hand grip rule is the only thing you need.
We can look at a few example to get familiar with it.

\begin{example}
    Consider a ring inside a magnetic field whose
    magnitude is increasing in the out-of-paper direction.
    \begin{enumerate}
        \item Magnetic flux is getting "more out-of-paper" due to 
        increase in B-field's magnitude.
        \item Induced EMF want to "oppose" this magnetic flux change.
        \item To oppose an increasing out-of-paper flux, one needs to decrease it,
        i.e. create an into-paper flux to compensate the increase.
        \item By right hand rule, into-paper flux can be created
        if a current flow clockwise in the ring. 
        So the induced EMF must be clockwise. 

    \end{enumerate}

    \insertFig{B field direction}
    \insertFig{induce flux direction}
    \insertFig{bar chart}
\end{example}



\begin{example}
    Consider a ring inside a constant into-paper B-field.
    The ring is shrinking in radius.
    \begin{enumerate}
        \item Magnetic flux is getting "less into-paper" due to
        decrease in area.
        \item Induced EMF want to "oppose" this magnetic flux change.
        \item To oppose a decreasing into-paper flux, one needs to increase it,
        i.e. create an into-paper flux to compensate the decrease.
        \item By right hand rule, into-paper flux can be created
        if a current flow clockwise in the ring.
        So the induced EMF must be clockwise.
        
    \end{enumerate}


\end{example}
\linesep
% Section %%%%%%%%%%%%%%%%%%%%%%%%%%%%%%%%%%%%%%%%%%%%%%%%%%%%
\section{Solving Problems in Magnetic Induction}

Generally speaking, there are only two levels of questions related to magnetic induction.
\begin{itemize}
    \item Find EMF from the change in magnetic flux.
    \item Find induced E-field from the change in magnetic flux. 
\end{itemize}

The difference is in the difficulties - 
EMF is just a single number, the same everywhere in the coil.
But the E-field, when involving magnetic induction, 
is a vector function which depends on position AND time.

%%%%%%%%%%%%
\subsection{Finding EMF}

This is a straightforward calculation to the surface integral of Faraday's law.

\begin{itemize}
    \item \bf{\ul{For high school level}} - 
    B-field is usually given as a constant of position,
    so that the surface integral reduces to a multiplication.
    \aleq{
        \epsilon = -\dvv{t}\iint \vvec{B}\cdot \dd{\vvec{s}}
        \ \sim\ -\dvv{t}\qty(\vvec{B}(t)\cdot \qty(\text{Area})(t))
    }

    And because you are not expected to have studied differentiation chain rule,
    you will never see a situation where both $\vvec{B}(t)$ and Area$(t)$ are changing.
    It is always given the rate of change of one of them and the other is fixed.
    \aleq{
        \cus[red]{\epsilon = -\vvec{B} \cdot \dvv{t}(\text{Area}(t))}
            {\substack{\text{When the question is about}\\\text{motional EMF}}}
        \qquad\text{or}\qquad
        \cus[red]{\epsilon = -\dvv{\vvec{B}(t)}{t} \cdot (\text{Area})}
            {\substack{\text{When the question is about}\\\text{transformer EMF}}}
    }

    \vskip 1ex
    \item \bf{\ul{For university level}} - You are assumed to have already learnt Ampere's law,
    so more likely you are given the current and need to find B-field before EMF. 
    \aleq{
        \text{First}\quad
        \cus[blue]{\oint \vvec{B}(t)\cdot\dd{\vvec{l}} = \mu_0 I(t)}{\text{Ampere's law for }\vvec{B}}
        \ ,\quad \text{then}\quad 
        \cus[blue]{\epsilon = -\dvv{t}\iint_{\text{Area}(t)} \vvec{B}(t)\cdot \dd{\vvec{s}}}{\text{Faraday's law for }\epsilon}
    }
\end{itemize}

Asking for the EMF's direction is pretty common,
because applying Lenz's law does not involve any maths.
All you need is your right hand.

%%%%%%%%%%%%
\subsection{Finding induced E-field}

This is in fact the task of solving the PDE of Maxwell-Faraday equation.
\aleq{
    \curl \vvec{E} = -\pdvv{\vvec{B}}{t}
    \qquad\Rightarrow\qquad
    \vvec{E}(t) = \text{Some function of }\vvec{B}(t)
}

Similar to how we deal with Gauss's law or Ampere's law,
we can avoid solving PDE in some very symmetrical cases.
If these conditions are satisfied:
\begin{enumerate}
    \item E-field is of the same magnitude everywhere on the loop.
    \item E-field make the same angle with each line segment of the loop.
\end{enumerate}

Then we can reduce the integral form of Maxwell-Faraday equation in to multiplication.
\aleq{
    -\dvv{t}\iint\vvec{B}\cdot\dd{\vec{s}} &= \oint \vvec{E}\cdot\dd{\vvec{l}}\\
    &= \oint \tkn{ampere_dot}{\cul[green]{\norm{\vvec{E}}\norm{\dd{\vvec{l}}}\cos\theta}}\\[1ex]
    &= \tkn{ampere_B}{\cul[red]{\norm{\vvec{E}}}}\ \tkn{ampere_theta}{\cul[blue]{\cos\theta}}\ \oint \norm{\dd{\vvec{l}}}\\[2.5em]
    &= \norm{\vvec{E}}\ \cos\theta\ (\text{Perimeter of loop})\\[1em]
    \Aboxed{
        \norm{\vvec{E}} &= \frac{\qty(\text{Flux of }\vvec{B})}{(\text{Perimeter of loop})\cos\theta}
    }
}
\addArrow[green]{ampere_dot}{(5ex,0)}
{\scriptsize Just dot product\\[-1ex]\scriptsize $\vvec{a}\cdot\vvec{b}=\norm{\vvec{a}}\norm{\vvec{b}}\cos\theta$}
{(8ex,0)}{(6ex,0)}
\addBentArrow[red]{ampere_B}{(-8ex,-3.5ex)}
{Same magnitude everywhere\\[-0.5ex]Can move out of integral!}
{(0,-1.5ex)}{(-12.5ex,1ex)}
\addBentArrow[blue]{ampere_theta}{(8ex,-3.5ex)}
{Form same angle everywhere\\[-0.5ex]Can move out of integral!}
{(0,-1.5ex)}{(12.5ex,1ex)}

\begin{example}
    Consider a static rectangular loop next to an infinitely long wire, 
    which carries a time-varying current $I(t)$, same magnitude everywhere along the wire. 

    \insertFig{loop next to wire}

    \begin{enumerate}
        \item Always start with finding the magnetic flux through the loop.
        By Ampere's law, the B-field by an infinitely long wire is 
        \aleq{
            \vvec{B}(r,t) = \frac{\mu_0}{2\pi}\frac{I(t)}{r}
        }
        
        The magnetic flux through the loop can be calculated by 
        first dividing the loop's area into strips,
        then integrate the flux of all strips.
        \aleq{
            \Phi_B &= \iint \vvec{B}\cdot\dd{\vvec{s}}\\
            &= \int_{r_1}^{r_2} \frac{\mu_0 I(t)}{2\pi r} (L\dd{r})\\
            &= \frac{\mu_0 I(t) L}{2\pi}\qty[\ln(r_2)-\ln(r_1)]
        }

        \insertFig{split into strips}

        \item The total EMF generated in the loop is just the differentiation of the above. i.e.
        \aleq{
            \epsilon = \frac{\mu_0 L}{2\pi}\qty(\dvv{I(t)}{t})[\ln(r_2) - \ln(r_1)]
        }
        Direction of the EMF depends on how the current varies.
        For example, if $\dv{I(t)}{t}>0$, i.e. magnetic flux increasing in the into-paper direction.
        To oppose the change, EMF must be in anti-clockwise direction
        to produce an out-of-paper direction flux.

        \item On the other hand, to compute the E-field distribution, all we know is 
        \aleq{
            \abs{\epsilon} = \oint \vvec{E}\cdot \dd{\vvec{l}} 
            = E^{\parallel}_1 L + E^{\parallel}_2 (r_2-r_1) + E^{\parallel}_3 L + E^{\parallel}_4 (r_2-r_1)
        }

        \insertFig{show label of edge only parallel}
        
        After taking dot product, only the component parallel to the edge is left.\\

        To find the $E$ on each edge, we need symmetry arguments.
        \begin{itemize}
            \item By rotation symmetry, 
            the E-field is only a function of radial distance from the wire. 
            We can claim that $E^\parallel_1$ and $E^\parallel_3$ must have the same function form $\vvec{E}(r)$,
            
            \insertFig{rotation sym}
            
            \item By translational symmetry, 
            $E_2^\parallel$ and $E_4^\parallel$ must be the same. 
            Their contributions of dot product along the loop are cancelled.
        \end{itemize}
        
        So the E-field relation to EMF is reduced to
        \aleq{
            \abs{\epsilon} &= E^{\parallel}_3 L - E^{\parallel}_1 L \\
            &= E^{\parallel}(r_2) L - E^{\parallel}(r_1) L\\
            &\equiv \frac{\mu_0 L}{2\pi}\qty(\dvv{I(t)}{t}) \qty[\ln(r_2)-\ln(r_1)]
        }
        
        We can claim that the $\vvec{E}$ \cul[red]{'s component parallel to the wire} is
        \aleq{
            E^{\parallel}(r) = \frac{\mu_0 L}{2\pi}\qty(\dvv{I(t)}{t})\ln(r)
        }

        \item Note that in the above analysis,
        we cannot determine if the E-field has components perpendicular to the wire.

        \insertFig{parallel + perpendicular}

        If there is only parallel component\\
        If there is also perpendicular component. Is it physical?

        
        Recall from Gauss's law, 
        E-field can be perpendicular to the wire if the wire carries a \cul[red]{static} line charge density.
        So when the wire carries BOTH net charge and time-varying current,
        the total E-field will be in diagonal directions.

        \insertFig{sum of horizontal + vertical contribution = diagonal}

    \end{enumerate}

    
\end{example}


\linesep

\appendix
% Section %%%%%%%%%%%%%%%%%%%%%%%%%%%%%%%%%%%%%%%%%%%%%%%%%%%%
%\setcounter{section}{-1}
\section*{Appendix: A Brief History of Electromagnetism}

Electromagnetic induction is likely the most confusing topic in beginner E\&M.
In my opinion, taking reference of the history is helpful to unify the concepts you have learnt.

\begin{center}
    \begin{tabularx}{\textwidth}{
        >{\centering\arraybackslash}m{0.15\textwidth} 
        p{0.8\textwidth}
        }

        Year & \makecell[c]{Advancement} \\ 
        \hline
        Before 1500s & 
        Different electrostatics phenomena were known.
        But they were not unified or explained at all. \\
        %
        1600 &
        \makecell[tl]{
            \href{https://en.wikipedia.org/wiki/William_Gilbert_(physicist)}{William Gilbert}
            was the first person to use the word "electrical" to describe\\ electrostatics phenomena. 
            Also the first to propose that electrical effect is\\ due to flows of particles.
        }\\[1.5em]
        %
        1750 &
        \makecell[tl]{
            \href{https://en.wikipedia.org/wiki/Benjamin_Franklin}{Benjamin Franklin} 
            developed a one "fluid" theory of electricity, 
            and called\\ this fluid "charge".
        }\\[1.5em]
        %
        1784 &
        \makecell[tl]{
            \href{https://en.wikipedia.org/wiki/Charles-Augustin_de_Coulomb}{Charles-Augustin de Coulomb}
            experimentally showed that force between\\ charged objects $\propto \inv{r^2}$.
            \gray{(Coulomb's law $F = \inv{4\pi\epsilon_0}\frac{Qq}{r^2}$)}
        }\\[1.5em]
        %
        1800 &
        \makecell[tl]{
            \href{https://en.wikipedia.org/wiki/Alessandro_Volta}{Alessandro Volta}
            Made the first battery from electro-chemistry.\\
            \gray{(First time to have steady current.)}
        }\\[1.5em]
        %
    \end{tabularx}
\end{center}


\begin{center}
    \begin{tabularx}{\textwidth}{
        >{\centering\arraybackslash}m{0.15\textwidth} 
        p{0.8\textwidth}
        }
        
        1820 &
        \makecell[tl]{
            \href{https://en.wikipedia.org/wiki/Hans_Christian_\%C3\%98rsted}{Hans Christian Ørsted}
            discovered that current wire can deflect compress.\\
            \gray{(First time to relate electric and magnetic phenomena.)}
        }\\[2em]
        %
        1820 &
        \makecell[tl]{
            \href{https://en.wikipedia.org/wiki/Andr\%C3\%A9-Marie_Amp\%C3\%A8re}{André-Marie Ampère}
            formulated and verified the force between current wires.\\
            \gray{$(F = I\vvec{l}_1\cross \frac{\mu_0 I}{2\pi r}\vvec{l}_2)$}
        }\\[2em]
        %
        1831 &
        \makecell[tl]{
            \href{https://en.wikipedia.org/wiki/Michael_Faraday}{Michael Faraday}
            discovered magnetic induction. Experiments include:\\
            - Across iron core: Reading appears at the instant when switch is on/off.\\
            - Moving frame: Reading appears when wireframe moves, changes shape \\
            or when magnetic field change.\\
            \red{insertFig}
        }\\[1.5em]
        %
        1834 &
        \makecell[tl]{
            \href{https://en.wikipedia.org/wiki/Emil_Lenz}{Emil Lenz}
            Explained direction of induced current by energy conservation.\\
            \gray{(Lenz's Law)}
        }\\[1.5em]
        %
        1860 &
        \makecell[tl]{
            \href{https://en.wikipedia.org/wiki/James_Clerk_Maxwell}{James Clerk Maxwell}
            unified past discoveries into 20 equations, 
            and used\\ field description for the first time.\\
            \red{This was the first time $\vb{E}$ and $\vb{B}$ appeared in Physics.}\\
            \red{Before Maxwell, everything was described in terms of force.}\\ 
        }\\[4em]
        %
        1893 &
        \makecell[tl]{
            \href{https://en.wikipedia.org/wiki/Oliver_Heaviside}{Oliver Heaviside}
            combined Maxwell's 20 equations into 4, by vector calculus.\\
            \gray{(This is the version of Maxwell's equation we now know.)}
        }\\[1em]
        %
        1895 &
        \makecell[tl]{
            \href{https://en.wikipedia.org/wiki/Hendrik_Lorentz}{Hendrik Lorentz}
            derive the correct force on charges under both $\vvec{E}$ and $\vvec{B}$.\\
            \gray{(Lorentz force formula)}
        }\\[1em]

    \end{tabularx}
\end{center}


%%%
\theend
\end{document}