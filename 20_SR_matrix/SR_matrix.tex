\documentclass[class=article, crop=false, 12pt]{standalone}
\usepackage[subpreambles=true]{standalone}
\usepackage{../.common/common}


\author{Tony Shing}
%\pretitle{Supplementary}

\topic{T20 (Special Relativity)}
\title{Matrix Method for Special Relativity}

\version{2025} % leave blank for omitting

\begin{document}

\maketitle


\begin{overview}
    \begin{itemize}
        \item Lorentz transformation matrix
        \item Explaining relativistic phenomena with the matrix method
    \end{itemize}
\end{overview}



% content begins here
% Section %%%%%%%%%%%%%%%%%%%%%%%%%%%%%%%%%%%%%%%%%%%%%%%%%%%%
\section{Lorentz Transform}

%%%%%%%%%%%%%%
\subsection{Matrix as Linear Transformation}

First we shall re-visit matrix as a tool of coordinate transformation - 
By applying a matrix onto a position vector, 
we can change the \cul[red]{expression of a position} in one coordinate system 
into the expression in another coordinate system.
Here are some very common transformations that you need to remember:

\begin{itemize}
    \item \bf{\ul{Rotation matrix}}
    \begin{center}
        \vskip -2em
        \begin{minipage}{0.4\linewidth}
            \aleq{
                \bmat{x'\\ y'}
                =
                \bmat{
                    \cos{\theta} & \sin{\theta} \\
                    -\sin{\theta} & \cos{\theta}
                }
                \bmat{x\\y}
            }

        \end{minipage}
        \hspace{0.05\textwidth}
        \begin{minipage}{0.3\linewidth}
            \centering
            \includegraphics[width=\textwidth]{coor_rotate}
        \end{minipage}
    \end{center}
    
    \item \bf{\ul{Reflection matrix}}
    \begin{center}
        \vskip -2em
        \begin{minipage}{0.4\linewidth}
            \aleq{
                \bmat{x'\\y'}
                =
                \bmat{ 
                    -1 & 0 \\
                    0 & 1 
                }
                \bmat{x\\y}
            }

        \end{minipage}
        \hspace{0.05\textwidth}
        \begin{minipage}{0.3\linewidth}
            \centering
            \includegraphics[width=\textwidth]{coor_reflect}
        \end{minipage}
    \end{center}
    
\end{itemize}

Remember the very important fact about coordinate transform:
\begin{center}
    \bf{The point is always the same one, 
    but the point's coordinate can change \\  
    because we are free to choose the coordinate system.}
\end{center}

\begin{notation}[Side Note:]
    In fact, the coordinate transformation by a matrix $\mmat{A}$ will map the coordinate expression onto 
    the coordinate system spanned by the vectors $\qty{\inv{\lambda_1}\vvec{v}_1, \inv{\lambda_1}\vvec{v}_1, \dots, \inv{\lambda_n}\vvec{v}_n}$,
    where $\lambda_i$ are the eigenvalues of $\mmat{A}$ and $\vvec{v}_i$ are the corresponding eigenvectors.

    %\insertFig{random transform}

    %For example, $\mmat{A} = \bmat{1 & 1 \\ 4 & 1}$ 
\end{notation}

%%%%%%%%%%%%%%
\subsection{Lorentz Transformation Matrix}

%%%%%%%%%%%%%%
\subsubsection{Spacetime Coordinate}

The topic of relativity is to study 
the transform between \bf{spacetime coordinate system}:

\begin{itemize}
    \item Every "\bf{event}" in the spacetime can be labelled with a coordinate:
    \begin{itemize}
        \item An event happens at time $t$ at position $(x,y,z)$ is given the coordinate $(ct,x,y,z)$. 
        \item The time $t$ is multiplied by speed of light $c$ such that all 4 coordinates have the unit of positions.
    \end{itemize}

    
    \item Different \bf{observers} can describe the same event using their own coordinate systems,
    leading to different expressions of the same event. E.g. For the same event, 
    \begin{itemize}
        \item Observer A may describe it as $(ct,x,y,z)$, while
        \item Observer B describe it as $(ct',x',y',z')$.
    \end{itemize}
    
    \iffalse
    % move to bottom of this section
    \item 
    \fi
\end{itemize}

In special relativity, we only deals with observers in different \bf{inertial frames}, 
i.e. they do not experience accelerations.  
For simplicity, we can choose 
\begin{itemize}
    \item The relative velocity between observers is along the x-axis.
    \item The origins of the observers' coordinate "coincide", 
    i.e. $(ct, x,y,z) =(0,0,0,0)$ is the same point as $(ct', x', y',z')=(0,0,0,0)$.
\end{itemize}

Then the $y$/$z$ coordinate of an event will be the same when described by both observers. 
We can focus on the transformation on $t$ and $x$ coordinate only.\\

In the following section, we will derive the coordinate transform matrix -
the Lorentz matrix, which relates the observed $ct$ and $x$ coordinate between observers. 
\aleq{
    = \tkn{coorB}{\cul[red]{\bmat{ct'\\x'}}}
    = \tkn{lorentz}{\text{\Huge $(\mmat{\Lambda})$}} \tkn{coorA}{\cul[blue]{\bmat{ct\\x}}}
}
\addBentArrow[red]{coorB}{(-8ex,-3ex)}{An event's coordinate\\observed by B}{(0,-5ex)}{(-8ex,1ex)}
\addBentArrow[blue]{coorA}{(8ex,-3ex)}{An event's coordinate\\observed by A}{(0,-5ex)}{(8ex,1ex)}
\addArrow[gray]{lorentz}{(0,-10ex)}{The Lorentz Transformation\\A 2x2 matrix}{(0,-2ex)}{(0,-1ex)}

\vskip 4em
Geometrically, this transformation can be visualized 
as a change in the coordinate on a $t-x$ plane called \bf{Minkowski diagram}:
\begin{center}
    \begin{minipage}{0.35\linewidth}
        \centering
        \includegraphics[width=\textwidth]{coor_minkowski}
    \end{minipage}
\end{center}


%%%%%%%%%%%%%%
\subsubsection{2 Einstein's Posulates}

Special relativity is proposed based on only two principles.
They are essential to derive the expression of Lorentz transformation.

\begin{enumerate}
    \item \bf{\ul{Principle of Relativity}}\\[0.5ex]
    All physics must be the same to any inertial observers. 
    i.e. All the formula yield the correct results, 
    although values to be substituted are different for different observers.
    %For example, observers cannot tell if an object is moving relative to him or 
    %himself is moving relative to the object, when there is no acceleration. 
    
    \item \bf{\ul{Principle of invariant light speed}}\\[0.5ex]
    Speed of light is the same for all observers. 
    (This includes non-inertial frame observers.)

\end{enumerate}


\vskip 1em
%%%%%%%%%%%%%%
\subsubsection{Deriving the Matrix}

Let the two observers \blue{A}, \red{B} differ in relative velocity $v$. 
Bear in mind that when there is no acceleration, 
observers cannot distinguish if it is the object moving relative to him 
or him moving relative to the object.
\begin{itemize}
    \item \blue{A} always thinks that \red{B} is the one moving, while A himself never moves.
    
    \item \red{B} always thinks that \blue{A} is the one moving, while B himself never moves.

    \item If \blue{A} sees \red{B} moving with velocity $v$, 
    \red{B} sees \blue{A} moving with velocity $-v$.

\end{itemize}

\begin{center}
    \begin{minipage}{0.4\linewidth}
        \centering
        \includegraphics[width=\textwidth]{observer1}
    \end{minipage}
    \quad \vline\quad 
    \begin{minipage}{0.4\linewidth}
        \centering
        \includegraphics[width=\textwidth]{observer2}
    \end{minipage}
\end{center}

\vskip 1em
We are going to show that:
\begin{framed}
When \red{B} moves at velocity $v$ relative to \blue{A}
    \aleq{
        \tkn{coorB2}{\cul[red]{\bmat{ct'\\x'}}}
        = \tkn{lorentz2}{\cul[gray]{\bmat{\gamma & -\gamma\beta\\ -\gamma\beta & \gamma}}}\,
        \tkn{coorA2}{\cul[blue]{\bmat{ct\\x}}}
    }
    \addBentArrow[red]{coorB2}{(-8ex,-3ex)}{An event's coordinate\\observed by B}{(0,-5ex)}{(-8ex,1ex)}
    \addBentArrow[blue]{coorA2}{(8ex,-3ex)}{An event's coordinate\\observed by A}{(0,-5ex)}{(8ex,1ex)}
    \addArrow[gray]{lorentz2}{(0,-3ex)}{\bf{The Lorentz}\\\bf{transformation matrix $\mmat{\Lambda}$}}{(0,-4.5ex)}{(0,-1.5ex)}

    \vskip 6ex
    with $\gamma = \inv{\sqrt{1-\frac{v^2}{c^2}}}$ and $\beta = \frac{v}{c}$.
\end{framed}

\newpage
\underline{\textit{Proof}}\par
\vskip 0.5em

In general, A and B are no different other than they move with a speed $v$ relative to each other.
So the Lorentz transform between them should only be related to $v$.
We can write 
\aleq{
    \bmat{ct'\\x'}
    = \mmat{\Lambda}(v)\bmat{ct\\x}
    = \bmat{
        p_{\cul[red]{\cul[red]{v}}} & q_{\cul[red]{\cul[red]{v}}} \\
        r_{\cul[red]{\cul[red]{v}}} & s_{\cul[red]{\cul[red]{v}}}
    }
    \bmat{ct\\x}
}

The subscripts on $p, q, r, s$ indicate that they are \cul[red]{functions of $v$}. 

\begin{enumerate}
    
    \item
    \it{\ul{The inverse of $\mmat{\Lambda}$ must exist}}\\
    We should always be able to transfrom from B's coordinates back to A’s coordinates. 
    Since B sees A moving at velocity $-v$,
    the inverse of $\mmat{\Lambda}(v)$ should have the same expression
    but with all $v$ changed to $-v$
    \aleq{
        \bmat{ct\\x} 
        = \mmat{\Lambda}\minv(v) \bmat{ct'\\x'}
        &= 
        \mmat{\Lambda}(-v) \bmat{ct'\\x'}
        = \bmat{
            p_{\cul[blue]{\cul[blue]{-v}}} & q_{\cul[blue]{\cul[blue]{-v}}} \\
            r_{\cul[blue]{\cul[blue]{-v}}} & s_{\cul[blue]{\cul[blue]{-v}}}
        }
        \bmat{ct'\\x'}\\[1em]
        %
        \Aboxed{
            \mmat{\Lambda}\minv(v) &= \mmat{\Lambda}(-v)
        }
    }

    
    \item \it{\ul{From A transform to B}}\\
    We have previously chosen their coordinate systems to coincide, 
    i.e. $(ct,x)=(0,0)$ is the same point as $(ct',x')=(0,0)$.    
    When some time $T$ advanced in A's clock, A will see 
    \begin{itemize}
        \item A himself has not moved.
        \item B's position changed to $vT$ because A sees B moving with velocity $v$.
    \end{itemize}

    \begin{center}
        \begin{tabular}{>{\centering\arraybackslash}m{0.3\linewidth}|
            >{\centering\arraybackslash}m{0.2\linewidth}|
            >{\centering\arraybackslash}m{0.2\linewidth}}
            & A seen by A & B seen by A \\ 
            \hline \vskip 1ex 
            When A's clock shows $t=0$ & 
            $\bmat{0\\0}$ & 
            $\bmat{0\\0}$ 
            \\[4ex] \hline \vskip 1ex 
            When A's clock shows $t=T$ & 
            $\bmat{ cT\\0 }$ & 
            $\bmat{cT\\vT}$
        \end{tabular}
    \end{center}

    Multiplying Lorentz matrix to these coordinate will transform to what is seen by B:
    \begin{center}
        \begin{tabular}{>{\centering\arraybackslash}m{0.3\linewidth}|
            >{\centering\arraybackslash}m{0.2\linewidth}|
            >{\centering\arraybackslash}m{0.2\linewidth}}
            & A seen by \red{B} & B seen by \red{B}\\ 
            \hline \vskip 1ex 
            When A's clock shows $t=0$  & 
            $\red{\bmat{p_v & q_v \\ r_v & s_v}}\bmat{0\\0}$ & 
            $\red{\bmat{p_v & q_v \\ r_v & s_v}}\bmat{0\\0}$ 
            \\[4ex] \hline \vskip 1ex 
            When A's clock shows $t=T$  & 
            $\red{\bmat{p_v & q_v \\ r_v & s_v}}\bmat{cT\\0}$ & 
            \cul[gray]{\cul[gray]{$\red{\bmat{p_v & q_v \\ r_v & s_v}}\bmat{cT\\vT}$}}
        \end{tabular}
    \end{center}
    
    Notice the bottom right entry - when B looks at himself, 
    he should always see himself not moving, i.e. always at his origin $(x'=0)$. 
    \aleq{
        \bmat{
            p_v & q_v \\
            r_v & s_v
        }
        \bmat{cT\\vT} 
        &=
        \bmat{\text{\it{don't care}}\\\cul[red]{\cul[red]{0}}} \sim \bmat{ct'\\x'}
    }

    This gives us the first relations between the Lorentz matrix's elements. 
    \aleq{
        \bmat{
            \cdots & \cdots \\
            r_v & s_v
        }
        \bmat{cT\\vT}
        =
        \bmat{\cdots \\ r_vcT+ s_vvT}
        &=
        \bmat{\cdots\\0}\\[1ex]
        %
        \Aboxed{
            r_v &= -s_v\qty(\frac{v}{c})
        }
    }
    

    \item \it{\ul{From B transform to A}}\\
    We can interchange the A, B's roles by switching
    $x'\leftrightarrow x$, $t'\leftrightarrow t$, and $v\leftrightarrow -v$ to repeat the previous step.
    When some time $T'$ advanced in B's clock, B will see 
    \begin{itemize}
        \item B himself has not moved.
        \item A's position changed by $-vT'$ because B sees A moving with velocity $-v$.
    \end{itemize}

    \begin{center}
        \begin{tabular}{>{\centering\arraybackslash}m{0.3\linewidth}|
            >{\centering\arraybackslash}m{0.2\linewidth}|
            >{\centering\arraybackslash}m{0.2\linewidth}}
            & A seen by B & B seen by B \\ 
            \hline \vskip 1ex 
            When B's clock shows $t'=0$ & 
            $\bmat{0\\0}$ & 
            $\bmat{0\\0}$ 
            \\[4ex] \hline \vskip 1ex 
            When B's clock shows $t'=T'$ & 
            $\bmat{ cT'\\-vT'}$ & 
            $\bmat{cT'\\0}$
        \end{tabular}
    \end{center}

    Multiplying these coordinate with the \cul[blue]{\it{inverse}} Lorentz matrix, 
    i.e. use $-v$ instead of $v$, will transform to what is seen by A:
    \begin{center}
        \begin{tabular}{>{\centering\arraybackslash}m{0.3\linewidth}|
            >{\centering\arraybackslash}m{0.3\linewidth}|
            >{\centering\arraybackslash}m{0.3\linewidth}}
            & A seen by \blue{A} & B seen by \blue{A}\\ 
            \hline \vskip 1ex 
            When B's clock shows $t'=0$  & 
            $\blue{\bmat{p_{-v} & q_{-v} \\ r_{-v} & s_{-v}}}\bmat{0\\0}$ & 
            $\blue{\bmat{p_{-v} & q_{-v} \\ r_{-v} & s_{-v}}}\bmat{0\\0}$ 
            \\[4ex] \hline \vskip 1ex 
            When B's clock shows $t'=T'$  & 
            \cul[gray]{\cul[gray]{$\blue{\bmat{p_{-v} & q_{-v} \\ r_{-v} & s_{-v}}}\bmat{cT'\\-vT'}$}} & 
            $\blue{\bmat{p_{-v} & q_{-v} \\ r_{-v} & s_{-v}}}\bmat{cT'\\0}$
        \end{tabular}
    \end{center}

    Notice the bottom left entry - when A looks at himself, 
    he should always see himself not moving, i.e. always at his origin $(x=0)$. 
    \aleq{
        \bmat{
            p_{-v} & q_{-v} \\
            r_{-v} & s_{-v}
        }
        \bmat{cT'\\-vT'} 
        &=
        \bmat{\text{\it{don't care}}\\\cul[blue]{\cul[blue]{0}}} \sim \bmat{ct\\x}
    }

    This gives us the second relations between the Lorentz matrix's elements. 
    \aleq{
        \bmat{
            \cdots & \cdots \\
            r_{-v} & s_{-v}
        }
        \bmat{cT'\\vT'}
        =
        \bmat{\cdots \\ r_{-v}cT' - s_{-v}vT'}
        &=
        \bmat{\cdots\\0}\\[1ex]
        %
        \Aboxed{
            r_{-v} &= s_{-v}\qty(\frac{v}{c})
        }
    }
    
    \item \it{\ul{(Matrix) $\times$ (Its inverse) $= \mmat{I}$}}\\
    Substitute the results from step 2 and 3 to $\mmat{\Lambda}$ and $\mmat{\Lambda}\minv$,
    then multiply them:
    \aleq{
        \mmat{\Lambda}\minv\mmat{\Lambda}
        =
        \bmat{
            p_{-v} & q_{-v}\\
            s_{-v}\qty(\frac{v}{c}) & s_{-v}
        }
        \bmat{
            p_{v} & q_{v}\\
            s_{v}\qty(-\frac{v}{c}) & s_{v}
        }
        \ \equiv\ 
        \mqty(\imat{2}) = \mmat{I}
    }

    The bottom left entry gives us a relation between $p$ and $s$.
    \aleq{
        \bmat{\cdots &\cdots \\ s_{-v}\qty(\frac{v}{c}) & s_{-v}}
        \bmat{p_v & \cdots \\ s_{v}\qty(-\frac{v}{c}) & \cdots}
        &= \bmat{\cdots & \cdots \\ 0 & \cdots }\\[1ex]
        %
        s_{-v}\qty(\frac{v}{c})p_v + s_{-v}s_{v}\qty(-\frac{v}{c}) &= 0 \\[1ex]
        %
        \Aboxed{
            p_v = s_v
        }
    }

    \iffalse
    Bottom right entry:
    \aleq{
        \bmat{\cdots &\cdots \\ s_{-v}\qty(\frac{v}{c}) & s_{-v}}
        \bmat{\cdots & q_v \\ \cdots & s_v}
        &= \bmat{\cdots & \cdots \\ \cdots & 1 }\\[1ex]
        %
        s_{-v}\qty(\frac{v}{c})q_v + s_{-v}s_{v} &= 1 \\[1ex]
        %
        \Aboxed{
            q_v = \frac{1-s_vs_{-v}}{s_{-v}}\qty(\frac{v}{c})
        }
    }
    \fi
    
    \item \it{\ul{Principle of constant light speed}}\\
    Both A and B should see a light beam travelling at speed $=c$.
    
    \begin{center}
        \begin{minipage}{0.4\textwidth}
            \begin{tabular}{>{\centering\arraybackslash}m{0.45\linewidth}|
                >{\centering\arraybackslash}m{0.4\linewidth}}
                & Light beam seen by A\\ 
                \hline \vskip 1ex 
                When A's clock shows $t=0$ & 
                $\bmat{0\\0}$
                \\[4ex] \hline \vskip 1ex 
                When A's clock shows $t=T$ & 
                $\bmat{cT\\cT}$
            \end{tabular}
        \end{minipage}
        \ AND\quad 
        \begin{minipage}{0.4\textwidth}
            \begin{tabular}{>{\centering\arraybackslash}m{0.45\linewidth}|
                >{\centering\arraybackslash}m{0.4\linewidth}}
                & Light beam seen by B \\ 
                \hline \vskip 1ex 
                When B's clock shows $t'=0$ & 
                $\bmat{0\\0}$
                \\[4ex] \hline \vskip 1ex 
                When B's clock shows $t'=T'$ & 
                $\bmat{ cT'\\cT' }$
            \end{tabular}
        \end{minipage}
    \end{center}

    These tables should be true for ANY value of $T$ and $T'$. 
    We can choose the value of $T'$ such that the light beam being observed by both A and B as the same event.
    \begin{center}
        \begin{tabular}{>{\centering\arraybackslash}m{0.2\linewidth}|
            >{\centering\arraybackslash}m{0.3\linewidth}|
            >{\centering\arraybackslash}m{0.35\linewidth}}
            & Light beam seen by A & Light beam seen by \red{B} \\ 
            \hline \vskip 1ex 
            When A's clock shows $t=0$ & 
            $\bmat{0\\0}$ &
            $\red{\bmat{p_v & q_v \\r_v & s_v}}\bmat{0\\0} = \red{\bmat{0\\0}}$
            \\[4ex] \hline \vskip 1ex 
            When A's clock shows $t=T$ & 
            $\bmat{ cT\\cT }$ &
            \cul[gray]{\cul[gray]{$\red{\bmat{p_v & q_v \\r_v & s_v}}\bmat{cT\\cT} = \red{\bmat{cT'\\cT'}}$}}
        \end{tabular}
    \end{center}

    This gives us a relation between $q$ and $s$:
    \aleq{
        \bcase{
            (p_v + q_v)cT &= cT' \\
            (r_v + s_v)cT &= cT'
        }
    }
    
    In the previous steps, we have already found $p_v=s_v$ and $r_v = -s_v\qty(\frac{v}{c})$
    so it remains
    \aleq{
        \Aboxed{
            q_v = r_v = -s_v\qty(\frac{v}{c})
        }
    }

    
    \item \it{\ul{Choose $\det(\mmat{\Lambda})=1$}}\\
    So far we have found each entry in the Lorentz transformation matrix in terms of $s_v$.
    \aleq{
        \mmat{\Lambda} 
        = \bmat{p_v & q_v \\ r_v & s_v}
        = \bmat{s_v & -s_v\qty(\frac{v}{c}) \\ -s_v\qty(\frac{v}{c}) & s_v}
        = \bmat{1 & -\frac{v}{c} \\ -\frac{v}{c} & 1} s_v
    }

    All we are left is the form of $s_v$.
    From its inverse property $\mmat{\Lambda}\minv(v) = \mmat{\Lambda}(-v)$ and 
    $\mmat{\Lambda}\minv\mmat{\Lambda} = \mmat{I}$, 
    \aleq{
        \mmat{\Lambda}\minv\mmat{\Lambda} 
        = \bmat{1 & \frac{v}{c} \\ \frac{v}{c} & 1} s_{-v} \cdot 
            \bmat{1 & -\frac{v}{c} \\ -\frac{v}{c} & 1} s_v 
        = s_vs_{-v} \bmat{1 -\frac{v^2}{c^2} & 0 \\ 0 & 1 -\frac{v^2}{c^2}}
        = \mqty(\imat{2}) = \mmat{I}
    }

    $s_v$ can be ANY functional form as long as \ \fbox{$s_vs_{-v} = 1 -\dfrac{v^2}{c^2}$}\ . 
    For example, 
    \begin{itemize}
        \item $s_v=s_{-v}=\dfrac{1}{\sqrt{1-\frac{v^2}{c^2}}}$
        \item $s_v = \dfrac{1}{1-\frac{v}{c}}$ and  $s_{-v} = \dfrac{1}{1+\frac{v}{c}}$
        \item ...
    \end{itemize}
    
    Out of all the choice, we are \cul[red]{choosing} the $s_v$ which makes $\det(\mmat{\Lambda})=1$:
    \aleq{
        1 &= \det(\mmat{\Lambda}) = s_v^2\cdot \qty(1-\frac{v^2}{c^2}) \\
        \Aboxed{
            s_v &= \dfrac{1}{\sqrt{1-\frac{v^2}{c^2}}}
        }
    }

    The true reason behind this choice is because 
    we want to construct certain "spacetime" invariants that carry physical meaning.
    We shall explain more in later sections. 

    
\end{enumerate}


As a conclusion, we have derived the Lorentz transformation matrix as the form:
\aleq{
    \Aboxed{
        \mmat{\Lambda}(\gray{v})
        \ \defeq\
        \dinv{\sqrt{1-\frac{v^2}{c^2}}}
        \bmat{
            1 & -\dfrac{v}{c} \\
            -\dfrac{v}{c} & 1
        }
        \ \defeq\ \gamma_\gray{v} \bmat{1 & -\beta\\ -\beta & 1}_\gray{v}
        = \bmat{\gamma & -\gamma\beta\\ -\gamma\beta & \gamma}_\gray{v}
    }
} 

\vskip 1ex
The conventional form is denoted by these letters:
\aleq{
    \gamma = \inv{\sqrt{1-\beta^2}} 
    \quad \text{and} \quad
    \beta = \frac{v}{c}
}


%%%%%%%%%%%%%%
\subsubsection{Reading Minkowski Diagram}



The Minkowski diagram can be used to read the coordinate's value 
of the same point (event) according to different observer.

\begin{center}
    \begin{minipage}{0.4\linewidth}
        \centering
        \includegraphics[height=10em]{read_minkowski3}\\
        Coordinates seen by one observer uses the perpendicular axis
    \end{minipage}
    \hspace{0.05\textwidth}
    \begin{minipage}{0.4\linewidth}
        \centering
        \includegraphics[height=10em]{read_minkowski2}\\
        Coordinates seen by another observer uses the sloped axis
    \end{minipage}
\end{center}

To read the value on the sloped axis, 
we can draw a line that is "normal" to that axis and tells by the intercept.
Just like how we read the coordinates in rectangular and polar coordinate.

\vskip 1ex
\begin{center}
    \begin{minipage}{0.3\linewidth}
        \centering
        \includegraphics[width=\textwidth]{read_xy}
    \end{minipage}
    \begin{minipage}{0.6\linewidth}
        \begin{itemize}
            \item \blue{Find $x$ coordinate: Draw a line normal to $x$-axis\\
            $\Rightarrow$ The line needs to be parallel to $y$-axis}
            
            \item \red{Find $y$ coordinate: Draw a line normal to $y$-axis\\
            $\Rightarrow$ The line needs to be parallel to $x$-axis}

        \end{itemize}
    \end{minipage}
\end{center}

\vskip 1ex
\begin{center}
    \begin{minipage}{0.3\linewidth}
        \centering
        \includegraphics[width=\textwidth]{read_polar}
    \end{minipage}
    \begin{minipage}{0.6\linewidth}
        \begin{itemize}
            \item \blue{Find $r$ coordinate: Draw a line "normal" to $r$-axis\\
            $\Rightarrow$ The line is a circular arc, "parallel" to $\theta$-axis}
            
            \item \red{Find $\theta$ coordinate: Draw a line "normal" to $\theta$-axis\\
            $\Rightarrow$ The line is a radial line, "parallel" to $r$-axis}

        \end{itemize}
    \end{minipage}
\end{center}

It is similar for Minkowski diagram, to read the coordinate on the sloped axis:
\begin{center}
    \begin{minipage}{0.3\linewidth}
        \centering
        \includegraphics[width=\textwidth]{read_minkowski}
    \end{minipage}
    \begin{minipage}{0.6\linewidth}
        \begin{itemize}
            \item \blue{Find $t'$ coordinate: Draw a line "normal" to $\theta$-axis\\
            $\Rightarrow$ The line is a radial line, "parallel" to $r$-axis}

            \item \red{Find $x'$ coordinate: Draw a line "normal" to $x'$-axis\\
            $\Rightarrow$ The line has the same slope as the $t'$-axis}
            
        \end{itemize}
    \end{minipage}
\end{center}

\iffalse
For example, if A uses the $(ct,x)$ axes, B uses the $(ct',x')$ axes, 
and B is moving at a velocity $v$ relative to A, 
The spacetime coordinate of B are then
\begin{itemize}
    \item As seen by A: $\bmat{ct\\vt}$
    \item As seen by B: $\bmat{ct'\\0}$
\end{itemize}

\insertFig{B's position labeld by both axis}
\fi

We can also determine that the angle between $t,t'$ axes (or between $x,x'$ axes) to be
\aleq{
    \tan{\theta} = \frac{vt}{ct} 
    \qquad\Rightarrow\qquad
    \theta = \tan^{-1}\qty(\frac{v}{c}) = \tan^{-1}\beta
}



\linesep
% Section %%%%%%%%%%%%%%%%%%%%%%%%%%%%%%%%%%%%%%%%%%%%%%%%%%%%
\section{Relativistic Phenomena}

In the following section, 
we are going to examine these 4 phenomena with the matrix method:
\vskip 1em
\begin{minipage}{0.3\linewidth}
    \begin{itemize}
        \item Time dilation
        \item Length contraction
    \end{itemize}
\end{minipage}
\hspace{0.05\textwidth}
\begin{minipage}{0.5\linewidth}
    \begin{itemize}
        \item Relative velocity addition under relativity
        \item Relativistic Doppler effect 
    \end{itemize}
\end{minipage}

\vskip 1em
%%%%%%%%%%%%%%
\subsection{Time Dilation}

The standard setup is to have 2 events: "\circled{1}" and "\circled{2}"
described by two observers A, B: 
\begin{itemize}
    \item A is the \blue{\bf{"co-moving"}} observer - 
    he sees the \blue{two events happen at the same position $x$},
    but different time $t=t_1$ and $t=t_2$.

    \item B is the \red{\bf{"moving"}} observer - 
    he moves at \red{velocity $v$ relative to the co-moving observer}.  
\end{itemize} 

We can tabulate the spacetime coordinate of the two events as
\begin{center}
    \begin{tabular}{>{\centering\arraybackslash}m{0.2\linewidth}|
        >{\centering\arraybackslash}m{0.2\linewidth}|
        >{\centering\arraybackslash}m{0.35\linewidth}}
        & Seen by \blue{A} & Seen by \red{B}\\ 
        \hline \vskip 3ex 
        Event \circled{1} & 
        $\bmat{ct_1\\x}$ & 
        $\red{\bmat{\gamma & -\gamma\beta\\ -\gamma\beta & \gamma}_v}\bmat{ ct_1\\x}$
        \\[4ex] \hline \vskip 3ex 
        Event \circled{2} & 
        $\bmat{ct_2\\x}$  & 
        $\red{\bmat{\gamma & -\gamma\beta\\ -\gamma\beta & \gamma}_v}\bmat{ct_2\\x}$
        \\[4ex] \hline \vskip 1ex 
        Difference in coordinates & 
        $\bmat{c(t_2-t_1)\\\cul[blue]{\cul[blue]{\blue{0}}}}$  & 
        $\red{\bmat{\gamma & -\gamma\beta\\ -\gamma\beta & \gamma}_v}\bmat{c(t_2-t_1)\\0}$
    \end{tabular}
\end{center}

This explains what actually happens when we describe time dilation:
\begin{itemize}
    \item If the \blue{"co-moving"} observer sees two events happen with a time difference $t_2-t_1$ in between,
    \aleq{
        \bmat{c\cdot \Delta t \\ \Delta x}_\blue{A}
        = \bmat{c\cdot \cul[blue]{(t_2-t_1)}\\ \blue{0}}
    }

    \item Any \red{other moving observer}, with a speed $v$ relative to the co-moving observer, 
    will see a time difference $\gamma_v(t_2-t_1)$.
    \aleq{
        \bmat{c\cdot \Delta t \\ \Delta x}_\red{B}
        &= \bmat{\gamma & -\gamma\beta\\ -\gamma\beta & \gamma}_v\bmat{c(t_2-t_1)\\0}\\[1ex]
        &= \bmat{\gamma_v c(t_2-t_1)\\ -\gamma_v v(t_2-t_1)}\\[1ex]
        &= \bmat{c \cdot \cul[red]{\gamma_v \cdot (\text{Time diff. seen by A})}\\ \red{\text{\it{Something }}\neq 0}}
    }
\end{itemize}

Because $\gamma = \inv{\sqrt{1-\frac{v^2}{c^2}}} \geq 1$, 
this effect is described as \bf{time dilation}:

\begin{center}
    \bf{Time difference $\Delta t$ measured by a \cul[blue]{co-moving observer} is always the shortest,\\
    while \cul[red]{other observer} will measure a "longer" time difference $\gamma \Delta t$.}
\end{center}

\vskip 1em
You should remember that although the time scale is changed, 
it comes with a side effect: 
\begin{itemize}
    \item The events happens at the \blue{same position} in according to the \blue{co-moving observer},
    \item But the \red{positions are different} according to \red{other observers}.
\end{itemize}

\vskip 1em
We can visualize this effect on the Minkowski diagram:

\begin{center}
    \begin{minipage}{0.4\linewidth}
        \centering
        \ul{\blue{Co-moving} observer sees:}
    \end{minipage}
    \begin{minipage}{0.5\linewidth}
        \centering
        \phantom{aaaaaaaaa}\ul{\red{Moving} observer sees:}
    \end{minipage}
\end{center}
\begin{center}
    \begin{minipage}{0.4\linewidth}
        \centering
        \includegraphics[height=10em]{time_D1}\\
        \green{\scriptsize Two events happen at the same location}
    \end{minipage}
    \begin{minipage}{0.5\linewidth}
        \centering
        \includegraphics[height=13em]{time_D2}\\
        \phantom{\scriptsize abc}
    \end{minipage}
\end{center}

\begin{notation}[Side note:]
    In most introductory textbooks,
    "proper" is the word to describe what means by "co-moving" in this note.
    \begin{itemize}
        \item \bf{Proper observer} = The co-moving observer.
        \item \bf{Proper time} = The time difference / time scale of the co-moving observer.
        \item \bf{Proper length} = The position difference / length scale of the co-moving observer, i.e. "rest" length.
    \end{itemize}
    
    However, I personally prefer saying "co-moving" because it is the most accurate - 
    the observer and the object are literally "moving together".
    \begin{itemize}
        \item "Proper" is not a good adjective about motions (What is a "proper" motion?).
        \item "Rest" is just wrong, because the objects are not really at rest.  
    \end{itemize}
\end{notation}

%%%%%%%%%%%%%%
\subsection{Length Contraction}

The standard setup is by observing the two endings of a rod, labeled as "\circled{1}" and "\circled{2}",
by two observers A, B: 
\begin{itemize}
    \item A is the \blue{\bf{"co-moving"}} observer - 
    he \blue{moves together with the rod} - 
    the two ends of the rod are always at the same position $x=x_1$ and $x=x_2$,
    at any time $t$.

    \item B is the \red{\bf{"moving"}} observer - 
    he moves at \red{velocity $v$ relative to the co-moving observer}.  
\end{itemize} 

We can tabulate the spacetime coordinate of the two endings of the rod as
\begin{center}
    \begin{tabular}{>{\centering\arraybackslash}m{0.2\linewidth}|
        >{\centering\arraybackslash}m{0.2\linewidth}|
        >{\centering\arraybackslash}m{0.35\linewidth}}
        & Seen by \blue{A} & Seen by \red{B}\\ 
        \hline \vskip 3ex 
        \makecell{\blue{A} reads position of \\ Ending \circled{1}} & 
        $\bmat{ct\\x_1}$ & 
        $\red{\bmat{\gamma & -\gamma\beta\\ -\gamma\beta & \gamma}_v}\bmat{ct\\x_1}$
        \\[4ex] \hline \vskip 3ex 
        \makecell{\blue{A} reads position of \\ Ending \circled{2}}  & 
        $\bmat{ct\\x_2}$  & 
        $\red{\bmat{\gamma & -\gamma\beta\\ -\gamma\beta & \gamma}_v}\bmat{ct\\x_2}$
        \\[4ex] \hline \vskip 1ex 
        Difference in coordinates & 
        $\bmat{0\\x_2-x_1}$  & 
        $\red{\bmat{\gamma & -\gamma\beta\\ -\gamma\beta & \gamma}_v}\bmat{0\\x_2-x_1}$
    \end{tabular}
\end{center}

Here analyze what are observed:
\begin{itemize}
    \item If the \blue{"co-moving"} observer checks the position of the two endings at the \blue{same time},
    getting a length measurement to the rod as $x_2-x_1$:
    \aleq{
        \bmat{c\cdot \Delta t \\ \Delta x}_\blue{A}
        = \bmat{\blue{0} \\ \cul[blue]{x_2-x_1} }
    }

    \item Then according to the \red{other observer}, 
    the co-moving observers checks the positions of the two endings at different positions and \red{different time}:
    \aleq{
        \bmat{c\cdot \Delta t \\ \Delta x}_\red{B}
        &= \bmat{\gamma & -\gamma\beta\\ -\gamma\beta & \gamma}_v\bmat{0 \\ x_2-x_1}\\[1ex]
        &= \bmat{-\gamma_v \beta(x_2-x_1)\\ \gamma_v (x_2-x_1)}\\[1ex]
        &= \bmat{\red{\text{\it{Something }}\neq 0} \\ \cul[red]{\gamma_v \cdot (\text{Length measured by A})} }
    }
\end{itemize}

Obviously, the moving observer should not 
simply subtract his recorded positions to claim it as the measured length of the rod,
because the records are taken at different time!

\begin{center}
    \begin{minipage}{0.4\linewidth}
        \centering
        \ul{\blue{Co-moving} observer sees:}\\[1ex]
        Left/right record at the same time
    \end{minipage}
    \hspace{0.05\textwidth}
    \begin{minipage}{0.5\linewidth}
        \centering
        \ul{\red{Moving} observer sees:}\\[1ex]
        Record left $\rightarrow$ rod moves $\rightarrow$ record right
    \end{minipage}
\end{center}
\begin{center}
    \begin{minipage}{0.4\linewidth}
        \centering
        \includegraphics[width=0.5\textwidth]{rod1}\\
    \end{minipage}
    \hspace{0.05\textwidth}
    \begin{minipage}{0.2\linewidth}
        \centering
        \includegraphics[width=0.7\textwidth]{rod2} 
    \end{minipage}
    \quad$\Rightarrow$\quad
    \begin{minipage}{0.2\linewidth}
        \centering
        \includegraphics[width=0.85\textwidth]{rod3} 
    \end{minipage}
\end{center}

\vskip 1em
We may see it clearer with Minkowski diagram:
\begin{center}
    \begin{minipage}{0.4\linewidth}
        \centering
        \ul{\blue{Co-moving} observer sees:}
    \end{minipage}
    \begin{minipage}{0.5\linewidth}
        \centering
        \phantom{aaaaaaaaa}\ul{\red{Moving} observer sees:}
    \end{minipage}
\end{center}
\begin{center}
    \begin{minipage}{0.4\linewidth}
        \centering
        \includegraphics[height=10em]{length_C1}\\
        \green{\scriptsize Two measurements happen at the same time}
    \end{minipage}
    \begin{minipage}{0.5\linewidth}
        \centering
        \includegraphics[height=13em]{length_C2}\\
        \phantom{\scriptsize abc}
    \end{minipage}
\end{center}

For B to take correct measurement, 
we require his measurements to be taken \red{at the same time}. 
Then we can use the inverse Lorentz transform to tell what is observed by A:
\begin{center}
    \begin{tabular}{>{\centering\arraybackslash}m{0.2\linewidth}|
        >{\centering\arraybackslash}m{0.35\linewidth}|
        >{\centering\arraybackslash}m{0.2\linewidth}}
        & Seen by \blue{A} & Seen by \red{B}\\ 
        \hline \vskip 3ex 
        \makecell{\red{B} reads position of \\ Ending \circled{1}} & 
        $\blue{\bmat{\gamma & -\gamma\beta\\ -\gamma\beta & \gamma}_{-v}}\bmat{ct'\\x'_1}$ &
        $\bmat{ct'\\x'_1}$
        \\[4ex] \hline \vskip 3ex 
        \makecell{\red{B} reads position of \\ Ending \circled{2}}  & 
        $\blue{\bmat{\gamma & -\gamma\beta\\ -\gamma\beta & \gamma}_{-v}}\bmat{ct'\\x'_2}$ &
        $\bmat{ct'\\x'_2}$ 
        \\[4ex] \hline \vskip 1ex 
        Difference in coordinates & 
        $\blue{\bmat{\gamma & -\gamma\beta\\ -\gamma\beta & \gamma}_{-v}}\bmat{0\\x'_2-x'_1}$ &
        $\bmat{\cul[red]{\cul[red]{0}}\\x'_2-x'_1}$
    \end{tabular}
\end{center}

But note that \blue{A} is the co-moving observer -
he will always find the position of Ending \circled{1} at $x=x_1$ and Ending \circled{2} at $x=x_2$,
at ANY time! So we must have
\aleq{
    \bmat{\gamma & -\gamma\beta\\ -\gamma\beta & \gamma}_{-v}\bmat{ct'\\x'_1}
    = \bmat{\cdots \\ \cul[blue]{\cul[blue]{x_1}}}
    \quad\text{and}\quad
    \bmat{\gamma & -\gamma\beta\\ -\gamma\beta & \gamma}_{-v}\bmat{ct'\\x'_2}
    = \bmat{\cdots \\ \cul[blue]{\cul[blue]{x_2}}}
}

The difference in coordinates give
\aleq{
    \bmat{c\cdot \Delta t\\\Delta x}_\blue{A}
    = \bmat{\cdots \\ x_2-x_1}
    &= \bmat{\gamma & -\gamma\beta\\ -\gamma\beta & \gamma}_{-v}\bmat{0\\x'_2-x'_1}\\[1ex]
    &= \bmat{-\gamma_{-v}\beta (x'_2 - x'_1) \\ \gamma_{-v}(x'_2-x'_1)}\\[1ex]
    &= \bmat{\blue{\text{\it{Something }}\neq 0} \\ \cul[blue]{\gamma_{-v}\cdot (\text{Length measured by B})}}
}

\vskip 1ex
i.e. If the co-moving observer measure a length $x_2-x_1$, 
any moving observer will measure a length $\inv{\gamma_{-v}}(x_2-x_1)$. 
Because $\inv{\gamma_v} = \sqrt{1-\frac{v^2}{c^2}} \leq 1$, 
this effect is described as \bf{length contraction}:

\begin{center}
    \bf{Position difference $\Delta x$ measured by a \cul[blue]{co-moving observer} is always the longest,\\
    while \cul[red]{other observer} will measure a "smaller" length difference $\inv{\gamma} \Delta x$.}
\end{center}

\vskip 1ex
You should remember that although the length scale is changed, 
it comes with a side effect: 
\begin{itemize}
    \item The positions of endings are recorded at the \red{same time} in according to the \red{moving observer},
    \item But the \blue{record time are different} according to \blue{co-moving observer}.
\end{itemize}

We can visualize this effect using the Minkowski diagram:
\begin{center}
    \begin{minipage}{0.4\linewidth}
        \centering
        \ul{\blue{Co-moving} observer sees:}
    \end{minipage}
    \begin{minipage}{0.5\linewidth}
        \centering
        \phantom{aaaaaaaaa}\ul{\red{Moving} observer sees:}
    \end{minipage}
\end{center}
\begin{center}
    \begin{minipage}{0.4\linewidth}
        \centering
        \includegraphics[height=10em]{length_C3}\\
        \green{\scriptsize Positions of the 2 ends never change}
    \end{minipage}
    \begin{minipage}{0.5\linewidth}
        \centering
        \includegraphics[height=13em]{length_C4}\\
        \phantom{\scriptsize abc}
    \end{minipage}
\end{center}


%%%%%%%%%%%%%%
\subsection{Velocity Addition}

Given 3 observers who are moving relative to each other:\\
\begin{minipage}{0.5\linewidth}
    \begin{itemize}
        \item B is moving at velocity $v$ relative to A
        \item C is moving at velocity $u$ relative to B
        \item C is moving at velocity $w$ relative to A 
    \end{itemize}
\end{minipage}
\hspace{0.05\textwidth}
\begin{minipage}{0.4\linewidth}
    \centering
    \includegraphics[width=\textwidth]{relative_v}
\end{minipage}

What are the relations between $v$, $u$ and $w$?
We can express the coordinate of C in terms of B's axes (ct', x'),
and then transform it through $\mmat{\Lambda}_{-v}$ to what is observed by A:
\begin{center}
    \begin{tabular}{>{\centering\arraybackslash}m{0.2\linewidth}|
        >{\centering\arraybackslash}m{0.2\linewidth}|
        >{\centering\arraybackslash}m{0.5\linewidth}}
        & C seen by \red{B} & C seen by \blue{A} \\ 
        \hline \vskip 1ex 
        When B's clock shows $t'=0$ & 
        $\bmat{0\\0}$ &
        $\blue{\bmat{\gamma & -\gamma\beta \\ -\gamma\beta & \gamma}_{-v}}\bmat{0\\0}= \bmat{0\\0}$
        \\[4ex] \hline \vskip 1ex 
        When B's clock shows $t'=T'$ & 
        $\bmat{ cT\\uT }$ &
        $\blue{\bmat{\gamma & -\gamma\beta \\ -\gamma\beta & \gamma}_{-v}}\bmat{cT\\uT} 
            = \bmat{\gamma_v cT+\gamma_v \frac{vu}{c}T \\ \gamma_v vT + \gamma_v uT}$
    \end{tabular}
\end{center}

Since C is moving at velocity $w$ relative to A, 
\aleq{
    w &= \frac{(\text{Change in position seen by A})}{(\text{Change in time seen by A})} \\[1ex]
    &= \frac{\gamma_v vT + \gamma_v uT}{\gamma_vT + \gamma_v \frac{vu}{c^2}T}\\[1ex]
    \Aboxed{
        w&= \frac{v+u}{1+\frac{vu}{c^2}}
    }
}

This is the relative velocity addition formula in relativity.
\vskip 1em
\begin{center}
    \begin{minipage}{0.4\linewidth}
        \centering
        \ul{Trajectory of C seen by \blue{A}}\\[1ex]
        \includegraphics[height=10em]{v_add1}\\
        Distance travel = $x_2-x_1$\\
        Time spent = $t_2-t_1$
    \end{minipage}
     \hspace{0.05\textwidth}
    \begin{minipage}{0.4\linewidth}
        \centering
        \ul{Trajectory of C seen by \red{B}}\\[1ex]
        \includegraphics[height=10em]{v_add2}\\
        Distance travel = $x'_2-x'_1$\\
        Time spent = $t'_2-t'_1$
    \end{minipage}
\end{center}
\vskip 1em
\begin{center}
    \begin{minipage}{0.4\linewidth}
        \centering
        \ul{Trajectory of \red{B} seen by \blue{A}}\\[1ex]
        \includegraphics[height=10em]{v_add3}\\
        Distance travel = $x''_2-x''_1$\\
        Time spent = $t''_2-t''_1$
    \end{minipage}
    %\hspace{0.05\textwidth}
    \begin{minipage}{0.45\linewidth}
        \centering
        In Minkowski diagram, \\
        speed is calculated as $\frac{\Delta x}{\Delta t} = \frac{c}{\text{Slope}}$ of a line.\\
        \vskip 1em
        Geometrically, velocity addition formula\\
        is the relation between the slopes\\ relative to different axes.  
    \end{minipage}
\end{center}


\begin{notation}[Side note:]
    Alternatively, we can show the velocity addition formula 
    by using Lorentz matrix as a tool to switch frame of reference. 
    
    \begin{itemize}
        \item For any coordinate observed by A, 
        we can multiply $\mmat{\Lambda}_\red{v}$ to change it into what is observed by B.
        \aleq{
            \bmat{ct_B\\x_B} = \mmat{\Lambda}_\red{v}\bmat{ct_A \\ x_A} 
        }

        \item For any coordinate observed by B, 
        we can multiply $\mmat{\Lambda}_\red{u}$ to change it into what is observed by C.
        \aleq{
            \bmat{ct_C\\x_C} = \mmat{\Lambda}_\red{u}\bmat{ct_B \\ x_B} 
        }

        \item For any coordinate observed by A, 
        we can multiply $\mmat{\Lambda}_\red{w}$ to change it into what is observed by C.
        \aleq{
            \bmat{ct_C\\x_C} = \mmat{\Lambda}_\red{w}\bmat{ct_A \\ x_A} 
        }
    \end{itemize}

    This gives a relation between different $\mmat{\Lambda}$:
    \aleq{
        \bmat{ct_C\\x_C} = \mmat{\Lambda}_\red{w}\bmat{ct_A \\ x_A}
        &= \mmat{\Lambda}_\red{u}\mmat{\Lambda}_\red{v}\bmat{ct_A \\ x_A} \\[1ex]
        \mmat{\Lambda}_\red{w} &= \mmat{\Lambda}_\red{u}\mmat{\Lambda}_\red{v}
    }

    We can use any of the entry to reach the velocity addition formula,
    for example,
    \aleq{
        \bmat{\gamma_w & \cdots\\\cdots & \cdots}
        &= \bmat{\gamma_u & -\gamma_u\beta_u\\\cdots & \cdots}
        \bmat{\gamma_v & \cdots\\-\gamma_v\beta_v & \cdots}\\[1ex]
        \gamma_w &= \gamma_u\gamma_v + \gamma_u\beta_u \gamma_v\beta_v \\[1ex]
        \inv{\sqrt{1-\frac{w^2}{c^2}}} 
        &= \inv{\sqrt{1-\frac{u^2}{c^2}}} \inv{\sqrt{1-\frac{v^2}{c^2}}}\qty(1+\frac{uv}{c^2})\\[1ex]
        \Rightarrow\quad w &= \frac{u+v}{1+\frac{uv}{c^2}}
    }
\end{notation}


%%%%%%%%%%
\subsection{Relativistic Doppler Effect}

Relativistic Doppler effect is only applicable to light - 
explaining what will be observed by different observers for an object that \it{by definition}
travels at the same speed to all observer. 
For objects that travels slower than light, 
you should use the classical Doppler effect formula.\\

Recall the terminologies in wave:
\begin{itemize}
    \item Period $T$ = Time separation ($\sim \Delta t$) between each pulse emission.
    \item Wavelength $\lambda$ = Distance travelled ($\sim \Delta x$) of the pulse within 1 period of time.
\end{itemize}

For light, its travelling speed $c$ is a constant to all observers. 
This makes $\lambda = cT$ and $\lambda'=cT'$. 
\begin{center}
    \begin{tabular}{>{\centering\arraybackslash}m{0.2\linewidth}|
        >{\centering\arraybackslash}m{0.2\linewidth}|
        >{\centering\arraybackslash}m{0.5\linewidth}}
        & Light seen by \blue{A} & Light seen by \red{B}  \\ 
        \hline \vskip 1ex 
        When A's clock shows $t=0$ & 
        $\bmat{0\\0}$ &
        $\red{\bmat{\gamma & -\gamma\beta \\ -\gamma\beta & \gamma}_v}\bmat{0\\0}= \bmat{0\\0}$
        \\[4ex] \hline \vskip 1ex 
        When A's clock shows $t= T$ & 
        $\bmat{ cT\\cT }$ &
        $\red{\bmat{\gamma & -\gamma\beta \\ -\gamma\beta & \gamma}_v}\bmat{cT\\cT} 
            = \cul[gray]{\cul[gray]{\bmat{cT' \\ cT'}}}$
    \end{tabular}
\end{center}

Therefore the observed period by B is 
\aleq{
    cT' &= \gamma (1-\beta)cT \\
    T' &= \frac{1-\beta}{\sqrt{1-\beta^2}}T \\
    &= \sqrt{\frac{1-\beta}{1+\beta}}T
}

and so the observed wavelength is
\aleq{
    \lambda' = cT' &= \sqrt{\frac{1-\beta}{1+\beta}}cT = \sqrt{\frac{1-\beta}{1+\beta}}\lambda
}

\vskip 1em
\begin{center}
    \begin{minipage}{0.4\linewidth}
        \centering
        \ul{Trajectory of light seen by \blue{A}}\\[1ex]
        \includegraphics[height=10em]{doppler1}\\
        Distance travel = $x_2-x_1$\\
        Time spent = $t_2-t_1$
    \end{minipage}
     \hspace{0.05\textwidth}
    \begin{minipage}{0.4\linewidth}
        \centering
        \ul{Trajectory of light seen by \red{B}}\\[1ex]
        \includegraphics[height=10em]{doppler2}\\
        Distance travel = $x'_2-x'_1$\\
        Time spent = $t'_2-t'_1$
    \end{minipage}
\end{center}

\vskip 1em
%%%
\theend
\end{document}