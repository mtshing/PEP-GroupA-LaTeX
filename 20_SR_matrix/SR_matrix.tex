\documentclass[class=article, crop=false, 12pt]{standalone}
\usepackage[subpreambles=true]{standalone}
\usepackage{../.common/common}


\author{Tony Shing}
%\pretitle{Supplementary}

\topic{T20 (Relativity)}
\title{Matrix Method for Special Relativity}

\version{2025} % leave blank for omitting

\begin{document}

\maketitle


\begin{overview}
    \begin{itemize}
        \item Lorentz transformation matrix
        \item Explaining relativistic phenomena: 
        \begin{itemize}
            \item Time dilation
            \item Length contraction
            \item Velocity addition
        \end{itemize}
    \end{itemize}
\end{overview}



% content begins here
% Section %%%%%%%%%%%%%%%%%%%%%%%%%%%%%%%%%%%%%%%%%%%%%%%%%%%%
\section{Lorentz Transform}

%%%%%%%%%%%%%%
\subsection{Matrix as Linear Transformation}

To begin with, first you need to be aware of that matrix can be used to describe transformation of coordinates:


%%%%%%%%%%%%%%
\subsection{Lorentz Transformation Matrix}

%%%%%%%%%%%%%%
\subsubsection{2 Einstein's Posulates}

%%%%%%%%%%%%%%
\subsubsection{Spacetime Coordinate}

%%%%%%%%%%%%%%
\subsubsection{The Derivation}


\linesep
% Section %%%%%%%%%%%%%%%%%%%%%%%%%%%%%%%%%%%%%%%%%%%%%%%%%%%%
\section{Relativistic Phenomena}

%%%%%%%%%%%%%%
\subsection{Time Dilation}

%%%%%%%%%%%%%%
\subsection{Length Contraction}

%%%%%%%%%%%%%%
\subsection{Velocity Addition}

%%%
\theend
\end{document}