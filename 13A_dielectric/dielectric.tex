\documentclass[class=article, crop=false, 12pt]{standalone}
\usepackage[subpreambles=true]{standalone}
\usepackage{../.common/common}


\author{Tony Shing}
%\pretitle{Supplementary}

\topic{T13A (Electromagnetism)}
\title{Dielectric \& Polarization}

\version{2025} % leave blank for omitting

\begin{document}

\maketitle


\begin{overview}
    \begin{itemize}
        \item Model of electric dipole
        \item How to describe dipole arrangement: Polarization field, bound charges
        \item Material under external $\vvec{E}$ field: 
        \begin{itemize}
            \item A special case of material model: \cul[red]{Linear dielectric} 
            \item Describe external field: Free charges, electric displacement field $\vvec{D}$
        \end{itemize}
    \end{itemize}
    \begin{center}
        \red{\bf{\ul{If linear dielectric is all you need, you can skip most of the maths.}}}
    \end{center}
\end{overview}




% content begins here
% Section %%%%%%%%%%%%%%%%%%%%%%%%%%%%%%%%%%%%%%%%%%%%%%%%%%%%
\section{Electric Dipole}

%%%%%%%%%%%%%%
\subsection{Electric Response in Insulators}

From chemistry, we have learnt that atoms in non-metalic substance join together by sharing electrons.
The difference in atoms' electronegativity (i.e. strength of attracting outer shell electrons) 
causes uneven charge distribution around the atoms. For example,

\begin{itemize}
    \item Ionic bond - One atom completely takes an electron from another, 
    forming a pair of positive and negative ions.
    \insertFig{ionic bond}

    \item Covalent bond - Bonding electrons are shared by atoms, 
    but it may be pulled closer to one atom than the other. 
    \insertFig{covalent bond}

\end{itemize}

In introductory physics, we will ignore the complicated atomic structure 
and treat the material to be made of units of \bf{electric dipoles},
which is made of a positive charged region and a negative charged region, 
bounded within a small distance.

\insertFig{complicated structure to single electric dipole}

To model electric responses in non-metalic materials (i.e. insulators, also called \bf{dielectrics}), 
we can first calculate the potential and field from a single electric dipole,
then sum the contributions from all dipoles based on the material configurations.

\insertFig{stacking dipoles in ionic bond/ covalent bond}




%%%%%%%%%%%%%%
\subsection{Potential from Electric Dipole}

This is the classical model of electric dipole.
Same calculations can be found in many textbooks.

\begin{itemize}
    \item Both ends of the dipole have equal but opposite \it{point} charge $+q$ and $-q$.
    \item Separation between the charges is labelled by a vector $\vvec{d}$,
    which points from the negative charge to the positive charge.
    \item We are only interested in the potential/field far away from the dipole, 
    i.e. $\norm{\vvec{r}} \gg \norm{\vvec{d}}$.
    The position vector $\vvec{r}$ is referenced from the center of the dipole.
\end{itemize}

\insertFig{electric dipole model}

The derivation of electric potential from a dipole is straightforward by Columb's law,
while applying Taylor approximation in the middle step. 

\begin{enumerate}
    \item The total potential contributions from the two charges towards position $\vvec{r}$ is
    \aleq{
        V(\vvec{r}) 
        &= \frac{1}{4\pi\epsilon_0} \qty( \frac{q}{\norm{\vvec{r}_+}} 
        + \frac{-q}{\norm{\vvec{r}_-}})\\
        &= \frac{q}{4\pi\epsilon_0} 
            \qty(\frac{1}{\tkn{cosLaw1}{\cul[blue]{\sqrt{\norm{\vvec{r}}^2 + \qty|\frac{\vvec{d}}{2}|^2 - 2\norm{\vvec{r}}\qty|\frac{\vvec{d}}{2}|\cos\theta}}}}
            - \frac{1}{\tkn{cosLaw2}{\cul[blue]{\sqrt{\norm{\vvec{r}}^2 + \qty|\frac{\vvec{d}}{2}|^2 + 2\norm{\vvec{r}}\qty|\frac{\vvec{d}}{2}|\cos\theta}}}})\\[2ex]
            %
        &= \frac{q}{4\pi\epsilon_0 \norm{\vvec{r}}}
            \qty(\inv{\sqrt{1 + \inv{4}\frac{\norm{\vvec{d}}^2}{\norm{\vvec{r}}^2} - \frac{\norm{\vvec{d}}}{\norm{\vvec{r}}}\cos\theta}}
            - \inv{\sqrt{1 + \inv{4}\frac{\norm{\vvec{d}}^2}{\norm{\vvec{r}}^2} + \frac{\norm{\vvec{d}}}{\norm{\vvec{r}}}\cos\theta}})
    }
    \addBentArrow[blue]{cosLaw1}{(8ex,-2ex)}{\scriptsize These are just cosine law}{(0,-3ex)}{(7.1ex,0.5ex)}
    \addBentArrow[blue]{cosLaw2}{(-8ex,-2ex)}{}{(0,-3ex)}

    \item Taylor expansion ``\cul[red]{$(1+x)^n \approx 1+nx$ for $x\ll 1$}" is always applied at this step. 
    Because we are only interested in the potential far from the dipole, $\frac{\norm{\vvec{d}}}{\norm{\vvec{r}}} \ll 1$, 
    \begin{itemize}
        \item $\inv{4}\frac{\norm{\vvec{d}}^2}{\norm{\vvec{r}}^2}\ \approx 0$
        \item $\inv{\sqrt{1 - \frac{\norm{\vvec{d}}}{\norm{\vvec{r}}}\cos\theta}} 
            = \qty(1 - \frac{\norm{\vvec{d}}}{\norm{\vvec{r}}}\cos\theta)^{-\half}
            \approx 1 + \half \cdot \frac{\norm{\vvec{d}}}{\norm{\vvec{r}}}\cos\theta$ 
    \end{itemize}

    So the potential becomes
    \aleq{
        V(\vvec{r}) &= \frac{q}{4\pi\epsilon_0 \norm{\vvec{r}}}
            \qty[\qty(1 + \half \frac{\norm{\vvec{d}}}{\norm{\vvec{r}}}\cos\theta)
            - \qty(1 - \half \frac{\norm{\vvec{d}}}{\norm{\vvec{r}}}\cos\theta)]\\[1ex]
        &= \frac{q}{4\pi\epsilon_0 \norm{\vvec{r}}}
            \, \frac{\norm{\vvec{d}}}{\norm{\vvec{r}}}\cos\theta
    }
    
    \item From geometry, $\theta$ is the angle between $\vvec{d}$ and $\vvec{r}$. 
    We can replace $\cos\theta$ by $\cos\theta = \frac{\vvec{d}\cdot\vvec{r}}{\norm{\vvec{d}}\norm{\vvec{r}}}$\
    \aleq{
        V(\vvec{r}) 
        &= \frac{q}{4\pi\epsilon_0}\, \frac{\norm{\vvec{d}}}{\norm{\vvec{r}}^2} 
            \, \frac{\vvec{d}\cdot\vvec{r}}{\norm{\vvec{d}}\norm{\vvec{r}}}\\
        &= \frac{1}{4\pi\epsilon_0} \, \frac{q\vvec{d}}{\norm{\vvec{r}}^2} 
            \, \cul[blue]{\qty(\tkn{unitVecR}{\frac{\vvec{r}}{\norm{\vvec{r}}}})}
    }
    \addBentArrow[blue]{unitVecR}{(2ex,-2ex)}{\scriptsize Unit vector of $\vvec{r}$}{(0,-3ex)}{(5.5ex,0.7ex)}
    
    \item To emphasize that we now treat an electric dipoles as ``one unit of source", 
    We define the \bf{electric dipole moment} $\vvec{p}$,
    \aleq{
        \Aboxed{\vvec{p}\ \defeq\ q\vvec{d}}
    }

    Finally we reach the standard formula of electric potential from an electric dipole:
    \aleq{
        \Aboxed{
            V(\vvec{r}) = \frac{1}{4\pi\epsilon_0}\frac{\vvec{p}\cdot\hhat{r}}{\norm{\vvec{r}}^2}
        }
    }
\end{enumerate}

As a comparison with Columb's law formula from point charge 

\aleq{
    V(\vvec{r}) = \frac{1}{4\pi\epsilon_0}\frac{q}{\norm{\vvec{r}}}
}

\insertFig{V from point charge source}

\begin{itemize}
    \item Charge is just a number, not directional.
    \item Dependance on distance is $\inv{r}$.
\end{itemize}

\aleq{
    V(\vvec{r}) = \frac{1}{4\pi\epsilon_0}\frac{\vvec{p}\cdot\hhat{r}}{\norm{\vvec{r}}^2}
}

\insertFig{V from dipole source}

\begin{itemize}
    \item Dipole is directional. 
    Magnitude of potential depends on the charges' magnitude AND the angle (dot product) between $\vvec{p}$ and $\hhat{r}$.
    \item Dependance on distance is $\inv{r^2}$.
\end{itemize}



%%%%%%%%%%%%%%
\subsection{E-field from Electric Dipole}

The standard derivation of E-field from dipole is through the relation $\vvec{E} = -\grad V$.
It is nothing more than some boring calculus.  
Here I quote the final result:
\aleq{
    \Aboxed{
        \vvec{E}(\vvec{r}) = \frac{1}{4\pi\epsilon_0}\frac{3(\vvec{p}\cdot\hhat{r})\hhat{r}-\vvec{p}}{\norm{\vvec{r}}^3}
    }
}

And this formula is rarely used because vector calculation is annoying. 

\begin{notation}[The boring derivation:]
    \begin{enumerate}
        \item Derivation is possible with high school calculus if we turn all vectors into x-y-z form: 
        $\vvec{r}= x\hhat{x} + y\hhat{y} + z\hhat{z}$, $\norm{\vvec{r}} = \sqrt{x^2 + y^2 + z^2}$
        and $\vvec{p} = p_x \hhat{x} + p_y \hhat{y} + p_z \hhat{z}$:
        \aleq{
            \frac{\vvec{p}\cdot\hhat{r}}{\norm{\vvec{r}}^2}
            \, =\, \frac{\vvec{p}}{\norm{\vvec{r}}^2}\cdot\frac{\vvec{r}}{\norm{\vvec{r}}}
            \, =\, \frac{1}{\norm{\vvec{r}}^3}(\vvec{p} \cdot \vvec{r})
            \, = \, \frac{1}{(x^2 + y^2 + z^2)^{\frac{3}{2}}}(p_x x + p_y y + p_z z)
        }

        \item First the $x$ term in gradient, $\hhat{x}\pdvv{x}(\cdots)$, can be calculated by:
        \addBentArrow[blue]{quotient}{(-6ex,6ex)}{\scriptsize This is just quotient rule}{(0,3ex)}{(16ex,-3ex)}
        \addArrow[blue]{dipoleE1}{(20ex,5.7ex)}{}{(2ex,1.7ex)}
        \addArrow[blue]{dipoleE2}{(30ex,5.7ex)}{}{(3ex,1.7ex)}
        \aleq{
            &\hhat{x}\pdvv{x}\qty[\frac{1}{(x^2 + y^2 + z^2)^{\frac{3}{2}}}(p_x x + p_y y + p_z z)]\\[0.5ex]
            =\ &\frac{\hhat{x}}{(x^2 + y^2 + z^2)^3}\qty[p_x (x^2 + y^2 + z^2)^{\frac{3}{2}} - (p_x x + p_y\tkm{quotient} y + p_z z) \cdot 3x (x^2 + y^2 + z^2)^{\half}]\\[0.5ex]
            =\ &\frac{1}{(x^2 + y^2 + z^2)^\red{\frac{5}{2}}}\qty[p_x\hhat{x} \cul[blue]{(x^2 + y^2 + z^2)} - \cul[blue]{(p_x x + p_y y + p_z z)} \cdot 3x\hhat{x}]\\[1ex]
            =\ &\frac{1}{\norm{\vvec{r}}^5}\qty[p_x\hhat{x} \tkn{dipoleE1}{\cul[blue]{\norm{\vvec{r}}^2}} - 3\tkn{dipoleE2}{\cul[blue]{(\vvec{p}\cdot\vvec{r})}}x\hhat{x}]
        }
        
        \item Similar for $y$ and $z$ terms:
        \aleq{
            \hhat{y}\pdvv{y}(\cdots) &= \frac{1}{\norm{\vvec{r}}^5}\qty[p_y\hhat{y} \norm{\vvec{r}}^2 - 3 (\vvec{p}\cdot\vvec{r})y\hhat{y}]\\
            \hhat{z}\pdvv{z}(\cdots) &= \frac{1}{\norm{\vvec{r}}^5}\qty[p_z\hhat{z} \norm{\vvec{r}}^2 - 3 (\vvec{p}\cdot\vvec{r})z\hhat{z}]
        }

        \item Summing all 3 components gives:
        \addArrow[blue]{dipoleE3}{(37ex,5.7ex)}{}{(1ex,1.7ex)}
        \addArrow[blue]{dipoleE4}{(-1ex,5ex)}{}{(0,2ex)}
        \addBentArrow[red]{dipoleE5}{(7ex,2.5ex)}{\scriptsize Take out magnitude to become unit vector}{(0,2ex)}{(14ex,-0.7ex)}
        \aleq{
            -\grad V(\vvec{r}) &= -\qty(\gradRec{V})\\[1ex]
            &= -\inv{4\pi\epsilon_0} \inv{\norm{\vvec{r}}^5}
                \qty[\cul[blue]{(p_x\hhat{x}+p_y\hhat{y}+p_z\hhat{z})} \norm{\vvec{r}}^2 - 3(\vvec{p}\cdot\vvec{r})\cul[blue]{(x\hhat{x}+y\hhat{y}+z\hhat{z})}]\\[1ex]
            &= \inv{4\pi\epsilon_0} \inv{\norm{\vvec{r}}^5} \qty[3(\vvec{p}\cdot\vvec{r})\tkn{dipoleE3}{\cul[blue]{\vvec{r}}} - \tkn{dipoleE4}{\cul[blue]{\vvec{p}}}\norm{\vvec{r}}^2]\\[1ex]
            &= \inv{4\pi\epsilon_0} \inv{\norm{\vvec{r}}^5} \qty[3(\vvec{p}\cdot\red{\hhat{r}})\tkm{dipoleE5}\red{\hhat{r}}\,\red{\norm{\vvec{r}}^2} - \vvec{p}\norm{\vvec{r}}^2]\\
            \Aboxed{
                \vvec{E}(\vvec{r}) &= \inv{4\pi\epsilon_0} \frac{3(\vvec{p}\cdot\hhat{r})\hhat{r} - \vvec{p}}{\norm{\vvec{r}}^3}
            }
        }
    \end{enumerate}
        
\end{notation}



\linesep
% Section %%%%%%%%%%%%%%%%%%%%%%%%%%%%%%%%%%%%%%%%%%%%%%%%%%%%
\section{Describing Dipole Arrangement}

In general, all dielectric materials are made of many tiny electric dipoles.
To analyze the material's electric properties, 
we can begin with the property of a single dipole,
then sum the contributions of all dipoles according to the dipole arrangement in the material.

\insertFig{NaCl, H2O by dipole arrangement}

For example, the electric potential created by the material can be treated as a sum of potentials from all dipoles.
\aleq{
    \cub[blue]{V(\vvec{r}) = \inv{4\pi\epsilon_0} \frac{\vvec{p} }{\norm{\vvec{r}}^2} \cdot \qty(\frac{\vvec{r}}{\norm{\vvec{r}}})}
        {\text{By a single dipole from origin}}
    \qquad\Rightarrow\qquad
    \cub[red]{V(\vvec{r}) = \inv{4\pi\epsilon_0} \sum_{\substack{\text{All dipoles } i}} 
        \frac{\vvec{p}_i }{\norm{\vvec{r}-\vvec{r}_i}^2} \cdot \qty(\frac{\vvec{r}-\vvec{r}_i}{\norm{\vvec{r}-\vvec{r}_i}})}
        {\text{By many dipoles at different positions }\vvec{r}_i}
}

\insertFig{single dipole point to r vs many dipoles at different ri to r}

Here we introduce two quantities that describe electric dipole arrangements in materials.
\begin{itemize}
    \item \bf{Polarization field} - $\vvec{P}(\vvec{r})$
    \item \bf{Bound charge distributions} - $\rho_b(\vvec{r})$ and $\sigma_b(\vvec{r})$
\end{itemize} 


%%%%%%%%%%%%%%
\subsection{Polarization Field}

When the dipoles in the material are so dense such that we can treat them as a continuous distribution,
we can replace 
\begin{itemize}
    \item Sum of all dipoles \ $\xRightarrow{\text{become}}$\  Volume integral over the whole object.
    \item Discrete dipoles source \ $\xRightarrow{\text{become}}$\ A vector distribution called \bf{polarization field} $\vvec{P}(\vvec{r})$.
\end{itemize}
\aleq{
    \Aboxed{
        V(\vvec{r}) = \inv{4\pi\epsilon_0} \underset{\substack{\substack{\text{Whole}\\\text{material}}}}{\iiint}
        \frac{\vvec{P}(\vvec{r}')}{\norm{\vvec{r}-\vvec{r}'}^2} \cdot \qty(\frac{\vvec{r}-\vvec{r}'}{\norm{\vvec{r}-\vvec{r}'}}) \dd[3]{\vvec{r}'}
    }
}

\insertFig{dispole arrangement is similar to some vector field in the object}

The polarization field $\vvec{P}(\vvec{r})$ can also be interpreted as \bf{electric dipole density} 
because its usage is similar to charge density $\rho(\vvec{r})$. 
Comparing with the Coulomb's law for electric potential:
\begin{itemize}
    \item When the source is made of charges:
    \aleq{
        V(\red{\vvec{r}}) = \inv{4\pi\epsilon_0}{\iiint}
            \frac{\rho(\blue{\vvec{r}'})}{\norm{\red{\vvec{r}}-\blue{\vvec{r}'}}} \dd[3]{\blue{\vvec{r}'}}
        %
        \quad\sim\quad \inv{4\pi\epsilon_0} \sum \frac{(\text{charge density})}{(\text{distance})}
    }

    \item When the source is made of dipoles:
    \aleq{
        V(\red{\vvec{r}}) = \inv{4\pi\epsilon_0}{\iiint}
            \frac{\vvec{P}(\blue{\vvec{r}'})}{\norm{\red{\vvec{r}}-\blue{\vvec{r}'}}^2}\cdot 
            \qty(\frac{\red{\vvec{r}}-\blue{\vvec{r}'}}{\norm{\red{\vvec{r}}-\blue{\vvec{r}'}}})\dd[3]{\blue{\vvec{r}'}}
        %
        \quad\sim\quad \inv{4\pi\epsilon_0} \sum \frac{(\text{dipole density})}{(\text{distance})^2}\cdot \qty(\substack{\text{unit}\\\text{vector}})
    }
\end{itemize}

\insertFig{charge density vs dipole density in the same material}


%%%%%%%%%%%%%%
\subsection{Bound Charge Distribution}

Ultimately, electric dipoles are just pairs of charges.
If the dipoles are aligned non-uniformly in a material,
some regions may appear to have a higher density of positive heads than its surroundings, 
and so as negative tails.

\insertFig{dipole heads and tails non uniform}

These regions with extra dipole heads / tails are described as \bf{bound charges} distribution in the material -
they are always ``bounded" to regions where there are more dipole heads/tails.
So bound charge distribution can be used to describe electric dipole arrangement in the material.

%%%%%%%%%%%%%%
\subsubsection{Mathematical Origin}

With vector calculus, 
the potential formula can be rewritten into a ``charge densities form".
\aleq{
    V(\vvec{r}) &= \inv{4\pi\epsilon_0} \underset{\substack{\substack{\text{Whole}\\\text{material}}}}{\iiint}
        \frac{\vvec{P}(\vvec{r}')}{\norm{\vvec{r}-\vvec{r}'}^2} \cdot \qty(\frac{\vvec{r}-\vvec{r}'}{\norm{\vvec{r}-\vvec{r}'}}) \dd[3]{\vvec{r}'}\\[1ex]
    %
    &= (\cdots \text{ \it{After more boring vector calculus} } \cdots)\\[1ex]
    %
    &= \inv{4\pi\epsilon_0} \underset{\substack{\substack{\text{Whole}\\\text{material}}}}{\iiint}
        \frac{-\div \vvec{P}(\vvec{r}')}{\norm{\vvec{r}-\vvec{r}'}} \dd[3]{\vvec{r}'}
        + \inv{4\pi\epsilon_0} \underset{\substack{\substack{\text{Surface}\\\text{of material}}}}{\oiint}
        \frac{\vvec{P}(\vvec{r}')}{\norm{\vvec{r}-\vvec{r}'}} \cdot \dd[2]{\vvec{r}'}
}

By comparing with the Coulomb's law for electric potential 
$V\sim \inv{4\pi\epsilon_0} \sum \frac{(\text{charge density})}{(\text{distance})}$,
we identify the 2 source terms as the \bf{bound charge densities}:
\begin{itemize}
    \item \bf{\ul{The \nth{1} term:}} \quad
    \aleq{
        \inv{4\pi\epsilon_0} \underset{\substack{\substack{\text{Whole}\\\text{material}}}}{\iiint}
            \frac{-\div \vvec{P}(\vvec{r}')}{\norm{\vvec{r}-\vvec{r}'}} \dd[3]{\vvec{r}'}
        \quad\sim\quad
        \inv{4\pi\epsilon_0} \sum_{\substack{\text{Inside}\\\text{material}}} \frac{(-\div \vvec{P})}{(\text{distance})}
    }

    \cul[red]{$-\div \vvec{P}$} appears as some charge density distributed \cul[red]{inside} the material. 
    Therefore it is defined as the \bf{volume bound charge density} $\rho_b$.
    \aleq{
        \Aboxed{
            \rho_b(\vvec{r}) \ \defeq\ -\div \vvec{P}(\vvec{r})
        }
    }

    \vskip 1ex
    \item \bf{\ul{The \nth{2} term:}} \quad
    \aleq{    
        \inv{4\pi\epsilon_0} \underset{\substack{\substack{\text{Surface}\\\text{of material}}}}{\oiint}
        \frac{\vvec{P}(\vvec{r}')}{\norm{\vvec{r}-\vvec{r}'}} \cdot \dd[2]{\vvec{r}'}
        \quad\sim\quad
        \inv{4\pi\epsilon_0} \sum_{\substack{\text{On material}\\\text{surface}}} \frac{(\text{Flux of }\vvec{P})}{(\text{distance})}
    }

    \cul[red]{Flux of $\vvec{P}$ on the material surface} appears as some charge density distributed \cul[red]{on the surface} of the material. 
    Therefore it is defined as the \bf{surface bound charge density} $\sigma_b$.
    \aleq{
        \Aboxed{
            \sigma_b(\vvec{r}) \ \defeq\ \vvec{P}(\vvec{r})\cdot \hhat{n}
        }
    }
    where $\hhat{n}$ is the (outward) unit normal vector on the material surface.
    Dot product with it to represent out-flux.

\end{itemize}


Finally, the expression of the ``charge densities form" is nothing more than saying that
the electric potential from a material is the result of the two kinds of charges distributions.
\aleq{
    V(\vvec{r})
    &= \inv{4\pi\epsilon_0} \underset{\substack{\substack{\text{Whole}\\\text{material}}}}{\iiint}
        \frac{\red{\rho_b(\vvec{r}')}}{\norm{\vvec{r}-\vvec{r}'}} \dd[3]{\vvec{r}'}
        \ +\ \inv{4\pi\epsilon_0} \underset{\substack{\substack{\text{Surface}\\\text{of material}}}}{\oiint}
        \frac{\red{\sigma_b(\vvec{r}')}}{\norm{\vvec{r}-\vvec{r}'}} \dd[2]{\vvec{r}'}\\[1ex]
        %
    &= \ \ \,\qty(\mstack{\text{Contribution}\\\text{by charges inside}\\\text{the material}})
        \ \ +\ \  \qty(\mstack{\text{Contribution}\\\text{by charges on}\\\text{material's surface}})
}

\begin{notation}[The boring derivation:]
    \begin{enumerate}
        \item Deriving the ``charge densities" form begins with a vector calculus identity:
        \addArrow[blue]{potential1}{(-20ex,-4ex)}{\scriptsize Note: Differentiation\\[-1ex]\scriptsize is w.r.t. $x',y',z'$}
        {(0,-1ex)}{(3ex,-1ex)}
        \addBentArrow[blue]{potential2}{(5ex,2ex)}{\scriptsize This part is just a unit vector}
        {(0,4ex)}{(9.5ex,-0.7ex)}
        \vskip -3em
        \aleq{
            \grad \qty(\inv{\norm{\vvec{r}-\vvec{r}'}}) 
            &= \qty(\hhat{x}\pdvv{\tkm{potential1}\cul[blue]{x'}}+\hhat{y}\pdvv{\cul[blue]{y'}}+\hhat{z}\pdvv{\cul[blue]{z'}}) 
                \qty(\inv{\sqrt{(x - x')^2 + (y - y')^2 + (z - z')^2}})\\[0.5ex]
            &= \frac{\hhat{x}(x-x')+\hhat{y}(y-y')+\hhat{z}(z-z')}{\qty[(x - x')^2 + (y - y')^2 + (z - z')^2]^{\frac{3}{2}}}\\[0.5ex]
            &= \frac{\vvec{r}-\vvec{r}'}{\norm{\vvec{r}-\vvec{r}'}^3}\\
            &= \inv{\norm{\vvec{r}-\vvec{r}'}^2}\,\tkn{potential2}{\cul[blue]{\qty(\frac{\vvec{r}-\vvec{r}'}{\norm{\vvec{r}-\vvec{r}'}})}}
            \qquad\quad \red{\qty(\substack{\text{This is basically }\dv{r'}\qty(\inv{r-r'}) = \inv{(r-r')^2}\\ \text{ but in vector version}})}
        }

        \item So the potential formula can be rewritten as:
        \aleq{
            V(\vvec{r})
            &= \inv{4\pi\epsilon_0} {\iiint}
            \frac{\vvec{P}(\vvec{r}')}{\norm{\vvec{r}-\vvec{r}'}^2} \cdot \qty(\frac{\vvec{r}-\vvec{r}'}{\norm{\vvec{r}-\vvec{r}'}}) \dd[3]{\vvec{r}'} \\[1ex]
            &= \inv{4\pi\epsilon_0} {\iiint}
                \cus[blue]{\vvec{P}(\vvec{r}') \cdot \grad \qty(\inv{\norm{\vvec{r}-\vvec{r}'}})}{\scriptsize \vvec{G}\,\cdot\,\grad f} \dd[3]{\vvec{r}'}\\[0.5ex]
            &= \inv{4\pi\epsilon_0} {\iiint}
                \cus[blue]{\div \qty(\frac{\vvec{P}(\vvec{r}')}{\norm{\vvec{r}-\vvec{r}'}})}{\scriptsize \div(f\vvec{G})} 
                - \cus[blue]{\frac{\div \vvec{P}(\vvec{r}')}{\norm{\vvec{r}-\vvec{r}'}}}{\scriptsize -f(\div \vvec{G})} \dd[3]{\vvec{r}'}
        }
        \vskip -1em
        Here used the product rule of divergence \fbox{$\div{(f\vvec{G})} = f(\div\vvec{G}) + (\grad f)\cdot \vvec{G}$}, 
        where $f$ is a scalar function (like $\inv{\norm{\vvec{r}-\vvec{r}'}}$) and $\vvec{G}$ is a vector function (like $\vvec{P}$).

        \item Finally use divergence theorem to convert the \nth{1} term's volume integral into a surface integral over the volume's surface:
        \addArrow[blue]{potential3}{(1ex,5.5ex)}{\scriptsize Divergence theorem}
        {(0,2.5ex)}{(-9ex,-2.5ex)}
        \aleq{
            V(\vvec{r})
            &= \inv{4\pi\epsilon_0} \cul[blue]{{\iiint}
                \div \qty(\frac{\vvec{P}(\vvec{r}')}{\norm{\vvec{r}-\vvec{r}'}}) \dd[3]{\vvec{r}'}}
                \ -\  \inv{4\pi\epsilon_0} {\iiint}\frac{\div \vvec{P}(\vvec{r}')}{\norm{\vvec{r}-\vvec{r}'}} \dd[3]{\vvec{r}'}\\[2.5ex]
            &= \inv{4\pi\epsilon_0} \cul[blue]{{\oiint}\,
                \frac{\vvec{P}(\vvec{r}')}{\norm{\vvec{r}-\vvec{r}'}} \tkn{potential3}{\cdot} \dd[2]{\vvec{r}'}}
                \ -\  \inv{4\pi\epsilon_0} {\iiint}\frac{\div \vvec{P}(\vvec{r}')}{\norm{\vvec{r}-\vvec{r}'}} \dd[3]{\vvec{r}'}\\[1.5ex]
            &=\ \qty(\mstack{\text{Contribution}\\\text{from material's}\\\text{surface}})
                \ \ +\ \  \qty(\mstack{\text{Contribution}\\\text{from inside}\\\text{the material}})
        }

    \end{enumerate}
    Remind that we are integrating regions where dipoles exist. 
    So this volume integral corresponds to the whole material and surface integral corresponds to only the surface of the material.

    
    
\end{notation}



%%%%%%%%%%%%%%
\subsubsection{Visualization}

The bound charge densities are related to $\vvec{P}$ pretty much like normal charge density are related to $\vvec{E}$ in Gauss's law.

\begin{itemize}
    \item \bf{\ul{Volume bound charge}}: 
    By circling the regions with more dipole heads and treat the circle like a Gaussian surface, 
    we can see
    \aleq{
        \Aboxed{
            \mstack{\text{More dipole}\\\text{pointing into the circle} }
            \qquad\Leftrightarrow\qquad 
            \mstack{\text{More positive charge}\\\text{contained in the circle}}
        }
    }
    
    \insertFig{contain region dipole head}

    Recall that we can use sign of flux to identify positions with converging/diverging vector field,
    and divergence operator $\div$ is equivalent to finding flux per volume. 
    \aleq{
        \qty(\mstack{\text{Bound charge}\\\text{volume density}}) 
        \ \sim\ \rho_b = -\div \vvec{P} \ \sim\  
        \qty(\mstack{\text{\red{In}-flux of dipole}\\\text{per volume}} )
    }

    \vskip 2ex
    \item \bf{\ul{Surface bound charge}}:
    On the material surface, we can see that
    \aleq{
        \Aboxed{
            \mstack{\text{More dipole}\\\text{pointing outward} }
            \qquad\Leftrightarrow\qquad 
            \mstack{\text{More positive charge}\\\text{on the material surface}}
        }
    }
    \insertFig{surface dipole head}

    As we have defined the normal vector $\hhat{n}$ of the material surface to be pointing outward, 
    the more dipoles in the same direction as $\hhat{n}$, the more positive charges on the surface.
    Therefore the definition of surface bound charge density involve dot product with $\hhat{n}$.
    \aleq{
        \qty(\mstack{\text{Bound charge}\\\text{surface density}}) 
        \ \sim\ \sigma_b = \vvec{P}\cdot \hhat{n} \ \sim\  
        \qty(\mstack{\text{Out-flux of dipole}\\\text{per surface area}} )
    }

\end{itemize}

\begin{notation}[Side note:]
    Although all dielectric materials are made of dipoles, 
    we almost never observe the $\vvec{P}$ field or bound charges on daily-life objects because
    \begin{itemize}
        \item Magnitude of a dipole is too small.
        Typical atomic separation $d \sim 10^{-10}$ m, 
        and so a dipole moment is of magnitude $\vvec{p} = q\vvec{d} \sim 1.6\times 10^{-29}$ C$\cdot$m.

        \item The dipoles are usually randomly arranged, 
        or in patterns such that their effects cancel out.
        Especially when the material is large.
    \end{itemize}

    To most material, \red{polarization effect is neligible unless it is placed under an external E-field},
    which forces its dipoles to align and ``stretch" them harder (increase the $\vvec{d}$ in dipole moment).
\end{notation}



\linesep
% Section %%%%%%%%%%%%%%%%%%%%%%%%%%%%%%%%%%%%%%%%%%%%%%%%%%%%
\section{Material under External E-field}

Now we study what will happen when a material is polarized by an external E-field:
\begin{enumerate}
    \item External E-field acts on material, 
    aligns the dipoles and strengthens their dipole moments.
    
    \item New dipoles alignment create a new E-field, which also affects surrounding dipoles.
    
    \item Alignment continues, until it reaches an equilibrium between external field and interactions between dipoles.
    
\end{enumerate}

Theoretically, how the dipoles reponse to the external E-field is a complicated process 
when all the bondings and interactions are involved - it highly depends on the material's structure.
So the final dipole alignment might not always align with the external E-field! 

\insertFig{random spherical dipole + regular external E-field = irregular align ellipse dipole}

In a general theory about material's E-field response,
one may treat the polarization field $\vvec{P}$ (i.e. new dipole alignment) as some function to the total E-field $\vec{E}_\text{total}$,
\aleq{
    \vvec{P} \ =\ f(\vvec{E}_\text{total}) 
    \ \sim\  \cub[blue]{\vvec{a}^{(1)}_iE_i + \vvec{a}^{(2)}_{ij}E_iE_j + \vvec{a}^{(3)}_{ijk}E_iE_jE_k + \cdots}{\text{\scriptsize Like a Taylor expansion}}
}

This $f$ function \cul[red]{denotes a theoretical model that we have choosen} to investigate the material.
And the \cul[red]{model parameters $\vvec{a}^{(1)}, \vvec{a}^{(2)}, \vvec{a}^{(3)}$... shall be determined from experiment}.

%%%%%%%%%%%%%%
\subsection{Special Case: Linear Dielectric}

Linear dielectric is the simplest model of E-field reponse - where the dipoles are completely free to rotate,
such that new dipole alignment is directly proportional to the external E-field's magnitude and direction.
\cul[red]{This assumption is applicable to most daily life materials}.

\insertFig{dipoles uniform under external E-field}

The linear model only has 1 parameter to be determined from experiments - 
the proportionality constant between $\vvec{P}$ and $\vvec{E}_\text{total}$.
\aleq{
    \Aboxed{
        \vvec{P} = \epsilon_0\chi_e \vvec{E}_\text{total}
    }
}

Here the $\chi_e$ is called \bf{electric susceptibility}, 
a pure number (no unit) whose value depends on the type of material.
$\epsilon_0$ is multiplied to match the units. 
\begin{itemize}
    \item $\chi_e = 0$ for vacuum (by defintion), because there are no dipoles in a vacuum.
    \item $\chi_e>0$ for most normal materials, because dipoles normally follow E-field's direction.
\end{itemize}



%%%%%%%%%%%%%%
\subsection{Describing External Field}

The total E-field is a result of the external E-field plus the field induced by dipole alignment.
After we have chosen the model about dipole alignment,
now we look at the external E-field.


%%%%%%%%%%%%%%
\subsubsection{Free Charge Distribution}

In order to create an external E-field around the material, 
we need a ``setup" to build an external source of charges. 
For example, apply voltage between two very large parallel plates:

\insertFig{free charge vs bound charge in dielectric setup}

The charges that build up in the setup are given the name \bf{free charges},
to distinguish from the bound charges (dipole alignment) in the material. 
The are ``free" because we can always control them by varying the setup,
making them known quantities in calculation.

\insertFig{adjust parallel plate voltage -> adjust free charge amount}

\aleq{
    \frac{\sigma_{\text{free}}}{\epsilon_0} = \norm{\vvec{E}} = \frac{V}{d}
}


For calculation, free charge densities are denoted like bound charges:

\begin{itemize}
    \item \bf{\ul{Volume free charge density}} $\rho_f(\vvec{r})$ - 
    Charge distribution \cul[red]{inside} the setup.

    \item \bf{\ul{Surface free charge density}} $\sigma_f(\vvec{r})$ - 
    Charge \cul[red]{on any surfaces} of the setup.
\end{itemize}

\vskip 1ex
In real practices, a pair of parallel plates is the most common setup to create external E-field.
So surface free charge is mostly all you need. No reason to make things complicated.



%%%%%%%%%%%%%%
\subsubsection{Displacement Field $\vvec{D}$}

Previously, we have relate bound charges to a vector field quantity - the polarization field $\vvec{P}$.
Similarly, free charges distributions can be related with another vector field.
From Gauss's law, 
\aleq{
    \epsilon_0\div \vvec{E}_\text{total}\ 
    &=\ \ \ \rho_\text{total}\ \  =\ \qty(\mstack{\text{All}\\\text{charges}})\\[0.5ex]
    &=\ \rho_f + \rho_b \ =\ \qty(\mstack{\text{Free charges}\\\text{on setup}}) + \qty(\mstack{\text{Bound charges}\\\text{on material}})\\[0.5ex]
    &=\ \rho_f + (-\div \vvec{P})\\[0.5ex]
    \div(\epsilon_0 \vvec{E}_\text{total} + \vvec{P}) \ &=\ \rho_f
}

Here we define the \bf{displacement field} $\vvec{D}$:
\aleq{
    \Acboxed[red]{
        \vvec{D} \ \defeq\ \epsilon_0 \vvec{E}_\text{total} + \vvec{P}\tkm{D_field}
    }
}
\addArrow[red]{D_field}{(5ex,0)}{\scriptsize This relation connects\\[-1ex]\scriptsize all 3 field quantities}{(2ex,0.7ex)}{(7ex,0)}

\vskip -1em
Such that it is related to the free charge density by:
\aleq{
    \Aboxed{
        \bcase{
            \rho_f(\vvec{r}) \ &\defeq\ \div \vvec{D}(\vvec{r}) \\[0.5ex]
            \sigma_f(\vvec{r}) \ &\defeq\ \vvec{D}(\vvec{r})\cdot \hhat{n}
        }
    }
}
where $\hhat{n}$ is the (outward) unit normal vector on the equipment surface.
Dot product with it to represent out-flux.

Notice the relations of $\vvec{D}$ with free charges are almost identical to $\vvec{E}$ with total charges in Gauss's law.
\red{In the special case of linear dielectric}, 
we can solve for $\vvec{D}$ from the given free charges like the Gauss's law integral form or with Coulomb's law.
\aleq{
    Q_f &= \oiint \vvec{D}(\vvec{r})\cdot \dd{\vvec{s}}\\[1ex]
    &= \oiint \tkn{gauss_dot}{\cul[green]{\norm{\vvec{D}}\norm{\dd{\vvec{s}}}\cos\theta}}\\[1ex]
    &= \tkn{gauss_E}{\cul[red]{\norm{\vvec{D}}}}\ \tkn{gauss_theta}{\cul[blue]{\cos\theta}}\ \oiint \norm{\dd{\vvec{s}}}\\[2em]
    &= \norm{\vvec{D}}\ \cos\theta\ (\text{Total surface area})\\[2ex]
    \Aboxed{
        \norm{\vvec{D}} &= \frac{Q_f}{(\text{Total surface area})\cos\theta}
    }
}
\addArrow[green]{gauss_dot}{(5ex,0)}
{\scriptsize Just dot product\\[-1ex]\scriptsize $\vvec{a}\cdot\vvec{b}=\norm{\vvec{a}}\norm{\vvec{b}}\cos\theta$}
{(8ex,0)}{(6ex,0)}
\addBentArrow[red]{gauss_E}{(-8ex,-3.5ex)}
{\scriptsize Same magnitude everywhere\\[-1ex]\scriptsize Can move out of integral}
{(0,-1.5ex)}{(-8.5ex,1ex)}
\addBentArrow[blue]{gauss_theta}{(8ex,-3.5ex)}
{\scriptsize Form same angle everywhere\\[-1ex]\scriptsize Can move out of integral}
{(0,-1.5ex)}{(8.5ex,1ex)}

But note that \red{this can be wrong for other models of dielectrics}.


\begin{notation}[Side Note:]
    Although $\vvec{D}$ forms a PDE relationship to free charge just like $\vvec{E}_\text{total}$ with total charge,
    we cannot solve for $\vvec{D}$ exactly by free charge if we don't know the material's model. 
    This is because:
    \begin{itemize}
        \item $\vvec{E}_\text{total}$ is guarenteed to be ``curl-less" in electrostatics, 
        such that we can define a potential function $V$ and solve for $\vvec{E}_\text{total}$ uniquely.

        \item $\vvec{P}$ may not be ``curl-less" since the dipole arrangement can be arbituary, 
        e.g. if they arrange into some vortex-like pattern.
        Then by $\vvec{D} = \epsilon_0\vvec{E}_\text{total} + \vvec{P}$,
        there is no guarenteed that $\vvec{D}$ is ``curl-less" either.
    \end{itemize} 

    \insertFig{vortex dipole arrangement -> P field not curl 0}
  
    But in the \cul[red]{special case of linear dielectric},
    \aleq{
        \cbox{\mstack{\vvec{P} \text{ aligns}\\[0.2ex]\text{with }\vvec{E}_\text{total}}}
        \quad\Rightarrow\quad
        \cbox{\mstack{\vvec{P} \text{ is}\\[1ex]\text{``curl-less"}}}
        \quad\Rightarrow\quad
        \cbox{\mstack{\vvec{D} = \epsilon_0\vvec{E}_\text{total} + \vvec{P}\\[0.5ex]\text{is also ``curl-less"}}}
    }

    So it is fine to solve for $\vvec{D}$ like normal Gauss's law problems for linear dielectric.

    
\end{notation}


%%%%%%%%%%%%%%
\subsection{Measurement in Practice}

In problems that involve dielectric, 
\begin{itemize}
    \item The material's polarization (dipole alignment) is the cause of all troubles.
    But usually we are only told which material model we need to apply. 
    \it{(E.g. Given a linear dielectric ...)}

    \item Actual calculation requires measurements from equipment setup. 
    In electrostatic experiments, usually the control/measure-able parameters are
    \begin{itemize}
        \item \ul{Applied voltage} - Directly connect with a power source. All you need is a voltmeter.
        \item \ul{Free charge} - If the amount of charge is manually placed on the setup.
    \end{itemize}
\end{itemize}

\insertFig{control set up with voltage or free charge procedure}


Note that 
\begin{itemize}
    \item Applied voltage can be used to determine to total E-field $\vvec{E}_\text{total}$
    \item Free charge can be used to determine to displacement field $\vvec{D}$.
\end{itemize}

Once we have chosen a model of the material, we can connect the two measurables by:
\aleq{
    \vvec{D} = \epsilon_0\vvec{E}_\text{total} + f(\vvec{E}_\text{total})
}

Then the model parameters in $f(\cdots)$ can be determined by experiments. 
But in the special case of linear dielectric, 
since there is only 1 constant parameter $\chi_e$ for each material, 
\aleq{
    \vvec{D} &= \epsilon_0\vvec{E}_\text{total} + \epsilon_0\chi_e\vvec{E}_\text{total}\\
    &= \epsilon_0(1+\chi_e)\vvec{E}_\text{total}\\
    &= (\text{A constant})\cdot \vvec{E}_\text{total}
}

We usually use this new constant to represent the material's response to E-field,
by defining
\aleq{
    \boxed{
        \epsilon \ \defeq\ \epsilon_0(1+\chi_e)   \ \defeq\ \epsilon_0\epsilon_r 
    }
    \qquad\text{such that}\qquad
    \vvec{D} = \epsilon\vvec{E}_\text{total}
}
where
\begin{itemize}
    \item $\epsilon$ = \bf{Absolute permittivity}, or \bf{dielectric constant}. Has the same unit as $\epsilon_0$.

    \item $\epsilon_r$ = \bf{Relative permittivity}. A pure number that normally $>1$. 
    Material handbooks usually tabulate values of $\epsilon_r$ of different materials.

\end{itemize}

Material scientists prefer using $\epsilon$ instead of $\chi_e$ is likely for convenience - 
since $\vvec{D}$ and $\vvec{E}_\text{total}$ are the measureables (through free charge and voltage),
the slope in a $\norm{\vvec{D}}$ v.s. $\norm{\vvec{E}_\text{total}}$ graph immediately gives the value of $\epsilon$.
\it{Too lazy to convert it back to $\chi_e$.}

\insertFig{D v.s. E plot -> slope = epsilon}



%%%%%%%%%%%%%%
\subsection{Calculation Example}

Here are some examples of questions you may see in the chapter of linear dielectric.
In principle, one of the information about $\vvec{D}$ (free charge) or $\vvec{E}_\text{total}$ (voltage) will be given,
then you are asked to find the other, and so as $\vvec{P}$ and bound charges.

\begin{example}
    Consider two pieces of material,
    each with relativity permittivity $\epsilon_1$,$\epsilon_2$,
    being placed between two large parallel plates.
    The plates are initially charged up to a surface density $\pm\sigma$ before putting in the materials. 
    Assume the plates to be large enough such that the charged density and total E-field are uniform.

    \insertFig{material btw parallel plate}

    \begin{enumerate}
        \item A good start is to label all the charges locations, and if they are free/bounded.
        
        \insertFig{label free/bound charge}

        \item Always remember that $\vvec{D}$ is only related to the free charges. 
        In this configuration, $\pm \sigma$ is the exact amount of free charge density on the plates since they are the manually placed in.
        \cul[red]{This $\vvec{D}$ is the same everywhere between the plates.} 
        \aleq{
            \norm{\vvec{D}} = \sigma_f = \sigma
        }

        \insertFig{only free charge and D btw parallel plate}

        \item According to the linear dielectric model, 
        E-field is proportional to the $\vvec{D}$ by the dielectric constant at that location, 
        i.e. $\vvec{D}=\epsilon\vvec{E}$.
        Because we have the same $\vvec{D}$ everywhere,
        \begin{itemize}
            \item In vacuum, $\norm{\vvec{E}} = \dfrac{\norm{\vvec{D}}}{\epsilon_0} = \dfrac{\sigma}{\epsilon_0}$
            \item In material 1, $\norm{\vvec{E}} = \dfrac{\norm{\vvec{D}}}{\epsilon_1} = \dfrac{\sigma}{\epsilon_1}$
            \item In material 2, $\norm{\vvec{E}} = \dfrac{\norm{\vvec{D}}}{\epsilon_2} = \dfrac{\sigma}{\epsilon_2}$
        \end{itemize}

        \insertFig{different E at different region}

        \item Then we can find $\vvec{P}$ by $\vvec{P}=\vvec{D}-\epsilon_0\vvec{E}$:
        \begin{itemize}
            \item In vacuum, $\norm{\vvec{P}} = \norm{\vvec{D}} - \norm{\vvec{D}} = 0$\quad (Obviously)
            \item In material 1, $\norm{\vvec{P}} = \norm{\vvec{D}} - \epsilon_0\norm{\vvec{E}} = \norm{\vvec{D}} - \epsilon_0 \cdot \dfrac{\norm{\vvec{D}}}{\epsilon_1} = \qty(1 - \dfrac{\epsilon_0}{\epsilon_1})\sigma$
            \item In material 2, $\norm{\vvec{P}} = \norm{\vvec{D}} - \epsilon_0\norm{\vvec{E}} = \norm{\vvec{D}} - \epsilon_0 \cdot \dfrac{\norm{\vvec{D}}}{\epsilon_2} = \qty(1 - \dfrac{\epsilon_0}{\epsilon_2})\sigma$
        \end{itemize}

        According to $\vvec{P}\cdot\hhat{n} = \sigma_b$, 
        the $\norm{\vvec{P}}$ on each material are also the amount of surface bound charge densities on each of the materials' surfaces.

        \insertFig{different bound charge}

    \end{enumerate}

\end{example}



\begin{example}
    Consider two uniform spherical shells of radius $R_1$,$R_2$ connected by a voltage source $V$.
    A layer of material with dielectric constant $\epsilon$ and thickness $w$ is covering the inner shell. 

    \insertFig{spherical shell dielectric setup}

    \begin{enumerate}
        \item This time we are only given the applied voltage but not the free charge.
        However we can always tell that there must be some free charges on the shells. 
        Let the total free charges to be $\pm Q_f$.

        \insertFig{label free charge on spherical shell}

        \item By symmetry, $\vvec{D}$ must be uniform and radially outward.
        We can apply Gauss's law integral form for $\vvec{D}$:
        \aleq{
            Q_f = \oiint \vvec{D} \cdot \dd{\vvec{s}}
            = \norm{\vvec{D}} \cdot 4\pi r^2 
            \qquad\Rightarrow\qquad
            \norm{\vvec{D}} = \frac{Q_f}{4\pi r^2}
        }

        This is the $\vvec{D}$ everywhere between the shells, which is function to radial distance $r$.

        \item According to the linear dielectric model, 
        E-field is proportional to the $\vvec{D}$ by the dielectric constant at that location.
        \begin{itemize}
            \item In the dielectric layer, $\norm{\vvec{E}} = \dfrac{\norm{\vvec{D}}}{\epsilon} = \dfrac{Q_f}{4\pi \epsilon r^2}$
            \item In the vacuum layer, $\norm{\vvec{E}} = \dfrac{\norm{\vvec{D}}}{\epsilon_0} = \dfrac{Q_f}{4\pi \epsilon_0 r^2}$
        \end{itemize}

        \item Now we can find the free charge by relating the applied voltage $V$ to the E-field by integrating along radial direction:
        \aleq{
            V &= \int \vvec{E} \cdot \dd{\vvec{r}} \\[1ex]
            &= \int_{R_1}^{R_1+w} \frac{Q_f}{4\pi \epsilon r^2} \dd{r} \ +\ \int_{R_1+w}^{R_2} \frac{Q_f}{4\pi \epsilon_0 r^2} \dd{r} \\[1ex]
            &= \frac{Q_f}{4\pi \epsilon} \qty(\frac{1}{R_1} - \frac{1}{R_1+w}) \ +\ \frac{Q_f}{4\pi \epsilon_0} \qty(\frac{1}{R_1+w} - \frac{1}{R_2})\\[1ex]
            Q_f &= \frac{4\pi V}{\frac{1}{\epsilon}(\frac{1}{R_1} - \frac{1}{R_1+w}) + \frac{1}{\epsilon_0}(\frac{1}{R_1+w} - \frac{1}{R_2})} 
        }

        \item Also $\vvec{P}$ by $\vvec{P}=\vvec{D}-\epsilon_0\vvec{E}$:
        \begin{itemize}
            \item In the dielectric layer, $\norm{\vvec{P}} = \norm{\vvec{D}} - \epsilon_0\norm{\vvec{E}} = \norm{\vvec{D}} - \epsilon_0 \cdot \dfrac{\norm{\vvec{D}}}{\epsilon} = \qty(1 - \dfrac{\epsilon_0}{\epsilon})\dfrac{Q_f}{4\pi r^2}$
            \item In the vacuum layer, $\norm{\vvec{P}} = 0$\quad (Obviously)
        \end{itemize}

        \item This time $\vvec{P}$ is a function of $r$, so the surface bound charge \it{density} are different on the inner and outer surfaces.
        \begin{itemize}
            \item On inner surface: $\sigma_b = \norm{\vvec{P}}_{(r=R_1)} = \qty(1 - \dfrac{\epsilon_0}{\epsilon})\dfrac{Q_f}{4\pi R_1^2}$
            \item On outer surface: $\sigma_b = \norm{\vvec{P}}_{(r=R_1+w)} = \qty(1 - \dfrac{\epsilon_0}{\epsilon})\dfrac{Q_f}{4\pi (R_1+w)^2}$
        \end{itemize}
        
        But notice that the total bound charge on each surface are the same. 
        \aleq{
            Q_b &= \sigma_b \cdot (4\pi r^2) =  \qty(1 - \dfrac{\epsilon_0}{\epsilon})Q_f
        }

        \insertFig{different bound charge density on spherical shell}

        Also, checking for volume bound charge in the material is more annoying.
        \aleq{
            \rho_b &= -\div \vvec{P} \\[1ex]
            &= -\frac{1}{r^2}\pdv{r}\qty(r^2 \norm{\vvec{P}}) \\[1ex]
            &= -\frac{1}{r^2}\pdv{r}\qty(r^2 \qty(1 - \dfrac{\epsilon_0}{\epsilon})\dfrac{Q_f}{4\pi r^2}) \\[1ex]
            &= 0
        } 
        
    \end{enumerate}

\end{example}


%%%
\theend
\end{document}