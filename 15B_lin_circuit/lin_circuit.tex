\documentclass[class=article, crop=false, 12pt]{standalone}
\usepackage[subpreambles=true]{standalone}
\usepackage{../.common/common}


\author{Tony Shing}
%\pretitle{Supplementary}

\topic{T15B (Electromagnetism)}
\title{Linear Circuit}

\version{2025} % leave blank for omitting

\begin{document}

\maketitle


\begin{overview}
    \begin{itemize}
        \item Kirchhoff's law
        \item 3 methods of solving circuits: ODE, phaser, impedance
        \item Advanced techniques for simplifying circuits
    \end{itemize}
\end{overview}



% content begins here
% Section %%%%%%%%%%%%%%%%%%%%%%%%%%%%%%%%%%%%%%%%%%%%%%%%%%%%
\section{Kirchhoff's Law}

The two Kirchhoff's laws are the fundamentals to every circuit-solving problem. 
\begin{itemize}
    \item Kirchhoff's Current Law (KCL)
    \item Kirchhoff's Voltage Law (KVL)
\end{itemize}



\linesep
% Section %%%%%%%%%%%%%%%%%%%%%%%%%%%%%%%%%%%%%%%%%%%%%%%%%%%%
\section{Methods of Solving Circuit}

There are 3 different methods commonly found in textbooks:
\begin{itemize}
    \item Solving ODEs
    \item Phasor Diagram
    \item Impedance method
\end{itemize}

Solving with ODEs is the most fundamental approach,
but it is also the most annoying. 
So electronic engineers developed the latter two methods to simplify their jobs.

%%%%%%%%%%%%%%
\subsection{System of ODEs}

%%%%%%%%%%%%%%
\subsubsection{R-C Circuit}

%%%%%%%%%%%%%%
\subsubsection{R-L Circuit}

%%%%%%%%%%%%%%
\subsubsection{R-C-L Circuit}


%%%%%%%%%%%%%%
\subsection{Phasor Diagram}


%%%%%%%%%%%%%%
\subsection{Equivalent Impedance}

\linesep
% Section %%%%%%%%%%%%%%%%%%%%%%%%%%%%%%%%%%%%%%%%%%%%%%%%%%%%
\section{Advanced Circuit Techniques}

Here introduces several techniques for simplifying circuits that are used by electronic engineers very commonly,
but are never taught to physics students (so it is easier to make you suffer).

%%%%%%%%%%%%%%
\subsection{Thevenin - Norton Equivalency}

%%%%%%%%%%%%%%
\subsubsection{Linear Circuit Blackbox}

%%%%%%%%%%%%%%
\subsubsection{The Equivalent Circuit}



%%%%%%%%%%%%%%
\subsection{Superposition Theorem}




%%%%%%%%%%%%%%
\subsection{Y-$\Delta$ Transform}

%%%
\theend
\end{document}